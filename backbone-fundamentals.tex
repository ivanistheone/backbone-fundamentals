\documentclass[9pt]{book}
\usepackage[T1]{fontenc}
\usepackage{lmodern}
\usepackage{amssymb,amsmath}
\usepackage{ifxetex,ifluatex}
\usepackage{fixltx2e} % provides \textsubscript
% use upquote if available, for straight quotes in verbatim environments
\IfFileExists{upquote.sty}{\usepackage{upquote}}{}
\ifnum 0\ifxetex 1\fi\ifluatex 1\fi=0 % if pdftex
  \usepackage[utf8]{inputenc}
\else % if luatex or xelatex
  \ifxetex
    \usepackage{mathspec}
    \usepackage{xltxtra,xunicode}
  \else
    \usepackage{fontspec}
  \fi
  \defaultfontfeatures{Mapping=tex-text,Scale=MatchLowercase}
  \newcommand{\euro}{€}
\fi
% use microtype if available
\IfFileExists{microtype.sty}{\usepackage{microtype}}{}
\usepackage[papersize={5.5in,8.5in}, 
headsep=0.3cm,tmargin=1.5cm, lmargin=2.25cm,rmargin=1.15cm,bmargin=1.3cm,footskip=0.4cm]{geometry}
\usepackage{color}
\usepackage{fancyvrb}
\fvset{commandchars=?\[\],fontfamily=courier,fontsize=\scriptsize} %, frame=single}
\newcommand{\VerbBar}{|}
\newcommand{\VERB}{\Verb[commandchars=\\\{\}]}
\DefineVerbatimEnvironment{Highlighting}{Verbatim}{commandchars=\\\{\}}
% Add ',fontsize=\small' for more characters per line
\newenvironment{Shaded}{}{}
\newcommand{\KeywordTok}[1]{\textcolor[rgb]{0.00,0.44,0.13}{\textbf{{#1}}}}
\newcommand{\DataTypeTok}[1]{\textcolor[rgb]{0.56,0.13,0.00}{{#1}}}
\newcommand{\DecValTok}[1]{\textcolor[rgb]{0.25,0.63,0.44}{{#1}}}
\newcommand{\BaseNTok}[1]{\textcolor[rgb]{0.25,0.63,0.44}{{#1}}}
\newcommand{\FloatTok}[1]{\textcolor[rgb]{0.25,0.63,0.44}{{#1}}}
\newcommand{\CharTok}[1]{\textcolor[rgb]{0.25,0.44,0.63}{{#1}}}
\newcommand{\StringTok}[1]{\textcolor[rgb]{0.25,0.44,0.63}{{#1}}}
\newcommand{\CommentTok}[1]{\textcolor[rgb]{0.38,0.63,0.69}{\textit{{#1}}}}
\newcommand{\OtherTok}[1]{\textcolor[rgb]{0.00,0.44,0.13}{{#1}}}
\newcommand{\AlertTok}[1]{\textcolor[rgb]{1.00,0.00,0.00}{\textbf{{#1}}}}
\newcommand{\FunctionTok}[1]{\textcolor[rgb]{0.02,0.16,0.49}{{#1}}}
\newcommand{\RegionMarkerTok}[1]{{#1}}
\newcommand{\ErrorTok}[1]{\textcolor[rgb]{1.00,0.00,0.00}{\textbf{{#1}}}}
\newcommand{\NormalTok}[1]{{#1}}
\usepackage{graphicx}
% Redefine \includegraphics so that, unless explicit options are
% given, the image width will not exceed the width of the page.
% Images get their normal width if they fit onto the page, but
% are scaled down if they would overflow the margins.
\makeatletter
\def\ScaleIfNeeded{%
  \ifdim\Gin@nat@width>\linewidth
    \linewidth
  \else
    \Gin@nat@width
  \fi
}
\makeatother
\let\Oldincludegraphics\includegraphics
{%
 \catcode`\@=11\relax%
 \gdef\includegraphics{\@ifnextchar[{\Oldincludegraphics}{\Oldincludegraphics[width=\ScaleIfNeeded]}}%
}%
\ifxetex
  \usepackage[setpagesize=false, % page size defined by xetex
              unicode=false, % unicode breaks when used with xetex
              xetex]{hyperref}
\else
  \usepackage[unicode=true]{hyperref}
\fi
\hypersetup{breaklinks=true,
            bookmarks=true,
            pdfauthor={},
            pdftitle={},
            colorlinks=true,
            citecolor=blue,
            urlcolor=blue,
            linkcolor=magenta,
            pdfborder={0 0 0}}
\urlstyle{same}  % don't use monospace font for urls
\setlength{\parindent}{0pt}
\setlength{\parskip}{6pt plus 2pt minus 1pt}
\setlength{\emergencystretch}{3em}  % prevent overfull lines
\setcounter{secnumdepth}{0}

\title{Developing Backbone.js Applications}
\author{Addy Osmani }
\date{July 7\textsuperscript{th}, 2014}

\begin{document}

\maketitle


{
\hypersetup{linkcolor=black}
\setcounter{tocdepth}{3}
\tableofcontents
}
\subsection{Prelude}\label{prelude}

\begin{figure}[htbp]
\centering
\includegraphics{img/logo.jpg}
\end{figure}

Not so long ago, ``data-rich web application'' was an oxymoron. Today,
these applications are everywhere and you need to know how to build
them.

Traditionally, web applications left the heavy-lifting of data to
servers that pushed HTML to the browser in complete page loads. The use
of client-side JavaScript was limited to improving the user experience.
Now this relationship has been inverted - client applications pull raw
data from the server and render it into the browser when and where it is
needed.

Think of the Ajax shopping cart which doesn't require a refresh on the
page when adding an item to your basket. Initially, jQuery became the
go-to library for this paradigm. Its nature was to make Ajax requests
then update text on the page and so on. However, this pattern with
jQuery revealed that we have implicit model data on the client side.
With the server no longer being the only place that knows about our item
count, it was a hint that there was a natural tension and pull of this
evolution.

The rise of arbitrary code on the client-side which can talk to the
server however it sees fit has meant an increase in client-side
complexity. Good architecture on the client has gone from an
afterthought to essential - you can't just hack together some jQuery
code and expect it to scale as your application grows. Most likely, you
would end up with a nightmarish tangle of UI callbacks entwined with
business logic, destined to be discarded by the poor soul who inherits
your code.

Thankfully, there are a growing number of JavaScript libraries that can
help improve the structure and maintainability of your code, making it
easier to build ambitious interfaces without a great deal of effort.
\href{http://documentcloud.github.com/backbone/}{Backbone.js} has
quickly become one of the most popular open-source solutions to these
issues and in this book we will take you through an in-depth walkthrough
of it.

Begin with the fundamentals, work your way through the exercises, and
learn how to build an application that is both cleanly organized and
maintainable. If you are a developer looking to write code that can be
more easily read, structured, and extended - this guide can help.

Improving developer education is important to me, which is why this book
is released under a Creative Commons
Attribution-NonCommercial-ShareAlike 3.0 Unported
\href{http://creativecommons.org/licenses/by-nc-sa/3.0/}{license}. This
means you can purchase or grab a copy of the book for
\href{http://addyosmani.github.com/backbone-fundamentals/}{free} or help
to further
\href{https://github.com/addyosmani/backbone-fundamentals/}{improve} it.
Corrections to existing material are always welcome and I hope that
together we can provide the community with an up-to-date resource that
is of help.

My extended thanks go out to \href{https://github.com/jashkenas}{Jeremy
Ashkenas} and \href{http://www.documentcloud.org}{DocumentCloud} for
creating Backbone.js and
\href{https://github.com/addyosmani/backbone-fundamentals/contributors}{these}
members of the community for their assistance making this project far
better than I could have imagined.

\subsection{Target Audience}\label{target-audience}

This book is targeted at novice to intermediate developers wishing to
learn how to better structure their client-side code. An understanding
of JavaScript fundamentals is required to get the most out of it,
however we have tried to provide a basic description of these concepts
where possible.


\subsection{Target Version}\label{target-version}

Developing Backbone.js Applications targets Backbone.js 1.1.x (and
Underscore 1.6.x) and will actively attempt to stay up to date with more
recent versions of these libraries. Where possible, if you find using a
newer version of Backbone breaks an example, please consult the official
guide to \href{http://backbonejs.org/\#upgrading}{upgrading} as it
contains instructions for how to work around breaking changes.
StackOverflow also contains many excellent examples of how other users
are handling updating their code.

\subsection{Reading}\label{reading}

I assume your level of knowledge about JavaScript goes beyond the basics
and as such certain topics such as object literals are skipped. If you
need to learn more about the language, I am happy to suggest:

\begin{itemize}
\itemsep1pt\parskip0pt\parsep0pt
\item
  \href{http://eloquentjavascript.net/}{Eloquent JavaScript}
\item
  \href{http://shop.oreilly.com/product/9780596805531.do}{JavaScript:
  The Definitive Guide} by David Flanagan (O'Reilly)
\item
  \href{http://www.informit.com/store/effective-javascript-68-specific-ways-to-harness-the-9780321812186}{Effective
  JavaScript} by David Herman (Pearson)
\item
  \href{http://shop.oreilly.com/product/9780596517748.do}{JavaScript:
  The Good Parts} by Douglas Crockford (O'Reilly)
\item
  \href{http://www.amazon.com/Object-Oriented-Javascript-Stoyan-Stefanov/dp/1847194141}{Object-Oriented
  JavaScript} by Stoyan Stefanov (Packt Publishing)
\end{itemize}

\section{Introduction}\label{introduction}

Frank Lloyd Wright once said ``You can't make an architect. You can
however open the doors and windows toward the light as you see it.'' In
this book, I hope to shed some light on how to improve the structure of
your web applications, opening doors to what will hopefully be more
maintainable, readable applications in your future.

The goal of all architecture is to build something well; in our case, to
craft code that is enduring and delights both ourselves and the
developers who will maintain our code long after we are gone. We all
want our architecture to be simple, yet beautiful.

Modern JavaScript frameworks and libraries can bring structure and
organization to your projects, establishing a maintainable foundation
right from the start. They build on the trials and tribulations of
developers who have had to work around callback chaos similar to that
which you are facing now or may in the near future.

When developing applications using just jQuery, the piece missing is a
way to structure and organize your code. It's very easy to create a
JavaScript app that ends up a tangled mess of jQuery selectors and
callbacks, all desperately trying to keep data in sync between the HTML
for your UI, the logic in your JavaScript, and calls to your API for
data.

Without something to help tame the mess, you're likely to string
together a set of independent plugins and libraries to make up the
functionality or build everything yourself from scratch and have to
maintain it yourself. Backbone solves this problem for you, providing a
way to cleanly organize code, separating responsibilities into
recognizable pieces that are easy to maintain.

In ``Developing Backbone.js Applications,'' I and a number of other
experienced authors will show you how to improve your web application
structure using the popular JavaScript library, Backbone.js

\subsubsection{What Is MVC?}\label{what-is-mvc}

A number of modern JavaScript frameworks provide developers an easy path
to organizing their code using variations of a pattern known as MVC
(Model-View-Controller). MVC separates the concerns in an application
into three parts:

\begin{itemize}
\itemsep1pt\parskip0pt\parsep0pt
\item
  Models represent the domain-specific knowledge and data in an
  application. Think of this as being a `type' of data you can model ---
  like a User, Photo, or Todo note. Models can notify observers when
  their state changes.
\item
  Views typically constitute the user interface in an application (e.g.,
  markup and templates), but don't have to be. They observe Models, but
  don't directly communicate with them.
\item
  Controllers handle input (e.g., clicks, user actions) and update
  Models.
\end{itemize}

Thus, in an MVC application, user input is acted upon by Controllers
which update Models. Views observe Models and update the user interface
when changes occur.

JavaScript MVC frameworks don't always strictly follow the above
pattern. Some solutions (including Backbone.js) merge the responsibility
of the Controller into the View, while other approaches add additional
components into the mix.

For this reason we refer to such frameworks as following the MV*
pattern; that is, you're likely to have a Model and a View, but a
distinct Controller might not be present and other components may come
into play.

\subsubsection{What is Backbone.js?}\label{what-is-backbone.js}

\begin{figure}[htbp]
\centering
\includegraphics{img/backbonejsorg.jpg}
\end{figure}

Backbone.js is a lightweight JavaScript library that adds structure to
your client-side code. It makes it easy to manage and decouple concerns
in your application, leaving you with code that is more maintainable in
the long term.

Developers commonly use libraries like Backbone.js to create single-page
applications (SPAs). SPAs are web applications that load into the
browser and then react to data changes on the client side without
requiring complete page refreshes from the server.

Backbone is mature, popular, and has both a vibrant developer community
as well as a wealth of plugins and extensions available that build upon
it. It has been used to create non-trivial applications by companies
such as Disqus, Walmart, SoundCloud and LinkedIn.

Backbone focuses on giving you helpful methods for querying and
manipulating your data rather than re-inventing the JavaScript object
model. It's a library, rather than a framework, that plays well with
others and scales well, from embedded widgets to large-scale
applications.

As it's small, there is also less your users have to download on mobile
or slower connections. The entire Backbone source can be read and
understood in just a few hours.

\subsubsection{When Do I Need A JavaScript MVC
Framework?}\label{when-do-i-need-a-javascript-mvc-framework}

When building a single-page application using JavaScript, whether it
involves a complex user interface or is simply trying to reduce the
number of HTTP requests required for new Views, you will likely find
yourself inventing many of the pieces that make up an MV* framework.

At the outset, it isn't terribly difficult to write your own application
framework that offers some opinionated way to avoid spaghetti code;
however, to say that it is equally as trivial to write something as
robust as Backbone would be a grossly incorrect assumption.

There's a lot more that goes into structuring an application than tying
together a DOM manipulation library, templating, and routing. Mature MV*
frameworks typically include not only the pieces you would find yourself
writing, but also include solutions to problems you'll find yourself
running into later on down the road. This is a time-saver that you
shouldn't underestimate the value of.

So, where will you likely need an MV* framework and where won't you?

If you're writing an application where much of the heavy lifting for
view rendering and data manipulation will be occurring in the browser,
you may find a JavaScript MV* framework useful. Examples of applications
that fall into this category are GMail, NewsBlur and the LinkedIn mobile
app.

These types of applications typically download a single payload
containing all the scripts, stylesheets, and markup users need for
common tasks and then perform a lot of additional behavior in the
background. For instance, it's trivial to switch between reading an
email or document to writing one without sending a new page request to
the server.

If, however, you're building an application that still relies on the
server for most of the heavy-lifting of page/view rendering and you're
just using a little JavaScript or jQuery to make things more
interactive, an MV* framework may be overkill. There certainly are
complex Web applications where the partial rendering of views can be
coupled with a single-page application effectively, but for everything
else, you may find yourself better sticking to a simpler setup.

Maturity in software (framework) development isn't simply about how long
a framework has been around. It's about how solid the framework is and
more importantly how well it's evolved to fill its role. Has it become
more effective at solving common problems? Does it continue to improve
as developers build larger and more complex applications with it?

\subsubsection{Why Consider
Backbone.js?}\label{why-consider-backbone.js}

Backbone provides a minimal set of data-structuring (Models,
Collections) and user interface (Views, URLs) primitives that are
helpful when building dynamic applications using JavaScript. It's not
opinionated, meaning you have the freedom and flexibility to build the
best experience for your web application how you see fit. You can either
use the prescribed architecture it offers out of the box or extend it to
meet your requirements.

The library doesn't focus on widgets or replacing the way you structure
objects - it just supplies you with utilities for manipulating and
querying data in your application. It also doesn't prescribe a specific
template engine - while you are free to use the Micro-templating offered
by Underscore.js (one of its dependencies), views can bind to HTML
constructed using your templating solution of choice.

Looking at the \href{http://backbonejs.org/\#examples}{large} number of
applications built with Backbone, it's clear that it scales well.
Backbone also works quite well with other libraries, meaning you can
embed Backbone widgets in an application written with AngularJS, use it
with TypeScript, or just use an individual class (like Models) as a data
backer for simpler apps.

There are no performance drawbacks to using Backbone to structure your
application. It avoids run loops, two-way binding, and constant polling
of your data structures for updates and tries to keep things simple
where possible. That said, should you wish to go against the grain, you
can of course implement such things on top of it. Backbone won't stop
you.

With a vibrant community of plugin and extension authors, there's a
likelihood that if you're looking to achieve some behavior Backbone is
lacking, a complementary project exists that works well with it. This is
made simpler by Backbone offering literate documentation of its source
code, allowing anyone an opportunity to easily understand what is going
on behind the scenes.

Having been refined over two and a half years of development, Backbone
is a mature library that will continue to offer a minimalist solution
for building better web applications. I regularly use it and hope that
you find it as useful an addition to your toolbelt as I have.

\subsubsection{Setting Expectations}\label{setting-expectations}

The goal of this book is to create an authoritative and centralized
repository of information that can help those developing real-world apps
with Backbone. If you come across a section or topic which you think
could be improved or expanded on, please feel free to submit an issue
(or better yet, a pull-request) on the book's
\href{https://github.com/addyosmani/backbone-fundamentals}{GitHub site}.
It won't take long and you'll be helping other developers avoid the
problems you ran into.

Topics will include MVC theory and how to build applications using
Backbone's Models, Views, Collections, and Routers. I'll also be taking
you through advanced topics like modular development with Backbone.js
and AMD (via RequireJS), solutions to common problems like nested views,
how to solve routing problems with Backbone and jQuery Mobile, and much
more.

Here is a peek at what you will be learning in each chapter:

Chapter 2, Fundamentals traces the history of the MVC design pattern and
introduces how it is implemented by Backbone.js and other JavaScript
frameworks.

Chapter 3, Backbone Basics covers the major features of the Backbone.js
core and the technologies and techniques you will need to know in order
to apply it.

Chapter 4, Exercise 1: Todos - Your First Backbone.js App takes you
step-by-step through development of a simple client-side Todo List
application.

Chapter 5, Exercise 2: Book Library - Your First RESTful Backbone.js App
walks you through development of a Book Library application which
persists its model to a server using a REST API.

Chapter 6, Backbone Extensions describes Backbone.Marionette and Thorax,
two extension frameworks which add features to Backbone.js that are
useful for developing large-scale applications.

Chapter 7, Common Problems and Solutions reviews common issues you may
encounter when using Backbone.js and ways of addressing them.

Chapter 8, Modular Development looks at how AMD modules and RequireJS
can be used to modularize your code.

Chapter 9, Exercise 3: Todos - Your First Modular Backbone + RequireJS
App takes you through rewriting the app created in Exercise 1 to be more
modular with the help of RequireJS.

Chapter 10, Paginating Backbone Requests \& Collections walks through
how to use the Backbone.Paginator plugin to paginate data for your
Collections.

Chapter 11, Backbone Boilerplate And Grunt BBB introduces powerful tools
you can use to bootstrap a new Backbone.js application with boilerplate
code.

Chapter 12, Mobile Applications addresses the issues that arise when
using Backbone with jQuery Mobile.

Chapter 13, Jasmine covers how to unit test Backbone code using the
Jasmine test framework.

Chapter 14, QUnit discusses how to use QUnit for unit testing.

Chapter 15, SinonJS discusses how to use SinonJS for unit testing your
Backbone apps.

Chapter 16, Resources provides references to additional Backbone-related
resources.

Chapter 17, Conclusions wraps up our tour through the world of
Backbone.js development.

Chapter 18, Appendix returns to our design pattern discussion by
contrasting MVC with the Model-View-Presenter (MVP) pattern and examines
how Backbone.js relates to both. A walkthrough of writing a
Backbone-like library from scratch and other topics are also covered.

\section{Fundamentals}\label{fundamentals}

Design patterns are proven solutions to common development problems that
can help us improve the organization and structure of our applications.
By using patterns, we benefit from the collective experience of skilled
developers who have repeatedly solved similar problems.

Historically, developers creating desktop and server-class applications
have had a wealth of design patterns available for them to lean on, but
it's only been in the past few years that such patterns have been
applied to client-side development.

In this chapter, we're going to explore the evolution of the
Model-View-Controller (MVC) design pattern and get our first look at how
Backbone.js allows us to apply this pattern to client-side development.

\subsection{MVC}\label{mvc}

MVC is an architectural design pattern that encourages improved
application organization through a separation of concerns. It enforces
the isolation of business data (Models) from user interfaces (Views),
with a third component (Controllers) traditionally managing logic,
user-input, and coordination of Models and Views. The pattern was
originally designed by
\href{http://en.wikipedia.org/wiki/Trygve_Reenskaug}{Trygve Reenskaug}
while working on Smalltalk-80 (1979), where it was initially called
Model-View-Controller-Editor. MVC was described in depth in
\href{http://www.amazon.co.uk/Design-patterns-elements-reusable-object-oriented/dp/0201633612}{``Design
Patterns: Elements of Reusable Object-Oriented Software''} (The ``GoF''
or ``Gang of Four'' book) in 1994, which played a role in popularizing
its use.

\subsubsection{Smalltalk-80 MVC}\label{smalltalk-80-mvc}

It's important to understand the issues that the original MVC pattern
was aiming to solve as it has changed quite heavily since the days of
its origin. Back in the 70's, graphical user-interfaces were few and far
between. An approach known as
\href{http://martinfowler.com/eaaDev/uiArchs.html}{Separated
Presentation} began to be used as a means to make a clear division
between domain objects which modeled concepts in the real world (e.g., a
photo, a person) and the presentation objects which were rendered to the
user's screen.

The Smalltalk-80 implementation of MVC took this concept further and had
an objective of separating out the application logic from the user
interface. The idea was that decoupling these parts of the application
would also allow the reuse of Models for other interfaces in the
application. There are some interesting points worth noting about
Smalltalk-80's MVC architecture:

\begin{itemize}
\itemsep1pt\parskip0pt\parsep0pt
\item
  A Domain element was known as a Model and was ignorant of the
  user-interface (Views and Controllers)
\item
  Presentation was taken care of by the View and the Controller, but
  there wasn't just a single View and Controller. A View-Controller pair
  was required for each element being displayed on the screen and so
  there was no true separation between them
\item
  The Controller's role in this pair was handling user input (such as
  key-presses and click events) and doing something sensible with them
\item
  The Observer pattern was used to update the View whenever the Model
  changed
\end{itemize}

Developers are sometimes surprised when they learn that the Observer
pattern (nowadays commonly implemented as a Publish/Subscribe system)
was included as a part of MVC's architecture decades ago. In
Smalltalk-80's MVC, the View and Controller both observe the Model:
anytime the Model changes, the Views react. A simple example of this is
an application backed by stock market data - for the application to show
real-time information, any change to the data in its Model should result
in the View being refreshed instantly.

Martin Fowler has done an excellent job of writing about the
\href{http://martinfowler.com/eaaDev/uiArchs.html}{origins} of MVC over
the years and if you are interested in further historical information
about Smalltalk-80's MVC, I recommend reading his work.

\subsubsection{MVC Applied To The Web}\label{mvc-applied-to-the-web}

The web heavily relies on the HTTP protocol, which is stateless. This
means that there is not a constantly open connection between the browser
and server; each request instantiates a new communication channel
between the two. Once the request initiator (e.g.~a browser) gets a
response the connection is closed. This fact creates a completely
different context when compared to the one of the operating systems on
which many of the original MVC ideas were developed. The MVC
implementation has to conform to the web context.

An example of a server-side web application framework which tries to
apply MVC to the web context is
\href{http://guides.rubyonrails.org/}{Ruby On Rails}.

\begin{figure}[htbp]
\centering
\includegraphics{img/rails_mvc.png}
\end{figure}

At its core are the three MVC components we would expect - the Model,
View and Controller architecture. In Rails:

\begin{itemize}
\itemsep1pt\parskip0pt\parsep0pt
\item
  Models represent the data in an application and are typically used to
  manage rules for interacting with a specific database table. You
  generally have one table corresponding to one model with much of your
  application's business logic living within these models.
\item
  Views represent your user interface, often taking the form of HTML
  that will be sent down to the browser. They're used to present
  application data to anything making requests from your application.
\item
  Controllers offer the glue between models and views. Their
  responsibility is to process requests from the browser, ask your
  models for data and then supply this data to views so that they may be
  presented to the browser.
\end{itemize}

Although there's a clear separation of concerns that is MVC-like in
Rails, it is actually using a different pattern called
\href{http://en.wikipedia.org/wiki/Model2}{Model2}. One reason for this
is that Rails does not notify views from the model or controllers - it
just passes model data directly to the view.

That said, even for the server-side workflow of receiving a request from
a URL, baking out an HTML page as a response and separating your
business logic from your interface has many benefits. In the same way
that keeping your UI cleanly separate from your database records is
useful in server-side frameworks, it's equally as useful to keep your UI
cleanly separated from your data models in JavaScript (as we will read
more about shortly).

Other server-side implementations of MVC (such as the PHP
\href{http://zend.com}{Zend} framework) also implement the
\href{http://en.wikipedia.org/wiki/Front_Controller_pattern}{Front
Controller} design pattern. This pattern layers an MVC stack behind a
single point of entry. This single point of entry means that all HTTP
requests (e.g., \texttt{http://www.example.com},
\texttt{http://www.example.com/whichever-page/}, etc.) are routed by the
server's configuration to the same handler, independent of the URI.

When the Front Controller receives an HTTP request it analyzes it and
decides which class (Controller) and method (Action) to invoke. The
selected Controller Action takes over and interacts with the appropriate
Model to fulfill the request. The Controller receives data back from the
Model, loads an appropriate View, injects the Model data into it, and
returns the response to the browser.

For example, let's say we have our blog on \texttt{www.example.com} and
we want to edit an article (with \texttt{id=43}) and request
\texttt{http://www.example.com/article/edit/43}:

On the server side, the Front Controller would analyze the URL and
invoke the Article Controller (corresponding to the \texttt{/article/}
part of the URI) and its Edit Action (corresponding to the
\texttt{/edit/} part of the URI). Within the Action there would be a
call to, let's say, the Articles Model and its
\texttt{Articles::getEntry(43)} method (43 corresponding to the
\texttt{/43} at the end of the URI). This would return the blog article
data from the database for editing. The Article Controller would then
load the (\texttt{article/edit}) View which would include logic for
injecting the article's data into a form suitable for editing its
content, title, and other (meta) data. Finally, the resulting HTML
response would be returned to the browser.

As you can imagine, a similar flow is necessary with POST requests after
we press a save button in a form. The POST action URI would look like
\texttt{/article/save/43}. The request would go through the same
Controller, but this time the Save Action would be invoked (due to the
\texttt{/save/} URI chunk), the Articles Model would save the edited
article to the database with \texttt{Articles::saveEntry(43)}, and the
browser would be redirected to the \texttt{/article/edit/43} URI for
further editing.

Finally, if the user requested \texttt{http://www.example.com/} the
Front Controller would invoke the default Controller and Action; e.g.,
the Index Controller and its Index action. Within Index Action there
would be a call to the Articles model and its
\texttt{Articles::getLastEntries(10)} method which would return the last
10 blog posts. The Controller would load the blog/index View which would
have basic logic for listing the blog posts.

The picture below shows this typical HTTP request/response lifecycle for
server-side MVC:

\begin{figure}[htbp]
\centering
\includegraphics{img/webmvcflow_bacic.png}
\end{figure}

The Server receives an HTTP request and routes it through a single entry
point. At that entry point, the Front Controller analyzes the request
and based on it invokes an Action of the appropriate Controller. This
process is called routing. The Action Model is asked to return and/or
save submitted data. The Model communicates with the data source (e.g.,
database or API). Once the Model completes its work it returns data to
the Controller which then loads the appropriate View. The View executes
presentation logic (loops through articles and prints titles, content,
etc.) using the supplied data. In the end, an HTTP response is returned
to the browser.

\subsubsection{Client-Side MVC \& Single Page
Apps}\label{client-side-mvc-single-page-apps}

Several
\href{http://radar.oreilly.com/2009/07/velocity-making-your-site-fast.html}{studies}
have confirmed that improvements to latency can have a positive impact
on the usage and user engagement of sites and apps. This is at odds with
the traditional approach to web app development which is very
server-centric, requiring a complete page reload to move from one page
to the next. Even with heavy caching in place, the browser still has to
parse the CSS, JavaScript, and HTML and render the interface to the
screen.

In addition to resulting in a great deal of duplicated content being
served back to the user, this approach affects both latency and the
general responsiveness of the user experience. A trend to improve
perceived latency in the past few years has been to move towards
building Single Page Applications (SPAs) - apps which after an initial
page load are able to handle subsequent navigations and requests for
data without the need for a complete reload.

When a user navigates to a new view, additional content required for the
view is requested using an XHR (XMLHttpRequest), typically communicating
with a server-side REST API or endpoint.
\href{https://en.wikipedia.org/wiki/Ajax_(programming)}{Ajax}
(Asynchronous JavaScript and XML) makes communication with the server
asynchronous so that data is transferred and processed in the
background, allowing the user to interact with other parts of a page
without interruption. This improves usability and responsiveness.

SPAs can also take advantage of browser features like the
\href{http://diveintohtml5.info/history.html}{History API} to update the
address seen in the location bar when moving from one view to another.
These URLs also make it possible to bookmark and share a particular
application state, without the need to navigate to completely new pages.

The typical SPA consists of smaller pieces of interface representing
logical entities, all of which have their own UI, business logic and
data. A good example is a basket in a shopping web application which can
have items added to it. This basket might be presented to the user in a
box in the top right corner of the page (see the picture below):

\begin{figure}[htbp]
\centering
\includegraphics{img/wireframe_e_commerce.png}
\end{figure}

The basket and its data are presented in HTML. The data and its
associated View in HTML changes over time. There was a time when we used
jQuery (or a similar DOM manipulation library) and a bunch of Ajax calls
and callbacks to keep the two in sync. That often produced code that was
not well-structured or easy to maintain. Bugs were frequent and perhaps
even unavoidable.

The need for fast, complex, and responsive Ajax-powered web applications
demands replication of a lot of this logic on the client side,
dramatically increasing the size and complexity of the code residing
there. Eventually this has brought us to the point where we need MVC (or
a similar architecture) implemented on the client side to better
structure the code and make it easier to maintain and further extend
during the application life-cycle.

Through evolution and trial and error, JavaScript developers have
harnessed the power of the traditional MVC pattern, leading to the
development of several MVC-inspired JavaScript frameworks, such as
Backbone.js.

\subsubsection{Client-Side MVC - Backbone
Style}\label{client-side-mvc---backbone-style}

Let's take our first look at how Backbone.js brings the benefits of MVC
to client-side development using a Todo application as our example. We
will build on this example in the coming chapters when we explore
Backbone's features but for now we will just focus on the core
components' relationships to MVC.

Our example will need a div element to which we can attach a list of
Todo's. It will also need an HTML template containing a placeholder for
a Todo item title and a completion checkbox which can be instantiated
for Todo item instances. These are provided by the following HTML:

\begin{Shaded}
\begin{Highlighting}[]
\ErrorTok{<}\NormalTok{!doctype html>}
\KeywordTok{<html}\OtherTok{ lang=}\StringTok{"en"}\KeywordTok{>}
\KeywordTok{<head>}
  \KeywordTok{<meta}\OtherTok{ charset=}\StringTok{"utf-8"}\KeywordTok{>}
  \KeywordTok{<title></title>}
  \KeywordTok{<meta}\OtherTok{ name=}\StringTok{"description"}\OtherTok{ content=}\StringTok{""}\KeywordTok{>}
\KeywordTok{</head>}
\KeywordTok{<body>}
  \KeywordTok{<div}\OtherTok{ id=}\StringTok{"todo"}\KeywordTok{>}
  \KeywordTok{</div>}
  \KeywordTok{<script}\OtherTok{ type=}\StringTok{"text/template"}\OtherTok{ id=}\StringTok{"item-template"}\KeywordTok{>}
    \NormalTok{<div>}
      \NormalTok{<input id=}\StringTok{"todo_complete"} \NormalTok{type=}\StringTok{"checkbox"} \NormalTok{<%= completed ? }\StringTok{'checked="checked"'} \NormalTok{: }\StringTok{''} \NormalTok{%> }\OtherTok{/>}
\OtherTok{      <%= title %>}
\OtherTok{    </div}\NormalTok{>}
  \KeywordTok{</script>}
  \KeywordTok{<script}\OtherTok{ src=}\StringTok{"jquery.js"}\KeywordTok{></script>}
  \KeywordTok{<script}\OtherTok{ src=}\StringTok{"underscore.js"}\KeywordTok{></script>}
  \KeywordTok{<script}\OtherTok{ src=}\StringTok{"backbone.js"}\KeywordTok{></script>}
  \KeywordTok{<script}\OtherTok{ src=}\StringTok{"demo.js"}\KeywordTok{></script>}
\KeywordTok{</body>}
\KeywordTok{</html>}
\end{Highlighting}
\end{Shaded}

In our Todo application (demo.js), Backbone Model instances are used to
hold the data for each Todo item:

\begin{Shaded}
\begin{Highlighting}[]
\CommentTok{// Define a Todo Model}
\KeywordTok{var} \NormalTok{Todo = }\OtherTok{Backbone}\NormalTok{.}\OtherTok{Model}\NormalTok{.}\FunctionTok{extend}\NormalTok{(\{}
  \CommentTok{// Default todo attribute values}
  \DataTypeTok{defaults}\NormalTok{: \{}
    \DataTypeTok{title}\NormalTok{: }\StringTok{''}\NormalTok{,}
    \DataTypeTok{completed}\NormalTok{: }\KeywordTok{false}
  \NormalTok{\}}
\NormalTok{\});}

\CommentTok{// Instantiate the Todo Model with a title, with the completed attribute}
\CommentTok{// defaulting to false}
\KeywordTok{var} \NormalTok{myTodo = }\KeywordTok{new} \FunctionTok{Todo}\NormalTok{(\{}
  \DataTypeTok{title}\NormalTok{: }\StringTok{'Check attributes property of the logged models in the console.'}
\NormalTok{\});}
\end{Highlighting}
\end{Shaded}

Our Todo Model extends Backbone.Model and simply defines default values
for two data attributes. As you will discover in the upcoming chapters,
Backbone Models provide many more features but this simple Model
illustrates that first and foremost a Model is a data container.

Each Todo instance will be rendered on the page by a TodoView:

\begin{Shaded}
\begin{Highlighting}[]
\KeywordTok{var} \NormalTok{TodoView = }\OtherTok{Backbone}\NormalTok{.}\OtherTok{View}\NormalTok{.}\FunctionTok{extend}\NormalTok{(\{}

  \DataTypeTok{tagName}\NormalTok{:  }\StringTok{'li'}\NormalTok{,}

  \CommentTok{// Cache the template function for a single item.}
  \DataTypeTok{todoTpl}\NormalTok{: }\OtherTok{_}\NormalTok{.}\FunctionTok{template}\NormalTok{( }\FunctionTok{$}\NormalTok{(}\StringTok{'#item-template'}\NormalTok{).}\FunctionTok{html}\NormalTok{() ),}

  \DataTypeTok{events}\NormalTok{: \{}
    \StringTok{'dblclick label'}\NormalTok{: }\StringTok{'edit'}\NormalTok{,}
    \StringTok{'keypress .edit'}\NormalTok{: }\StringTok{'updateOnEnter'}\NormalTok{,}
    \StringTok{'blur .edit'}\NormalTok{:   }\StringTok{'close'}
  \NormalTok{\},}

  \CommentTok{// Called when the view is first created}
  \DataTypeTok{initialize}\NormalTok{: }\KeywordTok{function}\NormalTok{() \{}
    \KeywordTok{this}\NormalTok{.}\FunctionTok{$el} \NormalTok{= }\FunctionTok{$}\NormalTok{(}\StringTok{'#todo'}\NormalTok{);}
    \CommentTok{// Later we'll look at:}
    \CommentTok{// this.listenTo(someCollection, 'all', this.render);}
    \CommentTok{// but you can actually run this example right now by}
    \CommentTok{// calling todoView.render();}
  \NormalTok{\},}

  \CommentTok{// Re-render the titles of the todo item.}
  \DataTypeTok{render}\NormalTok{: }\KeywordTok{function}\NormalTok{() \{}
    \KeywordTok{this}\NormalTok{.}\OtherTok{$el}\NormalTok{.}\FunctionTok{html}\NormalTok{( }\KeywordTok{this}\NormalTok{.}\FunctionTok{todoTpl}\NormalTok{( }\KeywordTok{this}\NormalTok{.}\OtherTok{model}\NormalTok{.}\FunctionTok{attributes} \NormalTok{) );}
    \CommentTok{// $el here is a reference to the jQuery element}
    \CommentTok{// associated with the view, todoTpl is a reference}
    \CommentTok{// to an Underscore template and model.attributes}
    \CommentTok{// contains the attributes of the model.}
    \CommentTok{// Altogether, the statement is replacing the HTML of}
    \CommentTok{// a DOM element with the result of instantiating a}
    \CommentTok{// template with the model's attributes.}
    \KeywordTok{this}\NormalTok{.}\FunctionTok{input} \NormalTok{= }\KeywordTok{this}\NormalTok{.}\FunctionTok{$}\NormalTok{(}\StringTok{'.edit'}\NormalTok{);}
    \KeywordTok{return} \KeywordTok{this}\NormalTok{;}
  \NormalTok{\},}

  \DataTypeTok{edit}\NormalTok{: }\KeywordTok{function}\NormalTok{() \{}
    \CommentTok{// executed when todo label is double clicked}
  \NormalTok{\},}

  \DataTypeTok{close}\NormalTok{: }\KeywordTok{function}\NormalTok{() \{}
    \CommentTok{// executed when todo loses focus}
  \NormalTok{\},}

  \DataTypeTok{updateOnEnter}\NormalTok{: }\KeywordTok{function}\NormalTok{( e ) \{}
    \CommentTok{// executed on each keypress when in todo edit mode,}
    \CommentTok{// but we'll wait for enter to get in action}
  \NormalTok{\}}
\NormalTok{\});}

\CommentTok{// create a view for a todo}
\KeywordTok{var} \NormalTok{todoView = }\KeywordTok{new} \FunctionTok{TodoView}\NormalTok{(\{}\DataTypeTok{model}\NormalTok{: myTodo\});}
\end{Highlighting}
\end{Shaded}

TodoView is defined by extending Backbone.View and is instantiated with
an associated Model. In our example, the \texttt{render()} method uses a
template to construct the HTML for the Todo item which is placed inside
an li element. Each call to \texttt{render()} will replace the content
of the li element using the current Model data. Thus, a View instance
renders the content of a DOM element using the attributes of an
associated Model. Later we will see how a View can bind its
\texttt{render()} method to Model change events, causing the View to
re-render whenever the Model changes.

So far, we have seen that Backbone.Model implements the Model aspect of
MVC and Backbone.View implements the View. However, as we noted earlier,
Backbone departs from traditional MVC when it comes to Controllers -
there is no Backbone.Controller!

Instead, the Controller responsibility is addressed within the View.
Recall that Controllers respond to requests and perform appropriate
actions which may result in changes to the Model and updates to the
View. In a single-page application, rather than having requests in the
traditional sense, we have events. Events can be traditional browser DOM
events (e.g., clicks) or internal application events such as Model
changes.

In our TodoView, the \texttt{events} attribute fulfills the role of the
Controller configuration, defining how events occurring within the
View's DOM element are to be routed to event-handling methods defined in
the View.

While in this instance events help us relate Backbone to the MVC
pattern, we will see them playing a much larger role in our SPA
applications. Backbone.Event is a fundamental Backbone component which
is mixed into both Backbone.Model and Backbone.View, providing them with
rich event management capabilities. Note that the traditional controller
role (Smalltalk-80 style) is performed by the template, not by the
Backbone.View.

This completes our first encounter with Backbone.js. The remainder of
this book will explore the many features which build on these simple
constructs. Before moving on, let's take a look at common features of
JavaScript MV* libraries and frameworks.

\subsubsection{Implementation Specifics}\label{implementation-specifics}

A SPA is loaded into the browser using a normal HTTP request and
response. The page may simply be an HTML file, as in our example above,
or it could be a view constructed by a server-side MVC implementation.

Once loaded, a client-side Router intercepts URLs and invokes
client-side logic in place of sending a new request to the server. The
picture below shows typical request handling for client-side MVC as
implemented by Backbone:

\begin{figure}[htbp]
\centering
\includegraphics{img/backbone_mvc.png}
\end{figure}

URL routing, DOM events (e.g., mouse clicks), and Model events (e.g.,
attribute changes) all trigger handling logic in the View. The handlers
update the DOM and Models, which may trigger additional events. Models
are synced with Data Sources which may involve communicating with
back-end servers.

\paragraph{Models}\label{models}

\begin{itemize}
\item
  The built-in capabilities of Models vary across frameworks; however,
  it's common for them to support validation of attributes, where
  attributes represent the properties of the Model, such as a Model
  identifier.
\item
  When using Models in real-world applications we generally also need a
  way of persisting Models. Persistence allows us to edit and update
  Models with the knowledge that their most recent states will be saved
  somewhere, for example in a web browser's localStorage data-store or
  synchronized with a database.
\item
  A Model may have multiple Views observing it for changes. By
  \emph{observing} we mean that a View has registered an interest in
  being informed whenever an update is made to the Model. This allows
  the View to ensure that what is displayed on screen is kept in sync
  with the data contained in the model. Depending on your requirements,
  you might create a single View displaying all Model attributes, or
  create separate Views displaying different attributes. The important
  point is that the Model doesn't care how these Views are organized, it
  simply announces updates to its data as necessary through the
  framework's event system.
\item
  It is not uncommon for modern MVC/MV* frameworks to provide a means of
  grouping Models together. In Backbone, these groups are called
  Collections. Managing Models in groups allows us to write application
  logic based on notifications from the group when a Model within the
  group changes. This avoids the need to manually observe individual
  Model instances. We'll see this in action later in the book.
  Collections are also useful for performing any aggregate computations
  across more than one model.
\end{itemize}

\paragraph{Views}\label{views}

\begin{itemize}
\item
  Users interact with Views, which usually means reading and editing
  Model data. For example, in our Todo application, Todo Model viewing
  happens in the user interface in the list of all Todo items. Within
  it, each Todo is rendered with its title and completed checkbox. Model
  editing is done through an ``edit'' View where a user who has selected
  a specific Todo edits its title in a form.
\item
  We define a \texttt{render()} utility within our View which is
  responsible for rendering the contents of the \texttt{Model} using a
  JavaScript templating engine (provided by Underscore.js) and updating
  the contents of our View, referenced by \texttt{this.\$el}.
\item
  We then add our \texttt{render()} callback as a Model subscriber, so
  the View can be triggered to update when the Model changes.
\item
  You may wonder where user interaction comes into play here. When users
  click on a Todo element within the View, it's not the View's
  responsibility to know what to do next. A Controller makes this
  decision. In Backbone, this is achieved by adding an event listener to
  the Todo's element which delegates handling of the click to an event
  handler.
\end{itemize}

\textbf{Templating}

In the context of JavaScript frameworks that support MVC/MV*, it is
worth looking more closely at JavaScript templating and its relationship
to Views.

It has long been considered bad practice (and computationally expensive)
to manually create large blocks of HTML markup in-memory through string
concatenation. Developers using this technique often find themselves
iterating through their data, wrapping it in nested divs and using
outdated techniques such as \texttt{document.write} to inject the
`template' into the DOM. This approach often means keeping scripted
markup inline with standard markup, which can quickly become difficult
to read and maintain, especially when building large applications.

JavaScript templating libraries (such as Mustache or Handlebars.js) are
often used to define templates for Views as HTML markup containing
template variables. These template blocks can be either stored
externally or within script tags with a custom type (e.g
`text/template'). Variables are delimited using a variable syntax (e.g
\texttt{\textless{}\%= title \%\textgreater{}} for Underscore and
\texttt{\{\{title\}\}} for Handlebars).

JavaScript template libraries typically accept data in a number of
formats, including JSON; a serialisation format that is always a string.
The grunt work of populating templates with data is generally taken care
of by the framework itself. This has several benefits, particularly when
opting to store templates externally which enables applications to load
templates dynamically on an as-needed basis.

Let's compare two examples of HTML templates. One is implemented using
the popular Handlebars.js library, and the other uses Underscore's
`microtemplates'.

\textbf{Handlebars.js:}

\begin{Shaded}
\begin{Highlighting}[]
\KeywordTok{<div}\OtherTok{ class=}\StringTok{"view"}\KeywordTok{>}
  \KeywordTok{<input}\OtherTok{ class=}\StringTok{"toggle"}\OtherTok{ type=}\StringTok{"checkbox"} \ErrorTok{\{\{#if}\OtherTok{ completed}\ErrorTok{\}\}} \ErrorTok{"checked"} \ErrorTok{\{\{/if\}\}}\KeywordTok{>}
  \KeywordTok{<label>}\NormalTok{\{\{title\}\}}\KeywordTok{</label>}
  \KeywordTok{<button}\OtherTok{ class=}\StringTok{"destroy"}\KeywordTok{></button>}
\KeywordTok{</div>}
\KeywordTok{<input}\OtherTok{ class=}\StringTok{"edit"}\OtherTok{ value=}\StringTok{"\{\{title\}\}"}\KeywordTok{>}
\end{Highlighting}
\end{Shaded}

\textbf{Underscore.js Microtemplates:}

\begin{Shaded}
\begin{Highlighting}[]
\KeywordTok{<div}\OtherTok{ class=}\StringTok{"view"}\KeywordTok{>}
  \KeywordTok{<input}\OtherTok{ class=}\StringTok{"toggle"}\OtherTok{ type=}\StringTok{"checkbox"} \ErrorTok{<%}\OtherTok{=} \StringTok{completed} \ErrorTok{?} \ErrorTok{'checked'}\OtherTok{ :} \ErrorTok{''} \ErrorTok{%}\KeywordTok{>}\NormalTok{>}
  \KeywordTok{<label>}\ErrorTok{<}\NormalTok{%= title %>}\KeywordTok{</label>}
  \KeywordTok{<button}\OtherTok{ class=}\StringTok{"destroy"}\KeywordTok{></button>}
\KeywordTok{</div>}
\KeywordTok{<input}\OtherTok{ class=}\StringTok{"edit"}\OtherTok{ value=}\StringTok{"}\ErrorTok{<}\StringTok{%= title %>"}\KeywordTok{>}
\end{Highlighting}
\end{Shaded}

You may also use double curly brackets (i.e \texttt{\{\{\}\}}) (or any
other tag you feel comfortable with) in Microtemplates. In the case of
curly brackets, this can be done by setting the Underscore
\texttt{templateSettings} attribute as follows:

\begin{Shaded}
\begin{Highlighting}[]
\OtherTok{_}\NormalTok{.}\FunctionTok{templateSettings} \NormalTok{= \{ }\DataTypeTok{interpolate }\NormalTok{: }\OtherTok{/}\FloatTok{\textbackslash{}\{\textbackslash{}\{(}\OtherTok{.}\FloatTok{+?)\textbackslash{}\}\textbackslash{}\}}\OtherTok{/g} \NormalTok{\};}
\end{Highlighting}
\end{Shaded}

\textbf{A note on Navigation and State}

It is also worth noting that in classical web development, navigating
between independent views required the use of a page refresh. In
single-page JavaScript applications, however, once data is fetched from
a server via Ajax, it can be dynamically rendered in a new view within
the same page. Since this doesn't automatically update the URL, the role
of navigation thus falls to a ``router'', which assists in managing
application state (e.g., allowing users to bookmark a particular view
they have navigated to). As routers are neither a part of MVC nor
present in every MVC-like framework, I will not be going into them in
greater detail in this section.

\paragraph{Controllers}\label{controllers}

In our Todo application, a Controller would be responsible for handling
changes the user made in the edit View for a particular Todo, updating a
specific Todo Model when a user has finished editing.

It's with Controllers that most JavaScript MVC frameworks depart from
the traditional interpretation of the MVC pattern. The reasons for this
vary, but in my opinion, JavaScript framework authors likely initially
looked at server-side interpretations of MVC (such as Ruby on Rails),
realized that the approach didn't translate 1:1 on the client-side, and
so re-interpreted the C in MVC to solve their state management problem.
This was a clever approach, but it can make it hard for developers
coming to MVC for the first time to understand both the classical MVC
pattern and the ``proper'' role of Controllers in other JavaScript
frameworks.

So does Backbone.js have Controllers? Not really. Backbone's Views
typically contain ``Controller'' logic, and Routers are used to help
manage application state, but neither are true Controllers according to
classical MVC.

In this respect, contrary to what might be mentioned in the official
documentation or in blog posts, Backbone isn't truly an MVC library.
It's in fact better to see it a member of the MV* family which
approaches architecture in its own way. There is of course nothing wrong
with this, but it is important to distinguish between classical MVC and
MV* should you be relying on discussions of MVC to help with your
Backbone projects.

\subsection{What does MVC give us?}\label{what-does-mvc-give-us}

To summarize, the MVC pattern helps you keep your application logic
separate from your user interface, making it easier to change and
maintain both. Thanks to this separation of logic, it is more clear
where changes to your data, interface, or business logic need to be made
and for what your unit tests should be written.

\subsubsection{Delving Deeper into MVC}\label{delving-deeper-into-mvc}

Right now, you likely have a basic understanding of what the MVC pattern
provides, but for the curious, we'll explore it a little further.

The GoF (Gang of Four) do not refer to MVC as a design pattern, but
rather consider it a ``set of classes to build a user interface.'' In
their view, it's actually a variation of three other classical design
patterns: the Observer (Publish/Subscribe), Strategy, and Composite
patterns. Depending on how MVC has been implemented in a framework, it
may also use the Factory and Decorator patterns. I've covered some of
these patterns in my other free book, ``JavaScript Design Patterns For
Beginners'' if you would like to read about them further.

As we've discussed, Models represent application data, while Views
handle what the user is presented on screen. As such, MVC relies on
Publish/Subscribe for some of its core communication (something that
surprisingly isn't covered in many articles about the MVC pattern). When
a Model is changed it ``publishes'' to the rest of the application that
it has been updated. The ``subscriber,'' generally a Controller, then
updates the View accordingly. The observer-viewer nature of this
relationship is what facilitates multiple Views being attached to the
same Model.

For developers interested in knowing more about the decoupled nature of
MVC (once again, depending on the implementation), one of the goals of
the pattern is to help define one-to-many relationships between a topic
and its observers. When a topic changes, its observers are updated.
Views and Controllers have a slightly different relationship.
Controllers facilitate Views' responses to different user input and are
an example of the Strategy pattern.

\subsubsection{Summary}\label{summary}

Having reviewed the classical MVC pattern, you should now understand how
it allows developers to cleanly separate concerns in an application. You
should also now appreciate how JavaScript MVC frameworks may differ in
their interpretation of MVC, and how they share some of the fundamental
concepts of the original pattern.

When reviewing a new JavaScript MVC/MV* framework, remember - it can be
useful to step back and consider how it's opted to approach Models,
Views, Controllers or other alternatives, as this can better help you
understand how the framework is intended to be used.

\subsubsection{Further reading}\label{further-reading}

If you are interested in learning more about the variation of MVC which
Backbone.js uses, please see the MVP (Model-View-Presenter) section in
the appendix.

\subsection{Fast facts}\label{fast-facts}

\subsubsection{Backbone.js}\label{backbone.js}

\begin{itemize}
\itemsep1pt\parskip0pt\parsep0pt
\item
  Core components: Model, View, Collection, Router. Enforces its own
  flavor of MV*
\item
  Event-driven communication between Views and Models. As we'll see,
  it's relatively straight-forward to add event listeners to any
  attribute in a Model, giving developers fine-grained control over what
  changes in the View
\item
  Supports data bindings through manual events or a separate Key-value
  observing (KVO) library
\item
  Support for RESTful interfaces out of the box, so Models can be easily
  tied to a backend
\item
  Extensive eventing system. It's
  \href{http://lostechies.com/derickbailey/2011/07/19/references-routing-and-the-event-aggregator-coordinating-views-in-backbone-js/}{trivial}
  to add support for pub/sub in Backbone
\item
  Prototypes are instantiated with the \texttt{new} keyword, which some
  developers prefer
\item
  Agnostic about templating frameworks, however Underscore's
  micro-templating is available by default
\item
  Clear and flexible conventions for structuring applications. Backbone
  doesn't force usage of all of its components and can work with only
  those needed
\end{itemize}




\section{Backbone Basics}\label{backbone-basics}

In this section, you'll learn the essentials of Backbone's models,
views, collections, events, and routers. This isn't by any means a
replacement for the official documentation, but it will help you
understand many of the core concepts behind Backbone before you start
building applications using it.

\subsubsection{Getting set up}\label{getting-set-up}

Before we dive into more code examples, let's define some boilerplate
markup you can use to specify the dependencies Backbone requires. This
boilerplate can be reused in many ways with little to no alteration and
will allow you to run code from examples with ease.

You can paste the following into your text editor of choice, replacing
the commented line between the script tags with the JavaScript from any
given example:

\begin{Shaded}
\begin{Highlighting}[]
\DataTypeTok{<!DOCTYPE }\NormalTok{HTML}\DataTypeTok{>}
\KeywordTok{<html>}
\KeywordTok{<head>}
    \KeywordTok{<meta}\OtherTok{ charset=}\StringTok{"UTF-8"}\KeywordTok{>}
    \KeywordTok{<title>}\NormalTok{Title}\KeywordTok{</title>}
\KeywordTok{</head>}
\KeywordTok{<body>}

\KeywordTok{<script}\OtherTok{ src=}\StringTok{"https://ajax.googleapis.com/ajax/libs/jquery/1.9.1/jquery.min.js"}\KeywordTok{></script>}
\KeywordTok{<script}\OtherTok{ src=}\StringTok{"http://documentcloud.github.com/underscore/underscore-min.js"}\KeywordTok{></script>}
\KeywordTok{<script}\OtherTok{ src=}\StringTok{"http://documentcloud.github.com/backbone/backbone-min.js"}\KeywordTok{></script>}
\KeywordTok{<script>}
  \CommentTok{// Your code goes here}
\KeywordTok{</script>}
\KeywordTok{</body>}
\KeywordTok{</html>}
\end{Highlighting}
\end{Shaded}

You can then save and run the file in your browser of choice, such as
Chrome or Firefox. Alternatively, if you prefer working with an online
code editor, \href{http://jsfiddle.net/jnf8B/}{jsFiddle} and
\href{http://jsbin.com/iwiwox/1/edit}{jsBin} versions of this
boilerplate are also available.

Most examples can also be run directly from within the console in your
browser's developer tools, assuming you've loaded the boilerplate HTML
page so that Backbone and its dependencies are available for use.

For Chrome, you can open up the DevTools via the Chrome menu in the top
right hand corner: select ``Tools \textgreater{} Developer Tools'' or
alternatively use the Control + Shift + I shortcut on Windows/Linux or
Command + Option + I on Mac.

\begin{figure}[htbp]
\centering
\includegraphics{img/devtools.png}
\end{figure}

Next, switch to the Console tab, from where you can enter in and run any
piece of JavaScript code by hitting the return key. You can also use the
Console as a multi-line editor using the Shift + Enter shortcut on
Windows, or Ctrl + Enter shortcut on Mac to move from the end of one
line to the start of another.

\subsection{Models}\label{models-1}

Backbone models contain data for an application as well as the logic
around this data. For example, we can use a model to represent the
concept of a todo item including its attributes like title (todo
content) and completed (current state of the todo).

Models can be created by extending \texttt{Backbone.Model} as follows:

\begin{Shaded}
\begin{Highlighting}[]
\KeywordTok{var} \NormalTok{Todo = }\OtherTok{Backbone}\NormalTok{.}\OtherTok{Model}\NormalTok{.}\FunctionTok{extend}\NormalTok{(\{\});}

\CommentTok{// We can then create our own concrete instance of a (Todo) model}
\CommentTok{// with no values at all:}
\KeywordTok{var} \NormalTok{todo1 = }\KeywordTok{new} \FunctionTok{Todo}\NormalTok{();}
\CommentTok{// Following logs: \{\}}
\OtherTok{console}\NormalTok{.}\FunctionTok{log}\NormalTok{(}\OtherTok{JSON}\NormalTok{.}\FunctionTok{stringify}\NormalTok{(todo1));}

\CommentTok{// or with some arbitrary data:}
\KeywordTok{var} \NormalTok{todo2 = }\KeywordTok{new} \FunctionTok{Todo}\NormalTok{(\{}
  \DataTypeTok{title}\NormalTok{: }\StringTok{'Check the attributes of both model instances in the console.'}\NormalTok{,}
  \DataTypeTok{completed}\NormalTok{: }\KeywordTok{true}
\NormalTok{\});}

\CommentTok{// Following logs: \{"title":"Check the attributes of both model instances in the console.","completed":true\}}
\OtherTok{console}\NormalTok{.}\FunctionTok{log}\NormalTok{(}\OtherTok{JSON}\NormalTok{.}\FunctionTok{stringify}\NormalTok{(todo2));}
\end{Highlighting}
\end{Shaded}

\paragraph{Initialization}\label{initialization}

The \texttt{initialize()} method is called when a new instance of a
model is created. Its use is optional; however you'll see why it's good
practice to use it below.

\begin{Shaded}
\begin{Highlighting}[]
\KeywordTok{var} \NormalTok{Todo = }\OtherTok{Backbone}\NormalTok{.}\OtherTok{Model}\NormalTok{.}\FunctionTok{extend}\NormalTok{(\{}
  \DataTypeTok{initialize}\NormalTok{: }\KeywordTok{function}\NormalTok{()\{}
      \OtherTok{console}\NormalTok{.}\FunctionTok{log}\NormalTok{(}\StringTok{'This model has been initialized.'}\NormalTok{);}
  \NormalTok{\}}
\NormalTok{\});}

\KeywordTok{var} \NormalTok{myTodo = }\KeywordTok{new} \FunctionTok{Todo}\NormalTok{();}
\CommentTok{// Logs: This model has been initialized.}
\end{Highlighting}
\end{Shaded}

\textbf{Default values}

There are times when you want your model to have a set of default values
(e.g., in a scenario where a complete set of data isn't provided by the
user). This can be set using a property called \texttt{defaults} in your
model.

\begin{Shaded}
\begin{Highlighting}[]
\KeywordTok{var} \NormalTok{Todo = }\OtherTok{Backbone}\NormalTok{.}\OtherTok{Model}\NormalTok{.}\FunctionTok{extend}\NormalTok{(\{}
  \CommentTok{// Default todo attribute values}
  \DataTypeTok{defaults}\NormalTok{: \{}
    \DataTypeTok{title}\NormalTok{: }\StringTok{''}\NormalTok{,}
    \DataTypeTok{completed}\NormalTok{: }\KeywordTok{false}
  \NormalTok{\}}
\NormalTok{\});}

\CommentTok{// Now we can create our concrete instance of the model}
\CommentTok{// with default values as follows:}
\KeywordTok{var} \NormalTok{todo1 = }\KeywordTok{new} \FunctionTok{Todo}\NormalTok{();}

\CommentTok{// Following logs: \{"title":"","completed":false\}}
\OtherTok{console}\NormalTok{.}\FunctionTok{log}\NormalTok{(}\OtherTok{JSON}\NormalTok{.}\FunctionTok{stringify}\NormalTok{(todo1));}

\CommentTok{// Or we could instantiate it with some of the attributes (e.g., with custom title):}
\KeywordTok{var} \NormalTok{todo2 = }\KeywordTok{new} \FunctionTok{Todo}\NormalTok{(\{}
  \DataTypeTok{title}\NormalTok{: }\StringTok{'Check attributes of the logged models in the console.'}
\NormalTok{\});}

\CommentTok{// Following logs: \{"title":"Check attributes of the logged models in the console.","completed":false\}}
\OtherTok{console}\NormalTok{.}\FunctionTok{log}\NormalTok{(}\OtherTok{JSON}\NormalTok{.}\FunctionTok{stringify}\NormalTok{(todo2));}

\CommentTok{// Or override all of the default attributes:}
\KeywordTok{var} \NormalTok{todo3 = }\KeywordTok{new} \FunctionTok{Todo}\NormalTok{(\{}
  \DataTypeTok{title}\NormalTok{: }\StringTok{'This todo is done, so take no action on this one.'}\NormalTok{,}
  \DataTypeTok{completed}\NormalTok{: }\KeywordTok{true}
\NormalTok{\});}

\CommentTok{// Following logs: \{"title":"This todo is done, so take no action on this one.","completed":true\} }
\OtherTok{console}\NormalTok{.}\FunctionTok{log}\NormalTok{(}\OtherTok{JSON}\NormalTok{.}\FunctionTok{stringify}\NormalTok{(todo3));}
\end{Highlighting}
\end{Shaded}

\paragraph{Getters \& Setters}\label{getters-setters}

\textbf{Model.get()}

\texttt{Model.get()} provides easy access to a model's attributes.

\begin{Shaded}
\begin{Highlighting}[]
\KeywordTok{var} \NormalTok{Todo = }\OtherTok{Backbone}\NormalTok{.}\OtherTok{Model}\NormalTok{.}\FunctionTok{extend}\NormalTok{(\{}
  \CommentTok{// Default todo attribute values}
  \DataTypeTok{defaults}\NormalTok{: \{}
    \DataTypeTok{title}\NormalTok{: }\StringTok{''}\NormalTok{,}
    \DataTypeTok{completed}\NormalTok{: }\KeywordTok{false}
  \NormalTok{\}}
\NormalTok{\});}

\KeywordTok{var} \NormalTok{todo1 = }\KeywordTok{new} \FunctionTok{Todo}\NormalTok{();}
\OtherTok{console}\NormalTok{.}\FunctionTok{log}\NormalTok{(}\OtherTok{todo1}\NormalTok{.}\FunctionTok{get}\NormalTok{(}\StringTok{'title'}\NormalTok{)); }\CommentTok{// empty string}
\OtherTok{console}\NormalTok{.}\FunctionTok{log}\NormalTok{(}\OtherTok{todo1}\NormalTok{.}\FunctionTok{get}\NormalTok{(}\StringTok{'completed'}\NormalTok{)); }\CommentTok{// false}

\KeywordTok{var} \NormalTok{todo2 = }\KeywordTok{new} \FunctionTok{Todo}\NormalTok{(\{}
  \DataTypeTok{title}\NormalTok{: }\StringTok{"Retrieved with model's get() method."}\NormalTok{,}
  \DataTypeTok{completed}\NormalTok{: }\KeywordTok{true}
\NormalTok{\});}
\OtherTok{console}\NormalTok{.}\FunctionTok{log}\NormalTok{(}\OtherTok{todo2}\NormalTok{.}\FunctionTok{get}\NormalTok{(}\StringTok{'title'}\NormalTok{)); }\CommentTok{// Retrieved with model's get() method.}
\OtherTok{console}\NormalTok{.}\FunctionTok{log}\NormalTok{(}\OtherTok{todo2}\NormalTok{.}\FunctionTok{get}\NormalTok{(}\StringTok{'completed'}\NormalTok{)); }\CommentTok{// true}
\end{Highlighting}
\end{Shaded}

If you need to read or clone all of a model's data attributes, use its
\texttt{toJSON()} method. This method returns a copy of the attributes
as an object (not a JSON string despite its name). (When
\texttt{JSON.stringify()} is passed an object with a \texttt{toJSON()}
method, it stringifies the return value of \texttt{toJSON()} instead of
the original object. The examples in the previous section took advantage
of this feature when they called \texttt{JSON.stringify()} to log model
instances.)

\begin{Shaded}
\begin{Highlighting}[]
\KeywordTok{var} \NormalTok{Todo = }\OtherTok{Backbone}\NormalTok{.}\OtherTok{Model}\NormalTok{.}\FunctionTok{extend}\NormalTok{(\{}
  \CommentTok{// Default todo attribute values}
  \DataTypeTok{defaults}\NormalTok{: \{}
    \DataTypeTok{title}\NormalTok{: }\StringTok{''}\NormalTok{,}
    \DataTypeTok{completed}\NormalTok{: }\KeywordTok{false}
  \NormalTok{\}}
\NormalTok{\});}

\KeywordTok{var} \NormalTok{todo1 = }\KeywordTok{new} \FunctionTok{Todo}\NormalTok{();}
\KeywordTok{var} \NormalTok{todo1Attributes = }\OtherTok{todo1}\NormalTok{.}\FunctionTok{toJSON}\NormalTok{();}
\CommentTok{// Following logs: \{"title":"","completed":false\} }
\OtherTok{console}\NormalTok{.}\FunctionTok{log}\NormalTok{(todo1Attributes);}

\KeywordTok{var} \NormalTok{todo2 = }\KeywordTok{new} \FunctionTok{Todo}\NormalTok{(\{}
  \DataTypeTok{title}\NormalTok{: }\StringTok{"Try these examples and check results in console."}\NormalTok{,}
  \DataTypeTok{completed}\NormalTok{: }\KeywordTok{true}
\NormalTok{\});}

\CommentTok{// logs: \{"title":"Try these examples and check results in console.","completed":true\}}
\OtherTok{console}\NormalTok{.}\FunctionTok{log}\NormalTok{(}\OtherTok{todo2}\NormalTok{.}\FunctionTok{toJSON}\NormalTok{());}
\end{Highlighting}
\end{Shaded}

\textbf{Model.set()}

\texttt{Model.set()} sets a hash containing one or more attributes on
the model. When any of these attributes alter the state of the model, a
``change'' event is triggered on it. Change events for each attribute
are also triggered and can be bound to (e.g. \texttt{change:name},
\texttt{change:age}).

\begin{Shaded}
\begin{Highlighting}[]
\KeywordTok{var} \NormalTok{Todo = }\OtherTok{Backbone}\NormalTok{.}\OtherTok{Model}\NormalTok{.}\FunctionTok{extend}\NormalTok{(\{}
  \CommentTok{// Default todo attribute values}
  \DataTypeTok{defaults}\NormalTok{: \{}
    \DataTypeTok{title}\NormalTok{: }\StringTok{''}\NormalTok{,}
    \DataTypeTok{completed}\NormalTok{: }\KeywordTok{false}
  \NormalTok{\}}
\NormalTok{\});}

\CommentTok{// Setting the value of attributes via instantiation}
\KeywordTok{var} \NormalTok{myTodo = }\KeywordTok{new} \FunctionTok{Todo}\NormalTok{(\{}
  \DataTypeTok{title}\NormalTok{: }\StringTok{"Set through instantiation."}
\NormalTok{\});}
\OtherTok{console}\NormalTok{.}\FunctionTok{log}\NormalTok{(}\StringTok{'Todo title: '} \NormalTok{+ }\OtherTok{myTodo}\NormalTok{.}\FunctionTok{get}\NormalTok{(}\StringTok{'title'}\NormalTok{)); }\CommentTok{// Todo title: Set through instantiation.}
\OtherTok{console}\NormalTok{.}\FunctionTok{log}\NormalTok{(}\StringTok{'Completed: '} \NormalTok{+ }\OtherTok{myTodo}\NormalTok{.}\FunctionTok{get}\NormalTok{(}\StringTok{'completed'}\NormalTok{)); }\CommentTok{// Completed: false}

\CommentTok{// Set single attribute value at a time through Model.set():}
\OtherTok{myTodo}\NormalTok{.}\FunctionTok{set}\NormalTok{(}\StringTok{"title"}\NormalTok{, }\StringTok{"Title attribute set through Model.set()."}\NormalTok{);}
\OtherTok{console}\NormalTok{.}\FunctionTok{log}\NormalTok{(}\StringTok{'Todo title: '} \NormalTok{+ }\OtherTok{myTodo}\NormalTok{.}\FunctionTok{get}\NormalTok{(}\StringTok{'title'}\NormalTok{)); }\CommentTok{// Todo title: Title attribute set through Model.set().}
\OtherTok{console}\NormalTok{.}\FunctionTok{log}\NormalTok{(}\StringTok{'Completed: '} \NormalTok{+ }\OtherTok{myTodo}\NormalTok{.}\FunctionTok{get}\NormalTok{(}\StringTok{'completed'}\NormalTok{)); }\CommentTok{// Completed: false}

\CommentTok{// Set map of attributes through Model.set():}
\OtherTok{myTodo}\NormalTok{.}\FunctionTok{set}\NormalTok{(\{}
  \DataTypeTok{title}\NormalTok{: }\StringTok{"Both attributes set through Model.set()."}\NormalTok{,}
  \DataTypeTok{completed}\NormalTok{: }\KeywordTok{true}
\NormalTok{\});}
\OtherTok{console}\NormalTok{.}\FunctionTok{log}\NormalTok{(}\StringTok{'Todo title: '} \NormalTok{+ }\OtherTok{myTodo}\NormalTok{.}\FunctionTok{get}\NormalTok{(}\StringTok{'title'}\NormalTok{)); }\CommentTok{// Todo title: Both attributes set through Model.set().}
\OtherTok{console}\NormalTok{.}\FunctionTok{log}\NormalTok{(}\StringTok{'Completed: '} \NormalTok{+ }\OtherTok{myTodo}\NormalTok{.}\FunctionTok{get}\NormalTok{(}\StringTok{'completed'}\NormalTok{)); }\CommentTok{// Completed: true}
\end{Highlighting}
\end{Shaded}

\textbf{Direct access}

Models expose an \texttt{.attributes} attribute which represents an
internal hash containing the state of that model. This is generally in
the form of a JSON object similar to the model data you might find on
the server but can take other forms.

Setting values through the \texttt{.attributes} attribute on a model
bypasses triggers bound to the model.

Passing \texttt{\{silent:true\}} on change doesn't delay individual
\texttt{"change:attr"} events. Instead they are silenced entirely:

\begin{Shaded}
\begin{Highlighting}[]
\KeywordTok{var} \NormalTok{Person = }\KeywordTok{new} \OtherTok{Backbone}\NormalTok{.}\FunctionTok{Model}\NormalTok{();}
\OtherTok{Person}\NormalTok{.}\FunctionTok{on}\NormalTok{(}\StringTok{"change:name"}\NormalTok{, }\KeywordTok{function}\NormalTok{() \{ }\OtherTok{console}\NormalTok{.}\FunctionTok{log}\NormalTok{(}\StringTok{'Name changed'}\NormalTok{); \});}
\OtherTok{Person}\NormalTok{.}\FunctionTok{set}\NormalTok{(\{}\DataTypeTok{name}\NormalTok{: }\StringTok{'Andrew'}\NormalTok{\});}
\CommentTok{// log entry: Name changed}

\OtherTok{Person}\NormalTok{.}\FunctionTok{set}\NormalTok{(\{}\DataTypeTok{name}\NormalTok{: }\StringTok{'Jeremy'}\NormalTok{\}, \{}\DataTypeTok{silent}\NormalTok{: }\KeywordTok{true}\NormalTok{\});}
\CommentTok{// no log entry}

\OtherTok{console}\NormalTok{.}\FunctionTok{log}\NormalTok{(}\OtherTok{Person}\NormalTok{.}\FunctionTok{hasChanged}\NormalTok{(}\StringTok{"name"}\NormalTok{));}
\CommentTok{// true: change was recorded}
\OtherTok{console}\NormalTok{.}\FunctionTok{log}\NormalTok{(}\OtherTok{Person}\NormalTok{.}\FunctionTok{hasChanged}\NormalTok{(}\KeywordTok{null}\NormalTok{));}
\CommentTok{// true: something (anything) has changed}
\end{Highlighting}
\end{Shaded}

Remember where possible it is best practice to use \texttt{Model.set()},
or direct instantiation as explained earlier.

\paragraph{Listening for changes to your
model}\label{listening-for-changes-to-your-model}

If you want to receive a notification when a Backbone model changes you
can bind a listener to the model for its change event. A convenient
place to add listeners is in the \texttt{initialize()} function as shown
below:

\begin{Shaded}
\begin{Highlighting}[]
\KeywordTok{var} \NormalTok{Todo = }\OtherTok{Backbone}\NormalTok{.}\OtherTok{Model}\NormalTok{.}\FunctionTok{extend}\NormalTok{(\{}
  \CommentTok{// Default todo attribute values}
  \DataTypeTok{defaults}\NormalTok{: \{}
    \DataTypeTok{title}\NormalTok{: }\StringTok{''}\NormalTok{,}
    \DataTypeTok{completed}\NormalTok{: }\KeywordTok{false}
  \NormalTok{\},}
  \DataTypeTok{initialize}\NormalTok{: }\KeywordTok{function}\NormalTok{()\{}
    \OtherTok{console}\NormalTok{.}\FunctionTok{log}\NormalTok{(}\StringTok{'This model has been initialized.'}\NormalTok{);}
    \KeywordTok{this}\NormalTok{.}\FunctionTok{on}\NormalTok{(}\StringTok{'change'}\NormalTok{, }\KeywordTok{function}\NormalTok{()\{}
        \OtherTok{console}\NormalTok{.}\FunctionTok{log}\NormalTok{(}\StringTok{'- Values for this model have changed.'}\NormalTok{);}
    \NormalTok{\});}
  \NormalTok{\}}
\NormalTok{\});}

\KeywordTok{var} \NormalTok{myTodo = }\KeywordTok{new} \FunctionTok{Todo}\NormalTok{();}

\OtherTok{myTodo}\NormalTok{.}\FunctionTok{set}\NormalTok{(}\StringTok{'title'}\NormalTok{, }\StringTok{'The listener is triggered whenever an attribute value changes.'}\NormalTok{);}
\OtherTok{console}\NormalTok{.}\FunctionTok{log}\NormalTok{(}\StringTok{'Title has changed: '} \NormalTok{+ }\OtherTok{myTodo}\NormalTok{.}\FunctionTok{get}\NormalTok{(}\StringTok{'title'}\NormalTok{));}


\OtherTok{myTodo}\NormalTok{.}\FunctionTok{set}\NormalTok{(}\StringTok{'completed'}\NormalTok{, }\KeywordTok{true}\NormalTok{);}
\OtherTok{console}\NormalTok{.}\FunctionTok{log}\NormalTok{(}\StringTok{'Completed has changed: '} \NormalTok{+ }\OtherTok{myTodo}\NormalTok{.}\FunctionTok{get}\NormalTok{(}\StringTok{'completed'}\NormalTok{));}

\OtherTok{myTodo}\NormalTok{.}\FunctionTok{set}\NormalTok{(\{}
  \DataTypeTok{title}\NormalTok{: }\StringTok{'Changing more than one attribute at the same time only triggers the listener once.'}\NormalTok{,}
  \DataTypeTok{completed}\NormalTok{: }\KeywordTok{true}
\NormalTok{\});}

\CommentTok{// Above logs:}
\CommentTok{// This model has been initialized.}
\CommentTok{// - Values for this model have changed.}
\CommentTok{// Title has changed: The listener is triggered whenever an attribute value changes.}
\CommentTok{// - Values for this model have changed.}
\CommentTok{// Completed has changed: true}
\CommentTok{// - Values for this model have changed.}
\end{Highlighting}
\end{Shaded}

You can also listen for changes to individual attributes in a Backbone
model. In the following example, we log a message whenever a specific
attribute (the title of our Todo model) is altered.

\begin{Shaded}
\begin{Highlighting}[]
\KeywordTok{var} \NormalTok{Todo = }\OtherTok{Backbone}\NormalTok{.}\OtherTok{Model}\NormalTok{.}\FunctionTok{extend}\NormalTok{(\{}
  \CommentTok{// Default todo attribute values}
  \DataTypeTok{defaults}\NormalTok{: \{}
    \DataTypeTok{title}\NormalTok{: }\StringTok{''}\NormalTok{,}
    \DataTypeTok{completed}\NormalTok{: }\KeywordTok{false}
  \NormalTok{\},}

  \DataTypeTok{initialize}\NormalTok{: }\KeywordTok{function}\NormalTok{()\{}
    \OtherTok{console}\NormalTok{.}\FunctionTok{log}\NormalTok{(}\StringTok{'This model has been initialized.'}\NormalTok{);}
    \KeywordTok{this}\NormalTok{.}\FunctionTok{on}\NormalTok{(}\StringTok{'change:title'}\NormalTok{, }\KeywordTok{function}\NormalTok{()\{}
        \OtherTok{console}\NormalTok{.}\FunctionTok{log}\NormalTok{(}\StringTok{'Title value for this model has changed.'}\NormalTok{);}
    \NormalTok{\});}
  \NormalTok{\},}

  \DataTypeTok{setTitle}\NormalTok{: }\KeywordTok{function}\NormalTok{(newTitle)\{}
    \KeywordTok{this}\NormalTok{.}\FunctionTok{set}\NormalTok{(\{ }\DataTypeTok{title}\NormalTok{: newTitle \});}
  \NormalTok{\}}
\NormalTok{\});}

\KeywordTok{var} \NormalTok{myTodo = }\KeywordTok{new} \FunctionTok{Todo}\NormalTok{();}

\CommentTok{// Both of the following changes trigger the listener:}
\OtherTok{myTodo}\NormalTok{.}\FunctionTok{set}\NormalTok{(}\StringTok{'title'}\NormalTok{, }\StringTok{'Check what}\CharTok{\textbackslash{}'}\StringTok{s logged.'}\NormalTok{);}
\OtherTok{myTodo}\NormalTok{.}\FunctionTok{setTitle}\NormalTok{(}\StringTok{'Go fishing on Sunday.'}\NormalTok{);}

\CommentTok{// But, this change type is not observed, so no listener is triggered:}
\OtherTok{myTodo}\NormalTok{.}\FunctionTok{set}\NormalTok{(}\StringTok{'completed'}\NormalTok{, }\KeywordTok{true}\NormalTok{);}
\OtherTok{console}\NormalTok{.}\FunctionTok{log}\NormalTok{(}\StringTok{'Todo set as completed: '} \NormalTok{+ }\OtherTok{myTodo}\NormalTok{.}\FunctionTok{get}\NormalTok{(}\StringTok{'completed'}\NormalTok{));}

\CommentTok{// Above logs:}
\CommentTok{// This model has been initialized.}
\CommentTok{// Title value for this model has changed.}
\CommentTok{// Title value for this model has changed.}
\CommentTok{// Todo set as completed: true}
\end{Highlighting}
\end{Shaded}

\paragraph{Validation}\label{validation}

Backbone supports model validation through \texttt{model.validate()},
which allows checking the attribute values for a model prior to setting
them. By default, validation occurs when the model is persisted using
the \texttt{save()} method or when \texttt{set()} is called if
\texttt{\{validate:true\}} is passed as an argument.

\begin{Shaded}
\begin{Highlighting}[]
\KeywordTok{var} \NormalTok{Person = }\KeywordTok{new} \OtherTok{Backbone}\NormalTok{.}\FunctionTok{Model}\NormalTok{(\{}\DataTypeTok{name}\NormalTok{: }\StringTok{'Jeremy'}\NormalTok{\});}

\CommentTok{// Validate the model name}
\OtherTok{Person}\NormalTok{.}\FunctionTok{validate} \NormalTok{= }\KeywordTok{function}\NormalTok{(attrs) \{}
  \KeywordTok{if} \NormalTok{(!}\OtherTok{attrs}\NormalTok{.}\FunctionTok{name}\NormalTok{) \{}
    \KeywordTok{return} \StringTok{'I need your name'}\NormalTok{;}
  \NormalTok{\}}
\NormalTok{\};}

\CommentTok{// Change the name}
\OtherTok{Person}\NormalTok{.}\FunctionTok{set}\NormalTok{(\{}\DataTypeTok{name}\NormalTok{: }\StringTok{'Samuel'}\NormalTok{\});}
\OtherTok{console}\NormalTok{.}\FunctionTok{log}\NormalTok{(}\OtherTok{Person}\NormalTok{.}\FunctionTok{get}\NormalTok{(}\StringTok{'name'}\NormalTok{));}
\CommentTok{// 'Samuel'}

\CommentTok{// Remove the name attribute, force validation}
\OtherTok{Person}\NormalTok{.}\FunctionTok{unset}\NormalTok{(}\StringTok{'name'}\NormalTok{, \{}\DataTypeTok{validate}\NormalTok{: }\KeywordTok{true}\NormalTok{\});}
\CommentTok{// false}
\end{Highlighting}
\end{Shaded}

Above, we also use the \texttt{unset()} method, which removes an
attribute by deleting it from the internal model attributes hash.

Validation functions can be as simple or complex as necessary. If the
attributes provided are valid, nothing should be returned from
\texttt{.validate()}. If they are invalid, an error value should be
returned instead.

Should an error be returned:

\begin{itemize}
\itemsep1pt\parskip0pt\parsep0pt
\item
  An \texttt{invalid} event will be triggered, setting the
  \texttt{validationError} property on the model with the value which is
  returned by this method.
\item
  \texttt{.save()} will not continue and the attributes of the model
  will not be modified on the server.
\end{itemize}

A more complete validation example can be seen below:

\begin{Shaded}
\begin{Highlighting}[]
\KeywordTok{var} \NormalTok{Todo = }\OtherTok{Backbone}\NormalTok{.}\OtherTok{Model}\NormalTok{.}\FunctionTok{extend}\NormalTok{(\{}
  \DataTypeTok{defaults}\NormalTok{: \{}
    \DataTypeTok{completed}\NormalTok{: }\KeywordTok{false}
  \NormalTok{\},}

  \DataTypeTok{validate}\NormalTok{: }\KeywordTok{function}\NormalTok{(attributes)\{}
    \KeywordTok{if}\NormalTok{(}\OtherTok{attributes}\NormalTok{.}\FunctionTok{title} \NormalTok{=== }\KeywordTok{undefined}\NormalTok{)\{}
        \KeywordTok{return} \StringTok{"Remember to set a title for your todo."}\NormalTok{;}
    \NormalTok{\}}
  \NormalTok{\},}

  \DataTypeTok{initialize}\NormalTok{: }\KeywordTok{function}\NormalTok{()\{}
    \OtherTok{console}\NormalTok{.}\FunctionTok{log}\NormalTok{(}\StringTok{'This model has been initialized.'}\NormalTok{);}
    \KeywordTok{this}\NormalTok{.}\FunctionTok{on}\NormalTok{(}\StringTok{"invalid"}\NormalTok{, }\KeywordTok{function}\NormalTok{(model, error)\{}
        \OtherTok{console}\NormalTok{.}\FunctionTok{log}\NormalTok{(error);}
    \NormalTok{\});}
  \NormalTok{\}}
\NormalTok{\});}

\KeywordTok{var} \NormalTok{myTodo = }\KeywordTok{new} \FunctionTok{Todo}\NormalTok{();}
\OtherTok{myTodo}\NormalTok{.}\FunctionTok{set}\NormalTok{(}\StringTok{'completed'}\NormalTok{, }\KeywordTok{true}\NormalTok{, \{}\DataTypeTok{validate}\NormalTok{: }\KeywordTok{true}\NormalTok{\}); }\CommentTok{// logs: Remember to set a title for your todo.}
\OtherTok{console}\NormalTok{.}\FunctionTok{log}\NormalTok{(}\StringTok{'completed: '} \NormalTok{+ }\OtherTok{myTodo}\NormalTok{.}\FunctionTok{get}\NormalTok{(}\StringTok{'completed'}\NormalTok{)); }\CommentTok{// completed: false}
\end{Highlighting}
\end{Shaded}

\textbf{Note}: the \texttt{attributes} object passed to the
\texttt{validate} function represents what the attributes would be after
completing the current \texttt{set()} or \texttt{save()}. This object is
distinct from the current attributes of the model and from the
parameters passed to the operation. Since it is created by shallow copy,
it is not possible to change any Number, String, or Boolean attribute of
the input within the function, but it \emph{is} possible to change
attributes in nested objects.

An example of this (by @fivetanley) is available
\href{http://jsfiddle.net/2NdDY/270/}{here}.

Note also, that validation on initialization is possible but of limited
use, as the object being constructed is internally marked invalid but
nevertheless passed back to the caller (continuing the above example):

\begin{Shaded}
\begin{Highlighting}[]
\KeywordTok{var} \NormalTok{emptyTodo = }\KeywordTok{new} \FunctionTok{Todo}\NormalTok{(}\KeywordTok{null}\NormalTok{, \{}\DataTypeTok{validate}\NormalTok{: }\KeywordTok{true}\NormalTok{\});}
\OtherTok{console}\NormalTok{.}\FunctionTok{log}\NormalTok{(}\OtherTok{emptyTodo}\NormalTok{.}\FunctionTok{validationError}\NormalTok{);}
\end{Highlighting}
\end{Shaded}

\subsection{Views}\label{views-1}

Views in Backbone don't contain the HTML markup for your application;
they contain the logic behind the presentation of the model's data to
the user. This is usually achieved using JavaScript templating (e.g.,
Underscore Microtemplates, Mustache, jQuery-tmpl, etc.). A view's
\texttt{render()} method can be bound to a model's \texttt{change()}
event, enabling the view to instantly reflect model changes without
requiring a full page refresh.

\paragraph{Creating new views}\label{creating-new-views}

Creating a new view is relatively straightforward and similar to
creating new models. To create a new View, simply extend
\texttt{Backbone.View}. We introduced the sample TodoView below in the
previous chapter; now let's take a closer look at how it works:

\begin{Shaded}
\begin{Highlighting}[]
\KeywordTok{var} \NormalTok{TodoView = }\OtherTok{Backbone}\NormalTok{.}\OtherTok{View}\NormalTok{.}\FunctionTok{extend}\NormalTok{(\{}

  \DataTypeTok{tagName}\NormalTok{:  }\StringTok{'li'}\NormalTok{,}

  \CommentTok{// Cache the template function for a single item.}
  \DataTypeTok{todoTpl}\NormalTok{: }\OtherTok{_}\NormalTok{.}\FunctionTok{template}\NormalTok{( }\StringTok{"An example template"} \NormalTok{),}

  \DataTypeTok{events}\NormalTok{: \{}
    \StringTok{'dblclick label'}\NormalTok{: }\StringTok{'edit'}\NormalTok{,}
    \StringTok{'keypress .edit'}\NormalTok{: }\StringTok{'updateOnEnter'}\NormalTok{,}
    \StringTok{'blur .edit'}\NormalTok{:   }\StringTok{'close'}
  \NormalTok{\},}

  \DataTypeTok{initialize}\NormalTok{: }\KeywordTok{function} \NormalTok{(options) \{}
    \CommentTok{// In Backbone 1.1.0, if you want to access passed options in}
    \CommentTok{// your view, you will need to save them as follows:}
    \KeywordTok{this}\NormalTok{.}\FunctionTok{options} \NormalTok{= options || \{\};}
  \NormalTok{\},}

  \CommentTok{// Re-render the title of the todo item.}
  \DataTypeTok{render}\NormalTok{: }\KeywordTok{function}\NormalTok{() \{}
    \KeywordTok{this}\NormalTok{.}\OtherTok{$el}\NormalTok{.}\FunctionTok{html}\NormalTok{( }\KeywordTok{this}\NormalTok{.}\FunctionTok{todoTpl}\NormalTok{( }\KeywordTok{this}\NormalTok{.}\OtherTok{model}\NormalTok{.}\FunctionTok{attributes} \NormalTok{) );}
    \KeywordTok{this}\NormalTok{.}\FunctionTok{input} \NormalTok{= }\KeywordTok{this}\NormalTok{.}\FunctionTok{$}\NormalTok{(}\StringTok{'.edit'}\NormalTok{);}
    \KeywordTok{return} \KeywordTok{this}\NormalTok{;}
  \NormalTok{\},}

  \DataTypeTok{edit}\NormalTok{: }\KeywordTok{function}\NormalTok{() \{}
    \CommentTok{// executed when todo label is double clicked}
  \NormalTok{\},}

  \DataTypeTok{close}\NormalTok{: }\KeywordTok{function}\NormalTok{() \{}
    \CommentTok{// executed when todo loses focus}
  \NormalTok{\},}

  \DataTypeTok{updateOnEnter}\NormalTok{: }\KeywordTok{function}\NormalTok{( e ) \{}
    \CommentTok{// executed on each keypress when in todo edit mode,}
    \CommentTok{// but we'll wait for enter to get in action}
  \NormalTok{\}}
\NormalTok{\});}

\KeywordTok{var} \NormalTok{todoView = }\KeywordTok{new} \FunctionTok{TodoView}\NormalTok{();}

\CommentTok{// log reference to a DOM element that corresponds to the view instance}
\OtherTok{console}\NormalTok{.}\FunctionTok{log}\NormalTok{(}\OtherTok{todoView}\NormalTok{.}\FunctionTok{el}\NormalTok{); }\CommentTok{// logs <li></li>}
\end{Highlighting}
\end{Shaded}

\paragraph{What is \texttt{el}?}\label{what-is-el}

The central property of a view is \texttt{el} (the value logged in the
last statement of the example). What is \texttt{el} and how is it
defined?

\texttt{el} is basically a reference to a DOM element and all views must
have one. Views can use \texttt{el} to compose their element's content
and then insert it into the DOM all at once, which makes for faster
rendering because the browser performs the minimum required number of
reflows and repaints.

There are two ways to associate a DOM element with a view: a new element
can be created for the view and subsequently added to the DOM or a
reference can be made to an element which already exists in the page.

If you want to create a new element for your view, set any combination
of the following properties on the view: \texttt{tagName}, \texttt{id},
and \texttt{className}. A new element will be created for you by the
framework and a reference to it will be available at the \texttt{el}
property. If nothing is specified \texttt{tagName} defaults to
\texttt{div}.

In the example above, \texttt{tagName} is set to `li', resulting in
creation of an li element. The following example creates a ul element
with id and class attributes:

\begin{Shaded}
\begin{Highlighting}[]
\KeywordTok{var} \NormalTok{TodosView = }\OtherTok{Backbone}\NormalTok{.}\OtherTok{View}\NormalTok{.}\FunctionTok{extend}\NormalTok{(\{}
  \DataTypeTok{tagName}\NormalTok{: }\StringTok{'ul'}\NormalTok{, }\CommentTok{// required, but defaults to 'div' if not set}
  \DataTypeTok{className}\NormalTok{: }\StringTok{'container'}\NormalTok{, }\CommentTok{// optional, you can assign multiple classes to }
                          \CommentTok{// this property like so: 'container homepage'}
  \DataTypeTok{id}\NormalTok{: }\StringTok{'todos'} \CommentTok{// optional}
\NormalTok{\});}

\KeywordTok{var} \NormalTok{todosView = }\KeywordTok{new} \FunctionTok{TodosView}\NormalTok{();}
\OtherTok{console}\NormalTok{.}\FunctionTok{log}\NormalTok{(}\OtherTok{todosView}\NormalTok{.}\FunctionTok{el}\NormalTok{); }\CommentTok{// logs <ul id="todos" class="container"></ul>}
\end{Highlighting}
\end{Shaded}

The above code creates the DOM element below but doesn't append it to
the DOM.

\begin{Shaded}
\begin{Highlighting}[]
\KeywordTok{<ul}\OtherTok{ id=}\StringTok{"todos"}\OtherTok{ class=}\StringTok{"container"}\KeywordTok{></ul>}
\end{Highlighting}
\end{Shaded}

If the element already exists in the page, you can set \texttt{el} as a
CSS selector that matches the element.

\begin{Shaded}
\begin{Highlighting}[]
\NormalTok{el: }\StringTok{'#footer'}
\end{Highlighting}
\end{Shaded}

Alternatively, you can set \texttt{el} to an existing element when
creating the view:

\begin{Shaded}
\begin{Highlighting}[]
\KeywordTok{var} \NormalTok{todosView = }\KeywordTok{new} \FunctionTok{TodosView}\NormalTok{(\{}\DataTypeTok{el}\NormalTok{: }\FunctionTok{$}\NormalTok{(}\StringTok{'#footer'}\NormalTok{)\});}
\end{Highlighting}
\end{Shaded}

Note: When declaring a View, \texttt{options}, \texttt{el},
\texttt{tagName}, \texttt{id} and \texttt{className} may be defined as
functions, if you want their values to be determined at runtime.

\textbf{$el and $()}

View logic often needs to invoke jQuery or Zepto functions on the
\texttt{el} element and elements nested within it. Backbone makes it
easy to do so by defining the \texttt{\$el} property and \texttt{\$()}
function. The \texttt{view.\$el} property is equivalent to
\texttt{\$(view.el)} and \texttt{view.\$(selector)} is equivalent to
\texttt{\$(view.el).find(selector)}. In our TodoView example's render
method, we see \texttt{this.\$el} used to set the HTML of the element
and \texttt{this.\$()} used to find subelements of class `edit'.

\textbf{setElement}

If you need to apply an existing Backbone view to a different DOM
element \texttt{setElement} can be used for this purpose. Overriding
this.el needs to both change the DOM reference and re-bind events to the
new element (and unbind from the old).

\texttt{setElement} will create a cached \texttt{\$el} reference for
you, moving the delegated events for a view from the old element to the
new one.

\begin{Shaded}
\begin{Highlighting}[]

\CommentTok{// We create two DOM elements representing buttons}
\CommentTok{// which could easily be containers or something else}
\KeywordTok{var} \NormalTok{button1 = }\FunctionTok{$}\NormalTok{(}\StringTok{'<button></button>'}\NormalTok{);}
\KeywordTok{var} \NormalTok{button2 = }\FunctionTok{$}\NormalTok{(}\StringTok{'<button></button>'}\NormalTok{);}

\CommentTok{// Define a new view}
\KeywordTok{var} \NormalTok{View = }\OtherTok{Backbone}\NormalTok{.}\OtherTok{View}\NormalTok{.}\FunctionTok{extend}\NormalTok{(\{}
      \DataTypeTok{events}\NormalTok{: \{}
        \DataTypeTok{click}\NormalTok{: }\KeywordTok{function}\NormalTok{(e) \{}
          \OtherTok{console}\NormalTok{.}\FunctionTok{log}\NormalTok{(}\OtherTok{view}\NormalTok{.}\FunctionTok{el} \NormalTok{=== }\OtherTok{e}\NormalTok{.}\FunctionTok{target}\NormalTok{);}
        \NormalTok{\}}
      \NormalTok{\}}
    \NormalTok{\});}

\CommentTok{// Create a new instance of the view, applying it}
\CommentTok{// to button1}
\KeywordTok{var} \NormalTok{view = }\KeywordTok{new} \FunctionTok{View}\NormalTok{(\{}\DataTypeTok{el}\NormalTok{: button1\});}

\CommentTok{// Apply the view to button2 using setElement}
\OtherTok{view}\NormalTok{.}\FunctionTok{setElement}\NormalTok{(button2);}

\OtherTok{button1}\NormalTok{.}\FunctionTok{trigger}\NormalTok{(}\StringTok{'click'}\NormalTok{); }
\OtherTok{button2}\NormalTok{.}\FunctionTok{trigger}\NormalTok{(}\StringTok{'click'}\NormalTok{); }\CommentTok{// returns true}
\end{Highlighting}
\end{Shaded}

The ``el'' property represents the markup portion of the view that will
be rendered; to get the view to actually render to the page, you need to
add it as a new element or append it to an existing element.

\begin{Shaded}
\begin{Highlighting}[]

\CommentTok{// We can also provide raw markup to setElement}
\CommentTok{// as follows (just to demonstrate it can be done):}
\KeywordTok{var} \NormalTok{view = }\KeywordTok{new} \OtherTok{Backbone}\NormalTok{.}\FunctionTok{View}\NormalTok{;}
\OtherTok{view}\NormalTok{.}\FunctionTok{setElement}\NormalTok{(}\StringTok{'<p><a><b>test</b></a></p>'}\NormalTok{);}
\OtherTok{console}\NormalTok{.}\FunctionTok{log}\NormalTok{(}\OtherTok{view}\NormalTok{.}\FunctionTok{$}\NormalTok{(}\StringTok{'a b'}\NormalTok{).}\FunctionTok{html}\NormalTok{()); }\CommentTok{// outputs "test"}
\end{Highlighting}
\end{Shaded}

\textbf{Understanding \texttt{render()}}

\texttt{render()} is an optional function that defines the logic for
rendering a template. We'll use Underscore's micro-templating in these
examples, but remember you can use other templating frameworks if you
prefer. Our example will reference the following HTML markup:

\begin{Shaded}
\begin{Highlighting}[]
\ErrorTok{<}\NormalTok{!doctype html>}
\KeywordTok{<html}\OtherTok{ lang=}\StringTok{"en"}\KeywordTok{>}
\KeywordTok{<head>}
  \KeywordTok{<meta}\OtherTok{ charset=}\StringTok{"utf-8"}\KeywordTok{>}
  \KeywordTok{<title></title>}
  \KeywordTok{<meta}\OtherTok{ name=}\StringTok{"description"}\OtherTok{ content=}\StringTok{""}\KeywordTok{>}
\KeywordTok{</head>}
\KeywordTok{<body>}
  \KeywordTok{<div}\OtherTok{ id=}\StringTok{"todo"}\KeywordTok{>}
  \KeywordTok{</div>}
  \KeywordTok{<script}\OtherTok{ type=}\StringTok{"text/template"}\OtherTok{ id=}\StringTok{"item-template"}\KeywordTok{>}
    \NormalTok{<div>}
      \NormalTok{<input id=}\StringTok{"todo_complete"} \NormalTok{type=}\StringTok{"checkbox"} \NormalTok{<%= completed ? }\StringTok{'checked="checked"'} \NormalTok{: }\StringTok{''} \NormalTok{%>>}
      \NormalTok{<%= title %>}
    \NormalTok{<}\OtherTok{/div>}
\OtherTok{  </script}\NormalTok{>}
  \NormalTok{<script src=}\StringTok{"underscore-min.js"}\NormalTok{>}\KeywordTok{</script>}
  \KeywordTok{<script}\OtherTok{ src=}\StringTok{"backbone-min.js"}\KeywordTok{></script>}
  \KeywordTok{<script}\OtherTok{ src=}\StringTok{"jquery-min.js"}\KeywordTok{></script>}
  \KeywordTok{<script}\OtherTok{ src=}\StringTok{"example.js"}\KeywordTok{></script>}
\KeywordTok{</body>}
\KeywordTok{</html>}
\end{Highlighting}
\end{Shaded}

The \texttt{\_.template} method in Underscore compiles JavaScript
templates into functions which can be evaluated for rendering. In the
TodoView, I'm passing the markup from the template with id
\texttt{item-template} to \texttt{\_.template()} to be compiled and
stored in the todoTpl property when the view is created.

The \texttt{render()} method uses this template by passing it the
\texttt{toJSON()} encoding of the attributes of the model associated
with the view. The template returns its markup after using the model's
title and completed flag to evaluate the expressions containing them. I
then set this markup as the HTML content of the \texttt{el} DOM element
using the \texttt{\$el} property.

Presto! This populates the template, giving you a data-complete set of
markup in just a few short lines of code.

A common Backbone convention is to return \texttt{this} at the end of
\texttt{render()}. This is useful for a number of reasons, including:

\begin{itemize}
\itemsep1pt\parskip0pt\parsep0pt
\item
  Making views easily reusable in other parent views.
\item
  Creating a list of elements without rendering and painting each of
  them individually, only to be drawn once the entire list is populated.
\end{itemize}

Let's try to implement the latter of these. The \texttt{render} method
of a simple ListView which doesn't use an ItemView for each item could
be written:

\begin{Shaded}
\begin{Highlighting}[]

\KeywordTok{var} \NormalTok{ListView = }\OtherTok{Backbone}\NormalTok{.}\OtherTok{View}\NormalTok{.}\FunctionTok{extend}\NormalTok{(\{}

  \CommentTok{// Compile a template for this view. In this case '...'}
  \CommentTok{// is a placeholder for a template such as }
  \CommentTok{// $("#list_template").html() }
  \DataTypeTok{template}\NormalTok{: }\OtherTok{_}\NormalTok{.}\FunctionTok{template}\NormalTok{(…),}
  
  \DataTypeTok{render}\NormalTok{: }\KeywordTok{function}\NormalTok{() \{}
    \KeywordTok{this}\NormalTok{.}\OtherTok{$el}\NormalTok{.}\FunctionTok{html}\NormalTok{(}\KeywordTok{this}\NormalTok{.}\FunctionTok{template}\NormalTok{(}\KeywordTok{this}\NormalTok{.}\OtherTok{model}\NormalTok{.}\FunctionTok{attributes}\NormalTok{));}
    \KeywordTok{return} \KeywordTok{this}\NormalTok{;}
  \NormalTok{\}}
\NormalTok{\});}
\end{Highlighting}
\end{Shaded}

Simple enough. Let's now assume a decision is made to construct the
items using an ItemView to provide enhanced behaviour to our list. The
ItemView could be written:

\begin{Shaded}
\begin{Highlighting}[]

\KeywordTok{var} \NormalTok{ItemView = }\OtherTok{Backbone}\NormalTok{.}\OtherTok{View}\NormalTok{.}\FunctionTok{extend}\NormalTok{(\{}
  \DataTypeTok{events}\NormalTok{: \{\},}
  \DataTypeTok{render}\NormalTok{: }\KeywordTok{function}\NormalTok{()\{}
    \KeywordTok{this}\NormalTok{.}\OtherTok{$el}\NormalTok{.}\FunctionTok{html}\NormalTok{(}\KeywordTok{this}\NormalTok{.}\FunctionTok{template}\NormalTok{(}\KeywordTok{this}\NormalTok{.}\OtherTok{model}\NormalTok{.}\FunctionTok{attributes}\NormalTok{));}
    \KeywordTok{return} \KeywordTok{this}\NormalTok{;}
  \NormalTok{\}}
\NormalTok{\});}
\end{Highlighting}
\end{Shaded}

Note the usage of \texttt{return this;} at the end of \texttt{render}.
This common pattern enables us to reuse the view as a sub-view. We can
also use it to pre-render the view prior to rendering. Using this
requires that we make a change to our ListView's \texttt{render} method
as follows:

\begin{Shaded}
\begin{Highlighting}[]

\KeywordTok{var} \NormalTok{ListView = }\OtherTok{Backbone}\NormalTok{.}\OtherTok{View}\NormalTok{.}\FunctionTok{extend}\NormalTok{(\{}
  \DataTypeTok{render}\NormalTok{: }\KeywordTok{function}\NormalTok{()\{}

    \CommentTok{// Assume our model exposes the items we will}
    \CommentTok{// display in our list}
    \KeywordTok{var} \NormalTok{items = }\KeywordTok{this}\NormalTok{.}\OtherTok{model}\NormalTok{.}\FunctionTok{get}\NormalTok{(}\StringTok{'items'}\NormalTok{);}

    \CommentTok{// Loop through each of our items using the Underscore}
    \CommentTok{// _.each iterator}
    \OtherTok{_}\NormalTok{.}\FunctionTok{each}\NormalTok{(items, }\KeywordTok{function}\NormalTok{(item)\{}

      \CommentTok{// Create a new instance of the ItemView, passing }
      \CommentTok{// it a specific model item}
      \KeywordTok{var} \NormalTok{itemView = }\KeywordTok{new} \FunctionTok{ItemView}\NormalTok{(\{ }\DataTypeTok{model}\NormalTok{: item \});}
      \CommentTok{// The itemView's DOM element is appended after it}
      \CommentTok{// has been rendered. Here, the 'return this' is helpful}
      \CommentTok{// as the itemView renders its model. Later, we ask for }
      \CommentTok{// its output ("el")}
      \KeywordTok{this}\NormalTok{.}\OtherTok{$el}\NormalTok{.}\FunctionTok{append}\NormalTok{( }\OtherTok{itemView}\NormalTok{.}\FunctionTok{render}\NormalTok{().}\FunctionTok{el} \NormalTok{);}
    \NormalTok{\}, }\KeywordTok{this}\NormalTok{);}
  \NormalTok{\}}
\NormalTok{\});}
\end{Highlighting}
\end{Shaded}

\textbf{The \texttt{events} hash}

The Backbone \texttt{events} hash allows us to attach event listeners to
either \texttt{el}-relative custom selectors, or directly to \texttt{el}
if no selector is provided. An event takes the form of a key-value pair
\texttt{'eventName selector': 'callbackFunction'} and a number of DOM
event-types are supported, including \texttt{click}, \texttt{submit},
\texttt{mouseover}, \texttt{dblclick} and more.

\begin{Shaded}
\begin{Highlighting}[]

\CommentTok{// A sample view}
\KeywordTok{var} \NormalTok{TodoView = }\OtherTok{Backbone}\NormalTok{.}\OtherTok{View}\NormalTok{.}\FunctionTok{extend}\NormalTok{(\{}
  \DataTypeTok{tagName}\NormalTok{:  }\StringTok{'li'}\NormalTok{,}

  \CommentTok{// with an events hash containing DOM events}
  \CommentTok{// specific to an item:}
  \DataTypeTok{events}\NormalTok{: \{}
    \StringTok{'click .toggle'}\NormalTok{: }\StringTok{'toggleCompleted'}\NormalTok{,}
    \StringTok{'dblclick label'}\NormalTok{: }\StringTok{'edit'}\NormalTok{,}
    \StringTok{'keypress .edit'}\NormalTok{: }\StringTok{'updateOnEnter'}\NormalTok{,}
    \StringTok{'click .destroy'}\NormalTok{: }\StringTok{'clear'}\NormalTok{,}
    \StringTok{'blur .edit'}\NormalTok{: }\StringTok{'close'}
  \NormalTok{\},}
\end{Highlighting}
\end{Shaded}

What isn't instantly obvious is that while Backbone uses jQuery's
\texttt{.delegate()} underneath, it goes further by extending it so that
\texttt{this} always refers to the current view object within callback
functions. The only thing to really keep in mind is that any string
callback supplied to the events attribute must have a corresponding
function with the same name within the scope of your view.

The declarative, delegated jQuery events means that you don't have to
worry about whether a particular element has been rendered to the DOM
yet or not. Usually with jQuery you have to worry about ``presence or
absence in the DOM'' all the time when binding events.

In our TodoView example, the edit callback is invoked when the user
double-clicks a label element within the \texttt{el} element,
updateOnEnter is called for each keypress in an element with class
`edit', and close executes when an element with class `edit' loses
focus. Each of these callback functions can use \texttt{this} to refer
to the TodoView object.

Note that you can also bind methods yourself using
\texttt{\_.bind(this.viewEvent, this)}, which is effectively what the
value in each event's key-value pair is doing. Below we use
\texttt{\_.bind} to re-render our view when a model changes.

\begin{Shaded}
\begin{Highlighting}[]

\KeywordTok{var} \NormalTok{TodoView = }\OtherTok{Backbone}\NormalTok{.}\OtherTok{View}\NormalTok{.}\FunctionTok{extend}\NormalTok{(\{}
  \DataTypeTok{initialize}\NormalTok{: }\KeywordTok{function}\NormalTok{() \{}
    \KeywordTok{this}\NormalTok{.}\OtherTok{model}\NormalTok{.}\FunctionTok{bind}\NormalTok{(}\StringTok{'change'}\NormalTok{, }\OtherTok{_}\NormalTok{.}\FunctionTok{bind}\NormalTok{(}\KeywordTok{this}\NormalTok{.}\FunctionTok{render}\NormalTok{, }\KeywordTok{this}\NormalTok{));}
  \NormalTok{\}}
\NormalTok{\});}
\end{Highlighting}
\end{Shaded}

\texttt{\_.bind} only works on one method at a time, but effectively
binds a function to an object so that anytime the function is called the
value of \texttt{this} will be the object. \texttt{\_.bind} also
supports passing in arguments to the function in order to fill them in
advance - a technique known as
\href{http://benalman.com/news/2012/09/partial-application-in-javascript/}{partial
application}.

\subsection{Collections}\label{collections}

Collections are sets of Models and are created by extending
\texttt{Backbone.Collection}.

Normally, when creating a collection you'll also want to define a
property specifying the type of model that your collection will contain,
along with any instance properties required.

In the following example, we create a TodoCollection that will contain
our Todo models:

\begin{Shaded}
\begin{Highlighting}[]
\KeywordTok{var} \NormalTok{Todo = }\OtherTok{Backbone}\NormalTok{.}\OtherTok{Model}\NormalTok{.}\FunctionTok{extend}\NormalTok{(\{}
  \DataTypeTok{defaults}\NormalTok{: \{}
    \DataTypeTok{title}\NormalTok{: }\StringTok{''}\NormalTok{,}
    \DataTypeTok{completed}\NormalTok{: }\KeywordTok{false}
  \NormalTok{\}}
\NormalTok{\});}

\KeywordTok{var} \NormalTok{TodosCollection = }\OtherTok{Backbone}\NormalTok{.}\OtherTok{Collection}\NormalTok{.}\FunctionTok{extend}\NormalTok{(\{}
  \DataTypeTok{model}\NormalTok{: Todo}
\NormalTok{\});}

\KeywordTok{var} \NormalTok{myTodo = }\KeywordTok{new} \FunctionTok{Todo}\NormalTok{(\{}\DataTypeTok{title}\NormalTok{:}\StringTok{'Read the whole book'}\NormalTok{, }\DataTypeTok{id}\NormalTok{: }\DecValTok{2}\NormalTok{\});}

\CommentTok{// pass array of models on collection instantiation}
\KeywordTok{var} \NormalTok{todos = }\KeywordTok{new} \FunctionTok{TodosCollection}\NormalTok{([myTodo]);}
\OtherTok{console}\NormalTok{.}\FunctionTok{log}\NormalTok{(}\StringTok{"Collection size: "} \NormalTok{+ }\OtherTok{todos}\NormalTok{.}\FunctionTok{length}\NormalTok{); }\CommentTok{// Collection size: 1}
\end{Highlighting}
\end{Shaded}

\paragraph{Adding and Removing Models}\label{adding-and-removing-models}

The preceding example populated the collection using an array of models
when it was instantiated. After a collection has been created, models
can be added and removed using the \texttt{add()} and \texttt{remove()}
methods:

\begin{Shaded}
\begin{Highlighting}[]
\KeywordTok{var} \NormalTok{Todo = }\OtherTok{Backbone}\NormalTok{.}\OtherTok{Model}\NormalTok{.}\FunctionTok{extend}\NormalTok{(\{}
  \DataTypeTok{defaults}\NormalTok{: \{}
    \DataTypeTok{title}\NormalTok{: }\StringTok{''}\NormalTok{,}
    \DataTypeTok{completed}\NormalTok{: }\KeywordTok{false}
  \NormalTok{\}}
\NormalTok{\});}

\KeywordTok{var} \NormalTok{TodosCollection = }\OtherTok{Backbone}\NormalTok{.}\OtherTok{Collection}\NormalTok{.}\FunctionTok{extend}\NormalTok{(\{}
  \DataTypeTok{model}\NormalTok{: Todo,}
\NormalTok{\});}

\KeywordTok{var} \NormalTok{a = }\KeywordTok{new} \FunctionTok{Todo}\NormalTok{(\{ }\DataTypeTok{title}\NormalTok{: }\StringTok{'Go to Jamaica.'}\NormalTok{\}),}
    \NormalTok{b = }\KeywordTok{new} \FunctionTok{Todo}\NormalTok{(\{ }\DataTypeTok{title}\NormalTok{: }\StringTok{'Go to China.'}\NormalTok{\}),}
    \NormalTok{c = }\KeywordTok{new} \FunctionTok{Todo}\NormalTok{(\{ }\DataTypeTok{title}\NormalTok{: }\StringTok{'Go to Disneyland.'}\NormalTok{\});}

\KeywordTok{var} \NormalTok{todos = }\KeywordTok{new} \FunctionTok{TodosCollection}\NormalTok{([a,b]);}
\OtherTok{console}\NormalTok{.}\FunctionTok{log}\NormalTok{(}\StringTok{"Collection size: "} \NormalTok{+ }\OtherTok{todos}\NormalTok{.}\FunctionTok{length}\NormalTok{);}
\CommentTok{// Logs: Collection size: 2}

\OtherTok{todos}\NormalTok{.}\FunctionTok{add}\NormalTok{(c);}
\OtherTok{console}\NormalTok{.}\FunctionTok{log}\NormalTok{(}\StringTok{"Collection size: "} \NormalTok{+ }\OtherTok{todos}\NormalTok{.}\FunctionTok{length}\NormalTok{);}
\CommentTok{// Logs: Collection size: 3}

\OtherTok{todos}\NormalTok{.}\FunctionTok{remove}\NormalTok{([a,b]);}
\OtherTok{console}\NormalTok{.}\FunctionTok{log}\NormalTok{(}\StringTok{"Collection size: "} \NormalTok{+ }\OtherTok{todos}\NormalTok{.}\FunctionTok{length}\NormalTok{);}
\CommentTok{// Logs: Collection size: 1}

\OtherTok{todos}\NormalTok{.}\FunctionTok{remove}\NormalTok{(c);}
\OtherTok{console}\NormalTok{.}\FunctionTok{log}\NormalTok{(}\StringTok{"Collection size: "} \NormalTok{+ }\OtherTok{todos}\NormalTok{.}\FunctionTok{length}\NormalTok{);}
\CommentTok{// Logs: Collection size: 0}
\end{Highlighting}
\end{Shaded}

Note that \texttt{add()} and \texttt{remove()} accept both individual
models and lists of models.

Also note that when using \texttt{add()} on a collection, passing
\texttt{\{merge: true\}} causes duplicate models to have their
attributes merged in to the existing models, instead of being ignored.

\begin{Shaded}
\begin{Highlighting}[]
\KeywordTok{var} \NormalTok{items = }\KeywordTok{new} \OtherTok{Backbone}\NormalTok{.}\FunctionTok{Collection}\NormalTok{;}
\OtherTok{items}\NormalTok{.}\FunctionTok{add}\NormalTok{([\{ }\DataTypeTok{id }\NormalTok{: }\DecValTok{1}\NormalTok{, }\DataTypeTok{name}\NormalTok{: }\StringTok{"Dog"} \NormalTok{, }\DataTypeTok{age}\NormalTok{: }\DecValTok{3}\NormalTok{\}, \{ }\DataTypeTok{id }\NormalTok{: }\DecValTok{2}\NormalTok{, }\DataTypeTok{name}\NormalTok{: }\StringTok{"cat"} \NormalTok{, }\DataTypeTok{age}\NormalTok{: }\DecValTok{2}\NormalTok{\}]);}
\OtherTok{items}\NormalTok{.}\FunctionTok{add}\NormalTok{([\{ }\DataTypeTok{id }\NormalTok{: }\DecValTok{1}\NormalTok{, }\DataTypeTok{name}\NormalTok{: }\StringTok{"Bear"} \NormalTok{\}], \{}\DataTypeTok{merge}\NormalTok{: }\KeywordTok{true} \NormalTok{\});}
\OtherTok{items}\NormalTok{.}\FunctionTok{add}\NormalTok{([\{ }\DataTypeTok{id }\NormalTok{: }\DecValTok{2}\NormalTok{, }\DataTypeTok{name}\NormalTok{: }\StringTok{"lion"} \NormalTok{\}]); }\CommentTok{// merge: false}
 
\OtherTok{console}\NormalTok{.}\FunctionTok{log}\NormalTok{(}\OtherTok{JSON}\NormalTok{.}\FunctionTok{stringify}\NormalTok{(}\OtherTok{items}\NormalTok{.}\FunctionTok{toJSON}\NormalTok{()));}
\CommentTok{// [\{"id":1,"name":"Bear","age":3\},\{"id":2,"name":"cat","age":2\}]}
\end{Highlighting}
\end{Shaded}

\paragraph{Retrieving Models}\label{retrieving-models}

There are a few different ways to retrieve a model from a collection.
The most straight-forward is to use \texttt{Collection.get()} which
accepts a single id as follows:

\begin{Shaded}
\begin{Highlighting}[]
\KeywordTok{var} \NormalTok{myTodo = }\KeywordTok{new} \FunctionTok{Todo}\NormalTok{(\{}\DataTypeTok{title}\NormalTok{:}\StringTok{'Read the whole book'}\NormalTok{, }\DataTypeTok{id}\NormalTok{: }\DecValTok{2}\NormalTok{\});}

\CommentTok{// pass array of models on collection instantiation}
\KeywordTok{var} \NormalTok{todos = }\KeywordTok{new} \FunctionTok{TodosCollection}\NormalTok{([myTodo]);}

\KeywordTok{var} \NormalTok{todo2 = }\OtherTok{todos}\NormalTok{.}\FunctionTok{get}\NormalTok{(}\DecValTok{2}\NormalTok{);}

\CommentTok{// Models, as objects, are passed by reference}
\OtherTok{console}\NormalTok{.}\FunctionTok{log}\NormalTok{(todo2 === myTodo); }\CommentTok{// true}
\end{Highlighting}
\end{Shaded}

In client-server applications, collections contain models obtained from
the server. Anytime you're exchanging data between the client and a
server, you will need a way to uniquely identify models. In Backbone,
this is done using the \texttt{id}, \texttt{cid}, and
\texttt{idAttribute} properties.

Each model in Backbone has an \texttt{id}, which is a unique identifier
that is either an integer or string (e.g., a UUID). Models also have a
\texttt{cid} (client id) which is automatically generated by Backbone
when the model is created. Either identifier can be used to retrieve a
model from a collection.

The main difference between them is that the \texttt{cid} is generated
by Backbone; it is helpful when you don't have a true id - this may be
the case if your model has yet to be saved to the server or you aren't
saving it to a database.

The \texttt{idAttribute} is the identifying attribute name of the model
returned from the server (i.e., the \texttt{id} in your database). This
tells Backbone which data field from the server should be used to
populate the \texttt{id} property (think of it as a mapper). By default,
it assumes \texttt{id}, but this can be customized as needed. For
instance, if your server sets a unique attribute on your model named
``userId'' then you would set \texttt{idAttribute} to ``userId'' in your
model definition.

The value of a model's idAttribute should be set by the server when the
model is saved. After this point you shouldn't need to set it manually,
unless further control is required.

Internally, \texttt{Backbone.Collection} contains an array of models
enumerated by their \texttt{id} property, if the model instances happen
to have one. When \texttt{collection.get(id)} is called, this array is
checked for existence of the model instance with the corresponding
\texttt{id}.

\begin{Shaded}
\begin{Highlighting}[]
\CommentTok{// extends the previous example}

\KeywordTok{var} \NormalTok{todoCid = }\OtherTok{todos}\NormalTok{.}\FunctionTok{get}\NormalTok{(}\OtherTok{todo2}\NormalTok{.}\FunctionTok{cid}\NormalTok{);}

\CommentTok{// As mentioned in previous example, }
\CommentTok{// models are passed by reference}
\OtherTok{console}\NormalTok{.}\FunctionTok{log}\NormalTok{(todoCid === myTodo); }\CommentTok{// true}
\end{Highlighting}
\end{Shaded}

\paragraph{Listening for events}\label{listening-for-events}

As collections represent a group of items, we can listen for
\texttt{add} and \texttt{remove} events which occur when models are
added to or removed from a collection. Here's an example:

\begin{Shaded}
\begin{Highlighting}[]
\KeywordTok{var} \NormalTok{TodosCollection = }\KeywordTok{new} \OtherTok{Backbone}\NormalTok{.}\FunctionTok{Collection}\NormalTok{();}

\OtherTok{TodosCollection}\NormalTok{.}\FunctionTok{on}\NormalTok{(}\StringTok{"add"}\NormalTok{, }\KeywordTok{function}\NormalTok{(todo) \{}
  \OtherTok{console}\NormalTok{.}\FunctionTok{log}\NormalTok{(}\StringTok{"I should "} \NormalTok{+ }\OtherTok{todo}\NormalTok{.}\FunctionTok{get}\NormalTok{(}\StringTok{"title"}\NormalTok{) + }\StringTok{". Have I done it before? "}  \NormalTok{+ (}\OtherTok{todo}\NormalTok{.}\FunctionTok{get}\NormalTok{(}\StringTok{"completed"}\NormalTok{) ? }\StringTok{'Yeah!'}\NormalTok{: }\StringTok{'No.'} \NormalTok{));}
\NormalTok{\});}

\OtherTok{TodosCollection}\NormalTok{.}\FunctionTok{add}\NormalTok{([}
  \NormalTok{\{ }\DataTypeTok{title}\NormalTok{: }\StringTok{'go to Jamaica'}\NormalTok{, }\DataTypeTok{completed}\NormalTok{: }\KeywordTok{false} \NormalTok{\},}
  \NormalTok{\{ }\DataTypeTok{title}\NormalTok{: }\StringTok{'go to China'}\NormalTok{, }\DataTypeTok{completed}\NormalTok{: }\KeywordTok{false} \NormalTok{\},}
  \NormalTok{\{ }\DataTypeTok{title}\NormalTok{: }\StringTok{'go to Disneyland'}\NormalTok{, }\DataTypeTok{completed}\NormalTok{: }\KeywordTok{true} \NormalTok{\}}
\NormalTok{]);}

\CommentTok{// The above logs:}
\CommentTok{// I should go to Jamaica. Have I done it before? No.}
\CommentTok{// I should go to China. Have I done it before? No.}
\CommentTok{// I should go to Disneyland. Have I done it before? Yeah!}
\end{Highlighting}
\end{Shaded}

In addition, we're also able to bind to a \texttt{change} event to
listen for changes to any of the models in the collection.

\begin{Shaded}
\begin{Highlighting}[]
\KeywordTok{var} \NormalTok{TodosCollection = }\KeywordTok{new} \OtherTok{Backbone}\NormalTok{.}\FunctionTok{Collection}\NormalTok{();}

\CommentTok{// log a message if a model in the collection changes}
\OtherTok{TodosCollection}\NormalTok{.}\FunctionTok{on}\NormalTok{(}\StringTok{"change:title"}\NormalTok{, }\KeywordTok{function}\NormalTok{(model) \{}
    \OtherTok{console}\NormalTok{.}\FunctionTok{log}\NormalTok{(}\StringTok{"Changed my mind! I should "} \NormalTok{+ }\OtherTok{model}\NormalTok{.}\FunctionTok{get}\NormalTok{(}\StringTok{'title'}\NormalTok{));}
\NormalTok{\});}

\OtherTok{TodosCollection}\NormalTok{.}\FunctionTok{add}\NormalTok{([}
  \NormalTok{\{ }\DataTypeTok{title}\NormalTok{: }\StringTok{'go to Jamaica.'}\NormalTok{, }\DataTypeTok{completed}\NormalTok{: }\KeywordTok{false}\NormalTok{, }\DataTypeTok{id}\NormalTok{: }\DecValTok{3} \NormalTok{\},}
\NormalTok{]);}

\KeywordTok{var} \NormalTok{myTodo = }\OtherTok{TodosCollection}\NormalTok{.}\FunctionTok{get}\NormalTok{(}\DecValTok{3}\NormalTok{);}

\OtherTok{myTodo}\NormalTok{.}\FunctionTok{set}\NormalTok{(}\StringTok{'title'}\NormalTok{, }\StringTok{'go fishing'}\NormalTok{);}
\CommentTok{// Logs: Changed my mind! I should go fishing}
\end{Highlighting}
\end{Shaded}

jQuery-style event maps of the form \texttt{obj.on(\{click: action\})}
can also be used. These can be clearer than needing three separate calls
to \texttt{.on} and should align better with the events hash used in
Views:

\begin{Shaded}
\begin{Highlighting}[]

\KeywordTok{var} \NormalTok{Todo = }\OtherTok{Backbone}\NormalTok{.}\OtherTok{Model}\NormalTok{.}\FunctionTok{extend}\NormalTok{(\{}
  \DataTypeTok{defaults}\NormalTok{: \{}
    \DataTypeTok{title}\NormalTok{: }\StringTok{''}\NormalTok{,}
    \DataTypeTok{completed}\NormalTok{: }\KeywordTok{false}
  \NormalTok{\}}
\NormalTok{\});}

\KeywordTok{var} \NormalTok{myTodo = }\KeywordTok{new} \FunctionTok{Todo}\NormalTok{();}
\OtherTok{myTodo}\NormalTok{.}\FunctionTok{set}\NormalTok{(\{}\DataTypeTok{title}\NormalTok{: }\StringTok{'Buy some cookies'}\NormalTok{, }\DataTypeTok{completed}\NormalTok{: }\KeywordTok{true}\NormalTok{\});}

\OtherTok{myTodo}\NormalTok{.}\FunctionTok{on}\NormalTok{(\{}
   \StringTok{'change:title'} \NormalTok{: titleChanged,}
   \StringTok{'change:completed'} \NormalTok{: stateChanged}
\NormalTok{\});}

\KeywordTok{function} \FunctionTok{titleChanged}\NormalTok{()\{}
  \OtherTok{console}\NormalTok{.}\FunctionTok{log}\NormalTok{(}\StringTok{'The title was changed!'}\NormalTok{);}
\NormalTok{\}}

\KeywordTok{function} \FunctionTok{stateChanged}\NormalTok{()\{}
  \OtherTok{console}\NormalTok{.}\FunctionTok{log}\NormalTok{(}\StringTok{'The state was changed!'}\NormalTok{);}
\NormalTok{\}}

\OtherTok{myTodo}\NormalTok{.}\FunctionTok{set}\NormalTok{(\{}\DataTypeTok{title}\NormalTok{: }\StringTok{'Get the groceries'}\NormalTok{\});}
\CommentTok{// The title was changed! }
\end{Highlighting}
\end{Shaded}

Backbone events also support a
\href{http://backbonejs.org/\#Events-once}{once()} method, which ensures
that a callback only fires one time when a notification arrives. It is
similar to Node's
\href{http://nodejs.org/api/events.html\#events_emitter_once_event_listener}{once},
or jQuery's \href{http://api.jquery.com/one/}{one}. This is particularly
useful for when you want to say ``the next time something happens, do
this''.

\begin{Shaded}
\begin{Highlighting}[]
\CommentTok{// Define an object with two counters}
\KeywordTok{var} \NormalTok{TodoCounter = \{ }\DataTypeTok{counterA}\NormalTok{: }\DecValTok{0}\NormalTok{, }\DataTypeTok{counterB}\NormalTok{: }\DecValTok{0} \NormalTok{\};}
\CommentTok{// Mix in Backbone Events}
\OtherTok{_}\NormalTok{.}\FunctionTok{extend}\NormalTok{(TodoCounter, }\OtherTok{Backbone}\NormalTok{.}\FunctionTok{Events}\NormalTok{);}

\CommentTok{// Increment counterA, triggering an event}
\KeywordTok{var} \NormalTok{incrA = }\KeywordTok{function}\NormalTok{()\{ }
  \OtherTok{TodoCounter}\NormalTok{.}\FunctionTok{counterA} \NormalTok{+= }\DecValTok{1}\NormalTok{; }
  \CommentTok{// This triggering will not }
  \CommentTok{// produce any efect on the counters}
  \OtherTok{TodoCounter}\NormalTok{.}\FunctionTok{trigger}\NormalTok{(}\StringTok{'event'}\NormalTok{); }
\NormalTok{\};}

\CommentTok{// Increment counterB}
\KeywordTok{var} \NormalTok{incrB = }\KeywordTok{function}\NormalTok{()\{ }
  \OtherTok{TodoCounter}\NormalTok{.}\FunctionTok{counterB} \NormalTok{+= }\DecValTok{1}\NormalTok{; }
\NormalTok{\};}

\CommentTok{// Use once rather than having to explicitly unbind}
\CommentTok{// our event listener}
\OtherTok{TodoCounter}\NormalTok{.}\FunctionTok{once}\NormalTok{(}\StringTok{'event'}\NormalTok{, incrA);}
\OtherTok{TodoCounter}\NormalTok{.}\FunctionTok{once}\NormalTok{(}\StringTok{'event'}\NormalTok{, incrB);}

\CommentTok{// Trigger the event for the first time}
\OtherTok{TodoCounter}\NormalTok{.}\FunctionTok{trigger}\NormalTok{(}\StringTok{'event'}\NormalTok{);}

\CommentTok{// Check out output}
\OtherTok{console}\NormalTok{.}\FunctionTok{log}\NormalTok{(}\OtherTok{TodoCounter}\NormalTok{.}\FunctionTok{counterA} \NormalTok{=== }\DecValTok{1}\NormalTok{); }\CommentTok{// true}
\OtherTok{console}\NormalTok{.}\FunctionTok{log}\NormalTok{(}\OtherTok{TodoCounter}\NormalTok{.}\FunctionTok{counterB} \NormalTok{=== }\DecValTok{1}\NormalTok{); }\CommentTok{// true}
\end{Highlighting}
\end{Shaded}

\texttt{counterA} and \texttt{counterB} should only have been
incremented once.

\paragraph{Resetting/Refreshing
Collections}\label{resettingrefreshing-collections}

Rather than adding or removing models individually, you might want to
update an entire collection at once. \texttt{Collection.set()} takes an
array of models and performs the necessary add, remove, and change
operations required to update the collection.

\begin{Shaded}
\begin{Highlighting}[]
\KeywordTok{var} \NormalTok{TodosCollection = }\KeywordTok{new} \OtherTok{Backbone}\NormalTok{.}\FunctionTok{Collection}\NormalTok{();}

\OtherTok{TodosCollection}\NormalTok{.}\FunctionTok{add}\NormalTok{([}
    \NormalTok{\{ }\DataTypeTok{id}\NormalTok{: }\DecValTok{1}\NormalTok{, }\DataTypeTok{title}\NormalTok{: }\StringTok{'go to Jamaica.'}\NormalTok{, }\DataTypeTok{completed}\NormalTok{: }\KeywordTok{false} \NormalTok{\},}
    \NormalTok{\{ }\DataTypeTok{id}\NormalTok{: }\DecValTok{2}\NormalTok{, }\DataTypeTok{title}\NormalTok{: }\StringTok{'go to China.'}\NormalTok{, }\DataTypeTok{completed}\NormalTok{: }\KeywordTok{false} \NormalTok{\},}
    \NormalTok{\{ }\DataTypeTok{id}\NormalTok{: }\DecValTok{3}\NormalTok{, }\DataTypeTok{title}\NormalTok{: }\StringTok{'go to Disneyland.'}\NormalTok{, }\DataTypeTok{completed}\NormalTok{: }\KeywordTok{true} \NormalTok{\}}
\NormalTok{]);}

\CommentTok{// we can listen for add/change/remove events}
\OtherTok{TodosCollection}\NormalTok{.}\FunctionTok{on}\NormalTok{(}\StringTok{"add"}\NormalTok{, }\KeywordTok{function}\NormalTok{(model) \{}
  \OtherTok{console}\NormalTok{.}\FunctionTok{log}\NormalTok{(}\StringTok{"Added "} \NormalTok{+ }\OtherTok{model}\NormalTok{.}\FunctionTok{get}\NormalTok{(}\StringTok{'title'}\NormalTok{));}
\NormalTok{\});}

\OtherTok{TodosCollection}\NormalTok{.}\FunctionTok{on}\NormalTok{(}\StringTok{"remove"}\NormalTok{, }\KeywordTok{function}\NormalTok{(model) \{}
  \OtherTok{console}\NormalTok{.}\FunctionTok{log}\NormalTok{(}\StringTok{"Removed "} \NormalTok{+ }\OtherTok{model}\NormalTok{.}\FunctionTok{get}\NormalTok{(}\StringTok{'title'}\NormalTok{));}
\NormalTok{\});}

\OtherTok{TodosCollection}\NormalTok{.}\FunctionTok{on}\NormalTok{(}\StringTok{"change:completed"}\NormalTok{, }\KeywordTok{function}\NormalTok{(model) \{}
  \OtherTok{console}\NormalTok{.}\FunctionTok{log}\NormalTok{(}\StringTok{"Completed "} \NormalTok{+ }\OtherTok{model}\NormalTok{.}\FunctionTok{get}\NormalTok{(}\StringTok{'title'}\NormalTok{));}
\NormalTok{\});}

\OtherTok{TodosCollection}\NormalTok{.}\FunctionTok{set}\NormalTok{([}
    \NormalTok{\{ }\DataTypeTok{id}\NormalTok{: }\DecValTok{1}\NormalTok{, }\DataTypeTok{title}\NormalTok{: }\StringTok{'go to Jamaica.'}\NormalTok{, }\DataTypeTok{completed}\NormalTok{: }\KeywordTok{true} \NormalTok{\},}
    \NormalTok{\{ }\DataTypeTok{id}\NormalTok{: }\DecValTok{2}\NormalTok{, }\DataTypeTok{title}\NormalTok{: }\StringTok{'go to China.'}\NormalTok{, }\DataTypeTok{completed}\NormalTok{: }\KeywordTok{false} \NormalTok{\},}
    \NormalTok{\{ }\DataTypeTok{id}\NormalTok{: }\DecValTok{4}\NormalTok{, }\DataTypeTok{title}\NormalTok{: }\StringTok{'go to Disney World.'}\NormalTok{, }\DataTypeTok{completed}\NormalTok{: }\KeywordTok{false} \NormalTok{\}}
\NormalTok{]);}

\CommentTok{// Above logs:}
\CommentTok{// Completed go to Jamaica.}
\CommentTok{// Removed go to Disneyland.}
\CommentTok{// Added go to Disney World.}
\end{Highlighting}
\end{Shaded}

If you need to simply replace the entire content of the collection then
\texttt{Collection.reset()} can be used:

\begin{Shaded}
\begin{Highlighting}[]
\KeywordTok{var} \NormalTok{TodosCollection = }\KeywordTok{new} \OtherTok{Backbone}\NormalTok{.}\FunctionTok{Collection}\NormalTok{();}

\CommentTok{// we can listen for reset events}
\OtherTok{TodosCollection}\NormalTok{.}\FunctionTok{on}\NormalTok{(}\StringTok{"reset"}\NormalTok{, }\KeywordTok{function}\NormalTok{() \{}
  \OtherTok{console}\NormalTok{.}\FunctionTok{log}\NormalTok{(}\StringTok{"Collection reset."}\NormalTok{);}
\NormalTok{\});}

\OtherTok{TodosCollection}\NormalTok{.}\FunctionTok{add}\NormalTok{([}
  \NormalTok{\{ }\DataTypeTok{title}\NormalTok{: }\StringTok{'go to Jamaica.'}\NormalTok{, }\DataTypeTok{completed}\NormalTok{: }\KeywordTok{false} \NormalTok{\},}
  \NormalTok{\{ }\DataTypeTok{title}\NormalTok{: }\StringTok{'go to China.'}\NormalTok{, }\DataTypeTok{completed}\NormalTok{: }\KeywordTok{false} \NormalTok{\},}
  \NormalTok{\{ }\DataTypeTok{title}\NormalTok{: }\StringTok{'go to Disneyland.'}\NormalTok{, }\DataTypeTok{completed}\NormalTok{: }\KeywordTok{true} \NormalTok{\}}
\NormalTok{]);}

\OtherTok{console}\NormalTok{.}\FunctionTok{log}\NormalTok{(}\StringTok{'Collection size: '} \NormalTok{+ }\OtherTok{TodosCollection}\NormalTok{.}\FunctionTok{length}\NormalTok{); }\CommentTok{// Collection size: 3}

\OtherTok{TodosCollection}\NormalTok{.}\FunctionTok{reset}\NormalTok{([}
  \NormalTok{\{ }\DataTypeTok{title}\NormalTok{: }\StringTok{'go to Cuba.'}\NormalTok{, }\DataTypeTok{completed}\NormalTok{: }\KeywordTok{false} \NormalTok{\}}
\NormalTok{]);}
\CommentTok{// Above logs 'Collection reset.'}

\OtherTok{console}\NormalTok{.}\FunctionTok{log}\NormalTok{(}\StringTok{'Collection size: '} \NormalTok{+ }\OtherTok{TodosCollection}\NormalTok{.}\FunctionTok{length}\NormalTok{); }\CommentTok{// Collection size: 1}
\end{Highlighting}
\end{Shaded}

Another useful tip is to use \texttt{reset} with no arguments to clear
out a collection completely. This is handy when dynamically loading a
new page of results where you want to blank out the current page of
results.

\begin{Shaded}
\begin{Highlighting}[]
\OtherTok{myCollection}\NormalTok{.}\FunctionTok{reset}\NormalTok{();}
\end{Highlighting}
\end{Shaded}

Note that using \texttt{Collection.reset()} doesn't fire any
\texttt{add} or \texttt{remove} events. A \texttt{reset} event is fired
instead as shown in the previous example. The reason you might want to
use this is to perform super-optimized rendering in extreme cases where
individual events are too expensive.

Also note that listening to a
\href{http://backbonejs.org/\#Collection-reset}{reset} event, the list
of previous models is available in \texttt{options.previousModels}, for
convenience.

\begin{Shaded}
\begin{Highlighting}[]
\KeywordTok{var} \NormalTok{todo = }\KeywordTok{new} \OtherTok{Backbone}\NormalTok{.}\FunctionTok{Model}\NormalTok{();}
\KeywordTok{var} \NormalTok{todos = }\KeywordTok{new} \OtherTok{Backbone}\NormalTok{.}\FunctionTok{Collection}\NormalTok{([todo])}
\NormalTok{.}\FunctionTok{on}\NormalTok{(}\StringTok{'reset'}\NormalTok{, }\KeywordTok{function}\NormalTok{(todos, options) \{}
  \OtherTok{console}\NormalTok{.}\FunctionTok{log}\NormalTok{(}\OtherTok{options}\NormalTok{.}\FunctionTok{previousModels}\NormalTok{);}
  \OtherTok{console}\NormalTok{.}\FunctionTok{log}\NormalTok{([todo]);}
  \OtherTok{console}\NormalTok{.}\FunctionTok{log}\NormalTok{(}\OtherTok{options}\NormalTok{.}\FunctionTok{previousModels}\NormalTok{[}\DecValTok{0}\NormalTok{] === todo); }\CommentTok{// true}
\NormalTok{\});}
\OtherTok{todos}\NormalTok{.}\FunctionTok{reset}\NormalTok{([]);}
\end{Highlighting}
\end{Shaded}

The \texttt{set()} method available for Collections can also be used for
``smart'' updating of sets of models. This method attempts to perform
smart updating of a collection using a specified list of models. When a
model in this list isn't present in the collection, it is added. If it's
present, its attributes will be merged. Models which are present in the
collection but not in the list are removed.

\begin{Shaded}
\begin{Highlighting}[]

\CommentTok{// Define a model of type 'Beatle' with a 'job' attribute}
\KeywordTok{var} \NormalTok{Beatle = }\OtherTok{Backbone}\NormalTok{.}\OtherTok{Model}\NormalTok{.}\FunctionTok{extend}\NormalTok{(\{}
  \DataTypeTok{defaults}\NormalTok{: \{}
    \DataTypeTok{job}\NormalTok{: }\StringTok{'musician'}
  \NormalTok{\}}
\NormalTok{\});}

\CommentTok{// Create models for each member of the Beatles}
\KeywordTok{var} \NormalTok{john = }\KeywordTok{new} \FunctionTok{Beatle}\NormalTok{(\{ }\DataTypeTok{firstName}\NormalTok{: }\StringTok{'John'}\NormalTok{, }\DataTypeTok{lastName}\NormalTok{: }\StringTok{'Lennon'}\NormalTok{\});}
\KeywordTok{var} \NormalTok{paul = }\KeywordTok{new} \FunctionTok{Beatle}\NormalTok{(\{ }\DataTypeTok{firstName}\NormalTok{: }\StringTok{'Paul'}\NormalTok{, }\DataTypeTok{lastName}\NormalTok{: }\StringTok{'McCartney'}\NormalTok{\});}
\KeywordTok{var} \NormalTok{george = }\KeywordTok{new} \FunctionTok{Beatle}\NormalTok{(\{ }\DataTypeTok{firstName}\NormalTok{: }\StringTok{'George'}\NormalTok{, }\DataTypeTok{lastName}\NormalTok{: }\StringTok{'Harrison'}\NormalTok{\});}
\KeywordTok{var} \NormalTok{ringo = }\KeywordTok{new} \FunctionTok{Beatle}\NormalTok{(\{ }\DataTypeTok{firstName}\NormalTok{: }\StringTok{'Ringo'}\NormalTok{, }\DataTypeTok{lastName}\NormalTok{: }\StringTok{'Starr'}\NormalTok{\});}

\CommentTok{// Create a collection using our models}
\KeywordTok{var} \NormalTok{theBeatles = }\KeywordTok{new} \OtherTok{Backbone}\NormalTok{.}\FunctionTok{Collection}\NormalTok{([john, paul, george, ringo]);}

\CommentTok{// Create a separate model for Pete Best}
\KeywordTok{var} \NormalTok{pete = }\KeywordTok{new} \FunctionTok{Beatle}\NormalTok{(\{ }\DataTypeTok{firstName}\NormalTok{: }\StringTok{'Pete'}\NormalTok{, }\DataTypeTok{lastName}\NormalTok{: }\StringTok{'Best'}\NormalTok{\});}

\CommentTok{// Update the collection}
\OtherTok{theBeatles}\NormalTok{.}\FunctionTok{set}\NormalTok{([john, paul, george, pete]);}

\CommentTok{// Fires a `remove` event for 'Ringo', and an `add` event for 'Pete'.}
\CommentTok{// Updates any of John, Paul and Georges's attributes that may have}
\CommentTok{// changed over the years.}
\end{Highlighting}
\end{Shaded}

\paragraph{Underscore utility
functions}\label{underscore-utility-functions}

Backbone takes full advantage of its hard dependency on Underscore by
making many of its utilities directly available on collections:

\textbf{\texttt{forEach}: iterate over collections}

\begin{Shaded}
\begin{Highlighting}[]
\KeywordTok{var} \NormalTok{todos = }\KeywordTok{new} \OtherTok{Backbone}\NormalTok{.}\FunctionTok{Collection}\NormalTok{();}

\OtherTok{todos}\NormalTok{.}\FunctionTok{add}\NormalTok{([}
  \NormalTok{\{ }\DataTypeTok{title}\NormalTok{: }\StringTok{'go to Belgium.'}\NormalTok{, }\DataTypeTok{completed}\NormalTok{: }\KeywordTok{false} \NormalTok{\},}
  \NormalTok{\{ }\DataTypeTok{title}\NormalTok{: }\StringTok{'go to China.'}\NormalTok{, }\DataTypeTok{completed}\NormalTok{: }\KeywordTok{false} \NormalTok{\},}
  \NormalTok{\{ }\DataTypeTok{title}\NormalTok{: }\StringTok{'go to Austria.'}\NormalTok{, }\DataTypeTok{completed}\NormalTok{: }\KeywordTok{true} \NormalTok{\}}
\NormalTok{]);}

\CommentTok{// iterate over models in the collection}
\OtherTok{todos}\NormalTok{.}\FunctionTok{forEach}\NormalTok{(}\KeywordTok{function}\NormalTok{(model)\{}
  \OtherTok{console}\NormalTok{.}\FunctionTok{log}\NormalTok{(}\OtherTok{model}\NormalTok{.}\FunctionTok{get}\NormalTok{(}\StringTok{'title'}\NormalTok{));}
\NormalTok{\});}
\CommentTok{// Above logs:}
\CommentTok{// go to Belgium.}
\CommentTok{// go to China.}
\CommentTok{// go to Austria.}
\end{Highlighting}
\end{Shaded}

\textbf{\texttt{sortBy()}: sort a collection on a specific attribute}

\begin{Shaded}
\begin{Highlighting}[]
\CommentTok{// sort collection}
\KeywordTok{var} \NormalTok{sortedByAlphabet = }\OtherTok{todos}\NormalTok{.}\FunctionTok{sortBy}\NormalTok{(}\KeywordTok{function} \NormalTok{(todo) \{}
    \KeywordTok{return} \OtherTok{todo}\NormalTok{.}\FunctionTok{get}\NormalTok{(}\StringTok{"title"}\NormalTok{).}\FunctionTok{toLowerCase}\NormalTok{();}
\NormalTok{\});}

\OtherTok{console}\NormalTok{.}\FunctionTok{log}\NormalTok{(}\StringTok{"- Now sorted: "}\NormalTok{);}

\OtherTok{sortedByAlphabet}\NormalTok{.}\FunctionTok{forEach}\NormalTok{(}\KeywordTok{function}\NormalTok{(model)\{}
  \OtherTok{console}\NormalTok{.}\FunctionTok{log}\NormalTok{(}\OtherTok{model}\NormalTok{.}\FunctionTok{get}\NormalTok{(}\StringTok{'title'}\NormalTok{));}
\NormalTok{\});}
\CommentTok{// Above logs:}
\CommentTok{// - Now sorted:}
\CommentTok{// go to Austria.}
\CommentTok{// go to Belgium.}
\CommentTok{// go to China.}
\end{Highlighting}
\end{Shaded}

\textbf{\texttt{map()}: iterate through a collection, mapping each value
through a transformation function}

\begin{Shaded}
\begin{Highlighting}[]
\KeywordTok{var} \NormalTok{count = }\DecValTok{1}\NormalTok{;}
\OtherTok{console}\NormalTok{.}\FunctionTok{log}\NormalTok{(}\OtherTok{todos}\NormalTok{.}\FunctionTok{map}\NormalTok{(}\KeywordTok{function}\NormalTok{(model)\{}
  \KeywordTok{return} \NormalTok{count++ + }\StringTok{". "} \NormalTok{+ }\OtherTok{model}\NormalTok{.}\FunctionTok{get}\NormalTok{(}\StringTok{'title'}\NormalTok{);}
\NormalTok{\}));}
\CommentTok{// Above logs:}
\CommentTok{//1. go to Belgium.}
\CommentTok{//2. go to China.}
\CommentTok{//3. go to Austria.}
\end{Highlighting}
\end{Shaded}

\textbf{\texttt{min()}/\texttt{max()}: retrieve item with the min or max
value of an attribute}

\begin{Shaded}
\begin{Highlighting}[]
\OtherTok{todos}\NormalTok{.}\FunctionTok{max}\NormalTok{(}\KeywordTok{function}\NormalTok{(model)\{}
  \KeywordTok{return} \OtherTok{model}\NormalTok{.}\FunctionTok{id}\NormalTok{;}
\NormalTok{\}).}\FunctionTok{id}\NormalTok{;}

\OtherTok{todos}\NormalTok{.}\FunctionTok{min}\NormalTok{(}\KeywordTok{function}\NormalTok{(model)\{}
  \KeywordTok{return} \OtherTok{model}\NormalTok{.}\FunctionTok{id}\NormalTok{;}
\NormalTok{\}).}\FunctionTok{id}\NormalTok{;}
\end{Highlighting}
\end{Shaded}

\textbf{\texttt{pluck()}: extract a specific attribute}

\begin{Shaded}
\begin{Highlighting}[]
\KeywordTok{var} \NormalTok{captions = }\OtherTok{todos}\NormalTok{.}\FunctionTok{pluck}\NormalTok{(}\StringTok{'caption'}\NormalTok{);}
\CommentTok{// returns list of captions}
\end{Highlighting}
\end{Shaded}

\textbf{\texttt{filter()}: filter a collection}

\emph{Filter by an array of model IDs}

\begin{Shaded}
\begin{Highlighting}[]
\KeywordTok{var} \NormalTok{Todos = }\OtherTok{Backbone}\NormalTok{.}\OtherTok{Collection}\NormalTok{.}\FunctionTok{extend}\NormalTok{(\{}
  \DataTypeTok{model}\NormalTok{: Todo,}
  \DataTypeTok{filterById}\NormalTok{: }\KeywordTok{function}\NormalTok{(ids)\{}
    \KeywordTok{return} \KeywordTok{this}\NormalTok{.}\OtherTok{models}\NormalTok{.}\FunctionTok{filter}\NormalTok{(}
      \KeywordTok{function}\NormalTok{(c) \{ }
        \KeywordTok{return} \OtherTok{_}\NormalTok{.}\FunctionTok{contains}\NormalTok{(ids, }\OtherTok{c}\NormalTok{.}\FunctionTok{id}\NormalTok{); }
      \NormalTok{\})}
  \NormalTok{\}}
\NormalTok{\});}
\end{Highlighting}
\end{Shaded}

\textbf{\texttt{indexOf()}: return the index of a particular item within
a collection}

\begin{Shaded}
\begin{Highlighting}[]
\KeywordTok{var} \NormalTok{people = }\KeywordTok{new} \OtherTok{Backbone}\NormalTok{.}\FunctionTok{Collection}\NormalTok{;}

\OtherTok{people}\NormalTok{.}\FunctionTok{comparator} \NormalTok{= }\KeywordTok{function}\NormalTok{(a, b) \{}
  \KeywordTok{return} \OtherTok{a}\NormalTok{.}\FunctionTok{get}\NormalTok{(}\StringTok{'name'}\NormalTok{) < }\OtherTok{b}\NormalTok{.}\FunctionTok{get}\NormalTok{(}\StringTok{'name'}\NormalTok{) ? -}\DecValTok{1} \NormalTok{: }\DecValTok{1}\NormalTok{;}
\NormalTok{\};}

\KeywordTok{var} \NormalTok{tom = }\KeywordTok{new} \OtherTok{Backbone}\NormalTok{.}\FunctionTok{Model}\NormalTok{(\{}\DataTypeTok{name}\NormalTok{: }\StringTok{'Tom'}\NormalTok{\});}
\KeywordTok{var} \NormalTok{rob = }\KeywordTok{new} \OtherTok{Backbone}\NormalTok{.}\FunctionTok{Model}\NormalTok{(\{}\DataTypeTok{name}\NormalTok{: }\StringTok{'Rob'}\NormalTok{\});}
\KeywordTok{var} \NormalTok{tim = }\KeywordTok{new} \OtherTok{Backbone}\NormalTok{.}\FunctionTok{Model}\NormalTok{(\{}\DataTypeTok{name}\NormalTok{: }\StringTok{'Tim'}\NormalTok{\});}

\OtherTok{people}\NormalTok{.}\FunctionTok{add}\NormalTok{(tom);}
\OtherTok{people}\NormalTok{.}\FunctionTok{add}\NormalTok{(rob);}
\OtherTok{people}\NormalTok{.}\FunctionTok{add}\NormalTok{(tim);}

\OtherTok{console}\NormalTok{.}\FunctionTok{log}\NormalTok{(}\OtherTok{people}\NormalTok{.}\FunctionTok{indexOf}\NormalTok{(rob) === }\DecValTok{0}\NormalTok{); }\CommentTok{// true}
\OtherTok{console}\NormalTok{.}\FunctionTok{log}\NormalTok{(}\OtherTok{people}\NormalTok{.}\FunctionTok{indexOf}\NormalTok{(tim) === }\DecValTok{1}\NormalTok{); }\CommentTok{// true}
\OtherTok{console}\NormalTok{.}\FunctionTok{log}\NormalTok{(}\OtherTok{people}\NormalTok{.}\FunctionTok{indexOf}\NormalTok{(tom) === }\DecValTok{2}\NormalTok{); }\CommentTok{// true}
\end{Highlighting}
\end{Shaded}

\textbf{\texttt{any()}: confirm if any of the values in a collection
pass an iterator truth test}

\begin{Shaded}
\begin{Highlighting}[]
\OtherTok{todos}\NormalTok{.}\FunctionTok{any}\NormalTok{(}\KeywordTok{function}\NormalTok{(model)\{}
  \KeywordTok{return} \OtherTok{model}\NormalTok{.}\FunctionTok{id} \NormalTok{=== }\DecValTok{100}\NormalTok{;}
\NormalTok{\});}

\CommentTok{// or}
\OtherTok{todos}\NormalTok{.}\FunctionTok{some}\NormalTok{(}\KeywordTok{function}\NormalTok{(model)\{}
  \KeywordTok{return} \OtherTok{model}\NormalTok{.}\FunctionTok{id} \NormalTok{=== }\DecValTok{100}\NormalTok{;}
\NormalTok{\});}
\end{Highlighting}
\end{Shaded}

\textbf{\texttt{size()}: return the size of a collection}

\begin{Shaded}
\begin{Highlighting}[]
\OtherTok{todos}\NormalTok{.}\FunctionTok{size}\NormalTok{();}

\CommentTok{// equivalent to}
\OtherTok{todos}\NormalTok{.}\FunctionTok{length}\NormalTok{;}
\end{Highlighting}
\end{Shaded}

\textbf{\texttt{isEmpty()}: determine whether a collection is empty}

\begin{Shaded}
\begin{Highlighting}[]
\KeywordTok{var} \NormalTok{isEmpty = }\OtherTok{todos}\NormalTok{.}\FunctionTok{isEmpty}\NormalTok{();}
\end{Highlighting}
\end{Shaded}

\textbf{\texttt{groupBy()}: group a collection into groups of like
items}

\begin{Shaded}
\begin{Highlighting}[]
\KeywordTok{var} \NormalTok{todos = }\KeywordTok{new} \OtherTok{Backbone}\NormalTok{.}\FunctionTok{Collection}\NormalTok{();}

\OtherTok{todos}\NormalTok{.}\FunctionTok{add}\NormalTok{([}
  \NormalTok{\{ }\DataTypeTok{title}\NormalTok{: }\StringTok{'go to Belgium.'}\NormalTok{, }\DataTypeTok{completed}\NormalTok{: }\KeywordTok{false} \NormalTok{\},}
  \NormalTok{\{ }\DataTypeTok{title}\NormalTok{: }\StringTok{'go to China.'}\NormalTok{, }\DataTypeTok{completed}\NormalTok{: }\KeywordTok{false} \NormalTok{\},}
  \NormalTok{\{ }\DataTypeTok{title}\NormalTok{: }\StringTok{'go to Austria.'}\NormalTok{, }\DataTypeTok{completed}\NormalTok{: }\KeywordTok{true} \NormalTok{\}}
\NormalTok{]);}

\CommentTok{// create groups of completed and incomplete models}
\KeywordTok{var} \NormalTok{byCompleted = }\OtherTok{todos}\NormalTok{.}\FunctionTok{groupBy}\NormalTok{(}\StringTok{'completed'}\NormalTok{);}
\KeywordTok{var} \NormalTok{completed = }\KeywordTok{new} \OtherTok{Backbone}\NormalTok{.}\FunctionTok{Collection}\NormalTok{(byCompleted[}\KeywordTok{true}\NormalTok{]);}
\OtherTok{console}\NormalTok{.}\FunctionTok{log}\NormalTok{(}\OtherTok{completed}\NormalTok{.}\FunctionTok{pluck}\NormalTok{(}\StringTok{'title'}\NormalTok{));}
\CommentTok{// logs: ["go to Austria."]}
\end{Highlighting}
\end{Shaded}

In addition, several of the Underscore operations on objects are
available as methods on Models.

\textbf{\texttt{pick()}: extract a set of attributes from a model}

\begin{Shaded}
\begin{Highlighting}[]
\KeywordTok{var} \NormalTok{Todo = }\OtherTok{Backbone}\NormalTok{.}\OtherTok{Model}\NormalTok{.}\FunctionTok{extend}\NormalTok{(\{}
  \DataTypeTok{defaults}\NormalTok{: \{}
    \DataTypeTok{title}\NormalTok{: }\StringTok{''}\NormalTok{,}
    \DataTypeTok{completed}\NormalTok{: }\KeywordTok{false}
  \NormalTok{\}}
\NormalTok{\});}

\KeywordTok{var} \NormalTok{todo = }\KeywordTok{new} \FunctionTok{Todo}\NormalTok{(\{}\DataTypeTok{title}\NormalTok{: }\StringTok{'go to Austria.'}\NormalTok{\});}
\OtherTok{console}\NormalTok{.}\FunctionTok{log}\NormalTok{(}\OtherTok{todo}\NormalTok{.}\FunctionTok{pick}\NormalTok{(}\StringTok{'title'}\NormalTok{));}
\CommentTok{// logs \{title: "go to Austria"\}}
\end{Highlighting}
\end{Shaded}

\textbf{\texttt{omit()}: extract all attributes from a model except
those listed}

\begin{Shaded}
\begin{Highlighting}[]
\KeywordTok{var} \NormalTok{todo = }\KeywordTok{new} \FunctionTok{Todo}\NormalTok{(\{}\DataTypeTok{title}\NormalTok{: }\StringTok{'go to Austria.'}\NormalTok{\});}
\OtherTok{console}\NormalTok{.}\FunctionTok{log}\NormalTok{(}\OtherTok{todo}\NormalTok{.}\FunctionTok{omit}\NormalTok{(}\StringTok{'title'}\NormalTok{));}
\CommentTok{// logs \{completed: false\}}
\end{Highlighting}
\end{Shaded}

\textbf{\texttt{keys()} and \texttt{values()}: get lists of attribute
names and values}

\begin{Shaded}
\begin{Highlighting}[]
\KeywordTok{var} \NormalTok{todo = }\KeywordTok{new} \FunctionTok{Todo}\NormalTok{(\{}\DataTypeTok{title}\NormalTok{: }\StringTok{'go to Austria.'}\NormalTok{\});}
\OtherTok{console}\NormalTok{.}\FunctionTok{log}\NormalTok{(}\OtherTok{todo}\NormalTok{.}\FunctionTok{keys}\NormalTok{());}
\CommentTok{// logs: ["title", "completed"]}

\OtherTok{console}\NormalTok{.}\FunctionTok{log}\NormalTok{(}\OtherTok{todo}\NormalTok{.}\FunctionTok{values}\NormalTok{());}
\CommentTok{//logs: ["go to Austria.", false]}
\end{Highlighting}
\end{Shaded}

\textbf{\texttt{pairs()}: get list of attributes as {[}key, value{]}
pairs}

\begin{Shaded}
\begin{Highlighting}[]
\KeywordTok{var} \NormalTok{todo = }\KeywordTok{new} \FunctionTok{Todo}\NormalTok{(\{}\DataTypeTok{title}\NormalTok{: }\StringTok{'go to Austria.'}\NormalTok{\});}
\KeywordTok{var} \NormalTok{pairs = }\OtherTok{todo}\NormalTok{.}\FunctionTok{pairs}\NormalTok{();}

\OtherTok{console}\NormalTok{.}\FunctionTok{log}\NormalTok{(pairs[}\DecValTok{0}\NormalTok{]);}
\CommentTok{// logs: ["title", "go to Austria."]}
\OtherTok{console}\NormalTok{.}\FunctionTok{log}\NormalTok{(pairs[}\DecValTok{1}\NormalTok{]);}
\CommentTok{// logs: ["completed", false]}
\end{Highlighting}
\end{Shaded}

\textbf{\texttt{invert()}: create object in which the values are keys
and the attributes are values}

\begin{Shaded}
\begin{Highlighting}[]
\KeywordTok{var} \NormalTok{todo = }\KeywordTok{new} \FunctionTok{Todo}\NormalTok{(\{}\DataTypeTok{title}\NormalTok{: }\StringTok{'go to Austria.'}\NormalTok{\});}
\OtherTok{console}\NormalTok{.}\FunctionTok{log}\NormalTok{(}\OtherTok{todo}\NormalTok{.}\FunctionTok{invert}\NormalTok{());}

\CommentTok{// logs: \{'go to Austria.': 'title', 'false': 'completed'\}}
\end{Highlighting}
\end{Shaded}

The complete list of what Underscore can do can be found in its official
\href{http://documentcloud.github.com/underscore/}{documentation}.

\paragraph{Chainable API}\label{chainable-api}

Speaking of utility methods, another bit of sugar in Backbone is its
support for Underscore's \texttt{chain()} method. Chaining is a common
idiom in object-oriented languages; a chain is a sequence of method
calls on the same object that are performed in a single statement. While
Backbone makes Underscore's array manipulation operations available as
methods of Collection objects, they cannot be directly chained since
they return arrays rather than the original Collection.

Fortunately, the inclusion of Underscore's \texttt{chain()} method
enables you to chain calls to these methods on Collections.

The \texttt{chain()} method returns an object that has all of the
Underscore array operations attached as methods which return that
object. The chain ends with a call to the \texttt{value()} method which
simply returns the resulting array value. In case you haven't seen it
before, the chainable API looks like this:

\begin{Shaded}
\begin{Highlighting}[]
\KeywordTok{var} \NormalTok{collection = }\KeywordTok{new} \OtherTok{Backbone}\NormalTok{.}\FunctionTok{Collection}\NormalTok{([}
  \NormalTok{\{ }\DataTypeTok{name}\NormalTok{: }\StringTok{'Tim'}\NormalTok{, }\DataTypeTok{age}\NormalTok{: }\DecValTok{5} \NormalTok{\},}
  \NormalTok{\{ }\DataTypeTok{name}\NormalTok{: }\StringTok{'Ida'}\NormalTok{, }\DataTypeTok{age}\NormalTok{: }\DecValTok{26} \NormalTok{\},}
  \NormalTok{\{ }\DataTypeTok{name}\NormalTok{: }\StringTok{'Rob'}\NormalTok{, }\DataTypeTok{age}\NormalTok{: }\DecValTok{55} \NormalTok{\}}
\NormalTok{]);}

\KeywordTok{var} \NormalTok{filteredNames = }\OtherTok{collection}\NormalTok{.}\FunctionTok{chain}\NormalTok{() }\CommentTok{// start chain, returns wrapper around collection's models}
  \NormalTok{.}\FunctionTok{filter}\NormalTok{(}\KeywordTok{function}\NormalTok{(item) \{ }\KeywordTok{return} \OtherTok{item}\NormalTok{.}\FunctionTok{get}\NormalTok{(}\StringTok{'age'}\NormalTok{) > }\DecValTok{10}\NormalTok{; \}) }\CommentTok{// returns wrapped array excluding Tim}
  \NormalTok{.}\FunctionTok{map}\NormalTok{(}\KeywordTok{function}\NormalTok{(item) \{ }\KeywordTok{return} \OtherTok{item}\NormalTok{.}\FunctionTok{get}\NormalTok{(}\StringTok{'name'}\NormalTok{); \}) }\CommentTok{// returns wrapped array containing remaining names}
  \NormalTok{.}\FunctionTok{value}\NormalTok{(); }\CommentTok{// terminates the chain and returns the resulting array}

\OtherTok{console}\NormalTok{.}\FunctionTok{log}\NormalTok{(filteredNames); }\CommentTok{// logs: ['Ida', 'Rob']}
\end{Highlighting}
\end{Shaded}

Some of the Backbone-specific methods do return \texttt{this}, which
means they can be chained as well:

\begin{Shaded}
\begin{Highlighting}[]
\KeywordTok{var} \NormalTok{collection = }\KeywordTok{new} \OtherTok{Backbone}\NormalTok{.}\FunctionTok{Collection}\NormalTok{();}

\NormalTok{collection}
    \NormalTok{.}\FunctionTok{add}\NormalTok{(\{ }\DataTypeTok{name}\NormalTok{: }\StringTok{'John'}\NormalTok{, }\DataTypeTok{age}\NormalTok{: }\DecValTok{23} \NormalTok{\})}
    \NormalTok{.}\FunctionTok{add}\NormalTok{(\{ }\DataTypeTok{name}\NormalTok{: }\StringTok{'Harry'}\NormalTok{, }\DataTypeTok{age}\NormalTok{: }\DecValTok{33} \NormalTok{\})}
    \NormalTok{.}\FunctionTok{add}\NormalTok{(\{ }\DataTypeTok{name}\NormalTok{: }\StringTok{'Steve'}\NormalTok{, }\DataTypeTok{age}\NormalTok{: }\DecValTok{41} \NormalTok{\});}

\KeywordTok{var} \NormalTok{names = }\OtherTok{collection}\NormalTok{.}\FunctionTok{pluck}\NormalTok{(}\StringTok{'name'}\NormalTok{);}

\OtherTok{console}\NormalTok{.}\FunctionTok{log}\NormalTok{(names); }\CommentTok{// logs: ['John', 'Harry', 'Steve']}
\end{Highlighting}
\end{Shaded}

\subsection{RESTful Persistence}\label{restful-persistence}

Thus far, all of our example data has been created in the browser. For
most single page applications, the models are derived from a data store
residing on a server. This is an area in which Backbone dramatically
simplifies the code you need to write to perform RESTful synchronization
with a server through a simple API on its models and collections.

\textbf{Fetching models from the server}

\texttt{Collections.fetch()} retrieves a set of models from the server
in the form of a JSON array by sending an HTTP GET request to the URL
specified by the collection's \texttt{url} property (which may be a
function). When this data is received, a \texttt{set()} will be executed
to update the collection.

\begin{Shaded}
\begin{Highlighting}[]
\KeywordTok{var} \NormalTok{Todo = }\OtherTok{Backbone}\NormalTok{.}\OtherTok{Model}\NormalTok{.}\FunctionTok{extend}\NormalTok{(\{}
  \DataTypeTok{defaults}\NormalTok{: \{}
    \DataTypeTok{title}\NormalTok{: }\StringTok{''}\NormalTok{,}
    \DataTypeTok{completed}\NormalTok{: }\KeywordTok{false}
  \NormalTok{\}}
\NormalTok{\});}

\KeywordTok{var} \NormalTok{TodosCollection = }\OtherTok{Backbone}\NormalTok{.}\OtherTok{Collection}\NormalTok{.}\FunctionTok{extend}\NormalTok{(\{}
  \DataTypeTok{model}\NormalTok{: Todo,}
  \DataTypeTok{url}\NormalTok{: }\StringTok{'/todos'}
\NormalTok{\});}

\KeywordTok{var} \NormalTok{todos = }\KeywordTok{new} \FunctionTok{TodosCollection}\NormalTok{();}
\OtherTok{todos}\NormalTok{.}\FunctionTok{fetch}\NormalTok{(); }\CommentTok{// sends HTTP GET to /todos}
\end{Highlighting}
\end{Shaded}

\textbf{Saving models to the server}

While Backbone can retrieve an entire collection of models from the
server at once, updates to models are performed individually using the
model's \texttt{save()} method. When \texttt{save()} is called on a
model that was fetched from the server, it constructs a URL by appending
the model's id to the collection's URL and sends an HTTP PUT to the
server. If the model is a new instance that was created in the browser
(i.e., it doesn't have an id) then an HTTP POST is sent to the
collection's URL. \texttt{Collections.create()} can be used to create a
new model, add it to the collection, and send it to the server in a
single method call.

\begin{Shaded}
\begin{Highlighting}[]
\KeywordTok{var} \NormalTok{Todo = }\OtherTok{Backbone}\NormalTok{.}\OtherTok{Model}\NormalTok{.}\FunctionTok{extend}\NormalTok{(\{}
  \DataTypeTok{defaults}\NormalTok{: \{}
    \DataTypeTok{title}\NormalTok{: }\StringTok{''}\NormalTok{,}
    \DataTypeTok{completed}\NormalTok{: }\KeywordTok{false}
  \NormalTok{\}}
\NormalTok{\});}

\KeywordTok{var} \NormalTok{TodosCollection = }\OtherTok{Backbone}\NormalTok{.}\OtherTok{Collection}\NormalTok{.}\FunctionTok{extend}\NormalTok{(\{}
  \DataTypeTok{model}\NormalTok{: Todo,}
  \DataTypeTok{url}\NormalTok{: }\StringTok{'/todos'}
\NormalTok{\});}

\KeywordTok{var} \NormalTok{todos = }\KeywordTok{new} \FunctionTok{TodosCollection}\NormalTok{();}
\OtherTok{todos}\NormalTok{.}\FunctionTok{fetch}\NormalTok{();}

\KeywordTok{var} \NormalTok{todo2 = }\OtherTok{todos}\NormalTok{.}\FunctionTok{get}\NormalTok{(}\DecValTok{2}\NormalTok{);}
\OtherTok{todo2}\NormalTok{.}\FunctionTok{set}\NormalTok{(}\StringTok{'title'}\NormalTok{, }\StringTok{'go fishing'}\NormalTok{);}
\OtherTok{todo2}\NormalTok{.}\FunctionTok{save}\NormalTok{(); }\CommentTok{// sends HTTP PUT to /todos/2}

\OtherTok{todos}\NormalTok{.}\FunctionTok{create}\NormalTok{(\{}\DataTypeTok{title}\NormalTok{: }\StringTok{'Try out code samples'}\NormalTok{\}); }\CommentTok{// sends HTTP POST to /todos and adds to collection}
\end{Highlighting}
\end{Shaded}

As mentioned earlier, a model's \texttt{validate()} method is called
automatically by \texttt{save()} and will trigger an \texttt{invalid}
event on the model if validation fails.

\textbf{Deleting models from the server}

A model can be removed from the containing collection and the server by
calling its \texttt{destroy()} method. Unlike
\texttt{Collection.remove()} which only removes a model from a
collection, \texttt{Model.destroy()} will also send an HTTP DELETE to
the collection's URL.

\begin{Shaded}
\begin{Highlighting}[]
\KeywordTok{var} \NormalTok{Todo = }\OtherTok{Backbone}\NormalTok{.}\OtherTok{Model}\NormalTok{.}\FunctionTok{extend}\NormalTok{(\{}
  \DataTypeTok{defaults}\NormalTok{: \{}
    \DataTypeTok{title}\NormalTok{: }\StringTok{''}\NormalTok{,}
    \DataTypeTok{completed}\NormalTok{: }\KeywordTok{false}
  \NormalTok{\}}
\NormalTok{\});}

\KeywordTok{var} \NormalTok{TodosCollection = }\OtherTok{Backbone}\NormalTok{.}\OtherTok{Collection}\NormalTok{.}\FunctionTok{extend}\NormalTok{(\{}
  \DataTypeTok{model}\NormalTok{: Todo,}
  \DataTypeTok{url}\NormalTok{: }\StringTok{'/todos'}
\NormalTok{\});}

\KeywordTok{var} \NormalTok{todos = }\KeywordTok{new} \FunctionTok{TodosCollection}\NormalTok{();}
\OtherTok{todos}\NormalTok{.}\FunctionTok{fetch}\NormalTok{();}

\KeywordTok{var} \NormalTok{todo2 = }\OtherTok{todos}\NormalTok{.}\FunctionTok{get}\NormalTok{(}\DecValTok{2}\NormalTok{);}
\OtherTok{todo2}\NormalTok{.}\FunctionTok{destroy}\NormalTok{(); }\CommentTok{// sends HTTP DELETE to /todos/2 and removes from collection}
\end{Highlighting}
\end{Shaded}

Calling \texttt{destroy} on a Model will return \texttt{false} if the
model \texttt{isNew}:

\begin{Shaded}
\begin{Highlighting}[]
\KeywordTok{var} \NormalTok{todo = }\KeywordTok{new} \OtherTok{Backbone}\NormalTok{.}\FunctionTok{Model}\NormalTok{();}
\OtherTok{console}\NormalTok{.}\FunctionTok{log}\NormalTok{(}\OtherTok{todo}\NormalTok{.}\FunctionTok{destroy}\NormalTok{());}
\CommentTok{// false}
\end{Highlighting}
\end{Shaded}

\textbf{Options}

Each RESTful API method accepts a variety of options. Most importantly,
all methods accept success and error callbacks which can be used to
customize the handling of server responses.

Specifying the \texttt{\{patch: true\}} option to \texttt{Model.save()}
will cause it to use HTTP PATCH to send only the changed attributes (i.e
partial updates) to the server instead of the entire model i.e
\texttt{model.save(attrs, \{patch: true\})}:

\begin{Shaded}
\begin{Highlighting}[]
\CommentTok{// Save partial using PATCH}
\OtherTok{model}\NormalTok{.}\FunctionTok{clear}\NormalTok{().}\FunctionTok{set}\NormalTok{(\{}\DataTypeTok{id}\NormalTok{: }\DecValTok{1}\NormalTok{, }\DataTypeTok{a}\NormalTok{: }\DecValTok{1}\NormalTok{, }\DataTypeTok{b}\NormalTok{: }\DecValTok{2}\NormalTok{, }\DataTypeTok{c}\NormalTok{: }\DecValTok{3}\NormalTok{, }\DataTypeTok{d}\NormalTok{: }\DecValTok{4}\NormalTok{\});}
\OtherTok{model}\NormalTok{.}\FunctionTok{save}\NormalTok{();}
\OtherTok{model}\NormalTok{.}\FunctionTok{save}\NormalTok{(\{}\DataTypeTok{b}\NormalTok{: }\DecValTok{2}\NormalTok{, }\DataTypeTok{d}\NormalTok{: }\DecValTok{4}\NormalTok{\}, \{}\DataTypeTok{patch}\NormalTok{: }\KeywordTok{true}\NormalTok{\});}
\OtherTok{console}\NormalTok{.}\FunctionTok{log}\NormalTok{(}\KeywordTok{this}\NormalTok{.}\OtherTok{syncArgs}\NormalTok{.}\FunctionTok{method}\NormalTok{);}
\CommentTok{// 'patch'}
\end{Highlighting}
\end{Shaded}

Similarly, passing the \texttt{\{reset: true\}} option to
\texttt{Collection.fetch()} will result in the collection being updated
using \texttt{reset()} rather than \texttt{set()}.

See the Backbone.js documentation for full descriptions of the supported
options.

\subsection{Events}\label{events}

Events are a basic inversion of control. Instead of having one function
call another by name, the second function is registered as a handler to
be called when a specific event occurs.

The part of your application that has to know how to call the other part
of your app has been inverted. This is the core thing that makes it
possible for your business logic to not have to know about how your user
interface works and is the most powerful thing about the Backbone Events
system.

Mastering events is one of the quickest ways to become more productive
with Backbone, so let's take a closer look at Backbone's event model.

\texttt{Backbone.Events} is mixed into the other Backbone ``classes'',
including:

\begin{itemize}
\itemsep1pt\parskip0pt\parsep0pt
\item
  Backbone
\item
  Backbone.Model
\item
  Backbone.Collection
\item
  Backbone.Router
\item
  Backbone.History
\item
  Backbone.View
\end{itemize}

Note that \texttt{Backbone.Events} is mixed into the \texttt{Backbone}
object. Since \texttt{Backbone} is globally visible, it can be used as a
simple event bus:

\begin{Shaded}
\begin{Highlighting}[]
\OtherTok{Backbone}\NormalTok{.}\FunctionTok{on}\NormalTok{(}\StringTok{'event'}\NormalTok{, }\KeywordTok{function}\NormalTok{() \{}\OtherTok{console}\NormalTok{.}\FunctionTok{log}\NormalTok{(}\StringTok{'Handled Backbone event'}\NormalTok{);\});}
\OtherTok{Backbone}\NormalTok{.}\FunctionTok{trigger}\NormalTok{(}\StringTok{'event'}\NormalTok{); }\CommentTok{// logs: Handled Backbone event}
\end{Highlighting}
\end{Shaded}

\paragraph{on(), off(), and trigger()}\label{on-off-and-trigger}

\texttt{Backbone.Events} can give any object the ability to bind and
trigger custom events. We can mix this module into any object easily and
there isn't a requirement for events to be declared before being bound
to a callback handler.

Example:

\begin{Shaded}
\begin{Highlighting}[]
\KeywordTok{var} \NormalTok{ourObject = \{\};}

\CommentTok{// Mixin}
\OtherTok{_}\NormalTok{.}\FunctionTok{extend}\NormalTok{(ourObject, }\OtherTok{Backbone}\NormalTok{.}\FunctionTok{Events}\NormalTok{);}

\CommentTok{// Add a custom event}
\OtherTok{ourObject}\NormalTok{.}\FunctionTok{on}\NormalTok{(}\StringTok{'dance'}\NormalTok{, }\KeywordTok{function}\NormalTok{(msg)\{}
  \OtherTok{console}\NormalTok{.}\FunctionTok{log}\NormalTok{(}\StringTok{'We triggered '} \NormalTok{+ msg);}
\NormalTok{\});}

\CommentTok{// Trigger the custom event}
\OtherTok{ourObject}\NormalTok{.}\FunctionTok{trigger}\NormalTok{(}\StringTok{'dance'}\NormalTok{, }\StringTok{'our event'}\NormalTok{);}
\end{Highlighting}
\end{Shaded}

If you're familiar with jQuery custom events or the concept of
Publish/Subscribe, \texttt{Backbone.Events} provides a system that is
very similar with \texttt{on} being analogous to \texttt{subscribe} and
\texttt{trigger} being similar to \texttt{publish}.

\texttt{on} binds a callback function to an object, as we've done with
\texttt{dance} in the above example. The callback is invoked whenever
the event is triggered.

The official Backbone.js documentation recommends namespacing event
names using colons if you end up using quite a few of these on your
page. e.g.:

\begin{Shaded}
\begin{Highlighting}[]
\KeywordTok{var} \NormalTok{ourObject = \{\};}

\CommentTok{// Mixin}
\OtherTok{_}\NormalTok{.}\FunctionTok{extend}\NormalTok{(ourObject, }\OtherTok{Backbone}\NormalTok{.}\FunctionTok{Events}\NormalTok{);}

\KeywordTok{function} \FunctionTok{dancing} \NormalTok{(msg) \{ }\OtherTok{console}\NormalTok{.}\FunctionTok{log}\NormalTok{(}\StringTok{"We started "} \NormalTok{+ msg); \}}

\CommentTok{// Add namespaced custom events}
\OtherTok{ourObject}\NormalTok{.}\FunctionTok{on}\NormalTok{(}\StringTok{"dance:tap"}\NormalTok{, dancing);}
\OtherTok{ourObject}\NormalTok{.}\FunctionTok{on}\NormalTok{(}\StringTok{"dance:break"}\NormalTok{, dancing);}

\CommentTok{// Trigger the custom events}
\OtherTok{ourObject}\NormalTok{.}\FunctionTok{trigger}\NormalTok{(}\StringTok{"dance:tap"}\NormalTok{, }\StringTok{"tap dancing. Yeah!"}\NormalTok{);}
\OtherTok{ourObject}\NormalTok{.}\FunctionTok{trigger}\NormalTok{(}\StringTok{"dance:break"}\NormalTok{, }\StringTok{"break dancing. Yeah!"}\NormalTok{);}

\CommentTok{// This one triggers nothing as no listener listens for it}
\OtherTok{ourObject}\NormalTok{.}\FunctionTok{trigger}\NormalTok{(}\StringTok{"dance"}\NormalTok{, }\StringTok{"break dancing. Yeah!"}\NormalTok{);}
\end{Highlighting}
\end{Shaded}

A special \texttt{all} event is made available in case you would like
notifications for every event that occurs on the object (e.g., if you
would like to screen events in a single location). The \texttt{all}
event can be used as follows:

\begin{Shaded}
\begin{Highlighting}[]
\KeywordTok{var} \NormalTok{ourObject = \{\};}

\CommentTok{// Mixin}
\OtherTok{_}\NormalTok{.}\FunctionTok{extend}\NormalTok{(ourObject, }\OtherTok{Backbone}\NormalTok{.}\FunctionTok{Events}\NormalTok{);}

\KeywordTok{function} \FunctionTok{dancing} \NormalTok{(msg) \{ }\OtherTok{console}\NormalTok{.}\FunctionTok{log}\NormalTok{(}\StringTok{"We started "} \NormalTok{+ msg); \}}

\OtherTok{ourObject}\NormalTok{.}\FunctionTok{on}\NormalTok{(}\StringTok{"all"}\NormalTok{, }\KeywordTok{function}\NormalTok{(eventName)\{}
  \OtherTok{console}\NormalTok{.}\FunctionTok{log}\NormalTok{(}\StringTok{"The name of the event passed was "} \NormalTok{+ eventName);}
\NormalTok{\});}

\CommentTok{// This time each event will be caught with a catch 'all' event listener}
\OtherTok{ourObject}\NormalTok{.}\FunctionTok{trigger}\NormalTok{(}\StringTok{"dance:tap"}\NormalTok{, }\StringTok{"tap dancing. Yeah!"}\NormalTok{);}
\OtherTok{ourObject}\NormalTok{.}\FunctionTok{trigger}\NormalTok{(}\StringTok{"dance:break"}\NormalTok{, }\StringTok{"break dancing. Yeah!"}\NormalTok{);}
\OtherTok{ourObject}\NormalTok{.}\FunctionTok{trigger}\NormalTok{(}\StringTok{"dance"}\NormalTok{, }\StringTok{"break dancing. Yeah!"}\NormalTok{);}
\end{Highlighting}
\end{Shaded}

\texttt{off} removes callback functions that were previously bound to an
object. Going back to our Publish/Subscribe comparison, think of it as
an \texttt{unsubscribe} for custom events.

To remove the \texttt{dance} event we previously bound to
\texttt{ourObject}, we would simply do:

\begin{Shaded}
\begin{Highlighting}[]
\KeywordTok{var} \NormalTok{ourObject = \{\};}

\CommentTok{// Mixin}
\OtherTok{_}\NormalTok{.}\FunctionTok{extend}\NormalTok{(ourObject, }\OtherTok{Backbone}\NormalTok{.}\FunctionTok{Events}\NormalTok{);}

\KeywordTok{function} \FunctionTok{dancing} \NormalTok{(msg) \{ }\OtherTok{console}\NormalTok{.}\FunctionTok{log}\NormalTok{(}\StringTok{"We "} \NormalTok{+ msg); \}}

\CommentTok{// Add namespaced custom events}
\OtherTok{ourObject}\NormalTok{.}\FunctionTok{on}\NormalTok{(}\StringTok{"dance:tap"}\NormalTok{, dancing);}
\OtherTok{ourObject}\NormalTok{.}\FunctionTok{on}\NormalTok{(}\StringTok{"dance:break"}\NormalTok{, dancing);}

\CommentTok{// Trigger the custom events. Each will be caught and acted upon.}
\OtherTok{ourObject}\NormalTok{.}\FunctionTok{trigger}\NormalTok{(}\StringTok{"dance:tap"}\NormalTok{, }\StringTok{"started tap dancing. Yeah!"}\NormalTok{);}
\OtherTok{ourObject}\NormalTok{.}\FunctionTok{trigger}\NormalTok{(}\StringTok{"dance:break"}\NormalTok{, }\StringTok{"started break dancing. Yeah!"}\NormalTok{);}

\CommentTok{// Removes event bound to the object}
\OtherTok{ourObject}\NormalTok{.}\FunctionTok{off}\NormalTok{(}\StringTok{"dance:tap"}\NormalTok{);}

\CommentTok{// Trigger the custom events again, but one is logged.}
\OtherTok{ourObject}\NormalTok{.}\FunctionTok{trigger}\NormalTok{(}\StringTok{"dance:tap"}\NormalTok{, }\StringTok{"stopped tap dancing."}\NormalTok{); }\CommentTok{// won't be logged as it's not listened for}
\OtherTok{ourObject}\NormalTok{.}\FunctionTok{trigger}\NormalTok{(}\StringTok{"dance:break"}\NormalTok{, }\StringTok{"break dancing. Yeah!"}\NormalTok{);}
\end{Highlighting}
\end{Shaded}

To remove all callbacks for the event we pass an event name (e.g.,
\texttt{move}) to the \texttt{off()} method on the object the event is
bound to. If we wish to remove a specific callback, we can pass that
callback as the second parameter:

\begin{Shaded}
\begin{Highlighting}[]
\KeywordTok{var} \NormalTok{ourObject = \{\};}

\CommentTok{// Mixin}
\OtherTok{_}\NormalTok{.}\FunctionTok{extend}\NormalTok{(ourObject, }\OtherTok{Backbone}\NormalTok{.}\FunctionTok{Events}\NormalTok{);}

\KeywordTok{function} \FunctionTok{dancing} \NormalTok{(msg) \{ }\OtherTok{console}\NormalTok{.}\FunctionTok{log}\NormalTok{(}\StringTok{"We are dancing. "} \NormalTok{+ msg); \}}
\KeywordTok{function} \FunctionTok{jumping} \NormalTok{(msg) \{ }\OtherTok{console}\NormalTok{.}\FunctionTok{log}\NormalTok{(}\StringTok{"We are jumping. "} \NormalTok{+ msg); \}}

\CommentTok{// Add two listeners to the same event}
\OtherTok{ourObject}\NormalTok{.}\FunctionTok{on}\NormalTok{(}\StringTok{"move"}\NormalTok{, dancing);}
\OtherTok{ourObject}\NormalTok{.}\FunctionTok{on}\NormalTok{(}\StringTok{"move"}\NormalTok{, jumping);}

\CommentTok{// Trigger the events. Both listeners are called.}
\OtherTok{ourObject}\NormalTok{.}\FunctionTok{trigger}\NormalTok{(}\StringTok{"move"}\NormalTok{, }\StringTok{"Yeah!"}\NormalTok{);}

\CommentTok{// Removes specified listener}
\OtherTok{ourObject}\NormalTok{.}\FunctionTok{off}\NormalTok{(}\StringTok{"move"}\NormalTok{, dancing);}

\CommentTok{// Trigger the events again. One listener left.}
\OtherTok{ourObject}\NormalTok{.}\FunctionTok{trigger}\NormalTok{(}\StringTok{"move"}\NormalTok{, }\StringTok{"Yeah, jump, jump!"}\NormalTok{);}
\end{Highlighting}
\end{Shaded}

Finally, as we have seen in our previous examples, \texttt{trigger}
triggers a callback for a specified event (or a space-separated list of
events). e.g.:

\begin{Shaded}
\begin{Highlighting}[]
\KeywordTok{var} \NormalTok{ourObject = \{\};}

\CommentTok{// Mixin}
\OtherTok{_}\NormalTok{.}\FunctionTok{extend}\NormalTok{(ourObject, }\OtherTok{Backbone}\NormalTok{.}\FunctionTok{Events}\NormalTok{);}

\KeywordTok{function} \FunctionTok{doAction} \NormalTok{(msg) \{ }\OtherTok{console}\NormalTok{.}\FunctionTok{log}\NormalTok{(}\StringTok{"We are "} \NormalTok{+ msg); \}}

\CommentTok{// Add event listeners}
\OtherTok{ourObject}\NormalTok{.}\FunctionTok{on}\NormalTok{(}\StringTok{"dance"}\NormalTok{, doAction);}
\OtherTok{ourObject}\NormalTok{.}\FunctionTok{on}\NormalTok{(}\StringTok{"jump"}\NormalTok{, doAction);}
\OtherTok{ourObject}\NormalTok{.}\FunctionTok{on}\NormalTok{(}\StringTok{"skip"}\NormalTok{, doAction);}

\CommentTok{// Single event}
\OtherTok{ourObject}\NormalTok{.}\FunctionTok{trigger}\NormalTok{(}\StringTok{"dance"}\NormalTok{, }\StringTok{'just dancing.'}\NormalTok{);}

\CommentTok{// Multiple events}
\OtherTok{ourObject}\NormalTok{.}\FunctionTok{trigger}\NormalTok{(}\StringTok{"dance jump skip"}\NormalTok{, }\StringTok{'very tired from so much action.'}\NormalTok{);}
\end{Highlighting}
\end{Shaded}

\texttt{trigger} can pass multiple arguments to the callback function:

\begin{Shaded}
\begin{Highlighting}[]
\KeywordTok{var} \NormalTok{ourObject = \{\};}

\CommentTok{// Mixin}
\OtherTok{_}\NormalTok{.}\FunctionTok{extend}\NormalTok{(ourObject, }\OtherTok{Backbone}\NormalTok{.}\FunctionTok{Events}\NormalTok{);}

\KeywordTok{function} \FunctionTok{doAction} \NormalTok{(action, duration) \{}
  \OtherTok{console}\NormalTok{.}\FunctionTok{log}\NormalTok{(}\StringTok{"We are "} \NormalTok{+ action + }\StringTok{' for '} \NormalTok{+ duration ); }
\NormalTok{\}}

\CommentTok{// Add event listeners}
\OtherTok{ourObject}\NormalTok{.}\FunctionTok{on}\NormalTok{(}\StringTok{"dance"}\NormalTok{, doAction);}
\OtherTok{ourObject}\NormalTok{.}\FunctionTok{on}\NormalTok{(}\StringTok{"jump"}\NormalTok{, doAction);}
\OtherTok{ourObject}\NormalTok{.}\FunctionTok{on}\NormalTok{(}\StringTok{"skip"}\NormalTok{, doAction);}

\CommentTok{// Passing multiple arguments to single event}
\OtherTok{ourObject}\NormalTok{.}\FunctionTok{trigger}\NormalTok{(}\StringTok{"dance"}\NormalTok{, }\StringTok{'dancing'}\NormalTok{, }\StringTok{"5 minutes"}\NormalTok{);}

\CommentTok{// Passing multiple arguments to multiple events}
\OtherTok{ourObject}\NormalTok{.}\FunctionTok{trigger}\NormalTok{(}\StringTok{"dance jump skip"}\NormalTok{, }\StringTok{'on fire'}\NormalTok{, }\StringTok{"15 minutes"}\NormalTok{);}
\end{Highlighting}
\end{Shaded}

\paragraph{listenTo() and
stopListening()}\label{listento-and-stoplistening}

While \texttt{on()} and \texttt{off()} add callbacks directly to an
observed object, \texttt{listenTo()} tells an object to listen for
events on another object, allowing the listener to keep track of the
events for which it is listening. \texttt{stopListening()} can
subsequently be called on the listener to tell it to stop listening for
events:

\begin{Shaded}
\begin{Highlighting}[]
\KeywordTok{var} \NormalTok{a = }\OtherTok{_}\NormalTok{.}\FunctionTok{extend}\NormalTok{(\{\}, }\OtherTok{Backbone}\NormalTok{.}\FunctionTok{Events}\NormalTok{);}
\KeywordTok{var} \NormalTok{b = }\OtherTok{_}\NormalTok{.}\FunctionTok{extend}\NormalTok{(\{\}, }\OtherTok{Backbone}\NormalTok{.}\FunctionTok{Events}\NormalTok{);}
\KeywordTok{var} \NormalTok{c = }\OtherTok{_}\NormalTok{.}\FunctionTok{extend}\NormalTok{(\{\}, }\OtherTok{Backbone}\NormalTok{.}\FunctionTok{Events}\NormalTok{);}

\CommentTok{// add listeners to A for events on B and C}
\OtherTok{a}\NormalTok{.}\FunctionTok{listenTo}\NormalTok{(b, }\StringTok{'anything'}\NormalTok{, }\KeywordTok{function}\NormalTok{(event)\{ }\OtherTok{console}\NormalTok{.}\FunctionTok{log}\NormalTok{(}\StringTok{"anything happened"}\NormalTok{); \});}
\OtherTok{a}\NormalTok{.}\FunctionTok{listenTo}\NormalTok{(c, }\StringTok{'everything'}\NormalTok{, }\KeywordTok{function}\NormalTok{(event)\{ }\OtherTok{console}\NormalTok{.}\FunctionTok{log}\NormalTok{(}\StringTok{"everything happened"}\NormalTok{); \});}

\CommentTok{// trigger an event}
\OtherTok{b}\NormalTok{.}\FunctionTok{trigger}\NormalTok{(}\StringTok{'anything'}\NormalTok{); }\CommentTok{// logs: anything happened}

\CommentTok{// stop listening}
\OtherTok{a}\NormalTok{.}\FunctionTok{stopListening}\NormalTok{();}

\CommentTok{// A does not receive these events}
\OtherTok{b}\NormalTok{.}\FunctionTok{trigger}\NormalTok{(}\StringTok{'anything'}\NormalTok{);}
\OtherTok{c}\NormalTok{.}\FunctionTok{trigger}\NormalTok{(}\StringTok{'everything'}\NormalTok{);}
\end{Highlighting}
\end{Shaded}

\texttt{stopListening()} can also be used to selectively stop listening
based on the event, model, or callback handler.

If you use \texttt{on} and \texttt{off} and remove views and their
corresponding models at the same time, there are generally no problems.
But a problem arises when you remove a view that had registered to be
notified about events on a model, but you don't remove the model or call
\texttt{off} to remove the view's event handler. Since the model has a
reference to the view's callback function, the JavaScript garbage
collector cannot remove the view from memory. This is called a ``ghost
view'' and is a form of memory leak which is common since the models
generally tend to outlive the corresponding views during an
application's lifecycle. For details on the topic and a solution, check
this
\href{http://lostechies.com/derickbailey/2011/09/15/zombies-run-managing-page-transitions-in-backbone-apps/}{excellent
article} by Derick Bailey.

Practically, every \texttt{on} called on an object also requires an
\texttt{off} to be called in order for the garbage collector to do its
job. \texttt{listenTo()} changes that, allowing Views to bind to Model
notifications and unbind from all of them with just one call -
\texttt{stopListening()}.

The default implementation of \texttt{View.remove()} makes a call to
\texttt{stopListening()}, ensuring that any listeners bound using
\texttt{listenTo()} are unbound before the view is destroyed.

\begin{Shaded}
\begin{Highlighting}[]
\KeywordTok{var} \NormalTok{view = }\KeywordTok{new} \OtherTok{Backbone}\NormalTok{.}\FunctionTok{View}\NormalTok{();}
\KeywordTok{var} \NormalTok{b = }\OtherTok{_}\NormalTok{.}\FunctionTok{extend}\NormalTok{(\{\}, }\OtherTok{Backbone}\NormalTok{.}\FunctionTok{Events}\NormalTok{);}

\OtherTok{view}\NormalTok{.}\FunctionTok{listenTo}\NormalTok{(b, }\StringTok{'all'}\NormalTok{, }\KeywordTok{function}\NormalTok{()\{ }\OtherTok{console}\NormalTok{.}\FunctionTok{log}\NormalTok{(}\KeywordTok{true}\NormalTok{); \});}
\OtherTok{b}\NormalTok{.}\FunctionTok{trigger}\NormalTok{(}\StringTok{'anything'}\NormalTok{);  }\CommentTok{// logs: true}

\OtherTok{view}\NormalTok{.}\FunctionTok{listenTo}\NormalTok{(b, }\StringTok{'all'}\NormalTok{, }\KeywordTok{function}\NormalTok{()\{ }\OtherTok{console}\NormalTok{.}\FunctionTok{log}\NormalTok{(}\KeywordTok{false}\NormalTok{); \});}
\OtherTok{view}\NormalTok{.}\FunctionTok{remove}\NormalTok{(); }\CommentTok{// stopListening() implicitly called}
\OtherTok{b}\NormalTok{.}\FunctionTok{trigger}\NormalTok{(}\StringTok{'anything'}\NormalTok{);  }\CommentTok{// does not log anything}
\end{Highlighting}
\end{Shaded}

\paragraph{Events and Views}\label{events-and-views}

Within a View, there are two types of events you can listen for: DOM
events and events triggered using the Event API. It is important to
understand the differences in how views bind to these events and the
context in which their callbacks are invoked.

DOM events can be bound to using the View's \texttt{events} property or
using \texttt{jQuery.on()}. Within callbacks bound using the
\texttt{events} property, \texttt{this} refers to the View object;
whereas any callbacks bound directly using jQuery will have
\texttt{this} set to the handling DOM element by jQuery. All DOM event
callbacks are passed an \texttt{event} object by jQuery. See
\texttt{delegateEvents()} in the Backbone documentation for additional
details.

Event API events are bound as described in this section. If the event is
bound using \texttt{on()} on the observed object, a context parameter
can be passed as the third argument. If the event is bound using
\texttt{listenTo()} then within the callback \texttt{this} refers to the
listener. The arguments passed to Event API callbacks depends on the
type of event. See the Catalog of Events in the Backbone documentation
for details.

The following example illustrates these differences:

\begin{Shaded}
\begin{Highlighting}[]
\KeywordTok{<div}\OtherTok{ id=}\StringTok{"todo"}\KeywordTok{>}
    \KeywordTok{<input}\OtherTok{ type=}\StringTok{'checkbox'} \KeywordTok{/>}
\KeywordTok{</div>}
\end{Highlighting}
\end{Shaded}

\begin{Shaded}
\begin{Highlighting}[]
\KeywordTok{var} \NormalTok{View = }\OtherTok{Backbone}\NormalTok{.}\OtherTok{View}\NormalTok{.}\FunctionTok{extend}\NormalTok{(\{}

    \DataTypeTok{el}\NormalTok{: }\StringTok{'#todo'}\NormalTok{,}

    \CommentTok{// bind to DOM event using events property}
    \DataTypeTok{events}\NormalTok{: \{}
        \StringTok{'click [type="checkbox"]'}\NormalTok{: }\StringTok{'clicked'}\NormalTok{,}
    \NormalTok{\},}

    \DataTypeTok{initialize}\NormalTok{: }\KeywordTok{function} \NormalTok{() \{}
        \CommentTok{// bind to DOM event using jQuery}
        \KeywordTok{this}\NormalTok{.}\OtherTok{$el}\NormalTok{.}\FunctionTok{click}\NormalTok{(}\KeywordTok{this}\NormalTok{.}\FunctionTok{jqueryClicked}\NormalTok{);}

        \CommentTok{// bind to API event}
        \KeywordTok{this}\NormalTok{.}\FunctionTok{on}\NormalTok{(}\StringTok{'apiEvent'}\NormalTok{, }\KeywordTok{this}\NormalTok{.}\FunctionTok{callback}\NormalTok{);}
    \NormalTok{\},}

    \CommentTok{// 'this' is view}
    \DataTypeTok{clicked}\NormalTok{: }\KeywordTok{function}\NormalTok{(event) \{}
        \OtherTok{console}\NormalTok{.}\FunctionTok{log}\NormalTok{(}\StringTok{"events handler for "} \NormalTok{+ }\KeywordTok{this}\NormalTok{.}\OtherTok{el}\NormalTok{.}\FunctionTok{outerHTML}\NormalTok{);}
        \KeywordTok{this}\NormalTok{.}\FunctionTok{trigger}\NormalTok{(}\StringTok{'apiEvent'}\NormalTok{, }\OtherTok{event}\NormalTok{.}\FunctionTok{type}\NormalTok{);}
    \NormalTok{\},}

    \CommentTok{// 'this' is handling DOM element}
    \DataTypeTok{jqueryClicked}\NormalTok{: }\KeywordTok{function}\NormalTok{(event) \{}
        \OtherTok{console}\NormalTok{.}\FunctionTok{log}\NormalTok{(}\StringTok{"jQuery handler for "} \NormalTok{+ }\KeywordTok{this}\NormalTok{.}\FunctionTok{outerHTML}\NormalTok{);}
    \NormalTok{\},}

    \DataTypeTok{callback}\NormalTok{: }\KeywordTok{function}\NormalTok{(eventType) \{}
        \OtherTok{console}\NormalTok{.}\FunctionTok{log}\NormalTok{(}\StringTok{"event type was "} \NormalTok{+ eventType);}
    \NormalTok{\}}

\NormalTok{\});}

\KeywordTok{var} \NormalTok{view = }\KeywordTok{new} \FunctionTok{View}\NormalTok{();}
\end{Highlighting}
\end{Shaded}

\subsection{Routers}\label{routers}

In Backbone, routers provide a way for you to connect URLs (either hash
fragments, or real) to parts of your application. Any piece of your
application that you want to be bookmarkable, shareable, and
back-button-able, needs a URL.

Some examples of routes using the hash mark may be seen below:

\begin{Shaded}
\begin{Highlighting}[]
\NormalTok{http:}\CommentTok{//example.com/#about}
\NormalTok{http:}\CommentTok{//example.com/#search/seasonal-horns/page2}
\end{Highlighting}
\end{Shaded}

An application will usually have at least one route mapping a URL route
to a function that determines what happens when a user reaches that
route. This relationship is defined as follows:

\begin{Shaded}
\begin{Highlighting}[]
\StringTok{'route'} \NormalTok{: }\StringTok{'mappedFunction'}
\end{Highlighting}
\end{Shaded}

Let's define our first router by extending \texttt{Backbone.Router}. For
the purposes of this guide, we're going to continue pretending we're
creating a complex todo application (something like a personal
organizer/planner) that requires a complex TodoRouter.

Note the inline comments in the code example below as they continue our
lesson on routers.

\begin{Shaded}
\begin{Highlighting}[]
\KeywordTok{var} \NormalTok{TodoRouter = }\OtherTok{Backbone}\NormalTok{.}\OtherTok{Router}\NormalTok{.}\FunctionTok{extend}\NormalTok{(\{}
    \CommentTok{/* define the route and function maps for this router */}
    \DataTypeTok{routes}\NormalTok{: \{}
        \StringTok{"about"} \NormalTok{: }\StringTok{"showAbout"}\NormalTok{,}
        \CommentTok{/* Sample usage: http://example.com/#about */}

        \StringTok{"todo/:id"} \NormalTok{: }\StringTok{"getTodo"}\NormalTok{,}
        \CommentTok{/* This is an example of using a ":param" variable which allows us to match}
\CommentTok{        any of the components between two URL slashes */}
        \CommentTok{/* Sample usage: http://example.com/#todo/5 */}

        \StringTok{"search/:query"} \NormalTok{: }\StringTok{"searchTodos"}\NormalTok{,}
        \CommentTok{/* We can also define multiple routes that are bound to the same map function,}
\CommentTok{        in this case searchTodos(). Note below how we're optionally passing in a}
\CommentTok{        reference to a page number if one is supplied */}
        \CommentTok{/* Sample usage: http://example.com/#search/job */}

        \StringTok{"search/:query/p:page"} \NormalTok{: }\StringTok{"searchTodos"}\NormalTok{,}
        \CommentTok{/* As we can see, URLs may contain as many ":param"s as we wish */}
        \CommentTok{/* Sample usage: http://example.com/#search/job/p1 */}

        \StringTok{"todos/:id/download/*documentPath"} \NormalTok{: }\StringTok{"downloadDocument"}\NormalTok{,}
        \CommentTok{/* This is an example of using a *splat. Splats are able to match any number of}
\CommentTok{        URL components and can be combined with ":param"s*/}
        \CommentTok{/* Sample usage: http://example.com/#todos/5/download/files/Meeting_schedule.doc */}

        \CommentTok{/* If you wish to use splats for anything beyond default routing, it's probably a good}
\CommentTok{        idea to leave them at the end of a URL otherwise you may need to apply regular}
\CommentTok{        expression parsing on your fragment */}

        \StringTok{"*other"}    \NormalTok{: }\StringTok{"defaultRoute"}\NormalTok{,}
        \CommentTok{/* This is a default route that also uses a *splat. Consider the}
\CommentTok{        default route a wildcard for URLs that are either not matched or where}
\CommentTok{        the user has incorrectly typed in a route path manually */}
        \CommentTok{/* Sample usage: http://example.com/# <anything> */}

        \StringTok{"optional(/:item)"}\NormalTok{: }\StringTok{"optionalItem"}\NormalTok{,}
        \StringTok{"named/optional/(y:z)"}\NormalTok{: }\StringTok{"namedOptionalItem"}
        \CommentTok{/* Router URLs also support optional parts via parentheses, without having}
\CommentTok{           to use a regex.  */}
    \NormalTok{\},}

    \DataTypeTok{showAbout}\NormalTok{: }\KeywordTok{function}\NormalTok{()\{}
    \NormalTok{\},}

    \DataTypeTok{getTodo}\NormalTok{: }\KeywordTok{function}\NormalTok{(id)\{}
        \CommentTok{/*}
\CommentTok{        Note that the id matched in the above route will be passed to this function}
\CommentTok{        */}
        \OtherTok{console}\NormalTok{.}\FunctionTok{log}\NormalTok{(}\StringTok{"You are trying to reach todo "} \NormalTok{+ id);}
    \NormalTok{\},}

    \DataTypeTok{searchTodos}\NormalTok{: }\KeywordTok{function}\NormalTok{(query, page)\{}
        \KeywordTok{var} \NormalTok{page_number = page || }\DecValTok{1}\NormalTok{;}
        \OtherTok{console}\NormalTok{.}\FunctionTok{log}\NormalTok{(}\StringTok{"Page number: "} \NormalTok{+ page_number + }\StringTok{" of the results for todos containing the word: "} \NormalTok{+ query);}
    \NormalTok{\},}

    \DataTypeTok{downloadDocument}\NormalTok{: }\KeywordTok{function}\NormalTok{(id, path)\{}
    \NormalTok{\},}

    \DataTypeTok{defaultRoute}\NormalTok{: }\KeywordTok{function}\NormalTok{(other)\{}
        \OtherTok{console}\NormalTok{.}\FunctionTok{log}\NormalTok{(}\StringTok{'Invalid. You attempted to reach:'} \NormalTok{+ other);}
    \NormalTok{\}}
\NormalTok{\});}

\CommentTok{/* Now that we have a router setup, we need to instantiate it */}

\KeywordTok{var} \NormalTok{myTodoRouter = }\KeywordTok{new} \FunctionTok{TodoRouter}\NormalTok{();}
\end{Highlighting}
\end{Shaded}

Backbone offers an opt-in for HTML5 pushState support via
\texttt{window.history.pushState}. This permits you to define routes
such as http://backbonejs.org/just/an/example. This will be supported
with automatic degradation when a user's browser doesn't support
pushState. Note that it is vastly preferred if you're capable of also
supporting pushState on the server side, although it is a little more
difficult to implement.

\textbf{Is there a limit to the number of routers I should be using?}

Andrew de Andrade has pointed out that DocumentCloud, the creators of
Backbone, usually only use a single router in most of their
applications. You're very likely to not require more than one or two
routers in your own projects; the majority of your application routing
can be kept organized in a single router without it getting unwieldy.

\paragraph{Backbone.history}\label{backbone.history}

Next, we need to initialize \texttt{Backbone.history} as it handles
\texttt{hashchange} events in our application. This will automatically
handle routes that have been defined and trigger callbacks when they've
been accessed.

The \texttt{Backbone.history.start()} method will simply tell Backbone
that it's okay to begin monitoring all \texttt{hashchange} events as
follows:

\begin{Shaded}
\begin{Highlighting}[]
\KeywordTok{var} \NormalTok{TodoRouter = }\OtherTok{Backbone}\NormalTok{.}\OtherTok{Router}\NormalTok{.}\FunctionTok{extend}\NormalTok{(\{}
  \CommentTok{/* define the route and function maps for this router */}
  \DataTypeTok{routes}\NormalTok{: \{}
    \StringTok{"about"} \NormalTok{: }\StringTok{"showAbout"}\NormalTok{,}
    \StringTok{"search/:query"} \NormalTok{: }\StringTok{"searchTodos"}\NormalTok{,}
    \StringTok{"search/:query/p:page"} \NormalTok{: }\StringTok{"searchTodos"}
  \NormalTok{\},}

  \DataTypeTok{showAbout}\NormalTok{: }\KeywordTok{function}\NormalTok{()\{\},}

  \DataTypeTok{searchTodos}\NormalTok{: }\KeywordTok{function}\NormalTok{(query, page)\{}
    \KeywordTok{var} \NormalTok{page_number = page || }\DecValTok{1}\NormalTok{;}
    \OtherTok{console}\NormalTok{.}\FunctionTok{log}\NormalTok{(}\StringTok{"Page number: "} \NormalTok{+ page_number + }\StringTok{" of the results for todos containing the word: "} \NormalTok{+ query);}
  \NormalTok{\}}
\NormalTok{\});}

\KeywordTok{var} \NormalTok{myTodoRouter = }\KeywordTok{new} \FunctionTok{TodoRouter}\NormalTok{();}

\OtherTok{Backbone}\NormalTok{.}\OtherTok{history}\NormalTok{.}\FunctionTok{start}\NormalTok{();}

\CommentTok{// Go to and check console:}
\CommentTok{// http://localhost/#search/job/p3   logs: Page number: 3 of the results for todos containing the word: job}
\CommentTok{// http://localhost/#search/job      logs: Page number: 1 of the results for todos containing the word: job }
\CommentTok{// etc.}
\end{Highlighting}
\end{Shaded}

Note: To run the last example, you'll need to create a local development
environment and test project, which we will cover later on in the book.

If you would like to update the URL to reflect the application state at
a particular point, you can use the router's \texttt{.navigate()}
method. By default, it simply updates your URL fragment without
triggering the \texttt{hashchange} event:

\begin{Shaded}
\begin{Highlighting}[]
\CommentTok{// Let's imagine we would like a specific fragment (edit) once a user opens a single todo}
\KeywordTok{var} \NormalTok{TodoRouter = }\OtherTok{Backbone}\NormalTok{.}\OtherTok{Router}\NormalTok{.}\FunctionTok{extend}\NormalTok{(\{}
  \DataTypeTok{routes}\NormalTok{: \{}
    \StringTok{"todo/:id"}\NormalTok{: }\StringTok{"viewTodo"}\NormalTok{,}
    \StringTok{"todo/:id/edit"}\NormalTok{: }\StringTok{"editTodo"}
    \CommentTok{// ... other routes}
  \NormalTok{\},}

  \DataTypeTok{viewTodo}\NormalTok{: }\KeywordTok{function}\NormalTok{(id)\{}
    \OtherTok{console}\NormalTok{.}\FunctionTok{log}\NormalTok{(}\StringTok{"View todo requested."}\NormalTok{);}
    \KeywordTok{this}\NormalTok{.}\FunctionTok{navigate}\NormalTok{(}\StringTok{"todo/"} \NormalTok{+ id + }\StringTok{'/edit'}\NormalTok{); }\CommentTok{// updates the fragment for us, but doesn't trigger the route}
  \NormalTok{\},}

  \DataTypeTok{editTodo}\NormalTok{: }\KeywordTok{function}\NormalTok{(id) \{}
    \OtherTok{console}\NormalTok{.}\FunctionTok{log}\NormalTok{(}\StringTok{"Edit todo opened."}\NormalTok{);}
  \NormalTok{\}}
\NormalTok{\});}

\KeywordTok{var} \NormalTok{myTodoRouter = }\KeywordTok{new} \FunctionTok{TodoRouter}\NormalTok{();}

\OtherTok{Backbone}\NormalTok{.}\OtherTok{history}\NormalTok{.}\FunctionTok{start}\NormalTok{();}

\CommentTok{// Go to: http://localhost/#todo/4}
\CommentTok{//}
\CommentTok{// URL is updated to: http://localhost/#todo/4/edit}
\CommentTok{// but editTodo() function is not invoked even though location we end up is mapped to it.}
\CommentTok{//}
\CommentTok{// logs: View todo requested.}
\end{Highlighting}
\end{Shaded}

It is also possible for \texttt{Router.navigate()} to trigger the route
along with updating the URL fragment by passing the
\texttt{trigger:true} option.

Note: This usage is discouraged. The recommended usage is the one
described above which creates a bookmarkable URL when your application
transitions to a specific state.

\begin{Shaded}
\begin{Highlighting}[]
\KeywordTok{var} \NormalTok{TodoRouter = }\OtherTok{Backbone}\NormalTok{.}\OtherTok{Router}\NormalTok{.}\FunctionTok{extend}\NormalTok{(\{}
  \DataTypeTok{routes}\NormalTok{: \{}
    \StringTok{"todo/:id"}\NormalTok{: }\StringTok{"viewTodo"}\NormalTok{,}
    \StringTok{"todo/:id/edit"}\NormalTok{: }\StringTok{"editTodo"}
    \CommentTok{// ... other routes}
  \NormalTok{\},}

  \DataTypeTok{viewTodo}\NormalTok{: }\KeywordTok{function}\NormalTok{(id)\{}
    \OtherTok{console}\NormalTok{.}\FunctionTok{log}\NormalTok{(}\StringTok{"View todo requested."}\NormalTok{);}
    \KeywordTok{this}\NormalTok{.}\FunctionTok{navigate}\NormalTok{(}\StringTok{"todo/"} \NormalTok{+ id + }\StringTok{'/edit'}\NormalTok{, \{}\DataTypeTok{trigger}\NormalTok{: }\KeywordTok{true}\NormalTok{\}); }\CommentTok{// updates the fragment and triggers the route as well}
  \NormalTok{\},}

  \DataTypeTok{editTodo}\NormalTok{: }\KeywordTok{function}\NormalTok{(id) \{}
    \OtherTok{console}\NormalTok{.}\FunctionTok{log}\NormalTok{(}\StringTok{"Edit todo opened."}\NormalTok{);}
  \NormalTok{\}}
\NormalTok{\});}

\KeywordTok{var} \NormalTok{myTodoRouter = }\KeywordTok{new} \FunctionTok{TodoRouter}\NormalTok{();}

\OtherTok{Backbone}\NormalTok{.}\OtherTok{history}\NormalTok{.}\FunctionTok{start}\NormalTok{();}

\CommentTok{// Go to: http://localhost/#todo/4}
\CommentTok{//}
\CommentTok{// URL is updated to: http://localhost/#todo/4/edit}
\CommentTok{// and this time editTodo() function is invoked.}
\CommentTok{//}
\CommentTok{// logs:}
\CommentTok{// View todo requested.}
\CommentTok{// Edit todo opened.}
\end{Highlighting}
\end{Shaded}

A ``route'' event is also triggered on the router in addition to being
fired on Backbone.history.

\begin{Shaded}
\begin{Highlighting}[]
\OtherTok{Backbone}\NormalTok{.}\OtherTok{history}\NormalTok{.}\FunctionTok{on}\NormalTok{(}\StringTok{'route'}\NormalTok{, onRoute);}

\CommentTok{// Trigger 'route' event on router instance.}
\OtherTok{router}\NormalTok{.}\FunctionTok{on}\NormalTok{(}\StringTok{'route'}\NormalTok{, }\KeywordTok{function}\NormalTok{(name, args) \{}
  \OtherTok{console}\NormalTok{.}\FunctionTok{log}\NormalTok{(name === }\StringTok{'routeEvent'}\NormalTok{); }
\NormalTok{\});}

\OtherTok{location}\NormalTok{.}\FunctionTok{replace}\NormalTok{(}\StringTok{'http://example.com#route-event/x'}\NormalTok{);}
\OtherTok{Backbone}\NormalTok{.}\OtherTok{history}\NormalTok{.}\FunctionTok{checkUrl}\NormalTok{();}
\end{Highlighting}
\end{Shaded}

\subsection{Backbone's Sync API}\label{backbones-sync-api}

We previously discussed how Backbone supports RESTful persistence via
its \texttt{fetch()} and \texttt{create()} methods on Collections and
\texttt{save()}, and \texttt{destroy()} methods on Models. Now we are
going to take a closer look at Backbone's sync method which underlies
these operations.

The Backbone.sync method is an integral part of Backbone.js. It assumes
a jQuery-like \texttt{\$.ajax()} method, so HTTP parameters are
organized based on jQuery's API. Since some legacy servers may not
support JSON-formatted requests and HTTP PUT and DELETE operations,
Backbone can be configured to emulate these capabilities using the two
configuration variables shown below with their default values:

\begin{Shaded}
\begin{Highlighting}[]
\OtherTok{Backbone}\NormalTok{.}\FunctionTok{emulateHTTP} \NormalTok{= }\KeywordTok{false}\NormalTok{; }\CommentTok{// set to true if server cannot handle HTTP PUT or HTTP DELETE}
\OtherTok{Backbone}\NormalTok{.}\FunctionTok{emulateJSON} \NormalTok{= }\KeywordTok{false}\NormalTok{; }\CommentTok{// set to true if server cannot handle application/json requests}
\end{Highlighting}
\end{Shaded}

The inline Backbone.emulateHTTP option should be set to true if extended
HTTP methods are not supported by the server. The Backbone.emulateJSON
option should be set to true if the server does not understand the MIME
type for JSON.

\begin{Shaded}
\begin{Highlighting}[]
\CommentTok{// Create a new library collection}
\KeywordTok{var} \NormalTok{Library = }\OtherTok{Backbone}\NormalTok{.}\OtherTok{Collection}\NormalTok{.}\FunctionTok{extend}\NormalTok{(\{}
    \DataTypeTok{url }\NormalTok{: }\KeywordTok{function}\NormalTok{() \{ }\KeywordTok{return} \StringTok{'/library'}\NormalTok{; \}}
\NormalTok{\});}

\CommentTok{// Define attributes for our model}
\KeywordTok{var} \NormalTok{attrs = \{}
    \DataTypeTok{title  }\NormalTok{: }\StringTok{"The Tempest"}\NormalTok{,}
    \DataTypeTok{author }\NormalTok{: }\StringTok{"Bill Shakespeare"}\NormalTok{,}
    \DataTypeTok{length }\NormalTok{: }\DecValTok{123}
\NormalTok{\};}
  
\CommentTok{// Create a new Library instance}
\KeywordTok{var} \NormalTok{library = }\KeywordTok{new} \NormalTok{Library;}

\CommentTok{// Create a new instance of a model within our collection}
\OtherTok{library}\NormalTok{.}\FunctionTok{create}\NormalTok{(attrs, \{}\DataTypeTok{wait}\NormalTok{: }\KeywordTok{false}\NormalTok{\});}
  
\CommentTok{// Update with just emulateHTTP}
\OtherTok{library}\NormalTok{.}\FunctionTok{first}\NormalTok{().}\FunctionTok{save}\NormalTok{(\{}\DataTypeTok{id}\NormalTok{: }\StringTok{'2-the-tempest'}\NormalTok{, }\DataTypeTok{author}\NormalTok{: }\StringTok{'Tim Shakespeare'}\NormalTok{\}, \{}
  \DataTypeTok{emulateHTTP}\NormalTok{: }\KeywordTok{true}
\NormalTok{\});}
    
\CommentTok{// Check the ajaxSettings being used for our request}
\OtherTok{console}\NormalTok{.}\FunctionTok{log}\NormalTok{(}\KeywordTok{this}\NormalTok{.}\OtherTok{ajaxSettings}\NormalTok{.}\FunctionTok{url} \NormalTok{=== }\StringTok{'/library/2-the-tempest'}\NormalTok{); }\CommentTok{// true}
\OtherTok{console}\NormalTok{.}\FunctionTok{log}\NormalTok{(}\KeywordTok{this}\NormalTok{.}\OtherTok{ajaxSettings}\NormalTok{.}\FunctionTok{type} \NormalTok{=== }\StringTok{'POST'}\NormalTok{); }\CommentTok{// true}
\OtherTok{console}\NormalTok{.}\FunctionTok{log}\NormalTok{(}\KeywordTok{this}\NormalTok{.}\OtherTok{ajaxSettings}\NormalTok{.}\FunctionTok{contentType} \NormalTok{=== }\StringTok{'application/json'}\NormalTok{); }\CommentTok{// true}

\CommentTok{// Parse the data for the request to confirm it is as expected}
\KeywordTok{var} \NormalTok{data = }\OtherTok{JSON}\NormalTok{.}\FunctionTok{parse}\NormalTok{(}\KeywordTok{this}\NormalTok{.}\OtherTok{ajaxSettings}\NormalTok{.}\FunctionTok{data}\NormalTok{);}
\OtherTok{console}\NormalTok{.}\FunctionTok{log}\NormalTok{(}\OtherTok{data}\NormalTok{.}\FunctionTok{id} \NormalTok{=== }\StringTok{'2-the-tempest'}\NormalTok{);  }\CommentTok{// true}
\OtherTok{console}\NormalTok{.}\FunctionTok{log}\NormalTok{(}\OtherTok{data}\NormalTok{.}\FunctionTok{author} \NormalTok{=== }\StringTok{'Tim Shakespeare'}\NormalTok{); }\CommentTok{// true}
\OtherTok{console}\NormalTok{.}\FunctionTok{log}\NormalTok{(}\OtherTok{data}\NormalTok{.}\FunctionTok{length} \NormalTok{=== }\DecValTok{123}\NormalTok{); }\CommentTok{// true}
\end{Highlighting}
\end{Shaded}

Similarly, we could just update using \texttt{emulateJSON}:

\begin{Shaded}
\begin{Highlighting}[]
\OtherTok{library}\NormalTok{.}\FunctionTok{first}\NormalTok{().}\FunctionTok{save}\NormalTok{(\{}\DataTypeTok{id}\NormalTok{: }\StringTok{'2-the-tempest'}\NormalTok{, }\DataTypeTok{author}\NormalTok{: }\StringTok{'Tim Shakespeare'}\NormalTok{\}, \{}
  \DataTypeTok{emulateJSON}\NormalTok{: }\KeywordTok{true}
\NormalTok{\});}

\OtherTok{console}\NormalTok{.}\FunctionTok{log}\NormalTok{(}\KeywordTok{this}\NormalTok{.}\OtherTok{ajaxSettings}\NormalTok{.}\FunctionTok{url} \NormalTok{=== }\StringTok{'/library/2-the-tempest'}\NormalTok{); }\CommentTok{// true}
\OtherTok{console}\NormalTok{.}\FunctionTok{log}\NormalTok{(}\KeywordTok{this}\NormalTok{.}\OtherTok{ajaxSettings}\NormalTok{.}\FunctionTok{type} \NormalTok{=== }\StringTok{'PUT'}\NormalTok{); }\CommentTok{// true}
\OtherTok{console}\NormalTok{.}\FunctionTok{log}\NormalTok{(}\KeywordTok{this}\NormalTok{.}\OtherTok{ajaxSettings}\NormalTok{.}\FunctionTok{contentType} \NormalTok{===}\StringTok{'application/x-www-form-urlencoded'}\NormalTok{); }\CommentTok{// true}

\KeywordTok{var} \NormalTok{data = }\OtherTok{JSON}\NormalTok{.}\FunctionTok{parse}\NormalTok{(}\KeywordTok{this}\NormalTok{.}\OtherTok{ajaxSettings}\NormalTok{.}\OtherTok{data}\NormalTok{.}\FunctionTok{model}\NormalTok{);}
\OtherTok{console}\NormalTok{.}\FunctionTok{log}\NormalTok{(}\OtherTok{data}\NormalTok{.}\FunctionTok{id} \NormalTok{=== }\StringTok{'2-the-tempest'}\NormalTok{);}
\OtherTok{console}\NormalTok{.}\FunctionTok{log}\NormalTok{(}\OtherTok{data}\NormalTok{.}\FunctionTok{author} \NormalTok{===}\StringTok{'Tim Shakespeare'}\NormalTok{);}
\OtherTok{console}\NormalTok{.}\FunctionTok{log}\NormalTok{(}\OtherTok{data}\NormalTok{.}\FunctionTok{length} \NormalTok{=== }\DecValTok{123}\NormalTok{);}
\end{Highlighting}
\end{Shaded}

\texttt{Backbone.sync} is called every time Backbone tries to read,
save, or delete models. It uses jQuery or Zepto's \texttt{\$.ajax()}
implementations to make these RESTful requests, however this can be
overridden as per your needs.

\textbf{Overriding Backbone.sync}

The \texttt{sync} function may be overridden globally as Backbone.sync,
or at a finer-grained level, by adding a sync function to a Backbone
collection or to an individual model.

Since all persistence is handled by the Backbone.sync function, an
alternative persistence layer can be used by simply overriding
Backbone.sync with a function that has the same signature:

\begin{Shaded}
\begin{Highlighting}[]
\OtherTok{Backbone}\NormalTok{.}\FunctionTok{sync} \NormalTok{= }\KeywordTok{function}\NormalTok{(method, model, options) \{}
\NormalTok{\};}
\end{Highlighting}
\end{Shaded}

The methodMap below is used by the standard sync implementation to map
the method parameter to an HTTP operation and illustrates the type of
action required by each method argument:

\begin{Shaded}
\begin{Highlighting}[]
\KeywordTok{var} \NormalTok{methodMap = \{}
  \StringTok{'create'}\NormalTok{: }\StringTok{'POST'}\NormalTok{,}
  \StringTok{'update'}\NormalTok{: }\StringTok{'PUT'}\NormalTok{,}
  \StringTok{'patch'}\NormalTok{:  }\StringTok{'PATCH'}\NormalTok{,}
  \StringTok{'delete'}\NormalTok{: }\StringTok{'DELETE'}\NormalTok{,}
  \StringTok{'read'}\NormalTok{:   }\StringTok{'GET'}
\NormalTok{\};}
\end{Highlighting}
\end{Shaded}

If we wanted to replace the standard \texttt{sync} implementation with
one that simply logged the calls to sync, we could do this:

\begin{Shaded}
\begin{Highlighting}[]
\KeywordTok{var} \NormalTok{id_counter = }\DecValTok{1}\NormalTok{;}
\OtherTok{Backbone}\NormalTok{.}\FunctionTok{sync} \NormalTok{= }\KeywordTok{function}\NormalTok{(method, model) \{}
  \OtherTok{console}\NormalTok{.}\FunctionTok{log}\NormalTok{(}\StringTok{"I've been passed "} \NormalTok{+ method + }\StringTok{" with "} \NormalTok{+ }\OtherTok{JSON}\NormalTok{.}\FunctionTok{stringify}\NormalTok{(model));}
  \KeywordTok{if}\NormalTok{(method === }\StringTok{'create'}\NormalTok{)\{ }\OtherTok{model}\NormalTok{.}\FunctionTok{set}\NormalTok{(}\StringTok{'id'}\NormalTok{, id_counter++); \}}
\NormalTok{\};}
\end{Highlighting}
\end{Shaded}

Note that we assign a unique id to any created models.

The Backbone.sync method is intended to be overridden to support other
persistence backends. The built-in method is tailored to a certain breed
of RESTful JSON APIs - Backbone was originally extracted from a Ruby on
Rails application, which uses HTTP methods like PUT in the same way.

The sync method is called with three parameters:

\begin{itemize}
\itemsep1pt\parskip0pt\parsep0pt
\item
  method: One of create, update, patch, delete, or read
\item
  model: The Backbone model object
\item
  options: May include success and error methods
\end{itemize}

Implementing a new sync method can use the following pattern:

\begin{Shaded}
\begin{Highlighting}[]
\OtherTok{Backbone}\NormalTok{.}\FunctionTok{sync} \NormalTok{= }\KeywordTok{function}\NormalTok{(method, model, options) \{}

  \KeywordTok{function} \FunctionTok{success}\NormalTok{(result) \{}
    \CommentTok{// Handle successful results from MyAPI}
    \KeywordTok{if} \NormalTok{(}\OtherTok{options}\NormalTok{.}\FunctionTok{success}\NormalTok{) \{}
      \OtherTok{options}\NormalTok{.}\FunctionTok{success}\NormalTok{(result);}
    \NormalTok{\}}
  \NormalTok{\}}

  \KeywordTok{function} \FunctionTok{error}\NormalTok{(result) \{}
    \CommentTok{// Handle error results from MyAPI}
    \KeywordTok{if} \NormalTok{(}\OtherTok{options}\NormalTok{.}\FunctionTok{error}\NormalTok{) \{}
      \OtherTok{options}\NormalTok{.}\FunctionTok{error}\NormalTok{(result);}
    \NormalTok{\}}
  \NormalTok{\}}

  \NormalTok{options || (options = \{\});}

  \KeywordTok{switch} \NormalTok{(method) \{}
    \KeywordTok{case} \StringTok{'create'}\NormalTok{:}
      \KeywordTok{return} \OtherTok{MyAPI}\NormalTok{.}\FunctionTok{create}\NormalTok{(model, success, error);}

    \KeywordTok{case} \StringTok{'update'}\NormalTok{:}
      \KeywordTok{return} \OtherTok{MyAPI}\NormalTok{.}\FunctionTok{update}\NormalTok{(model, success, error);}

    \KeywordTok{case} \StringTok{'patch'}\NormalTok{:}
      \KeywordTok{return} \OtherTok{MyAPI}\NormalTok{.}\FunctionTok{patch}\NormalTok{(model, success, error);}

    \KeywordTok{case} \StringTok{'delete'}\NormalTok{:}
      \KeywordTok{return} \OtherTok{MyAPI}\NormalTok{.}\FunctionTok{destroy}\NormalTok{(model, success, error);}

    \KeywordTok{case} \StringTok{'read'}\NormalTok{:}
      \KeywordTok{if} \NormalTok{(}\OtherTok{model}\NormalTok{.}\FunctionTok{cid}\NormalTok{) \{}
        \KeywordTok{return} \OtherTok{MyAPI}\NormalTok{.}\FunctionTok{find}\NormalTok{(model, success, error);}
      \NormalTok{\} }\KeywordTok{else} \NormalTok{\{}
        \KeywordTok{return} \OtherTok{MyAPI}\NormalTok{.}\FunctionTok{findAll}\NormalTok{(model, success, error);}
      \NormalTok{\}}
  \NormalTok{\}}
\NormalTok{\};}
\end{Highlighting}
\end{Shaded}

This pattern delegates API calls to a new object (MyAPI), which could be
a Backbone-style class that supports events. This can be safely tested
separately, and potentially used with libraries other than Backbone.

There are quite a few sync implementations out there. The following
examples are all available on GitHub:

\begin{itemize}
\itemsep1pt\parskip0pt\parsep0pt
\item
  Backbone localStorage: persists to the browser's local storage
\item
  Backbone offline: supports working offline
\item
  Backbone Redis: uses Redis key-value store
\item
  backbone-parse: integrates Backbone with Parse.com
\item
  backbone-websql: stores data to WebSQL
\item
  Backbone Caching Sync: uses local storage as cache for other sync
  implementations
\end{itemize}

\subsection{Dependencies}\label{dependencies}

The official Backbone.js \href{http://backbonejs.org/}{documentation}
states:

\begin{quote}
Backbone's only hard dependency is either Underscore.js (
\textgreater{}= 1.4.3) or Lo-Dash. For RESTful persistence, history
support via Backbone.Router and DOM manipulation with Backbone.View,
include json2.js, and either jQuery ( \textgreater{}= 1.7.0) or Zepto.
\end{quote}

What this translates to is that if you require working with anything
beyond models, you will need to include a DOM manipulation library such
as jQuery or Zepto. Underscore is primarily used for its utility methods
(which Backbone relies upon heavily) and json2.js for legacy browser
JSON support if Backbone.sync is used.

\subsection{Summary}\label{summary-1}

In this chapter we have introduced you to the components you will be
using to build applications with Backbone: Models, Views, Collections,
and Routers. We've also explored the Events mix-in that Backbone uses to
enhance all components with publish-subscribe capabilities and seen how
it can be used with arbitrary objects. Finally, we saw how Backbone
leverages the Underscore.js and jQuery/Zepto APIs to add rich
manipulation and persistence features to Backbone Collections and
Models.

Backbone has many operations and options beyond those we have covered
here and is always evolving, so be sure to visit the official
\href{http://backbonejs.org/}{documentation} for more details and the
latest features. In the next chapter you will start to get your hands
dirty as we walk you through the implementation of your first Backbone
application.

\section{Exercise 1: Todos - Your First Backbone.js
App}\label{exercise-1-todos---your-first-backbone.js-app}

Now that we've covered fundamentals, let's write our first Backbone.js
application. We'll build the Backbone Todo List application exhibited on
\href{http://todomvc.com}{TodoMVC.com}. Building a Todo List is a great
way to learn Backbone's conventions. It's a relatively simple
application, yet technical challenges surrounding binding, persisting
model data, routing, and template rendering provide opportunities to
illustrate some core Backbone features.

\begin{figure}[htbp]
\centering
\includegraphics{img/todos_a.png}
\end{figure}

Let's consider the application's architecture at a high level. We'll
need:

\begin{itemize}
\itemsep1pt\parskip0pt\parsep0pt
\item
  a \texttt{Todo} model to describe individual todo items
\item
  a \texttt{TodoList} collection to store and persist todos
\item
  a way of creating todos
\item
  a way to display a listing of todos
\item
  a way to edit existing todos
\item
  a way to mark a todo as completed
\item
  a way to delete todos
\item
  a way to filter the items that have been completed or are remaining
\end{itemize}

Essentially, these features are classic
\href{http://en.wikipedia.org/wiki/Create,_read,_update_and_delete}{CRUD}
methods. Let's get started!

\subsection{Static HTML}\label{static-html}

We'll place all of our HTML in a single file named index.html.

\paragraph{Header and Scripts}\label{header-and-scripts}

First, we'll set up the header and the basic application dependencies:
\href{http://jquery.com}{jQuery},
\href{http://underscorejs.org}{Underscore}, Backbone.js and the
\href{https://github.com/jeromegn/Backbone.localStorage}{Backbone
LocalStorage adapter}.

\begin{Shaded}
\begin{Highlighting}[]
\ErrorTok{<}\NormalTok{!doctype html>}
\KeywordTok{<html}\OtherTok{ lang=}\StringTok{"en"}\KeywordTok{>}
\KeywordTok{<head>}
  \KeywordTok{<meta}\OtherTok{ charset=}\StringTok{"utf-8"}\KeywordTok{>}
  \KeywordTok{<meta}\OtherTok{ http-equiv=}\StringTok{"X-UA-Compatible"}\OtherTok{ content=}\StringTok{"IE=edge,chrome=1"}\KeywordTok{>}
  \KeywordTok{<title>}\NormalTok{Backbone.js • TodoMVC}\KeywordTok{</title>}
  \KeywordTok{<link}\OtherTok{ rel=}\StringTok{"stylesheet"}\OtherTok{ href=}\StringTok{"assets/base.css"}\KeywordTok{>}
\KeywordTok{</head>}
\KeywordTok{<body>}
  \KeywordTok{<script}\OtherTok{ type=}\StringTok{"text/template"}\OtherTok{ id=}\StringTok{"item-template"}\KeywordTok{></script>}
  \KeywordTok{<script}\OtherTok{ type=}\StringTok{"text/template"}\OtherTok{ id=}\StringTok{"stats-template"}\KeywordTok{></script>}
  \KeywordTok{<script}\OtherTok{ src=}\StringTok{"js/lib/jquery.min.js"}\KeywordTok{></script>}
  \KeywordTok{<script}\OtherTok{ src=}\StringTok{"js/lib/underscore-min.js"}\KeywordTok{></script>}
  \KeywordTok{<script}\OtherTok{ src=}\StringTok{"js/lib/backbone-min.js"}\KeywordTok{></script>}
  \KeywordTok{<script}\OtherTok{ src=}\StringTok{"js/lib/backbone.localStorage.js"}\KeywordTok{></script>}
  \KeywordTok{<script}\OtherTok{ src=}\StringTok{"js/models/todo.js"}\KeywordTok{></script>}
  \KeywordTok{<script}\OtherTok{ src=}\StringTok{"js/collections/todos.js"}\KeywordTok{></script>}
  \KeywordTok{<script}\OtherTok{ src=}\StringTok{"js/views/todos.js"}\KeywordTok{></script>}
  \KeywordTok{<script}\OtherTok{ src=}\StringTok{"js/views/app.js"}\KeywordTok{></script>}
  \KeywordTok{<script}\OtherTok{ src=}\StringTok{"js/routers/router.js"}\KeywordTok{></script>}
  \KeywordTok{<script}\OtherTok{ src=}\StringTok{"js/app.js"}\KeywordTok{></script>}
\KeywordTok{</body>}
\KeywordTok{</html>}
\end{Highlighting}
\end{Shaded}

In addition to the aforementioned dependencies, note that a few other
application-specific files are also loaded. These are organized into
folders representing their application responsibilities: models, views,
collections, and routers. An app.js file is present to house central
initialization code.

Note: If you want to follow along, create a directory structure as
demonstrated in index.html:

\begin{enumerate}
\def\labelenumi{\arabic{enumi}.}
\itemsep1pt\parskip0pt\parsep0pt
\item
  Place the index.html in a top-level directory.
\item
  Download jQuery, Underscore, Backbone, and Backbone LocalStorage from
  their respective web sites and place them under js/lib
\item
  Create the directories js/models, js/collections, js/views, and
  js/routers
\end{enumerate}

You will also need
\href{https://raw2.github.com/tastejs/todomvc/gh-pages/architecture-examples/backbone/bower_components/todomvc-common/base.css}{base.css}
and
\href{https://raw2.github.com/tastejs/todomvc/gh-pages/architecture-examples/backbone/bower_components/todomvc-common/bg.png}{bg.png},
which should live in an assets directory. And remember that you can see
a demo of the final application at
\href{http://todomvc.com}{TodoMVC.com}.

We will be creating the application JavaScript files during the
tutorial. Don't worry about the two `text/template' script elements - we
will replace those soon!

\paragraph{Application HTML}\label{application-html}

Now let's populate the body of index.html. We'll need an
\texttt{\textless{}input\textgreater{}} for creating new todos, a
\texttt{\textless{}ul id="todo-list" /\textgreater{}} for listing the
actual todos, and a footer where we can later insert statistics and
links for performing operations such as clearing completed todos. We'll
add the following markup immediately inside our body tag before the
script elements:

\begin{Shaded}
\begin{Highlighting}[]
  \KeywordTok{<section}\OtherTok{ id=}\StringTok{"todoapp"}\KeywordTok{>}
    \KeywordTok{<header}\OtherTok{ id=}\StringTok{"header"}\KeywordTok{>}
      \KeywordTok{<h1>}\NormalTok{todos}\KeywordTok{</h1>}
      \KeywordTok{<input}\OtherTok{ id=}\StringTok{"new-todo"}\OtherTok{ placeholder=}\StringTok{"What needs to be done?"}\OtherTok{ autofocus}\KeywordTok{>}
    \KeywordTok{</header>}
    \KeywordTok{<section}\OtherTok{ id=}\StringTok{"main"}\KeywordTok{>}
      \KeywordTok{<input}\OtherTok{ id=}\StringTok{"toggle-all"}\OtherTok{ type=}\StringTok{"checkbox"}\KeywordTok{>}
      \KeywordTok{<label}\OtherTok{ for=}\StringTok{"toggle-all"}\KeywordTok{>}\NormalTok{Mark all as complete}\KeywordTok{</label>}
      \KeywordTok{<ul}\OtherTok{ id=}\StringTok{"todo-list"}\KeywordTok{></ul>}
    \KeywordTok{</section>}
    \KeywordTok{<footer}\OtherTok{ id=}\StringTok{"footer"}\KeywordTok{></footer>}
  \KeywordTok{</section>}
  \KeywordTok{<div}\OtherTok{ id=}\StringTok{"info"}\KeywordTok{>}
    \KeywordTok{<p>}\NormalTok{Double-click to edit a todo}\KeywordTok{</p>}
    \KeywordTok{<p>}\NormalTok{Written by }\KeywordTok{<a}\OtherTok{ href=}\StringTok{"https://github.com/addyosmani"}\KeywordTok{>}\NormalTok{Addy Osmani}\KeywordTok{</a></p>}
    \KeywordTok{<p>}\NormalTok{Part of }\KeywordTok{<a}\OtherTok{ href=}\StringTok{"http://todomvc.com"}\KeywordTok{>}\NormalTok{TodoMVC}\KeywordTok{</a></p>}
  \KeywordTok{</div>}
\end{Highlighting}
\end{Shaded}

\paragraph{Templates}\label{templates}

To complete index.html, we need to add the templates which we will use
to dynamically create HTML by injecting model data into their
placeholders. One way of including templates in the page is by using
custom script tags. These don't get evaluated by the browser, which just
interprets them as plain text. Underscore micro-templating can then
access the templates, rendering fragments of HTML.

We'll start by filling in the \#item-template which will be used to
display individual todo items.

\begin{Shaded}
\begin{Highlighting}[]
  \CommentTok{<!-- index.html -->}

  \KeywordTok{<script}\OtherTok{ type=}\StringTok{"text/template"}\OtherTok{ id=}\StringTok{"item-template"}\KeywordTok{>}
    \NormalTok{<div }\KeywordTok{class}\NormalTok{=}\StringTok{"view"}\NormalTok{>}
      \NormalTok{<input }\KeywordTok{class}\NormalTok{=}\StringTok{"toggle"} \NormalTok{type=}\StringTok{"checkbox"} \NormalTok{<%= completed ? }\StringTok{'checked'} \NormalTok{: }\StringTok{''} \NormalTok{%>>}
      \NormalTok{<label><%= title %><}\OtherTok{/label>}
\OtherTok{      <button class="destroy"></button}\NormalTok{>}
    \NormalTok{<}\OtherTok{/div>}
\OtherTok{    <input class="edit" value="<%= title %>">}
\OtherTok{  </script}\NormalTok{>}
\end{Highlighting}
\end{Shaded}

The template tags in the above markup, such as \texttt{\textless{}\%=}
and \texttt{\textless{}\%-}, are specific to Underscore.js and are
documented on the Underscore site. In your own applications, you have a
choice of template libraries, such as Mustache or Handlebars. Use
whichever you prefer, Backbone doesn't mind.

We also need to define the \#stats-template template which we will use
to populate the footer.

\begin{Shaded}
\begin{Highlighting}[]
  \CommentTok{<!-- index.html -->}

  \KeywordTok{<script}\OtherTok{ type=}\StringTok{"text/template"}\OtherTok{ id=}\StringTok{"stats-template"}\KeywordTok{>}
    \NormalTok{<span id=}\StringTok{"todo-count"}\NormalTok{><strong><%= remaining %><}\OtherTok{/strong> <%= remaining === 1 }\FloatTok{?}\OtherTok{ 'item' : 'items' %> left</span}\NormalTok{>}
    \NormalTok{<ul id=}\StringTok{"filters"}\NormalTok{>}
      \NormalTok{<li>}
        \NormalTok{<a }\KeywordTok{class}\NormalTok{=}\StringTok{"selected"} \NormalTok{href=}\StringTok{"#/"}\NormalTok{>All<}\OtherTok{/a>}
\OtherTok{      </li}\NormalTok{>}
      \NormalTok{<li>}
        \NormalTok{<a href=}\StringTok{"#/active"}\NormalTok{>Active<}\OtherTok{/a>}
\OtherTok{      </li}\NormalTok{>}
      \NormalTok{<li>}
        \NormalTok{<a href=}\StringTok{"#/completed"}\NormalTok{>Completed<}\OtherTok{/a>}
\OtherTok{      </li}\NormalTok{>}
    \NormalTok{<}\OtherTok{/ul>}
\OtherTok{    <% if }\FloatTok{(}\OtherTok{completed}\FloatTok{)}\OtherTok{ \{ %>}
\OtherTok{    <button id="clear-completed">Clear completed }\FloatTok{(}\OtherTok{<%= completed %>}\FloatTok{)}\OtherTok{</button}\NormalTok{>}
    \NormalTok{<% \} %>}
  \KeywordTok{</script>}
\end{Highlighting}
\end{Shaded}

The \#stats-template displays the number of remaining incomplete items
and contains a list of hyperlinks which will be used to perform actions
when we implement our router. It also contains a button which can be
used to clear all of the completed items.

Now that we have all the HTML that we will need, we'll start
implementing our application by returning to the fundamentals: a Todo
model.

\subsection{Todo model}\label{todo-model}

The \texttt{Todo} model is remarkably straightforward. First, a todo has
two attributes: a \texttt{title} stores a todo item's title and a
\texttt{completed} status indicates if it's complete. These attributes
are passed as defaults, as shown below:

\begin{Shaded}
\begin{Highlighting}[]

  \CommentTok{// js/models/todo.js}

  \KeywordTok{var} \NormalTok{app = app || \{\};}

  \CommentTok{// Todo Model}
  \CommentTok{// ----------}
  \CommentTok{// Our basic **Todo** model has `title` and `completed` attributes.}

  \OtherTok{app}\NormalTok{.}\FunctionTok{Todo} \NormalTok{= }\OtherTok{Backbone}\NormalTok{.}\OtherTok{Model}\NormalTok{.}\FunctionTok{extend}\NormalTok{(\{}

    \CommentTok{// Default attributes ensure that each todo created has `title` and `completed` keys.}
    \DataTypeTok{defaults}\NormalTok{: \{}
      \DataTypeTok{title}\NormalTok{: }\StringTok{''}\NormalTok{,}
      \DataTypeTok{completed}\NormalTok{: }\KeywordTok{false}
    \NormalTok{\},}

    \CommentTok{// Toggle the `completed` state of this todo item.}
    \DataTypeTok{toggle}\NormalTok{: }\KeywordTok{function}\NormalTok{() \{}
      \KeywordTok{this}\NormalTok{.}\FunctionTok{save}\NormalTok{(\{}
        \DataTypeTok{completed}\NormalTok{: !}\KeywordTok{this}\NormalTok{.}\FunctionTok{get}\NormalTok{(}\StringTok{'completed'}\NormalTok{)}
      \NormalTok{\});}
    \NormalTok{\}}

  \NormalTok{\});}
\end{Highlighting}
\end{Shaded}

Second, the Todo model has a \texttt{toggle()} method through which a
Todo item's completion status can be set and simultaneously persisted.

\subsection{Todo collection}\label{todo-collection}

Next, a \texttt{TodoList} collection is used to group our models. The
collection uses the LocalStorage adapter to override Backbone's default
\texttt{sync()} operation with one that will persist our Todo records to
HTML5 Local Storage. Through local storage, they're saved between page
requests.

\begin{Shaded}
\begin{Highlighting}[]

  \CommentTok{// js/collections/todos.js}

  \KeywordTok{var} \NormalTok{app = app || \{\};}

  \CommentTok{// Todo Collection}
  \CommentTok{// ---------------}

  \CommentTok{// The collection of todos is backed by *localStorage* instead of a remote}
  \CommentTok{// server.}
  \KeywordTok{var} \NormalTok{TodoList = }\OtherTok{Backbone}\NormalTok{.}\OtherTok{Collection}\NormalTok{.}\FunctionTok{extend}\NormalTok{(\{}

    \CommentTok{// Reference to this collection's model.}
    \DataTypeTok{model}\NormalTok{: }\OtherTok{app}\NormalTok{.}\FunctionTok{Todo}\NormalTok{,}

    \CommentTok{// Save all of the todo items under the `"todos-backbone"` namespace.}
    \DataTypeTok{localStorage}\NormalTok{: }\KeywordTok{new} \OtherTok{Backbone}\NormalTok{.}\FunctionTok{LocalStorage}\NormalTok{(}\StringTok{'todos-backbone'}\NormalTok{),}

    \CommentTok{// Filter down the list of all todo items that are finished.}
    \DataTypeTok{completed}\NormalTok{: }\KeywordTok{function}\NormalTok{() \{}
      \KeywordTok{return} \KeywordTok{this}\NormalTok{.}\FunctionTok{filter}\NormalTok{(}\KeywordTok{function}\NormalTok{( todo ) \{}
        \KeywordTok{return} \OtherTok{todo}\NormalTok{.}\FunctionTok{get}\NormalTok{(}\StringTok{'completed'}\NormalTok{);}
      \NormalTok{\});}
    \NormalTok{\},}

    \CommentTok{// Filter down the list to only todo items that are still not finished.}
    \DataTypeTok{remaining}\NormalTok{: }\KeywordTok{function}\NormalTok{() \{}
      \KeywordTok{return} \KeywordTok{this}\NormalTok{.}\OtherTok{without}\NormalTok{.}\FunctionTok{apply}\NormalTok{( }\KeywordTok{this}\NormalTok{, }\KeywordTok{this}\NormalTok{.}\FunctionTok{completed}\NormalTok{() );}
    \NormalTok{\},}

    \CommentTok{// We keep the Todos in sequential order, despite being saved by unordered}
    \CommentTok{// GUID in the database. This generates the next order number for new items.}
    \DataTypeTok{nextOrder}\NormalTok{: }\KeywordTok{function}\NormalTok{() \{}
      \KeywordTok{if} \NormalTok{( !}\KeywordTok{this}\NormalTok{.}\FunctionTok{length} \NormalTok{) \{}
        \KeywordTok{return} \DecValTok{1}\NormalTok{;}
      \NormalTok{\}}
      \KeywordTok{return} \KeywordTok{this}\NormalTok{.}\FunctionTok{last}\NormalTok{().}\FunctionTok{get}\NormalTok{(}\StringTok{'order'}\NormalTok{) + }\DecValTok{1}\NormalTok{;}
    \NormalTok{\},}

    \CommentTok{// Todos are sorted by their original insertion order.}
    \DataTypeTok{comparator}\NormalTok{: }\KeywordTok{function}\NormalTok{( todo ) \{}
      \KeywordTok{return} \OtherTok{todo}\NormalTok{.}\FunctionTok{get}\NormalTok{(}\StringTok{'order'}\NormalTok{);}
    \NormalTok{\}}
  \NormalTok{\});}

  \CommentTok{// Create our global collection of **Todos**.}
  \OtherTok{app}\NormalTok{.}\FunctionTok{Todos} \NormalTok{= }\KeywordTok{new} \FunctionTok{TodoList}\NormalTok{();}
\end{Highlighting}
\end{Shaded}

The collection's \texttt{completed()} and \texttt{remaining()} methods
return an array of finished and unfinished todos, respectively.

A \texttt{nextOrder()} method implements a sequence generator while a
\texttt{comparator()} sorts items by their insertion order.

\emph{Note}: \texttt{this.filter}, \texttt{this.without} and
\texttt{this.last} are Underscore methods that are mixed in to
Backbone.Collection so that the reader knows how to find out more about
them.

\subsection{Application View}\label{application-view}

Let's examine the core application logic which resides in the views.
Each view supports functionality such as edit-in-place, and therefore
contains a fair amount of logic. To help organize this logic, we'll use
the element controller pattern. The element controller pattern consists
of two views: one controls a collection of items while the other deals
with each individual item.

In our case, an \texttt{AppView} will handle the creation of new todos
and rendering of the initial todo list. Instances of \texttt{TodoView}
will be associated with each individual Todo record. Todo instances can
handle editing, updating, and destroying their associated todo.

To keep things short and simple, we won't be implementing all of the
application's features in this tutorial, we'll just cover enough to get
you started. Even so, there is a lot for us to cover in
\texttt{AppView}, so we'll split our discussion into two sections.

\begin{Shaded}
\begin{Highlighting}[]

  \CommentTok{// js/views/app.js}

  \KeywordTok{var} \NormalTok{app = app || \{\};}

  \CommentTok{// The Application}
  \CommentTok{// ---------------}

  \CommentTok{// Our overall **AppView** is the top-level piece of UI.}
  \OtherTok{app}\NormalTok{.}\FunctionTok{AppView} \NormalTok{= }\OtherTok{Backbone}\NormalTok{.}\OtherTok{View}\NormalTok{.}\FunctionTok{extend}\NormalTok{(\{}

    \CommentTok{// Instead of generating a new element, bind to the existing skeleton of}
    \CommentTok{// the App already present in the HTML.}
    \DataTypeTok{el}\NormalTok{: }\StringTok{'#todoapp'}\NormalTok{,}

    \CommentTok{// Our template for the line of statistics at the bottom of the app.}
    \DataTypeTok{statsTemplate}\NormalTok{: }\OtherTok{_}\NormalTok{.}\FunctionTok{template}\NormalTok{( }\FunctionTok{$}\NormalTok{(}\StringTok{'#stats-template'}\NormalTok{).}\FunctionTok{html}\NormalTok{() ),}

    \CommentTok{// At initialization we bind to the relevant events on the `Todos`}
    \CommentTok{// collection, when items are added or changed.}
    \DataTypeTok{initialize}\NormalTok{: }\KeywordTok{function}\NormalTok{() \{}
      \KeywordTok{this}\NormalTok{.}\FunctionTok{allCheckbox} \NormalTok{= }\KeywordTok{this}\NormalTok{.}\FunctionTok{$}\NormalTok{(}\StringTok{'#toggle-all'}\NormalTok{)[}\DecValTok{0}\NormalTok{];}
      \KeywordTok{this}\NormalTok{.}\FunctionTok{$input} \NormalTok{= }\KeywordTok{this}\NormalTok{.}\FunctionTok{$}\NormalTok{(}\StringTok{'#new-todo'}\NormalTok{);}
      \KeywordTok{this}\NormalTok{.}\FunctionTok{$footer} \NormalTok{= }\KeywordTok{this}\NormalTok{.}\FunctionTok{$}\NormalTok{(}\StringTok{'#footer'}\NormalTok{);}
      \KeywordTok{this}\NormalTok{.}\FunctionTok{$main} \NormalTok{= }\KeywordTok{this}\NormalTok{.}\FunctionTok{$}\NormalTok{(}\StringTok{'#main'}\NormalTok{);}

      \KeywordTok{this}\NormalTok{.}\FunctionTok{listenTo}\NormalTok{(}\OtherTok{app}\NormalTok{.}\FunctionTok{Todos}\NormalTok{, }\StringTok{'add'}\NormalTok{, }\KeywordTok{this}\NormalTok{.}\FunctionTok{addOne}\NormalTok{);}
      \KeywordTok{this}\NormalTok{.}\FunctionTok{listenTo}\NormalTok{(}\OtherTok{app}\NormalTok{.}\FunctionTok{Todos}\NormalTok{, }\StringTok{'reset'}\NormalTok{, }\KeywordTok{this}\NormalTok{.}\FunctionTok{addAll}\NormalTok{);}
    \NormalTok{\},}

    \CommentTok{// Add a single todo item to the list by creating a view for it, and}
    \CommentTok{// appending its element to the `<ul>`.}
    \DataTypeTok{addOne}\NormalTok{: }\KeywordTok{function}\NormalTok{( todo ) \{}
      \KeywordTok{var} \NormalTok{view = }\KeywordTok{new} \OtherTok{app}\NormalTok{.}\FunctionTok{TodoView}\NormalTok{(\{ }\DataTypeTok{model}\NormalTok{: todo \});}
      \FunctionTok{$}\NormalTok{(}\StringTok{'#todo-list'}\NormalTok{).}\FunctionTok{append}\NormalTok{( }\OtherTok{view}\NormalTok{.}\FunctionTok{render}\NormalTok{().}\FunctionTok{el} \NormalTok{);}
    \NormalTok{\},}

    \CommentTok{// Add all items in the **Todos** collection at once.}
    \DataTypeTok{addAll}\NormalTok{: }\KeywordTok{function}\NormalTok{() \{}
      \KeywordTok{this}\NormalTok{.}\FunctionTok{$}\NormalTok{(}\StringTok{'#todo-list'}\NormalTok{).}\FunctionTok{html}\NormalTok{(}\StringTok{''}\NormalTok{);}
      \OtherTok{app}\NormalTok{.}\OtherTok{Todos}\NormalTok{.}\FunctionTok{each}\NormalTok{(}\KeywordTok{this}\NormalTok{.}\FunctionTok{addOne}\NormalTok{, }\KeywordTok{this}\NormalTok{);}
    \NormalTok{\}}

  \NormalTok{\});}
\end{Highlighting}
\end{Shaded}

A few notable features are present in our initial version of AppView,
including a \texttt{statsTemplate}, an \texttt{initialize} method that's
implicitly called on instantiation, and several view-specific methods.

An \texttt{el} (element) property stores a selector targeting the DOM
element with an ID of \texttt{todoapp}. In the case of our application,
\texttt{el} refers to the matching
\texttt{\textless{}section id="todoapp" /\textgreater{}} element in
index.html.

The call to \_.template uses Underscore's micro-templating to construct
a statsTemplate object from our \#stats-template. We will use this
template later when we render our view.

Now let's take a look at the \texttt{initialize} function. First, it's
using jQuery to cache the elements it will be using into local
properties (recall that \texttt{this.\$()} finds elements relative to
\texttt{this.\$el}). Then it's binding to two events on the Todos
collection: \texttt{add} and \texttt{reset}. Since we're delegating
handling of updates and deletes to the \texttt{TodoView} view, we don't
need to worry about those here. The two pieces of logic are:

\begin{itemize}
\item
  When an \texttt{add} event is fired the \texttt{addOne()} method is
  called and passed the new model. \texttt{addOne()} creates an instance
  of TodoView view, renders it, and appends the resulting element to our
  Todo list.
\item
  When a \texttt{reset} event occurs (i.e., we update the collection in
  bulk as happens when the Todos are loaded from Local Storage),
  \texttt{addAll()} is called, which iterates over all of the Todos
  currently in our collection and fires \texttt{addOne()} for each item.
\end{itemize}

Note that we were able to use \texttt{this} within \texttt{addAll()} to
refer to the view because \texttt{listenTo()} implicitly set the
callback's context to the view when it created the binding.

Now, let's add some more logic to complete our AppView!

\begin{Shaded}
\begin{Highlighting}[]

  \CommentTok{// js/views/app.js}

  \KeywordTok{var} \NormalTok{app = app || \{\};}

  \CommentTok{// The Application}
  \CommentTok{// ---------------}

  \CommentTok{// Our overall **AppView** is the top-level piece of UI.}
  \OtherTok{app}\NormalTok{.}\FunctionTok{AppView} \NormalTok{= }\OtherTok{Backbone}\NormalTok{.}\OtherTok{View}\NormalTok{.}\FunctionTok{extend}\NormalTok{(\{}

    \CommentTok{// Instead of generating a new element, bind to the existing skeleton of}
    \CommentTok{// the App already present in the HTML.}
    \DataTypeTok{el}\NormalTok{: }\StringTok{'#todoapp'}\NormalTok{,}

    \CommentTok{// Our template for the line of statistics at the bottom of the app.}
    \DataTypeTok{statsTemplate}\NormalTok{: }\OtherTok{_}\NormalTok{.}\FunctionTok{template}\NormalTok{( }\FunctionTok{$}\NormalTok{(}\StringTok{'#stats-template'}\NormalTok{).}\FunctionTok{html}\NormalTok{() ),}

    \CommentTok{// New}
    \CommentTok{// Delegated events for creating new items, and clearing completed ones.}
    \DataTypeTok{events}\NormalTok{: \{}
      \StringTok{'keypress #new-todo'}\NormalTok{: }\StringTok{'createOnEnter'}\NormalTok{,}
      \StringTok{'click #clear-completed'}\NormalTok{: }\StringTok{'clearCompleted'}\NormalTok{,}
      \StringTok{'click #toggle-all'}\NormalTok{: }\StringTok{'toggleAllComplete'}
    \NormalTok{\},}

    \CommentTok{// At initialization we bind to the relevant events on the `Todos`}
    \CommentTok{// collection, when items are added or changed. Kick things off by}
    \CommentTok{// loading any preexisting todos that might be saved in *localStorage*.}
    \DataTypeTok{initialize}\NormalTok{: }\KeywordTok{function}\NormalTok{() \{}
      \KeywordTok{this}\NormalTok{.}\FunctionTok{allCheckbox} \NormalTok{= }\KeywordTok{this}\NormalTok{.}\FunctionTok{$}\NormalTok{(}\StringTok{'#toggle-all'}\NormalTok{)[}\DecValTok{0}\NormalTok{];}
      \KeywordTok{this}\NormalTok{.}\FunctionTok{$input} \NormalTok{= }\KeywordTok{this}\NormalTok{.}\FunctionTok{$}\NormalTok{(}\StringTok{'#new-todo'}\NormalTok{);}
      \KeywordTok{this}\NormalTok{.}\FunctionTok{$footer} \NormalTok{= }\KeywordTok{this}\NormalTok{.}\FunctionTok{$}\NormalTok{(}\StringTok{'#footer'}\NormalTok{);}
      \KeywordTok{this}\NormalTok{.}\FunctionTok{$main} \NormalTok{= }\KeywordTok{this}\NormalTok{.}\FunctionTok{$}\NormalTok{(}\StringTok{'#main'}\NormalTok{);}

      \KeywordTok{this}\NormalTok{.}\FunctionTok{listenTo}\NormalTok{(}\OtherTok{app}\NormalTok{.}\FunctionTok{Todos}\NormalTok{, }\StringTok{'add'}\NormalTok{, }\KeywordTok{this}\NormalTok{.}\FunctionTok{addOne}\NormalTok{);}
      \KeywordTok{this}\NormalTok{.}\FunctionTok{listenTo}\NormalTok{(}\OtherTok{app}\NormalTok{.}\FunctionTok{Todos}\NormalTok{, }\StringTok{'reset'}\NormalTok{, }\KeywordTok{this}\NormalTok{.}\FunctionTok{addAll}\NormalTok{);}

      \CommentTok{// New}
      \KeywordTok{this}\NormalTok{.}\FunctionTok{listenTo}\NormalTok{(}\OtherTok{app}\NormalTok{.}\FunctionTok{Todos}\NormalTok{, }\StringTok{'change:completed'}\NormalTok{, }\KeywordTok{this}\NormalTok{.}\FunctionTok{filterOne}\NormalTok{);}
      \KeywordTok{this}\NormalTok{.}\FunctionTok{listenTo}\NormalTok{(}\OtherTok{app}\NormalTok{.}\FunctionTok{Todos}\NormalTok{,}\StringTok{'filter'}\NormalTok{, }\KeywordTok{this}\NormalTok{.}\FunctionTok{filterAll}\NormalTok{);}
      \KeywordTok{this}\NormalTok{.}\FunctionTok{listenTo}\NormalTok{(}\OtherTok{app}\NormalTok{.}\FunctionTok{Todos}\NormalTok{, }\StringTok{'all'}\NormalTok{, }\KeywordTok{this}\NormalTok{.}\FunctionTok{render}\NormalTok{);}

      \OtherTok{app}\NormalTok{.}\OtherTok{Todos}\NormalTok{.}\FunctionTok{fetch}\NormalTok{();}
    \NormalTok{\},}

    \CommentTok{// New}
    \CommentTok{// Re-rendering the App just means refreshing the statistics -- the rest}
    \CommentTok{// of the app doesn't change.}
    \DataTypeTok{render}\NormalTok{: }\KeywordTok{function}\NormalTok{() \{}
      \KeywordTok{var} \NormalTok{completed = }\OtherTok{app}\NormalTok{.}\OtherTok{Todos}\NormalTok{.}\FunctionTok{completed}\NormalTok{().}\FunctionTok{length}\NormalTok{;}
      \KeywordTok{var} \NormalTok{remaining = }\OtherTok{app}\NormalTok{.}\OtherTok{Todos}\NormalTok{.}\FunctionTok{remaining}\NormalTok{().}\FunctionTok{length}\NormalTok{;}

      \KeywordTok{if} \NormalTok{( }\OtherTok{app}\NormalTok{.}\OtherTok{Todos}\NormalTok{.}\FunctionTok{length} \NormalTok{) \{}
        \KeywordTok{this}\NormalTok{.}\OtherTok{$main}\NormalTok{.}\FunctionTok{show}\NormalTok{();}
        \KeywordTok{this}\NormalTok{.}\OtherTok{$footer}\NormalTok{.}\FunctionTok{show}\NormalTok{();}

        \KeywordTok{this}\NormalTok{.}\OtherTok{$footer}\NormalTok{.}\FunctionTok{html}\NormalTok{(}\KeywordTok{this}\NormalTok{.}\FunctionTok{statsTemplate}\NormalTok{(\{}
          \DataTypeTok{completed}\NormalTok{: completed,}
          \DataTypeTok{remaining}\NormalTok{: remaining}
        \NormalTok{\}));}

        \KeywordTok{this}\NormalTok{.}\FunctionTok{$}\NormalTok{(}\StringTok{'#filters li a'}\NormalTok{)}
          \NormalTok{.}\FunctionTok{removeClass}\NormalTok{(}\StringTok{'selected'}\NormalTok{)}
          \NormalTok{.}\FunctionTok{filter}\NormalTok{(}\StringTok{'[href="#/'} \NormalTok{+ ( }\OtherTok{app}\NormalTok{.}\FunctionTok{TodoFilter} \NormalTok{|| }\StringTok{''} \NormalTok{) + }\StringTok{'"]'}\NormalTok{)}
          \NormalTok{.}\FunctionTok{addClass}\NormalTok{(}\StringTok{'selected'}\NormalTok{);}
      \NormalTok{\} }\KeywordTok{else} \NormalTok{\{}
        \KeywordTok{this}\NormalTok{.}\OtherTok{$main}\NormalTok{.}\FunctionTok{hide}\NormalTok{();}
        \KeywordTok{this}\NormalTok{.}\OtherTok{$footer}\NormalTok{.}\FunctionTok{hide}\NormalTok{();}
      \NormalTok{\}}

      \KeywordTok{this}\NormalTok{.}\OtherTok{allCheckbox}\NormalTok{.}\FunctionTok{checked} \NormalTok{= !remaining;}
    \NormalTok{\},}

    \CommentTok{// Add a single todo item to the list by creating a view for it, and}
    \CommentTok{// appending its element to the `<ul>`.}
    \DataTypeTok{addOne}\NormalTok{: }\KeywordTok{function}\NormalTok{( todo ) \{}
      \KeywordTok{var} \NormalTok{view = }\KeywordTok{new} \OtherTok{app}\NormalTok{.}\FunctionTok{TodoView}\NormalTok{(\{ }\DataTypeTok{model}\NormalTok{: todo \});}
      \FunctionTok{$}\NormalTok{(}\StringTok{'#todo-list'}\NormalTok{).}\FunctionTok{append}\NormalTok{( }\OtherTok{view}\NormalTok{.}\FunctionTok{render}\NormalTok{().}\FunctionTok{el} \NormalTok{);}
    \NormalTok{\},}

    \CommentTok{// Add all items in the **Todos** collection at once.}
    \DataTypeTok{addAll}\NormalTok{: }\KeywordTok{function}\NormalTok{() \{}
      \KeywordTok{this}\NormalTok{.}\FunctionTok{$}\NormalTok{(}\StringTok{'#todo-list'}\NormalTok{).}\FunctionTok{html}\NormalTok{(}\StringTok{''}\NormalTok{);}
      \OtherTok{app}\NormalTok{.}\OtherTok{Todos}\NormalTok{.}\FunctionTok{each}\NormalTok{(}\KeywordTok{this}\NormalTok{.}\FunctionTok{addOne}\NormalTok{, }\KeywordTok{this}\NormalTok{);}
    \NormalTok{\},}

    \CommentTok{// New}
    \DataTypeTok{filterOne }\NormalTok{: }\KeywordTok{function} \NormalTok{(todo) \{}
      \OtherTok{todo}\NormalTok{.}\FunctionTok{trigger}\NormalTok{(}\StringTok{'visible'}\NormalTok{);}
    \NormalTok{\},}

    \CommentTok{// New}
    \DataTypeTok{filterAll }\NormalTok{: }\KeywordTok{function} \NormalTok{() \{}
      \OtherTok{app}\NormalTok{.}\OtherTok{Todos}\NormalTok{.}\FunctionTok{each}\NormalTok{(}\KeywordTok{this}\NormalTok{.}\FunctionTok{filterOne}\NormalTok{, }\KeywordTok{this}\NormalTok{);}
    \NormalTok{\},}


    \CommentTok{// New}
    \CommentTok{// Generate the attributes for a new Todo item.}
    \DataTypeTok{newAttributes}\NormalTok{: }\KeywordTok{function}\NormalTok{() \{}
      \KeywordTok{return} \NormalTok{\{}
        \DataTypeTok{title}\NormalTok{: }\KeywordTok{this}\NormalTok{.}\OtherTok{$input}\NormalTok{.}\FunctionTok{val}\NormalTok{().}\FunctionTok{trim}\NormalTok{(),}
        \DataTypeTok{order}\NormalTok{: }\OtherTok{app}\NormalTok{.}\OtherTok{Todos}\NormalTok{.}\FunctionTok{nextOrder}\NormalTok{(),}
        \DataTypeTok{completed}\NormalTok{: }\KeywordTok{false}
      \NormalTok{\};}
    \NormalTok{\},}

    \CommentTok{// New}
    \CommentTok{// If you hit return in the main input field, create new Todo model,}
    \CommentTok{// persisting it to localStorage.}
    \DataTypeTok{createOnEnter}\NormalTok{: }\KeywordTok{function}\NormalTok{( event ) \{}
      \KeywordTok{if} \NormalTok{( }\OtherTok{event}\NormalTok{.}\FunctionTok{which} \NormalTok{!== ENTER_KEY || !}\KeywordTok{this}\NormalTok{.}\OtherTok{$input}\NormalTok{.}\FunctionTok{val}\NormalTok{().}\FunctionTok{trim}\NormalTok{() ) \{}
        \KeywordTok{return}\NormalTok{;}
      \NormalTok{\}}

      \OtherTok{app}\NormalTok{.}\OtherTok{Todos}\NormalTok{.}\FunctionTok{create}\NormalTok{( }\KeywordTok{this}\NormalTok{.}\FunctionTok{newAttributes}\NormalTok{() );}
      \KeywordTok{this}\NormalTok{.}\OtherTok{$input}\NormalTok{.}\FunctionTok{val}\NormalTok{(}\StringTok{''}\NormalTok{);}
    \NormalTok{\},}

    \CommentTok{// New}
    \CommentTok{// Clear all completed todo items, destroying their models.}
    \DataTypeTok{clearCompleted}\NormalTok{: }\KeywordTok{function}\NormalTok{() \{}
      \OtherTok{_}\NormalTok{.}\FunctionTok{invoke}\NormalTok{(}\OtherTok{app}\NormalTok{.}\OtherTok{Todos}\NormalTok{.}\FunctionTok{completed}\NormalTok{(), }\StringTok{'destroy'}\NormalTok{);}
      \KeywordTok{return} \KeywordTok{false}\NormalTok{;}
    \NormalTok{\},}

    \CommentTok{// New}
    \DataTypeTok{toggleAllComplete}\NormalTok{: }\KeywordTok{function}\NormalTok{() \{}
      \KeywordTok{var} \NormalTok{completed = }\KeywordTok{this}\NormalTok{.}\OtherTok{allCheckbox}\NormalTok{.}\FunctionTok{checked}\NormalTok{;}

      \OtherTok{app}\NormalTok{.}\OtherTok{Todos}\NormalTok{.}\FunctionTok{each}\NormalTok{(}\KeywordTok{function}\NormalTok{( todo ) \{}
        \OtherTok{todo}\NormalTok{.}\FunctionTok{save}\NormalTok{(\{}
          \StringTok{'completed'}\NormalTok{: completed}
        \NormalTok{\});}
      \NormalTok{\});}
    \NormalTok{\}}
  \NormalTok{\});}
\end{Highlighting}
\end{Shaded}

We have added the logic for creating new todos, editing them, and
filtering them based on their completed status.

\begin{itemize}
\item
  events: We've defined an \texttt{events} hash containing declarative
  callbacks for our DOM events. It binds those events to the following
  methods:
\item
  \texttt{createOnEnter()}: Creates a new Todo model and persists it in
  localStorage when a user hits enter inside the
  \texttt{\textless{}input/\textgreater{}} field. Also resets the main
  \texttt{\textless{}input/\textgreater{}} field value to prepare it for
  the next entry. The model is populated by newAttributes(), which
  returns an object literal composed of the title, order, and completed
  state of the new item. Note that \texttt{this} is referring to the
  view and not the DOM element since the callback was bound using the
  \texttt{events} hash.
\item
  \texttt{clearCompleted()}: Removes the items in the todo list that
  have been marked as completed when the user clicks the clear-completed
  checkbox (this checkbox will be in the footer populated by the
  \#stats-template).
\item
  \texttt{toggleAllComplete()}: Allows a user to mark all of the items
  in the todo list as completed by clicking the toggle-all checkbox.
\item
  \texttt{initialize()}: We've bound callbacks to several additional
  events:
\item
  We've bound a \texttt{filterOne()} callback on the Todos collection
  for a \texttt{change:completed} event. This listens for changes to the
  completed flag for any model in the collection. The affected todo is
  passed to the callback which triggers a custom \texttt{visible} event
  on the model.
\item
  We've bound a \texttt{filterAll()} callback for a \texttt{filter}
  event, which works a little similar to addOne() and addAll(). Its
  responsibility is to toggle which todo items are visible based on the
  filter currently selected in the UI (all, completed or remaining) via
  calls to filterOne().
\item
  We've used the special \texttt{all} event to bind any event triggered
  on the Todos collection to the view's render method (discussed below).
\end{itemize}

The \texttt{initialize()} method completes by fetching the previously
saved todos from localStorage.

\begin{itemize}
\itemsep1pt\parskip0pt\parsep0pt
\item
  \texttt{render()}: Several things are happening in our
  \texttt{render()} method:
\item
  The \#main and \#footer sections are displayed or hidden depending on
  whether there are any todos in the collection.
\item
  The footer is populated with the HTML produced by instantiating the
  statsTemplate with the number of completed and remaining todo items.
\item
  The HTML produced by the preceding step contains a list of filter
  links. The value of \texttt{app.TodoFilter}, which will be set by our
  router, is being used to apply the class `selected' to the link
  corresponding to the currently selected filter. This will result in
  conditional CSS styling being applied to that filter.
\item
  The allCheckbox is updated based on whether there are remaining todos.
\end{itemize}

\subsection{Individual Todo View}\label{individual-todo-view}

Now let's look at the \texttt{TodoView} view. This will be in charge of
individual Todo records, making sure the view updates when the todo
does. To enable this functionality, we will add event listeners to the
view that listen for events on an individual todo's HTML representation.

\begin{Shaded}
\begin{Highlighting}[]

  \CommentTok{// js/views/todos.js}

  \KeywordTok{var} \NormalTok{app = app || \{\};}

  \CommentTok{// Todo Item View}
  \CommentTok{// --------------}

  \CommentTok{// The DOM element for a todo item...}
  \OtherTok{app}\NormalTok{.}\FunctionTok{TodoView} \NormalTok{= }\OtherTok{Backbone}\NormalTok{.}\OtherTok{View}\NormalTok{.}\FunctionTok{extend}\NormalTok{(\{}

    \CommentTok{//... is a list tag.}
    \DataTypeTok{tagName}\NormalTok{: }\StringTok{'li'}\NormalTok{,}

    \CommentTok{// Cache the template function for a single item.}
    \DataTypeTok{template}\NormalTok{: }\OtherTok{_}\NormalTok{.}\FunctionTok{template}\NormalTok{( }\FunctionTok{$}\NormalTok{(}\StringTok{'#item-template'}\NormalTok{).}\FunctionTok{html}\NormalTok{() ),}

    \CommentTok{// The DOM events specific to an item.}
    \DataTypeTok{events}\NormalTok{: \{}
      \StringTok{'dblclick label'}\NormalTok{: }\StringTok{'edit'}\NormalTok{,}
      \StringTok{'keypress .edit'}\NormalTok{: }\StringTok{'updateOnEnter'}\NormalTok{,}
      \StringTok{'blur .edit'}\NormalTok{: }\StringTok{'close'}
    \NormalTok{\},}

    \CommentTok{// The TodoView listens for changes to its model, re-rendering. Since there's}
    \CommentTok{// a one-to-one correspondence between a **Todo** and a **TodoView** in this}
    \CommentTok{// app, we set a direct reference on the model for convenience.}
    \DataTypeTok{initialize}\NormalTok{: }\KeywordTok{function}\NormalTok{() \{}
      \KeywordTok{this}\NormalTok{.}\FunctionTok{listenTo}\NormalTok{(}\KeywordTok{this}\NormalTok{.}\FunctionTok{model}\NormalTok{, }\StringTok{'change'}\NormalTok{, }\KeywordTok{this}\NormalTok{.}\FunctionTok{render}\NormalTok{);}
    \NormalTok{\},}

    \CommentTok{// Re-renders the titles of the todo item.}
    \DataTypeTok{render}\NormalTok{: }\KeywordTok{function}\NormalTok{() \{}
      \KeywordTok{this}\NormalTok{.}\OtherTok{$el}\NormalTok{.}\FunctionTok{html}\NormalTok{( }\KeywordTok{this}\NormalTok{.}\FunctionTok{template}\NormalTok{( }\KeywordTok{this}\NormalTok{.}\OtherTok{model}\NormalTok{.}\FunctionTok{attributes} \NormalTok{) );}
      \KeywordTok{this}\NormalTok{.}\FunctionTok{$input} \NormalTok{= }\KeywordTok{this}\NormalTok{.}\FunctionTok{$}\NormalTok{(}\StringTok{'.edit'}\NormalTok{);}
      \KeywordTok{return} \KeywordTok{this}\NormalTok{;}
    \NormalTok{\},}

    \CommentTok{// Switch this view into `"editing"` mode, displaying the input field.}
    \DataTypeTok{edit}\NormalTok{: }\KeywordTok{function}\NormalTok{() \{}
      \KeywordTok{this}\NormalTok{.}\OtherTok{$el}\NormalTok{.}\FunctionTok{addClass}\NormalTok{(}\StringTok{'editing'}\NormalTok{);}
      \KeywordTok{this}\NormalTok{.}\OtherTok{$input}\NormalTok{.}\FunctionTok{focus}\NormalTok{();}
    \NormalTok{\},}

    \CommentTok{// Close the `"editing"` mode, saving changes to the todo.}
    \DataTypeTok{close}\NormalTok{: }\KeywordTok{function}\NormalTok{() \{}
      \KeywordTok{var} \NormalTok{value = }\KeywordTok{this}\NormalTok{.}\OtherTok{$input}\NormalTok{.}\FunctionTok{val}\NormalTok{().}\FunctionTok{trim}\NormalTok{();}

      \KeywordTok{if} \NormalTok{( value ) \{}
        \KeywordTok{this}\NormalTok{.}\OtherTok{model}\NormalTok{.}\FunctionTok{save}\NormalTok{(\{ }\DataTypeTok{title}\NormalTok{: value \});}
      \NormalTok{\}}

      \KeywordTok{this}\NormalTok{.}\OtherTok{$el}\NormalTok{.}\FunctionTok{removeClass}\NormalTok{(}\StringTok{'editing'}\NormalTok{);}
    \NormalTok{\},}

    \CommentTok{// If you hit `enter`, we're through editing the item.}
    \DataTypeTok{updateOnEnter}\NormalTok{: }\KeywordTok{function}\NormalTok{( e ) \{}
      \KeywordTok{if} \NormalTok{( }\OtherTok{e}\NormalTok{.}\FunctionTok{which} \NormalTok{=== ENTER_KEY ) \{}
        \KeywordTok{this}\NormalTok{.}\FunctionTok{close}\NormalTok{();}
      \NormalTok{\}}
    \NormalTok{\}}
  \NormalTok{\});}
\end{Highlighting}
\end{Shaded}

In the \texttt{initialize()} constructor, we set up a listener that
monitors a todo model's \texttt{change} event. As a result, when the
todo gets updated, the application will re-render the view and visually
reflect its changes. Note that the model passed in the arguments hash by
our AppView is automatically available to us as \texttt{this.model}.

In the \texttt{render()} method, we render our Underscore.js
\texttt{\#item-template}, which was previously compiled into
this.template using Underscore's \texttt{\_.template()} method. This
returns an HTML fragment that replaces the content of the view's element
(an li element was implicitly created for us based on the
\texttt{tagName} property). In other words, the rendered template is now
present under \texttt{this.el} and can be appended to the todo list in
the user interface. \texttt{render()} finishes by caching the input
element within the instantiated template into \texttt{this.input}.

Our events hash includes three callbacks:

\begin{itemize}
\itemsep1pt\parskip0pt\parsep0pt
\item
  \texttt{edit()}: changes the current view into editing mode when a
  user double-clicks on an existing item in the todo list. This allows
  them to change the existing value of the item's title attribute.
\item
  \texttt{updateOnEnter()}: checks that the user has hit the
  return/enter key and executes the close() function.
\item
  \texttt{close()}: trims the value of the current text in our
  \texttt{\textless{}input/\textgreater{}} field, ensuring that we don't
  process it further if it does not contain any text (e.g `'). If a
  valid value has been provided, we save the changes to the current todo
  model and close editing mode by removing the corresponding CSS class.
\end{itemize}

\subsection{Startup}\label{startup}

So now we have two views: \texttt{AppView} and \texttt{TodoView}. The
former needs to be instantiated on page load so its code gets executed.
This can be accomplished through jQuery's \texttt{ready()} utility,
which will execute a function when the DOM is loaded.

\begin{Shaded}
\begin{Highlighting}[]

  \CommentTok{// js/app.js}

  \KeywordTok{var} \NormalTok{app = app || \{\};}
  \KeywordTok{var} \NormalTok{ENTER_KEY = }\DecValTok{13}\NormalTok{;}

  \FunctionTok{$}\NormalTok{(}\KeywordTok{function}\NormalTok{() \{}

    \CommentTok{// Kick things off by creating the **App**.}
    \KeywordTok{new} \OtherTok{app}\NormalTok{.}\FunctionTok{AppView}\NormalTok{();}

  \NormalTok{\});}
\end{Highlighting}
\end{Shaded}

\subsection{In action}\label{in-action}

Let's pause and ensure that the work we've done so far functions as
intended.

If you are following along, open \texttt{file://*path*/index.html} in
your browser and monitor its console. If all is well, you shouldn't see
any JavaScript errors other than regarding the router.js file that we
haven't created yet. The todo list should be blank as we haven't yet
created any todos. Plus, there is some additional work we'll need to do
before the user interface fully functions.

However, a few things can be tested through the JavaScript console.

In the console, add a new todo item:
\texttt{app.Todos.create(\{ title: 'My first Todo item'\});} and hit
return.

\begin{figure}[htbp]
\centering
\includegraphics{img/todos_d.png}
\end{figure}

If all is functioning properly, this should log the new todo we've just
added to the todos collection. The newly created todo is also saved to
Local Storage and will be available on page refresh.

\texttt{app.Todos.create()} executes a collection method
(\texttt{Collection.create(attributes, {[}options{]})}) which
instantiates a new model item of the type passed into the collection
definition, in our case \texttt{app.Todo}:

\begin{Shaded}
\begin{Highlighting}[]

  \CommentTok{// from our js/collections/todos.js}

  \KeywordTok{var} \NormalTok{TodoList = }\OtherTok{Backbone}\NormalTok{.}\OtherTok{Collection}\NormalTok{.}\FunctionTok{extend}\NormalTok{(\{}

      \DataTypeTok{model}\NormalTok{: }\OtherTok{app}\NormalTok{.}\FunctionTok{Todo} \CommentTok{// the model type used by collection.create() to instantiate new model in the collection}
      \NormalTok{...}
  \NormalTok{)\};}
\end{Highlighting}
\end{Shaded}

Run the following in the console to check it out:

\begin{Shaded}
\begin{Highlighting}[]
\KeywordTok{var} \NormalTok{secondTodo = }\OtherTok{app}\NormalTok{.}\OtherTok{Todos}\NormalTok{.}\FunctionTok{create}\NormalTok{(\{ }\DataTypeTok{title}\NormalTok{: }\StringTok{'My second Todo item'}\NormalTok{\});}
\NormalTok{secondTodo }\KeywordTok{instanceof} \OtherTok{app}\NormalTok{.}\FunctionTok{Todo} \CommentTok{// returns true}
\end{Highlighting}
\end{Shaded}

Now refresh the page and we should be able to see the fruits of our
labour.

The todos added through the console should still appear in the list
since they are populated from the Local Storage. Also, we should be able
to create a new todo by typing a title and pressing enter.

\begin{figure}[htbp]
\centering
\includegraphics{img/todos_b.png}
\end{figure}

Excellent, we're making great progress, but what about completing and
deleting todos?

\subsection{Completing \& deleting
todos}\label{completing-deleting-todos}

The next part of our tutorial is going to cover completing and deleting
todos. These two actions are specific to each Todo item, so we need to
add this functionality to the TodoView view. We will do so by adding
\texttt{togglecompleted()} and \texttt{clear()} methods along with
corresponding entries in the \texttt{events} hash.

\begin{Shaded}
\begin{Highlighting}[]

  \CommentTok{// js/views/todos.js}

  \KeywordTok{var} \NormalTok{app = app || \{\};}

  \CommentTok{// Todo Item View}
  \CommentTok{// --------------}

  \CommentTok{// The DOM element for a todo item...}
  \OtherTok{app}\NormalTok{.}\FunctionTok{TodoView} \NormalTok{= }\OtherTok{Backbone}\NormalTok{.}\OtherTok{View}\NormalTok{.}\FunctionTok{extend}\NormalTok{(\{}

    \CommentTok{//... is a list tag.}
    \DataTypeTok{tagName}\NormalTok{: }\StringTok{'li'}\NormalTok{,}

    \CommentTok{// Cache the template function for a single item.}
    \DataTypeTok{template}\NormalTok{: }\OtherTok{_}\NormalTok{.}\FunctionTok{template}\NormalTok{( }\FunctionTok{$}\NormalTok{(}\StringTok{'#item-template'}\NormalTok{).}\FunctionTok{html}\NormalTok{() ),}

    \CommentTok{// The DOM events specific to an item.}
    \DataTypeTok{events}\NormalTok{: \{}
      \StringTok{'click .toggle'}\NormalTok{: }\StringTok{'togglecompleted'}\NormalTok{, }\CommentTok{// NEW}
      \StringTok{'dblclick label'}\NormalTok{: }\StringTok{'edit'}\NormalTok{,}
      \StringTok{'click .destroy'}\NormalTok{: }\StringTok{'clear'}\NormalTok{,           }\CommentTok{// NEW}
      \StringTok{'keypress .edit'}\NormalTok{: }\StringTok{'updateOnEnter'}\NormalTok{,}
      \StringTok{'blur .edit'}\NormalTok{: }\StringTok{'close'}
    \NormalTok{\},}

    \CommentTok{// The TodoView listens for changes to its model, re-rendering. Since there's}
    \CommentTok{// a one-to-one correspondence between a **Todo** and a **TodoView** in this}
    \CommentTok{// app, we set a direct reference on the model for convenience.}
    \DataTypeTok{initialize}\NormalTok{: }\KeywordTok{function}\NormalTok{() \{}
      \KeywordTok{this}\NormalTok{.}\FunctionTok{listenTo}\NormalTok{(}\KeywordTok{this}\NormalTok{.}\FunctionTok{model}\NormalTok{, }\StringTok{'change'}\NormalTok{, }\KeywordTok{this}\NormalTok{.}\FunctionTok{render}\NormalTok{);}
      \KeywordTok{this}\NormalTok{.}\FunctionTok{listenTo}\NormalTok{(}\KeywordTok{this}\NormalTok{.}\FunctionTok{model}\NormalTok{, }\StringTok{'destroy'}\NormalTok{, }\KeywordTok{this}\NormalTok{.}\FunctionTok{remove}\NormalTok{);        }\CommentTok{// NEW}
      \KeywordTok{this}\NormalTok{.}\FunctionTok{listenTo}\NormalTok{(}\KeywordTok{this}\NormalTok{.}\FunctionTok{model}\NormalTok{, }\StringTok{'visible'}\NormalTok{, }\KeywordTok{this}\NormalTok{.}\FunctionTok{toggleVisible}\NormalTok{); }\CommentTok{// NEW}
    \NormalTok{\},}

    \CommentTok{// Re-render the titles of the todo item.}
    \DataTypeTok{render}\NormalTok{: }\KeywordTok{function}\NormalTok{() \{}
      \KeywordTok{this}\NormalTok{.}\OtherTok{$el}\NormalTok{.}\FunctionTok{html}\NormalTok{( }\KeywordTok{this}\NormalTok{.}\FunctionTok{template}\NormalTok{( }\KeywordTok{this}\NormalTok{.}\OtherTok{model}\NormalTok{.}\FunctionTok{attributes} \NormalTok{) );}

      \KeywordTok{this}\NormalTok{.}\OtherTok{$el}\NormalTok{.}\FunctionTok{toggleClass}\NormalTok{( }\StringTok{'completed'}\NormalTok{, }\KeywordTok{this}\NormalTok{.}\OtherTok{model}\NormalTok{.}\FunctionTok{get}\NormalTok{(}\StringTok{'completed'}\NormalTok{) ); }\CommentTok{// NEW}
      \KeywordTok{this}\NormalTok{.}\FunctionTok{toggleVisible}\NormalTok{();                                             }\CommentTok{// NEW}

      \KeywordTok{this}\NormalTok{.}\FunctionTok{$input} \NormalTok{= }\KeywordTok{this}\NormalTok{.}\FunctionTok{$}\NormalTok{(}\StringTok{'.edit'}\NormalTok{);}
      \KeywordTok{return} \KeywordTok{this}\NormalTok{;}
    \NormalTok{\},}

    \CommentTok{// NEW - Toggles visibility of item}
    \DataTypeTok{toggleVisible }\NormalTok{: }\KeywordTok{function} \NormalTok{() \{}
      \KeywordTok{this}\NormalTok{.}\OtherTok{$el}\NormalTok{.}\FunctionTok{toggleClass}\NormalTok{( }\StringTok{'hidden'}\NormalTok{,  }\KeywordTok{this}\NormalTok{.}\FunctionTok{isHidden}\NormalTok{());}
    \NormalTok{\},}

    \CommentTok{// NEW - Determines if item should be hidden}
    \DataTypeTok{isHidden }\NormalTok{: }\KeywordTok{function} \NormalTok{() \{}
      \KeywordTok{var} \NormalTok{isCompleted = }\KeywordTok{this}\NormalTok{.}\OtherTok{model}\NormalTok{.}\FunctionTok{get}\NormalTok{(}\StringTok{'completed'}\NormalTok{);}
      \KeywordTok{return} \NormalTok{( }\CommentTok{// hidden cases only}
        \NormalTok{(!isCompleted && }\OtherTok{app}\NormalTok{.}\FunctionTok{TodoFilter} \NormalTok{=== }\StringTok{'completed'}\NormalTok{)}
        \NormalTok{|| (isCompleted && }\OtherTok{app}\NormalTok{.}\FunctionTok{TodoFilter} \NormalTok{=== }\StringTok{'active'}\NormalTok{)}
      \NormalTok{);}
    \NormalTok{\},}

    \CommentTok{// NEW - Toggle the `"completed"` state of the model.}
    \DataTypeTok{togglecompleted}\NormalTok{: }\KeywordTok{function}\NormalTok{() \{}
      \KeywordTok{this}\NormalTok{.}\OtherTok{model}\NormalTok{.}\FunctionTok{toggle}\NormalTok{();}
    \NormalTok{\},}

    \CommentTok{// Switch this view into `"editing"` mode, displaying the input field.}
    \DataTypeTok{edit}\NormalTok{: }\KeywordTok{function}\NormalTok{() \{}
      \KeywordTok{this}\NormalTok{.}\OtherTok{$el}\NormalTok{.}\FunctionTok{addClass}\NormalTok{(}\StringTok{'editing'}\NormalTok{);}
      \KeywordTok{this}\NormalTok{.}\OtherTok{$input}\NormalTok{.}\FunctionTok{focus}\NormalTok{();}
    \NormalTok{\},}

    \CommentTok{// Close the `"editing"` mode, saving changes to the todo.}
    \DataTypeTok{close}\NormalTok{: }\KeywordTok{function}\NormalTok{() \{}
      \KeywordTok{var} \NormalTok{value = }\KeywordTok{this}\NormalTok{.}\OtherTok{$input}\NormalTok{.}\FunctionTok{val}\NormalTok{().}\FunctionTok{trim}\NormalTok{();}

      \KeywordTok{if} \NormalTok{( value ) \{}
        \KeywordTok{this}\NormalTok{.}\OtherTok{model}\NormalTok{.}\FunctionTok{save}\NormalTok{(\{ }\DataTypeTok{title}\NormalTok{: value \});}
      \NormalTok{\} }\KeywordTok{else} \NormalTok{\{}
        \KeywordTok{this}\NormalTok{.}\FunctionTok{clear}\NormalTok{(); }\CommentTok{// NEW}
      \NormalTok{\}}

      \KeywordTok{this}\NormalTok{.}\OtherTok{$el}\NormalTok{.}\FunctionTok{removeClass}\NormalTok{(}\StringTok{'editing'}\NormalTok{);}
    \NormalTok{\},}

    \CommentTok{// If you hit `enter`, we're through editing the item.}
    \DataTypeTok{updateOnEnter}\NormalTok{: }\KeywordTok{function}\NormalTok{( e ) \{}
      \KeywordTok{if} \NormalTok{( }\OtherTok{e}\NormalTok{.}\FunctionTok{which} \NormalTok{=== ENTER_KEY ) \{}
        \KeywordTok{this}\NormalTok{.}\FunctionTok{close}\NormalTok{();}
      \NormalTok{\}}
    \NormalTok{\},}

    \CommentTok{// NEW - Remove the item, destroy the model from *localStorage* and delete its view.}
    \DataTypeTok{clear}\NormalTok{: }\KeywordTok{function}\NormalTok{() \{}
      \KeywordTok{this}\NormalTok{.}\OtherTok{model}\NormalTok{.}\FunctionTok{destroy}\NormalTok{();}
    \NormalTok{\}}
  \NormalTok{\});}
\end{Highlighting}
\end{Shaded}

The key part of this is the two event handlers we've added, a
togglecompleted event on the todo's checkbox, and a click event on the
todo's \texttt{\textless{}button class="destroy" /\textgreater{}}
button.

Let's look at the events that occur when we click the checkbox for a
todo item:

\begin{enumerate}
\def\labelenumi{\arabic{enumi}.}
\itemsep1pt\parskip0pt\parsep0pt
\item
  The \texttt{togglecompleted()} function is invoked which calls
  \texttt{toggle()} on the todo model.
\item
  \texttt{toggle()} toggles the completed status of the todo and calls
  \texttt{save()} on the model.
\item
  The save generates a \texttt{change} event on the model which is bound
  to our TodoView's \texttt{render()} method. We've added a statement in
  \texttt{render()} which toggles the completed class on the element
  depending on the model's completed state. The associated CSS changes
  the color of the title text and strikes a line through it when the
  todo is completed.
\item
  The save also results in a \texttt{change:completed} event on the
  model which is handled by the AppView's \texttt{filterOne()} method.
  If we look back at the AppView, we see that filterOne() will trigger a
  \texttt{visible} event on the model. This is used in conjunction with
  the filtering in our routes and collections so that we only display an
  item if its completed state falls in line with the current filter. In
  our update to the TodoView, we bound the model's visible event to the
  \texttt{toggleVisible()} method. This method uses the new
  \texttt{isHidden()} method to determine if the todo should be visible
  and updates it accordingly.
\end{enumerate}

Now let's look at what happens when we click on a todo's destroy button:

\begin{enumerate}
\def\labelenumi{\arabic{enumi}.}
\itemsep1pt\parskip0pt\parsep0pt
\item
  The \texttt{clear()} method is invoked which calls \texttt{destroy()}
  on the todo model.
\item
  The todo is deleted from local storage and a \texttt{destroy} event is
  triggered.
\item
  In our update to the TodoView, we bound the model's \texttt{destroy}
  event to the view's inherited \texttt{remove()} method. This method
  deletes the view and automatically removes the associated element from
  the DOM. Since we used \texttt{listenTo()} to bind the view's
  listeners to its model, \texttt{remove()} also unbinds the listening
  callbacks from the model ensuring that a memory leak does not occur.
\item
  \texttt{destroy()} also removes the model from the Todos collection,
  which triggers a \texttt{remove} event on the collection.
\item
  Since the AppView has its \texttt{render()} method bound to
  \texttt{all} events on the Todos collection, that view is rendered and
  the stats in the footer are updated.
\end{enumerate}

That's all there is to it!

If you want to see an example of those, see the
\href{https://github.com/tastejs/todomvc/tree/gh-pages/architecture-examples/backbone}{complete
source}.

\subsection{Todo routing}\label{todo-routing}

Finally, we move on to routing, which will allow us to easily filter the
list of items that are active as well as those which have been
completed. We'll be supporting the following routes:

\begin{verbatim}
#/ (all - default)
#/active
#/completed
\end{verbatim}

\begin{figure}[htbp]
\centering
\includegraphics{img/todos_e.png}
\end{figure}

When the route changes, the todo list will be filtered on a model level
and the selected class on the filter links in the footer will be toggled
as described above. When an item is updated while a filter is active it
will be updated accordingly (e.g., if the filter is active and the item
is checked, it will be hidden). The active filter is persisted on
reload.

\begin{Shaded}
\begin{Highlighting}[]

  \CommentTok{// js/routers/router.js}

  \CommentTok{// Todo Router}
  \CommentTok{// ----------}

  \KeywordTok{var} \NormalTok{Workspace = }\OtherTok{Backbone}\NormalTok{.}\OtherTok{Router}\NormalTok{.}\FunctionTok{extend}\NormalTok{(\{}
    \DataTypeTok{routes}\NormalTok{:\{}
      \StringTok{'*filter'}\NormalTok{: }\StringTok{'setFilter'}
    \NormalTok{\},}

    \DataTypeTok{setFilter}\NormalTok{: }\KeywordTok{function}\NormalTok{( param ) \{}
      \CommentTok{// Set the current filter to be used}
      \KeywordTok{if} \NormalTok{(param) \{}
        \NormalTok{param = }\OtherTok{param}\NormalTok{.}\FunctionTok{trim}\NormalTok{();}
      \NormalTok{\}}
      \OtherTok{app}\NormalTok{.}\FunctionTok{TodoFilter} \NormalTok{= param || }\StringTok{''}\NormalTok{;}

      \CommentTok{// Trigger a collection filter event, causing hiding/unhiding}
      \CommentTok{// of Todo view items}
      \OtherTok{app}\NormalTok{.}\OtherTok{Todos}\NormalTok{.}\FunctionTok{trigger}\NormalTok{(}\StringTok{'filter'}\NormalTok{);}
    \NormalTok{\}}
  \NormalTok{\});}

  \OtherTok{app}\NormalTok{.}\FunctionTok{TodoRouter} \NormalTok{= }\KeywordTok{new} \FunctionTok{Workspace}\NormalTok{();}
  \OtherTok{Backbone}\NormalTok{.}\OtherTok{history}\NormalTok{.}\FunctionTok{start}\NormalTok{();}
\end{Highlighting}
\end{Shaded}

Our router uses a *splat to set up a default route which passes the
string after `\#/' in the URL to \texttt{setFilter()} which sets
\texttt{app.TodoFilter} to that string.

As we can see in the line \texttt{app.Todos.trigger('filter')}, once the
filter has been set, we simply trigger `filter' on our Todos collection
to toggle which items are visible and which are hidden. Recall that our
AppView's \texttt{filterAll()} method is bound to the collection's
filter event and that any event on the collection will cause the AppView
to re-render.

Finally, we create an instance of our router and call
\texttt{Backbone.history.start()} to route the initial URL during page
load.

\subsection{Summary}\label{summary-2}

We've now built our first complete Backbone.js application. The latest
version of the full app can be viewed online at any time and the sources
are readily available via \href{http://www.todomvc.com}{TodoMVC}.

Later on in the book, we'll learn how to further modularize this
application using RequireJS, swap out our persistence layer to a
database back-end, and finally unit test the application with a few
different testing frameworks.

\section{Exercise 2: Book Library - Your First RESTful Backbone.js
App}\label{exercise-2-book-library---your-first-restful-backbone.js-app}

While our first application gave us a good taste of how Backbone.js
applications are made, most real-world applications will want to
communicate with a back-end of some sort. Let's reinforce what we have
already learned with another example, but this time we will also create
a RESTful API for our application to talk to.

In this exercise we will build a library application for managing
digital books using Backbone. For each book we will store the title,
author, date of release, and some keywords. We'll also show a picture of
the cover.

\subsection{Setting up}\label{setting-up}

First we need to create a folder structure for our project. To keep the
front-end and back-end separate, we will create a folder called
\emph{site} for our client in the project root. Within it we will create
css, img, and js directories.

As with the last example we will split our JavaScript files by their
function, so under the js directory create folders named lib, models,
collections, and views. Your directory hierarchy should look like this:

\begin{verbatim}
site/
    css/
    img/
    js/
        collections/
        lib/
        models/
        views/
\end{verbatim}

Download the Backbone, Underscore, and jQuery libraries and copy them to
your js/lib folder. We need a placeholder image for the book covers.
Save this image to your site/img folder:

\begin{figure}[htbp]
\centering
\includegraphics{img/placeholder.png}
\end{figure}

Just like before we need to load all of our dependencies in the
site/index.html file:

\begin{Shaded}
\begin{Highlighting}[]
\DataTypeTok{<!DOCTYPE }\NormalTok{html}\DataTypeTok{>}
\KeywordTok{<html}\OtherTok{ lang=}\StringTok{"en"}\KeywordTok{>}
    \KeywordTok{<head>}
        \KeywordTok{<meta}\OtherTok{ charset=}\StringTok{"UTF-8"}\KeywordTok{/>}
        \KeywordTok{<title>}\NormalTok{Backbone.js Library}\KeywordTok{</title>}
        \KeywordTok{<link}\OtherTok{ rel=}\StringTok{"stylesheet"}\OtherTok{ href=}\StringTok{"css/screen.css"}\KeywordTok{>}
    \KeywordTok{</head>}
    \KeywordTok{<body>}
        \KeywordTok{<script}\OtherTok{ src=}\StringTok{"js/lib/jquery.min.js"}\KeywordTok{></script>}
        \KeywordTok{<script}\OtherTok{ src=}\StringTok{"js/lib/underscore-min.js"}\KeywordTok{></script>}
        \KeywordTok{<script}\OtherTok{ src=}\StringTok{"js/lib/backbone-min.js"}\KeywordTok{></script>}
        \KeywordTok{<script}\OtherTok{ src=}\StringTok{"js/models/book.js"}\KeywordTok{></script>}
        \KeywordTok{<script}\OtherTok{ src=}\StringTok{"js/collections/library.js"}\KeywordTok{></script>}
        \KeywordTok{<script}\OtherTok{ src=}\StringTok{"js/views/book.js"}\KeywordTok{></script>}
        \KeywordTok{<script}\OtherTok{ src=}\StringTok{"js/views/library.js"}\KeywordTok{></script>}
        \KeywordTok{<script}\OtherTok{ src=}\StringTok{"js/app.js"}\KeywordTok{></script>}
    \KeywordTok{</body>}
\KeywordTok{</html>}
\end{Highlighting}
\end{Shaded}

We should also add in the HTML for the user interface. We'll want a form
for adding a new book so add the following immediately inside the
\texttt{body} element:

\begin{Shaded}
\begin{Highlighting}[]
\KeywordTok{<div}\OtherTok{ id=}\StringTok{"books"}\KeywordTok{>}
    \KeywordTok{<form}\OtherTok{ id=}\StringTok{"addBook"}\OtherTok{ action=}\StringTok{"#"}\KeywordTok{>}
        \KeywordTok{<div>}
            \KeywordTok{<label}\OtherTok{ for=}\StringTok{"coverImage"}\KeywordTok{>}\NormalTok{CoverImage: }\KeywordTok{</label><input}\OtherTok{ id=}\StringTok{"coverImage"}\OtherTok{ type=}\StringTok{"file"} \KeywordTok{/>}
            \KeywordTok{<label}\OtherTok{ for=}\StringTok{"title"}\KeywordTok{>}\NormalTok{Title: }\KeywordTok{</label><input}\OtherTok{ id=}\StringTok{"title"}\OtherTok{ type=}\StringTok{"text"} \KeywordTok{/>}
            \KeywordTok{<label}\OtherTok{ for=}\StringTok{"author"}\KeywordTok{>}\NormalTok{Author: }\KeywordTok{</label><input}\OtherTok{ id=}\StringTok{"author"}\OtherTok{ type=}\StringTok{"text"} \KeywordTok{/>}
            \KeywordTok{<label}\OtherTok{ for=}\StringTok{"releaseDate"}\KeywordTok{>}\NormalTok{Release date: }\KeywordTok{</label><input}\OtherTok{ id=}\StringTok{"releaseDate"}\OtherTok{ type=}\StringTok{"text"} \KeywordTok{/>}
            \KeywordTok{<label}\OtherTok{ for=}\StringTok{"keywords"}\KeywordTok{>}\NormalTok{Keywords: }\KeywordTok{</label><input}\OtherTok{ id=}\StringTok{"keywords"}\OtherTok{ type=}\StringTok{"text"} \KeywordTok{/>}
            \KeywordTok{<button}\OtherTok{ id=}\StringTok{"add"}\KeywordTok{>}\NormalTok{Add}\KeywordTok{</button>}
        \KeywordTok{</div>}
    \KeywordTok{</form>}
\KeywordTok{</div>}
\end{Highlighting}
\end{Shaded}

and we'll need a template for displaying each book which should be
placed before the \texttt{script} tags:

\begin{Shaded}
\begin{Highlighting}[]
\KeywordTok{<script}\OtherTok{ id=}\StringTok{"bookTemplate"}\OtherTok{ type=}\StringTok{"text/template"}\KeywordTok{>}
    \NormalTok{<img src=}\StringTok{"<%= coverImage %>"}\OtherTok{/>}
\OtherTok{    <ul>}
\OtherTok{        <li><%= title %></li}\NormalTok{>}
        \NormalTok{<li><%= author %><}\OtherTok{/li>}
\OtherTok{        <li><%= releaseDate %></li}\NormalTok{>}
        \NormalTok{<li><%= keywords %><}\OtherTok{/li>}
\OtherTok{    </ul}\NormalTok{>}

    \NormalTok{<button }\KeywordTok{class}\NormalTok{=}\StringTok{"delete"}\NormalTok{>Delete<}\OtherTok{/button>}
\OtherTok{</script}\NormalTok{>}
\end{Highlighting}
\end{Shaded}

To see what this will look like with some data in it, go ahead and add a
manually filled-in book to the \emph{books} div.

\begin{Shaded}
\begin{Highlighting}[]
\KeywordTok{<div}\OtherTok{ class=}\StringTok{"bookContainer"}\KeywordTok{>}
    \KeywordTok{<img}\OtherTok{ src=}\StringTok{"img/placeholder.png"}\KeywordTok{/>}
    \KeywordTok{<ul>}
        \KeywordTok{<li>}\NormalTok{Title}\KeywordTok{</li>}
        \KeywordTok{<li>}\NormalTok{Author}\KeywordTok{</li>}
        \KeywordTok{<li>}\NormalTok{Release Date}\KeywordTok{</li>}
        \KeywordTok{<li>}\NormalTok{Keywords}\KeywordTok{</li>}
    \KeywordTok{</ul>}

    \KeywordTok{<button}\OtherTok{ class=}\StringTok{"delete"}\KeywordTok{>}\NormalTok{Delete}\KeywordTok{</button>}
\KeywordTok{</div>}
\end{Highlighting}
\end{Shaded}

Open this file in a browser and it should look something like this:

\begin{figure}[htbp]
\centering
\includegraphics{img/chapter5-1.png}
\end{figure}

Not so great. This is not a CSS tutorial, but we still need to do some
formatting. Create a file named screen.css in your site/css folder:

\begin{Shaded}
\begin{Highlighting}[]
\NormalTok{body }\KeywordTok{\{}
    \KeywordTok{background-color:} \DataTypeTok{#eee}\KeywordTok{;}
\KeywordTok{\}}

\FloatTok{.bookContainer} \KeywordTok{\{}
    \KeywordTok{outline:} \DataTypeTok{1px} \DataTypeTok{solid} \DataTypeTok{#aaa}\KeywordTok{;}
    \KeywordTok{width:} \DataTypeTok{350px}\KeywordTok{;}
    \KeywordTok{height:} \DataTypeTok{130px}\KeywordTok{;}
    \KeywordTok{background-color:} \DataTypeTok{#fff}\KeywordTok{;}
    \KeywordTok{float:} \DataTypeTok{left}\KeywordTok{;}
    \KeywordTok{margin:} \DataTypeTok{5px}\KeywordTok{;}
\KeywordTok{\}}

\FloatTok{.bookContainer} \NormalTok{img }\KeywordTok{\{}
    \KeywordTok{float:} \DataTypeTok{left}\KeywordTok{;}
    \KeywordTok{margin:} \DataTypeTok{10px}\KeywordTok{;}
\KeywordTok{\}}

\FloatTok{.bookContainer} \NormalTok{ul }\KeywordTok{\{}
    \KeywordTok{list-style-type:} \DataTypeTok{none}\KeywordTok{;}
    \KeywordTok{margin-bottom:} \DataTypeTok{0}\KeywordTok{;}
\KeywordTok{\}}

\FloatTok{.bookContainer} \NormalTok{button }\KeywordTok{\{}
    \KeywordTok{float:} \DataTypeTok{right}\KeywordTok{;}
    \KeywordTok{margin:} \DataTypeTok{10px}\KeywordTok{;}
\KeywordTok{\}}

\FloatTok{#addBook} \NormalTok{label }\KeywordTok{\{}
    \KeywordTok{width:} \DataTypeTok{100px}\KeywordTok{;}
    \KeywordTok{margin-right:} \DataTypeTok{10px}\KeywordTok{;}
    \KeywordTok{text-align:} \DataTypeTok{right}\KeywordTok{;}
    \KeywordTok{line-height:} \DataTypeTok{25px}\KeywordTok{;}
\KeywordTok{\}}

\FloatTok{#addBook} \NormalTok{label, }\FloatTok{#addBook} \NormalTok{input }\KeywordTok{\{}
    \KeywordTok{display:} \DataTypeTok{block}\KeywordTok{;}
    \KeywordTok{margin-bottom:} \DataTypeTok{10px}\KeywordTok{;}
    \KeywordTok{float:} \DataTypeTok{left}\KeywordTok{;}
\KeywordTok{\}}

\FloatTok{#addBook} \NormalTok{label}\CharTok{[for=}\StringTok{"title"}\CharTok{]}\NormalTok{, }\FloatTok{#addBook} \NormalTok{label}\CharTok{[for=}\StringTok{"releaseDate"}\CharTok{]} \KeywordTok{\{}
    \KeywordTok{clear:} \DataTypeTok{both}\KeywordTok{;}
\KeywordTok{\}}

\FloatTok{#addBook} \NormalTok{button }\KeywordTok{\{}
    \KeywordTok{display:} \DataTypeTok{block}\KeywordTok{;}
    \KeywordTok{margin:} \DataTypeTok{5px} \DataTypeTok{20px} \DataTypeTok{10px} \DataTypeTok{10px}\KeywordTok{;}
    \KeywordTok{float:} \DataTypeTok{right}\KeywordTok{;}
    \KeywordTok{clear:} \DataTypeTok{both}\KeywordTok{;}
\KeywordTok{\}}

\FloatTok{#addBook} \NormalTok{div }\KeywordTok{\{}
    \KeywordTok{width:} \DataTypeTok{550px}\KeywordTok{;}
\KeywordTok{\}}

\FloatTok{#addBook} \NormalTok{div}\DecValTok{:after} \KeywordTok{\{}
    \KeywordTok{content:} \StringTok{""}\KeywordTok{;}
    \KeywordTok{display:} \DataTypeTok{block}\KeywordTok{;}
    \KeywordTok{height:} \DataTypeTok{0}\KeywordTok{;}
    \KeywordTok{visibility:} \DataTypeTok{hidden}\KeywordTok{;}
    \KeywordTok{clear:} \DataTypeTok{both}\KeywordTok{;}
    \KeywordTok{font-size:} \DataTypeTok{0}\KeywordTok{;}
    \KeywordTok{line-height:} \DataTypeTok{0}\KeywordTok{;}
\KeywordTok{\}}
\end{Highlighting}
\end{Shaded}

Now it looks a bit better:

\begin{figure}[htbp]
\centering
\includegraphics{img/chapter5-2.png}
\end{figure}

So this is what we want the final result to look like, but with more
books. Go ahead and copy the bookContainer div a few more times if you
would like to see what it looks like. Now we are ready to start
developing the actual application.

\paragraph{Creating the Model, Collection, Views, and
App}\label{creating-the-model-collection-views-and-app}

First, we'll need a model of a book and a collection to hold the list.
These are both very simple, with the model only declaring some defaults:

\begin{Shaded}
\begin{Highlighting}[]
\CommentTok{// site/js/models/book.js}

\KeywordTok{var} \NormalTok{app = app || \{\};}

\OtherTok{app}\NormalTok{.}\FunctionTok{Book} \NormalTok{= }\OtherTok{Backbone}\NormalTok{.}\OtherTok{Model}\NormalTok{.}\FunctionTok{extend}\NormalTok{(\{}
    \DataTypeTok{defaults}\NormalTok{: \{}
        \DataTypeTok{coverImage}\NormalTok{: }\StringTok{'img/placeholder.png'}\NormalTok{,}
        \DataTypeTok{title}\NormalTok{: }\StringTok{'No title'}\NormalTok{,}
        \DataTypeTok{author}\NormalTok{: }\StringTok{'Unknown'}\NormalTok{,}
        \DataTypeTok{releaseDate}\NormalTok{: }\StringTok{'Unknown'}\NormalTok{,}
        \DataTypeTok{keywords}\NormalTok{: }\StringTok{'None'}
    \NormalTok{\}}
\NormalTok{\});}
\end{Highlighting}
\end{Shaded}

\begin{Shaded}
\begin{Highlighting}[]
\CommentTok{// site/js/collections/library.js}

\KeywordTok{var} \NormalTok{app = app || \{\};}

\OtherTok{app}\NormalTok{.}\FunctionTok{Library} \NormalTok{= }\OtherTok{Backbone}\NormalTok{.}\OtherTok{Collection}\NormalTok{.}\FunctionTok{extend}\NormalTok{(\{}
    \DataTypeTok{model}\NormalTok{: }\OtherTok{app}\NormalTok{.}\FunctionTok{Book}
\NormalTok{\});}
\end{Highlighting}
\end{Shaded}

Next, in order to display books we'll need a view:

\begin{Shaded}
\begin{Highlighting}[]
\CommentTok{// site/js/views/book.js}

\KeywordTok{var} \NormalTok{app = app || \{\};}

\OtherTok{app}\NormalTok{.}\FunctionTok{BookView} \NormalTok{= }\OtherTok{Backbone}\NormalTok{.}\OtherTok{View}\NormalTok{.}\FunctionTok{extend}\NormalTok{(\{}
    \DataTypeTok{tagName}\NormalTok{: }\StringTok{'div'}\NormalTok{,}
    \DataTypeTok{className}\NormalTok{: }\StringTok{'bookContainer'}\NormalTok{,}
    \DataTypeTok{template}\NormalTok{: }\OtherTok{_}\NormalTok{.}\FunctionTok{template}\NormalTok{( }\FunctionTok{$}\NormalTok{( }\StringTok{'#bookTemplate'} \NormalTok{).}\FunctionTok{html}\NormalTok{() ),}

    \DataTypeTok{render}\NormalTok{: }\KeywordTok{function}\NormalTok{() \{}
        \CommentTok{//this.el is what we defined in tagName. use $el to get access to jQuery html() function}
        \KeywordTok{this}\NormalTok{.}\OtherTok{$el}\NormalTok{.}\FunctionTok{html}\NormalTok{( }\KeywordTok{this}\NormalTok{.}\FunctionTok{template}\NormalTok{( }\KeywordTok{this}\NormalTok{.}\OtherTok{model}\NormalTok{.}\FunctionTok{attributes} \NormalTok{) );}

        \KeywordTok{return} \KeywordTok{this}\NormalTok{;}
    \NormalTok{\}}
\NormalTok{\});}
\end{Highlighting}
\end{Shaded}

We'll also need a view for the list itself:

\begin{Shaded}
\begin{Highlighting}[]
\CommentTok{// site/js/views/library.js}

\KeywordTok{var} \NormalTok{app = app || \{\};}

\OtherTok{app}\NormalTok{.}\FunctionTok{LibraryView} \NormalTok{= }\OtherTok{Backbone}\NormalTok{.}\OtherTok{View}\NormalTok{.}\FunctionTok{extend}\NormalTok{(\{}
    \DataTypeTok{el}\NormalTok{: }\StringTok{'#books'}\NormalTok{,}

    \DataTypeTok{initialize}\NormalTok{: }\KeywordTok{function}\NormalTok{( initialBooks ) \{}
        \KeywordTok{this}\NormalTok{.}\FunctionTok{collection} \NormalTok{= }\KeywordTok{new} \OtherTok{app}\NormalTok{.}\FunctionTok{Library}\NormalTok{( initialBooks );}
        \KeywordTok{this}\NormalTok{.}\FunctionTok{render}\NormalTok{();}
    \NormalTok{\},}

    \CommentTok{// render library by rendering each book in its collection}
    \DataTypeTok{render}\NormalTok{: }\KeywordTok{function}\NormalTok{() \{}
        \KeywordTok{this}\NormalTok{.}\OtherTok{collection}\NormalTok{.}\FunctionTok{each}\NormalTok{(}\KeywordTok{function}\NormalTok{( item ) \{}
            \KeywordTok{this}\NormalTok{.}\FunctionTok{renderBook}\NormalTok{( item );}
        \NormalTok{\}, }\KeywordTok{this} \NormalTok{);}
    \NormalTok{\},}

    \CommentTok{// render a book by creating a BookView and appending the}
    \CommentTok{// element it renders to the library's element}
    \DataTypeTok{renderBook}\NormalTok{: }\KeywordTok{function}\NormalTok{( item ) \{}
        \KeywordTok{var} \NormalTok{bookView = }\KeywordTok{new} \OtherTok{app}\NormalTok{.}\FunctionTok{BookView}\NormalTok{(\{}
            \DataTypeTok{model}\NormalTok{: item}
        \NormalTok{\});}
        \KeywordTok{this}\NormalTok{.}\OtherTok{$el}\NormalTok{.}\FunctionTok{append}\NormalTok{( }\OtherTok{bookView}\NormalTok{.}\FunctionTok{render}\NormalTok{().}\FunctionTok{el} \NormalTok{);}
    \NormalTok{\}}
\NormalTok{\});}
\end{Highlighting}
\end{Shaded}

Note that in the initialize function we accept an array of data that we
pass to the app.Library constructor. We'll use this to populate our
collection with some sample data so that we can see everything is
working correctly. Finally, we have the entry point for our code, along
with the sample data:

\begin{Shaded}
\begin{Highlighting}[]
\CommentTok{// site/js/app.js}

\KeywordTok{var} \NormalTok{app = app || \{\};}

\FunctionTok{$}\NormalTok{(}\KeywordTok{function}\NormalTok{() \{}
    \KeywordTok{var} \NormalTok{books = [}
        \NormalTok{\{ }\DataTypeTok{title}\NormalTok{: }\StringTok{'JavaScript: The Good Parts'}\NormalTok{, }\DataTypeTok{author}\NormalTok{: }\StringTok{'Douglas Crockford'}\NormalTok{, }\DataTypeTok{releaseDate}\NormalTok{: }\StringTok{'2008'}\NormalTok{, }\DataTypeTok{keywords}\NormalTok{: }\StringTok{'JavaScript Programming'} \NormalTok{\},}
        \NormalTok{\{ }\DataTypeTok{title}\NormalTok{: }\StringTok{'The Little Book on CoffeeScript'}\NormalTok{, }\DataTypeTok{author}\NormalTok{: }\StringTok{'Alex MacCaw'}\NormalTok{, }\DataTypeTok{releaseDate}\NormalTok{: }\StringTok{'2012'}\NormalTok{, }\DataTypeTok{keywords}\NormalTok{: }\StringTok{'CoffeeScript Programming'} \NormalTok{\},}
        \NormalTok{\{ }\DataTypeTok{title}\NormalTok{: }\StringTok{'Scala for the Impatient'}\NormalTok{, }\DataTypeTok{author}\NormalTok{: }\StringTok{'Cay S. Horstmann'}\NormalTok{, }\DataTypeTok{releaseDate}\NormalTok{: }\StringTok{'2012'}\NormalTok{, }\DataTypeTok{keywords}\NormalTok{: }\StringTok{'Scala Programming'} \NormalTok{\},}
        \NormalTok{\{ }\DataTypeTok{title}\NormalTok{: }\StringTok{'American Psycho'}\NormalTok{, }\DataTypeTok{author}\NormalTok{: }\StringTok{'Bret Easton Ellis'}\NormalTok{, }\DataTypeTok{releaseDate}\NormalTok{: }\StringTok{'1991'}\NormalTok{, }\DataTypeTok{keywords}\NormalTok{: }\StringTok{'Novel Splatter'} \NormalTok{\},}
        \NormalTok{\{ }\DataTypeTok{title}\NormalTok{: }\StringTok{'Eloquent JavaScript'}\NormalTok{, }\DataTypeTok{author}\NormalTok{: }\StringTok{'Marijn Haverbeke'}\NormalTok{, }\DataTypeTok{releaseDate}\NormalTok{: }\StringTok{'2011'}\NormalTok{, }\DataTypeTok{keywords}\NormalTok{: }\StringTok{'JavaScript Programming'} \NormalTok{\}}
    \NormalTok{];}

    \KeywordTok{new} \OtherTok{app}\NormalTok{.}\FunctionTok{LibraryView}\NormalTok{( books );}
\NormalTok{\});}
\end{Highlighting}
\end{Shaded}

Our app just passes the sample data to a new instance of app.LibraryView
that it creates. Since the \texttt{initialize()} constructor in
LibraryView invokes the view's \texttt{render()} method, all the books
in the library will be displayed. Since we are passing our entry point
as a callback to jQuery (in the form of its \$ alias), the function will
execute when the DOM is ready.

If you view index.html in a browser you should see something like this:

\begin{figure}[htbp]
\centering
\includegraphics{img/chapter5-3.png}
\end{figure}

This is a complete Backbone application, though it doesn't yet do
anything interesting.

\subsection{Wiring in the interface}\label{wiring-in-the-interface}

Now we'll add some functionality to the useless form at the top and the
delete buttons on each book.

\subsubsection{Adding models}\label{adding-models}

When the user clicks the add button we want to take the data in the form
and use it to create a new model. In the LibraryView we need to add an
event handler for the click event:

\begin{Shaded}
\begin{Highlighting}[]
\NormalTok{events:\{}
    \StringTok{'click #add'}\NormalTok{:}\StringTok{'addBook'}
\NormalTok{\},}

\NormalTok{addBook: }\KeywordTok{function}\NormalTok{( e ) \{}
    \OtherTok{e}\NormalTok{.}\FunctionTok{preventDefault}\NormalTok{();}

    \KeywordTok{var} \NormalTok{formData = \{\};}

    \FunctionTok{$}\NormalTok{( }\StringTok{'#addBook div'} \NormalTok{).}\FunctionTok{children}\NormalTok{( }\StringTok{'input'} \NormalTok{).}\FunctionTok{each}\NormalTok{( }\KeywordTok{function}\NormalTok{( i, el ) \{}
        \KeywordTok{if}\NormalTok{( }\FunctionTok{$}\NormalTok{( el ).}\FunctionTok{val}\NormalTok{() != }\StringTok{''} \NormalTok{)}
        \NormalTok{\{}
            \NormalTok{formData[ }\OtherTok{el}\NormalTok{.}\FunctionTok{id} \NormalTok{] = }\FunctionTok{$}\NormalTok{( el ).}\FunctionTok{val}\NormalTok{();}
        \NormalTok{\}}
    \NormalTok{\});}

    \KeywordTok{this}\NormalTok{.}\OtherTok{collection}\NormalTok{.}\FunctionTok{add}\NormalTok{( }\KeywordTok{new} \OtherTok{app}\NormalTok{.}\FunctionTok{Book}\NormalTok{( formData ) );}
\NormalTok{\},}
\end{Highlighting}
\end{Shaded}

We select all the input elements of the form that have a value and
iterate over them using jQuery's each. Since we used the same names for
ids in our form as the keys on our Book model we can simply store them
directly in the formData object. We then create a new Book from the data
and add it to the collection. We skip fields without a value so that the
defaults will be applied.

Backbone passes an event object as a parameter to the event-handling
function. This is useful for us in this case since we don't want the
form to actually submit and reload the page. Adding a call to
\texttt{preventDefault} on the event in the \texttt{addBook} function
takes care of this for us.

Now we just need to make the view render again when a new model is
added. To do this, we put

\begin{Shaded}
\begin{Highlighting}[]
\KeywordTok{this}\NormalTok{.}\FunctionTok{listenTo}\NormalTok{( }\KeywordTok{this}\NormalTok{.}\FunctionTok{collection}\NormalTok{, }\StringTok{'add'}\NormalTok{, }\KeywordTok{this}\NormalTok{.}\FunctionTok{renderBook} \NormalTok{);}
\end{Highlighting}
\end{Shaded}

in the initialize function of LibraryView.

Now you should be ready to take the application for a spin.

\begin{figure}[htbp]
\centering
\includegraphics{img/chapter5-4.png}
\end{figure}

You may notice that the file input for the cover image isn't working,
but that is left as an exercise to the reader.

\subsubsection{Removing models}\label{removing-models}

Next, we need to wire up the delete button. Set up the event handler in
the BookView:

\begin{Shaded}
\begin{Highlighting}[]
    \NormalTok{events: \{}
        \StringTok{'click .delete'}\NormalTok{: }\StringTok{'deleteBook'}
    \NormalTok{\},}

    \NormalTok{deleteBook: }\KeywordTok{function}\NormalTok{() \{}
        \CommentTok{//Delete model}
        \KeywordTok{this}\NormalTok{.}\OtherTok{model}\NormalTok{.}\FunctionTok{destroy}\NormalTok{();}

        \CommentTok{//Delete view}
        \KeywordTok{this}\NormalTok{.}\FunctionTok{remove}\NormalTok{();}
    \NormalTok{\},}
\end{Highlighting}
\end{Shaded}

You should now be able to add and remove books from the library.

\subsection{Creating the back-end}\label{creating-the-back-end}

Now we need to make a small detour and set up a server with a REST api.
Since this is a JavaScript book we will use JavaScript to create the
server using node.js. If you are more comfortable in setting up a REST
server in another language, this is the API you need to conform to:

\begin{verbatim}
url             HTTP Method  Operation
/api/books      GET          Get an array of all books
/api/books/:id  GET          Get the book with id of :id
/api/books      POST         Add a new book and return the book with an id attribute added
/api/books/:id  PUT          Update the book with id of :id
/api/books/:id  DELETE       Delete the book with id of :id
\end{verbatim}

The outline for this section looks like this:

\begin{itemize}
\itemsep1pt\parskip0pt\parsep0pt
\item
  Install node.js, npm, and MongoDB
\item
  Install node modules
\item
  Create a simple web server
\item
  Connect to the database
\item
  Create the REST API
\end{itemize}

\subsubsection{Install node.js, npm, and
MongoDB}\label{install-node.js-npm-and-mongodb}

Download and install node.js from nodejs.org. The node package manager
(npm) will be installed as well.

Download, install, and run MongoDB from mongodb.org (you need Mongo to
be running to store data in a Mongo database). There are detailed
installation guides
\href{http://docs.mongodb.org/manual/installation/}{on the website}.

\subsubsection{Install node modules}\label{install-node-modules}

Create a file called \texttt{package.json} in the root of your project.
It should look like

\begin{Shaded}
\begin{Highlighting}[]
\NormalTok{\{}
    \StringTok{"name"}\NormalTok{: }\StringTok{"backbone-library"}\NormalTok{,}
    \StringTok{"version"}\NormalTok{: }\StringTok{"0.0.1"}\NormalTok{,}
    \StringTok{"description"}\NormalTok{: }\StringTok{"A simple library application using Backbone"}\NormalTok{,}
    \StringTok{"dependencies"}\NormalTok{: \{}
        \StringTok{"express"}\NormalTok{: }\StringTok{"~3.1.0"}\NormalTok{,}
        \StringTok{"path"}\NormalTok{: }\StringTok{"~0.4.9"}\NormalTok{,}
        \StringTok{"mongoose"}\NormalTok{: }\StringTok{"~3.5.5"}
    \NormalTok{\}}
\NormalTok{\}}
\end{Highlighting}
\end{Shaded}

Amongst other things, this file tells npm what the dependencies are for
our project. On the command line, from the root of your project, type:

\begin{verbatim}
npm install
\end{verbatim}

You should see npm fetch the dependencies that we listed in our
package.json and save them within a folder called node\_modules.

Your folder structure should look something like this:

\begin{verbatim}
node_modules/
  .bin/
  express/
  mongoose/
  path/
site/
  css/
  img/
  js/
  index.html
package.json
\end{verbatim}

\subsubsection{Create a simple web
server}\label{create-a-simple-web-server}

Create a file named server.js in the project root containing the
following code:

\begin{Shaded}
\begin{Highlighting}[]
\CommentTok{// Module dependencies.}
\KeywordTok{var} \NormalTok{application_root = __dirname,}
    \NormalTok{express = }\FunctionTok{require}\NormalTok{( }\StringTok{'express'} \NormalTok{), }\CommentTok{//Web framework}
    \NormalTok{path = }\FunctionTok{require}\NormalTok{( }\StringTok{'path'} \NormalTok{), }\CommentTok{//Utilities for dealing with file paths}
    \NormalTok{mongoose = }\FunctionTok{require}\NormalTok{( }\StringTok{'mongoose'} \NormalTok{); }\CommentTok{//MongoDB integration}

\CommentTok{//Create server}
\KeywordTok{var} \NormalTok{app = }\FunctionTok{express}\NormalTok{();}

\CommentTok{// Configure server}
\OtherTok{app}\NormalTok{.}\FunctionTok{configure}\NormalTok{( }\KeywordTok{function}\NormalTok{() \{}
    \CommentTok{//parses request body and populates request.body}
    \OtherTok{app}\NormalTok{.}\FunctionTok{use}\NormalTok{( }\OtherTok{express}\NormalTok{.}\FunctionTok{bodyParser}\NormalTok{() );}

    \CommentTok{//checks request.body for HTTP method overrides}
    \OtherTok{app}\NormalTok{.}\FunctionTok{use}\NormalTok{( }\OtherTok{express}\NormalTok{.}\FunctionTok{methodOverride}\NormalTok{() );}

    \CommentTok{//perform route lookup based on url and HTTP method}
    \OtherTok{app}\NormalTok{.}\FunctionTok{use}\NormalTok{( }\OtherTok{app}\NormalTok{.}\FunctionTok{router} \NormalTok{);}

    \CommentTok{//Where to serve static content}
    \OtherTok{app}\NormalTok{.}\FunctionTok{use}\NormalTok{( }\OtherTok{express}\NormalTok{.}\FunctionTok{static}\NormalTok{( }\OtherTok{path}\NormalTok{.}\FunctionTok{join}\NormalTok{( application_root, }\StringTok{'site'}\NormalTok{) ) );}

    \CommentTok{//Show all errors in development}
    \OtherTok{app}\NormalTok{.}\FunctionTok{use}\NormalTok{( }\OtherTok{express}\NormalTok{.}\FunctionTok{errorHandler}\NormalTok{(\{ }\DataTypeTok{dumpExceptions}\NormalTok{: }\KeywordTok{true}\NormalTok{, }\DataTypeTok{showStack}\NormalTok{: }\KeywordTok{true} \NormalTok{\}));}
\NormalTok{\});}

\CommentTok{//Start server}
\KeywordTok{var} \NormalTok{port = }\DecValTok{4711}\NormalTok{;}
\OtherTok{app}\NormalTok{.}\FunctionTok{listen}\NormalTok{( port, }\KeywordTok{function}\NormalTok{() \{}
    \OtherTok{console}\NormalTok{.}\FunctionTok{log}\NormalTok{( }\StringTok{'Express server listening on port %d in %s mode'}\NormalTok{, port, }\OtherTok{app}\NormalTok{.}\OtherTok{settings}\NormalTok{.}\FunctionTok{env} \NormalTok{);}
\NormalTok{\});}
\end{Highlighting}
\end{Shaded}

We start off by loading the modules required for this project: Express
for creating the HTTP server, Path for dealing with file paths, and
mongoose for connecting with the database. We then create an Express
server and configure it using an anonymous function. This is a pretty
standard configuration and for our application we don't actually need
the methodOverride part. It is used for issuing PUT and DELETE HTTP
requests directly from a form, since forms normally only support GET and
POST. Finally, we start the server by running the listen function. The
port number used, in this case 4711, could be any free port on your
system. I simply used 4711 since it is unlikely to have been used by
anything else. We are now ready to run our first server:

\begin{Shaded}
\begin{Highlighting}[]
\NormalTok{node }\OtherTok{server}\NormalTok{.}\FunctionTok{js}
\end{Highlighting}
\end{Shaded}

If you open a browser on http://localhost:4711 you should see something
like this:

\begin{figure}[htbp]
\centering
\includegraphics{img/chapter5-5.png}
\end{figure}

This is where we left off in Part 2, but we are now running on a server
instead of directly from the files. Great job! We can now start defining
routes (URLs) that the server should react to. This will be our REST
API. Routes are defined by using app followed by one of the HTTP verbs
get, put, post, and delete, which corresponds to Create, Read, Update
and Delete. Let us go back to server.js and define a simple route:

\begin{Shaded}
\begin{Highlighting}[]
\CommentTok{// Routes}
\OtherTok{app}\NormalTok{.}\FunctionTok{get}\NormalTok{( }\StringTok{'/api'}\NormalTok{, }\KeywordTok{function}\NormalTok{( request, response ) \{}
    \OtherTok{response}\NormalTok{.}\FunctionTok{send}\NormalTok{( }\StringTok{'Library API is running'} \NormalTok{);}
\NormalTok{\});}
\end{Highlighting}
\end{Shaded}

The get function takes a URL as the first parameter and a function as
the second. The function will be called with request and response
objects. Now you can restart node and go to our specified URL:

\begin{figure}[htbp]
\centering
\includegraphics{img/chapter5-6.png}
\end{figure}

\subsubsection{Connect to the database}\label{connect-to-the-database}

Fantastic. Now, since we want to store our data in MongoDB, we need to
define a schema. Add this to server.js:

\begin{Shaded}
\begin{Highlighting}[]
\CommentTok{//Connect to database}
\OtherTok{mongoose}\NormalTok{.}\FunctionTok{connect}\NormalTok{( }\StringTok{'mongodb://localhost/library_database'} \NormalTok{);}

\CommentTok{//Schemas}
\KeywordTok{var} \NormalTok{Book = }\KeywordTok{new} \OtherTok{mongoose}\NormalTok{.}\FunctionTok{Schema}\NormalTok{(\{}
    \DataTypeTok{title}\NormalTok{: String,}
    \DataTypeTok{author}\NormalTok{: String,}
    \DataTypeTok{releaseDate}\NormalTok{: Date}
\NormalTok{\});}

\CommentTok{//Models}
\KeywordTok{var} \NormalTok{BookModel = }\OtherTok{mongoose}\NormalTok{.}\FunctionTok{model}\NormalTok{( }\StringTok{'Book'}\NormalTok{, Book );}
\end{Highlighting}
\end{Shaded}

As you can see, schema definitions are quite straight forward. They can
be more advanced, but this will do for us. I also extracted a model
(BookModel) from Mongo. This is what we will be working with. Next up,
we define a GET operation for the REST API that will return all books:

\begin{Shaded}
\begin{Highlighting}[]
\CommentTok{//Get a list of all books}
\OtherTok{app}\NormalTok{.}\FunctionTok{get}\NormalTok{( }\StringTok{'/api/books'}\NormalTok{, }\KeywordTok{function}\NormalTok{( request, response ) \{}
    \KeywordTok{return} \OtherTok{BookModel}\NormalTok{.}\FunctionTok{find}\NormalTok{( }\KeywordTok{function}\NormalTok{( err, books ) \{}
        \KeywordTok{if}\NormalTok{( !err ) \{}
            \KeywordTok{return} \OtherTok{response}\NormalTok{.}\FunctionTok{send}\NormalTok{( books );}
        \NormalTok{\} }\KeywordTok{else} \NormalTok{\{}
            \KeywordTok{return} \OtherTok{console}\NormalTok{.}\FunctionTok{log}\NormalTok{( err );}
        \NormalTok{\}}
    \NormalTok{\});}
\NormalTok{\});}
\end{Highlighting}
\end{Shaded}

The find function of Model is defined like this:
\texttt{function find (conditions, fields, options, callback)} -- but
since we want a function that returns all books we only need the
callback parameter. The callback will be called with an error object and
an array of found objects. If there was no error we return the array of
objects to the client using the \texttt{send} function of the response
object, otherwise we log the error to the console.

To test our API we need to do a little typing in a JavaScript console.
Restart node and go to localhost:4711 in your browser. Open up the
JavaScript console. If you are using Google Chrome, go to
View-\textgreater{}Developer-\textgreater{}JavaScript Console. If you
are using Firefox, install Firebug and go to View-\textgreater{}Firebug.
Most other browsers will have a similar console. In the console type the
following:

\begin{Shaded}
\begin{Highlighting}[]
\OtherTok{jQuery}\NormalTok{.}\FunctionTok{get}\NormalTok{( }\StringTok{'/api/books/'}\NormalTok{, }\KeywordTok{function}\NormalTok{( data, textStatus, jqXHR ) \{}
    \OtherTok{console}\NormalTok{.}\FunctionTok{log}\NormalTok{( }\StringTok{'Get response:'} \NormalTok{);}
    \OtherTok{console}\NormalTok{.}\FunctionTok{dir}\NormalTok{( data );}
    \OtherTok{console}\NormalTok{.}\FunctionTok{log}\NormalTok{( textStatus );}
    \OtherTok{console}\NormalTok{.}\FunctionTok{dir}\NormalTok{( jqXHR );}
\NormalTok{\});}
\end{Highlighting}
\end{Shaded}

\ldots{}and press enter and you should get something like this:

\begin{figure}[htbp]
\centering
\includegraphics{img/chapter5-7.png}
\end{figure}

Here I used jQuery to make the call to our REST API, since it was
already loaded on the page. The returned array is obviously empty, since
we have not put anything into the database yet. Let's go and create a
POST route that enables adding new items in server.js:

\begin{Shaded}
\begin{Highlighting}[]
\CommentTok{//Insert a new book}
\OtherTok{app}\NormalTok{.}\FunctionTok{post}\NormalTok{( }\StringTok{'/api/books'}\NormalTok{, }\KeywordTok{function}\NormalTok{( request, response ) \{}
    \KeywordTok{var} \NormalTok{book = }\KeywordTok{new} \FunctionTok{BookModel}\NormalTok{(\{}
        \DataTypeTok{title}\NormalTok{: }\OtherTok{request}\NormalTok{.}\OtherTok{body}\NormalTok{.}\FunctionTok{title}\NormalTok{,}
        \DataTypeTok{author}\NormalTok{: }\OtherTok{request}\NormalTok{.}\OtherTok{body}\NormalTok{.}\FunctionTok{author}\NormalTok{,}
        \DataTypeTok{releaseDate}\NormalTok{: }\OtherTok{request}\NormalTok{.}\OtherTok{body}\NormalTok{.}\FunctionTok{releaseDate}
    \NormalTok{\});}
    
    \KeywordTok{return} \OtherTok{book}\NormalTok{.}\FunctionTok{save}\NormalTok{( }\KeywordTok{function}\NormalTok{( err ) \{}
        \KeywordTok{if}\NormalTok{( !err ) \{}
            \OtherTok{console}\NormalTok{.}\FunctionTok{log}\NormalTok{( }\StringTok{'created'} \NormalTok{);}
            
                        \KeywordTok{return} \OtherTok{response}\NormalTok{.}\FunctionTok{send}\NormalTok{( book );}
            \NormalTok{\} }\KeywordTok{else} \NormalTok{\{}
                \OtherTok{console}\NormalTok{.}\FunctionTok{log}\NormalTok{( err );}
            \NormalTok{\}}
    \NormalTok{\});}
\NormalTok{\});}
\end{Highlighting}
\end{Shaded}

We start by creating a new BookModel, passing an object with title,
author, and releaseDate attributes. The data are collected from
request.body. This means that anyone calling this operation in the API
needs to supply a JSON object containing the title, author, and
releaseDate attributes. Actually, the caller can omit any or all
attributes since we have not made any of them mandatory.

We then call the save function on the BookModel passing in a callback in
the same way as with the previous get route. Finally, we return the
saved BookModel. The reason we return the BookModel and not just
``success'' or similar string is that when the BookModel is saved it
will get an \_id attribute from MongoDB, which the client needs when
updating or deleting a specific book. Let's try it out again. Restart
node and go back to the console and type:

\begin{Shaded}
\begin{Highlighting}[]
\OtherTok{jQuery}\NormalTok{.}\FunctionTok{post}\NormalTok{( }\StringTok{'/api/books'}\NormalTok{, \{}
    \StringTok{'title'}\NormalTok{: }\StringTok{'JavaScript the good parts'}\NormalTok{,}
    \StringTok{'author'}\NormalTok{: }\StringTok{'Douglas Crockford'}\NormalTok{,}
    \StringTok{'releaseDate'}\NormalTok{: }\KeywordTok{new} \FunctionTok{Date}\NormalTok{( }\DecValTok{2008}\NormalTok{, }\DecValTok{4}\NormalTok{, }\DecValTok{1} \NormalTok{).}\FunctionTok{getTime}\NormalTok{()}
\NormalTok{\}, }\KeywordTok{function}\NormalTok{(data, textStatus, jqXHR) \{}
    \OtherTok{console}\NormalTok{.}\FunctionTok{log}\NormalTok{( }\StringTok{'Post response:'} \NormalTok{);}
    \OtherTok{console}\NormalTok{.}\FunctionTok{dir}\NormalTok{( data );}
    \OtherTok{console}\NormalTok{.}\FunctionTok{log}\NormalTok{( textStatus );}
    \OtherTok{console}\NormalTok{.}\FunctionTok{dir}\NormalTok{( jqXHR );}
\NormalTok{\});}
\end{Highlighting}
\end{Shaded}

..and then

\begin{Shaded}
\begin{Highlighting}[]
\OtherTok{jQuery}\NormalTok{.}\FunctionTok{get}\NormalTok{( }\StringTok{'/api/books/'}\NormalTok{, }\KeywordTok{function}\NormalTok{( data, textStatus, jqXHR ) \{}
    \OtherTok{console}\NormalTok{.}\FunctionTok{log}\NormalTok{( }\StringTok{'Get response:'} \NormalTok{);}
    \OtherTok{console}\NormalTok{.}\FunctionTok{dir}\NormalTok{( data );}
    \OtherTok{console}\NormalTok{.}\FunctionTok{log}\NormalTok{( textStatus );}
    \OtherTok{console}\NormalTok{.}\FunctionTok{dir}\NormalTok{( jqXHR );}
\NormalTok{\});}
\end{Highlighting}
\end{Shaded}

You should now get a one-element array back from our server. You may
wonder about this line:

\begin{Shaded}
\begin{Highlighting}[]
\StringTok{'releaseDate'}\NormalTok{: }\KeywordTok{new} \FunctionTok{Date}\NormalTok{(}\DecValTok{2008}\NormalTok{, }\DecValTok{4}\NormalTok{, }\DecValTok{1}\NormalTok{).}\FunctionTok{getTime}\NormalTok{()}
\end{Highlighting}
\end{Shaded}

MongoDB expects dates in UNIX time format (milliseconds from the start
of Jan 1st 1970 UTC), so we have to convert dates before posting. The
object we get back however, contains a JavaScript Date object. Also note
the \_id attribute of the returned object.

\begin{figure}[htbp]
\centering
\includegraphics{img/chapter5-8.png}
\end{figure}

Let's move on to creating a GET request that retrieves a single book in
server.js:

\begin{Shaded}
\begin{Highlighting}[]
\CommentTok{//Get a single book by id}
\OtherTok{app}\NormalTok{.}\FunctionTok{get}\NormalTok{( }\StringTok{'/api/books/:id'}\NormalTok{, }\KeywordTok{function}\NormalTok{( request, response ) \{}
    \KeywordTok{return} \OtherTok{BookModel}\NormalTok{.}\FunctionTok{findById}\NormalTok{( }\OtherTok{request}\NormalTok{.}\OtherTok{params}\NormalTok{.}\FunctionTok{id}\NormalTok{, }\KeywordTok{function}\NormalTok{( err, book ) \{}
        \KeywordTok{if}\NormalTok{( !err ) \{}
            \KeywordTok{return} \OtherTok{response}\NormalTok{.}\FunctionTok{send}\NormalTok{( book );}
        \NormalTok{\} }\KeywordTok{else} \NormalTok{\{}
            \KeywordTok{return} \OtherTok{console}\NormalTok{.}\FunctionTok{log}\NormalTok{( err );}
        \NormalTok{\}}
    \NormalTok{\});}
\NormalTok{\});}
\end{Highlighting}
\end{Shaded}

Here we use colon notation (:id) to tell Express that this part of the
route is dynamic. We also use the \texttt{findById} function on
BookModel to get a single result. If you restart node, you can get a
single book by adding the id previously returned to the URL like this:

\begin{Shaded}
\begin{Highlighting}[]
\OtherTok{jQuery}\NormalTok{.}\FunctionTok{get}\NormalTok{( }\StringTok{'/api/books/4f95a8cb1baa9b8a1b000006'}\NormalTok{, }\KeywordTok{function}\NormalTok{( data, textStatus, jqXHR ) \{}
    \OtherTok{console}\NormalTok{.}\FunctionTok{log}\NormalTok{( }\StringTok{'Get response:'} \NormalTok{);}
    \OtherTok{console}\NormalTok{.}\FunctionTok{dir}\NormalTok{( data );}
    \OtherTok{console}\NormalTok{.}\FunctionTok{log}\NormalTok{( textStatus );}
    \OtherTok{console}\NormalTok{.}\FunctionTok{dir}\NormalTok{( jqXHR );}
\NormalTok{\});}
\end{Highlighting}
\end{Shaded}

Let's create the PUT (update) function next:

\begin{Shaded}
\begin{Highlighting}[]
\CommentTok{//Update a book}
\OtherTok{app}\NormalTok{.}\FunctionTok{put}\NormalTok{( }\StringTok{'/api/books/:id'}\NormalTok{, }\KeywordTok{function}\NormalTok{( request, response ) \{}
    \OtherTok{console}\NormalTok{.}\FunctionTok{log}\NormalTok{( }\StringTok{'Updating book '} \NormalTok{+ }\OtherTok{request}\NormalTok{.}\OtherTok{body}\NormalTok{.}\FunctionTok{title} \NormalTok{);}
    \KeywordTok{return} \OtherTok{BookModel}\NormalTok{.}\FunctionTok{findById}\NormalTok{( }\OtherTok{request}\NormalTok{.}\OtherTok{params}\NormalTok{.}\FunctionTok{id}\NormalTok{, }\KeywordTok{function}\NormalTok{( err, book ) \{}
        \OtherTok{book}\NormalTok{.}\FunctionTok{title} \NormalTok{= }\OtherTok{request}\NormalTok{.}\OtherTok{body}\NormalTok{.}\FunctionTok{title}\NormalTok{;}
        \OtherTok{book}\NormalTok{.}\FunctionTok{author} \NormalTok{= }\OtherTok{request}\NormalTok{.}\OtherTok{body}\NormalTok{.}\FunctionTok{author}\NormalTok{;}
        \OtherTok{book}\NormalTok{.}\FunctionTok{releaseDate} \NormalTok{= }\OtherTok{request}\NormalTok{.}\OtherTok{body}\NormalTok{.}\FunctionTok{releaseDate}\NormalTok{;}

        \KeywordTok{return} \OtherTok{book}\NormalTok{.}\FunctionTok{save}\NormalTok{( }\KeywordTok{function}\NormalTok{( err ) \{}
            \KeywordTok{if}\NormalTok{( !err ) \{}
                \OtherTok{console}\NormalTok{.}\FunctionTok{log}\NormalTok{( }\StringTok{'book updated'} \NormalTok{);}
            \KeywordTok{return} \OtherTok{response}\NormalTok{.}\FunctionTok{send}\NormalTok{( book );}
        \NormalTok{\} }\KeywordTok{else} \NormalTok{\{}
            \OtherTok{console}\NormalTok{.}\FunctionTok{log}\NormalTok{( err );}
        \NormalTok{\}}
        \NormalTok{\});}
    \NormalTok{\});}
\NormalTok{\});}
\end{Highlighting}
\end{Shaded}

This is a little larger than previous ones, but is also pretty straight
forward -- we find a book by id, update its properties, save it, and
send it back to the client.

To test this we need to use the more general jQuery ajax function.
Again, in these examples you will need to replace the id property with
one that matches an item in your own database:

\begin{Shaded}
\begin{Highlighting}[]
\OtherTok{jQuery}\NormalTok{.}\FunctionTok{ajax}\NormalTok{(\{}
    \DataTypeTok{url}\NormalTok{: }\StringTok{'/api/books/4f95a8cb1baa9b8a1b000006'}\NormalTok{,}
    \DataTypeTok{type}\NormalTok{: }\StringTok{'PUT'}\NormalTok{,}
    \DataTypeTok{data}\NormalTok{: \{}
        \StringTok{'title'}\NormalTok{: }\StringTok{'JavaScript The good parts'}\NormalTok{,}
        \StringTok{'author'}\NormalTok{: }\StringTok{'The Legendary Douglas Crockford'}\NormalTok{,}
        \StringTok{'releaseDate'}\NormalTok{: }\KeywordTok{new} \FunctionTok{Date}\NormalTok{( }\DecValTok{2008}\NormalTok{, }\DecValTok{4}\NormalTok{, }\DecValTok{1} \NormalTok{).}\FunctionTok{getTime}\NormalTok{()}
    \NormalTok{\},}
    \DataTypeTok{success}\NormalTok{: }\KeywordTok{function}\NormalTok{( data, textStatus, jqXHR ) \{}
        \OtherTok{console}\NormalTok{.}\FunctionTok{log}\NormalTok{( }\StringTok{'Put response:'} \NormalTok{);}
        \OtherTok{console}\NormalTok{.}\FunctionTok{dir}\NormalTok{( data );}
        \OtherTok{console}\NormalTok{.}\FunctionTok{log}\NormalTok{( textStatus );}
        \OtherTok{console}\NormalTok{.}\FunctionTok{dir}\NormalTok{( jqXHR );}
    \NormalTok{\}}
\NormalTok{\});}
\end{Highlighting}
\end{Shaded}

Finally we create the delete route:

\begin{Shaded}
\begin{Highlighting}[]
\CommentTok{//Delete a book}
\OtherTok{app}\NormalTok{.}\FunctionTok{delete}\NormalTok{( }\StringTok{'/api/books/:id'}\NormalTok{, }\KeywordTok{function}\NormalTok{( request, response ) \{}
    \OtherTok{console}\NormalTok{.}\FunctionTok{log}\NormalTok{( }\StringTok{'Deleting book with id: '} \NormalTok{+ }\OtherTok{request}\NormalTok{.}\OtherTok{params}\NormalTok{.}\FunctionTok{id} \NormalTok{);}
    \KeywordTok{return} \OtherTok{BookModel}\NormalTok{.}\FunctionTok{findById}\NormalTok{( }\OtherTok{request}\NormalTok{.}\OtherTok{params}\NormalTok{.}\FunctionTok{id}\NormalTok{, }\KeywordTok{function}\NormalTok{( err, book ) \{}
        \KeywordTok{return} \OtherTok{book}\NormalTok{.}\FunctionTok{remove}\NormalTok{( }\KeywordTok{function}\NormalTok{( err ) \{}
            \KeywordTok{if}\NormalTok{( !err ) \{}
                \OtherTok{console}\NormalTok{.}\FunctionTok{log}\NormalTok{( }\StringTok{'Book removed'} \NormalTok{);}
                \KeywordTok{return} \OtherTok{response}\NormalTok{.}\FunctionTok{send}\NormalTok{( }\StringTok{''} \NormalTok{);}
            \NormalTok{\} }\KeywordTok{else} \NormalTok{\{}
                \OtherTok{console}\NormalTok{.}\FunctionTok{log}\NormalTok{( err );}
            \NormalTok{\}}
        \NormalTok{\});}
    \NormalTok{\});}
\NormalTok{\});}
\end{Highlighting}
\end{Shaded}

\ldots{}and try it out:

\begin{Shaded}
\begin{Highlighting}[]
\OtherTok{jQuery}\NormalTok{.}\FunctionTok{ajax}\NormalTok{(\{}
    \DataTypeTok{url}\NormalTok{: }\StringTok{'/api/books/4f95a5251baa9b8a1b000001'}\NormalTok{,}
    \DataTypeTok{type}\NormalTok{: }\StringTok{'DELETE'}\NormalTok{,}
    \DataTypeTok{success}\NormalTok{: }\KeywordTok{function}\NormalTok{( data, textStatus, jqXHR ) \{}
        \OtherTok{console}\NormalTok{.}\FunctionTok{log}\NormalTok{( }\StringTok{'Delete response:'} \NormalTok{);}
        \OtherTok{console}\NormalTok{.}\FunctionTok{dir}\NormalTok{( data );}
        \OtherTok{console}\NormalTok{.}\FunctionTok{log}\NormalTok{( textStatus );}
        \OtherTok{console}\NormalTok{.}\FunctionTok{dir}\NormalTok{( jqXHR );}
    \NormalTok{\}}
\NormalTok{\});}
\end{Highlighting}
\end{Shaded}

So now our REST API is complete -- we have support for all four HTTP
verbs. What's next? Well, until now I have left out the keywords part of
our books. This is a bit more complicated since a book could have
several keywords and we don't want to represent them as a string, but
rather an array of strings. To do that we need another schema. Add a
Keywords schema right above our Book schema:

\begin{Shaded}
\begin{Highlighting}[]
\CommentTok{//Schemas}
\KeywordTok{var} \NormalTok{Keywords = }\KeywordTok{new} \OtherTok{mongoose}\NormalTok{.}\FunctionTok{Schema}\NormalTok{(\{}
    \DataTypeTok{keyword}\NormalTok{: String}
\NormalTok{\});}
\end{Highlighting}
\end{Shaded}

To add a sub schema to an existing schema we use brackets notation like
so:

\begin{Shaded}
\begin{Highlighting}[]
\KeywordTok{var} \NormalTok{Book = }\KeywordTok{new} \OtherTok{mongoose}\NormalTok{.}\FunctionTok{Schema}\NormalTok{(\{}
    \DataTypeTok{title}\NormalTok{: String,}
    \DataTypeTok{author}\NormalTok{: String,}
    \DataTypeTok{releaseDate}\NormalTok{: Date,}
    \DataTypeTok{keywords}\NormalTok{: [ Keywords ]                       }\CommentTok{// NEW}
\NormalTok{\});}
\end{Highlighting}
\end{Shaded}

Also update POST and PUT:

\begin{Shaded}
\begin{Highlighting}[]
\CommentTok{//Insert a new book}
\OtherTok{app}\NormalTok{.}\FunctionTok{post}\NormalTok{( }\StringTok{'/api/books'}\NormalTok{, }\KeywordTok{function}\NormalTok{( request, response ) \{}
    \KeywordTok{var} \NormalTok{book = }\KeywordTok{new} \FunctionTok{BookModel}\NormalTok{(\{}
        \DataTypeTok{title}\NormalTok{: }\OtherTok{request}\NormalTok{.}\OtherTok{body}\NormalTok{.}\FunctionTok{title}\NormalTok{,}
        \DataTypeTok{author}\NormalTok{: }\OtherTok{request}\NormalTok{.}\OtherTok{body}\NormalTok{.}\FunctionTok{author}\NormalTok{,}
        \DataTypeTok{releaseDate}\NormalTok{: }\OtherTok{request}\NormalTok{.}\OtherTok{body}\NormalTok{.}\FunctionTok{releaseDate}\NormalTok{,}
        \DataTypeTok{keywords}\NormalTok{: }\OtherTok{request}\NormalTok{.}\OtherTok{body}\NormalTok{.}\FunctionTok{keywords}       \CommentTok{// NEW}
    \NormalTok{\});}
    \OtherTok{book}\NormalTok{.}\FunctionTok{save}\NormalTok{( }\KeywordTok{function}\NormalTok{( err ) \{}
        \KeywordTok{if}\NormalTok{( !err ) \{}
            \OtherTok{console}\NormalTok{.}\FunctionTok{log}\NormalTok{( }\StringTok{'created'} \NormalTok{);}
            \KeywordTok{return} \OtherTok{response}\NormalTok{.}\FunctionTok{send}\NormalTok{( book );}
        \NormalTok{\} }\KeywordTok{else} \NormalTok{\{}
            \KeywordTok{return} \OtherTok{console}\NormalTok{.}\FunctionTok{log}\NormalTok{( err );}
        \NormalTok{\}}
    \NormalTok{\});}
\NormalTok{\});}

\CommentTok{//Update a book}
\OtherTok{app}\NormalTok{.}\FunctionTok{put}\NormalTok{( }\StringTok{'/api/books/:id'}\NormalTok{, }\KeywordTok{function}\NormalTok{( request, response ) \{}
    \OtherTok{console}\NormalTok{.}\FunctionTok{log}\NormalTok{( }\StringTok{'Updating book '} \NormalTok{+ }\OtherTok{request}\NormalTok{.}\OtherTok{body}\NormalTok{.}\FunctionTok{title} \NormalTok{);}
    \KeywordTok{return} \OtherTok{BookModel}\NormalTok{.}\FunctionTok{findById}\NormalTok{( }\OtherTok{request}\NormalTok{.}\OtherTok{params}\NormalTok{.}\FunctionTok{id}\NormalTok{, }\KeywordTok{function}\NormalTok{( err, book ) \{}
        \OtherTok{book}\NormalTok{.}\FunctionTok{title} \NormalTok{= }\OtherTok{request}\NormalTok{.}\OtherTok{body}\NormalTok{.}\FunctionTok{title}\NormalTok{;}
        \OtherTok{book}\NormalTok{.}\FunctionTok{author} \NormalTok{= }\OtherTok{request}\NormalTok{.}\OtherTok{body}\NormalTok{.}\FunctionTok{author}\NormalTok{;}
        \OtherTok{book}\NormalTok{.}\FunctionTok{releaseDate} \NormalTok{= }\OtherTok{request}\NormalTok{.}\OtherTok{body}\NormalTok{.}\FunctionTok{releaseDate}\NormalTok{;}
        \OtherTok{book}\NormalTok{.}\FunctionTok{keywords} \NormalTok{= }\OtherTok{request}\NormalTok{.}\OtherTok{body}\NormalTok{.}\FunctionTok{keywords}\NormalTok{; }\CommentTok{// NEW}

        \KeywordTok{return} \OtherTok{book}\NormalTok{.}\FunctionTok{save}\NormalTok{( }\KeywordTok{function}\NormalTok{( err ) \{}
            \KeywordTok{if}\NormalTok{( !err ) \{}
                \OtherTok{console}\NormalTok{.}\FunctionTok{log}\NormalTok{( }\StringTok{'book updated'} \NormalTok{);}
            \NormalTok{\} }\KeywordTok{else} \NormalTok{\{}
                \OtherTok{console}\NormalTok{.}\FunctionTok{log}\NormalTok{( err );}
            \NormalTok{\}}
            \KeywordTok{return} \OtherTok{response}\NormalTok{.}\FunctionTok{send}\NormalTok{( book );}
        \NormalTok{\});}
    \NormalTok{\});}
\NormalTok{\});}
\end{Highlighting}
\end{Shaded}

There we are, that should be all we need, now we can try it out in the
console:

\begin{Shaded}
\begin{Highlighting}[]
\OtherTok{jQuery}\NormalTok{.}\FunctionTok{post}\NormalTok{( }\StringTok{'/api/books'}\NormalTok{, \{}
    \StringTok{'title'}\NormalTok{: }\StringTok{'Secrets of the JavaScript Ninja'}\NormalTok{,}
    \StringTok{'author'}\NormalTok{: }\StringTok{'John Resig'}\NormalTok{,}
    \StringTok{'releaseDate'}\NormalTok{: }\KeywordTok{new} \FunctionTok{Date}\NormalTok{( }\DecValTok{2008}\NormalTok{, }\DecValTok{3}\NormalTok{, }\DecValTok{12} \NormalTok{).}\FunctionTok{getTime}\NormalTok{(),}
    \StringTok{'keywords'}\NormalTok{:[}
        \NormalTok{\{ }\StringTok{'keyword'}\NormalTok{: }\StringTok{'JavaScript'} \NormalTok{\},}
        \NormalTok{\{ }\StringTok{'keyword'}\NormalTok{: }\StringTok{'Reference'} \NormalTok{\}}
    \NormalTok{]}
\NormalTok{\}, }\KeywordTok{function}\NormalTok{( data, textStatus, jqXHR ) \{}
    \OtherTok{console}\NormalTok{.}\FunctionTok{log}\NormalTok{( }\StringTok{'Post response:'} \NormalTok{);}
    \OtherTok{console}\NormalTok{.}\FunctionTok{dir}\NormalTok{( data );}
    \OtherTok{console}\NormalTok{.}\FunctionTok{log}\NormalTok{( textStatus );}
    \OtherTok{console}\NormalTok{.}\FunctionTok{dir}\NormalTok{( jqXHR );}
\NormalTok{\});}
\end{Highlighting}
\end{Shaded}

You now have a fully functional REST server that we can hook into from
our front-end.

\subsection{Talking to the server}\label{talking-to-the-server}

In this part we will cover connecting our Backbone application to the
server through the REST API.

As we mentioned in chapter 3 \emph{Backbone Basics}, we can retrieve
models from a server using \texttt{collection.fetch()} by setting
\texttt{collection.url} to be the URL of the API endpoint. Let's update
the Library collection to do that now:

\begin{Shaded}
\begin{Highlighting}[]
\KeywordTok{var} \NormalTok{app = app || \{\};}

\OtherTok{app}\NormalTok{.}\FunctionTok{Library} \NormalTok{= }\OtherTok{Backbone}\NormalTok{.}\OtherTok{Collection}\NormalTok{.}\FunctionTok{extend}\NormalTok{(\{}
    \DataTypeTok{model}\NormalTok{: }\OtherTok{app}\NormalTok{.}\FunctionTok{Book}\NormalTok{,}
    \DataTypeTok{url}\NormalTok{: }\StringTok{'/api/books'}     \CommentTok{// NEW}
\NormalTok{\});}
\end{Highlighting}
\end{Shaded}

This results in the default implementation of Backbone.sync assuming
that the API looks like this:

\begin{verbatim}
url             HTTP Method  Operation
/api/books      GET          Get an array of all books
/api/books/:id  GET          Get the book with id of :id
/api/books      POST         Add a new book and return the book with an id attribute added
/api/books/:id  PUT          Update the book with id of :id
/api/books/:id  DELETE       Delete the book with id of :id
\end{verbatim}

To have our application retrieve the Book models from the server on page
load we need to update the LibraryView. The Backbone documentation
recommends inserting all models when the page is generated on the server
side, rather than fetching them from the client side once the page is
loaded. Since this chapter is trying to give you a more complete picture
of how to communicate with a server, we will go ahead and ignore that
recommendation. Go to the LibraryView declaration and update the
initialize function as follows:

\begin{Shaded}
\begin{Highlighting}[]
\NormalTok{initialize: }\KeywordTok{function}\NormalTok{() \{}
    \KeywordTok{this}\NormalTok{.}\FunctionTok{collection} \NormalTok{= }\KeywordTok{new} \OtherTok{app}\NormalTok{.}\FunctionTok{Library}\NormalTok{();}
    \KeywordTok{this}\NormalTok{.}\OtherTok{collection}\NormalTok{.}\FunctionTok{fetch}\NormalTok{(\{}\DataTypeTok{reset}\NormalTok{: }\KeywordTok{true}\NormalTok{\}); }\CommentTok{// NEW}
    \KeywordTok{this}\NormalTok{.}\FunctionTok{render}\NormalTok{();}

    \KeywordTok{this}\NormalTok{.}\FunctionTok{listenTo}\NormalTok{( }\KeywordTok{this}\NormalTok{.}\FunctionTok{collection}\NormalTok{, }\StringTok{'add'}\NormalTok{, }\KeywordTok{this}\NormalTok{.}\FunctionTok{renderBook} \NormalTok{);}
    \KeywordTok{this}\NormalTok{.}\FunctionTok{listenTo}\NormalTok{( }\KeywordTok{this}\NormalTok{.}\FunctionTok{collection}\NormalTok{, }\StringTok{'reset'}\NormalTok{, }\KeywordTok{this}\NormalTok{.}\FunctionTok{render} \NormalTok{); }\CommentTok{// NEW}
\NormalTok{\},}
\end{Highlighting}
\end{Shaded}

Now that we are populating our Library from the database using
\texttt{this.collection.fetch()}, the \texttt{initialize()} function no
longer takes a set of sample data as an argument and doesn't pass
anything to the app.Library constructor. You can now remove the sample
data from site/js/app.js, which should reduce it to a single statement
which creates the LibraryView:

\begin{Shaded}
\begin{Highlighting}[]
\CommentTok{// site/js/app.js}

\KeywordTok{var} \NormalTok{app = app || \{\};}

\FunctionTok{$}\NormalTok{(}\KeywordTok{function}\NormalTok{() \{}
    \KeywordTok{new} \OtherTok{app}\NormalTok{.}\FunctionTok{LibraryView}\NormalTok{();}
\NormalTok{\});}
\end{Highlighting}
\end{Shaded}

We have also added a listener on the reset event. We need to do this
since the models are fetched asynchronously after the page is rendered.
When the fetch completes, Backbone fires the reset event, as requested
by the \texttt{reset: true} option, and our listener re-renders the
view. If you reload the page now you should see all books that are
stored on the server:

\begin{figure}[htbp]
\centering
\includegraphics{img/chapter5-9.png}
\end{figure}

As you can see the date and keywords look a bit weird. The date
delivered from the server is converted into a JavaScript Date object and
when applied to the underscore template it will use the toString()
function to display it. There isn't very good support for formatting
dates in JavaScript so we will use the dateFormat jQuery plugin to fix
this. Go ahead and download it from
\href{http://github.com/phstc/jquery-dateFormat}{here} and put it in
your site/js/lib folder. Update the book template so that the date is
displayed with:

\begin{Shaded}
\begin{Highlighting}[]
\KeywordTok{<li>}\ErrorTok{<}\NormalTok{%= $.format.date( new Date( releaseDate ), 'MMMM yyyy' ) %>}\KeywordTok{</li>}
\end{Highlighting}
\end{Shaded}

and add a script element for the plugin

\begin{Shaded}
\begin{Highlighting}[]
\KeywordTok{<script}\OtherTok{ src=}\StringTok{"js/lib/jquery-dateFormat-1.0.js"}\KeywordTok{></script>}
\end{Highlighting}
\end{Shaded}

Now the date on the page should look a bit better. How about the
keywords? Since we are receiving the keywords in an array we need to
execute some code that generates a string of separated keywords. To do
that we can omit the equals character in the template tag which will let
us execute code that doesn't display anything:

\begin{Shaded}
\begin{Highlighting}[]
\KeywordTok{<li>}\ErrorTok{<}\NormalTok{% _.each( keywords, function( keyobj ) \{%> }\ErrorTok{<}\NormalTok{%= keyobj.keyword %>}\ErrorTok{<}\NormalTok{% \} ); %>}\KeywordTok{</li>}
\end{Highlighting}
\end{Shaded}

Here I iterate over the keywords array using the Underscore
\texttt{each} function and print out every single keyword. Note that I
display the keyword using the \textless{}\%= tag. This will display the
keywords with spaces between them.

Reloading the page again should look quite decent:

\begin{figure}[htbp]
\centering
\includegraphics{img/chapter5-10.png}
\end{figure}

Now go ahead and delete a book and then reload the page: Tadaa! the
deleted book is back! Not cool, why is this? This happens because when
we get the BookModels from the server they have an \_id attribute
(notice the underscore), but Backbone expects an id attribute (no
underscore). Since no id attribute is present, Backbone sees this model
as new and deleting a new model doesn't need any synchronization.

To fix this we can use the parse function of Backbone.Model. The parse
function lets you edit the server response before it is passed to the
Model constructor. Add a parse method to the Book model:

\begin{Shaded}
\begin{Highlighting}[]
\NormalTok{parse: }\KeywordTok{function}\NormalTok{( response ) \{}
    \OtherTok{response}\NormalTok{.}\FunctionTok{id} \NormalTok{= }\OtherTok{response}\NormalTok{.}\FunctionTok{_id}\NormalTok{;}
    \KeywordTok{return} \NormalTok{response;}
\NormalTok{\}}
\end{Highlighting}
\end{Shaded}

Simply copy the value of \_id to the needed id attribute. If you reload
the page you will see that models are actually deleted on the server
when you press the delete button.

Another, simpler way of making Backbone recognize \emph{id as its unique
identifier is to set the idAttribute of the model to}id.

If you now try to add a new book using the form you'll notice that it is
a similar story to delete -- models won't get persisted on the server.
This is because Backbone.Collection.add doesn't automatically sync, but
it is easy to fix. In the LibraryView we find in
\texttt{views/library.js} change the line reading:

\begin{Shaded}
\begin{Highlighting}[]
\KeywordTok{this}\NormalTok{.}\OtherTok{collection}\NormalTok{.}\FunctionTok{add}\NormalTok{( }\KeywordTok{new} \FunctionTok{Book}\NormalTok{( formData ) );}
\end{Highlighting}
\end{Shaded}

\ldots{}to:

\begin{Shaded}
\begin{Highlighting}[]
\KeywordTok{this}\NormalTok{.}\OtherTok{collection}\NormalTok{.}\FunctionTok{create}\NormalTok{( formData );}
\end{Highlighting}
\end{Shaded}

Now newly created books will get persisted. Actually, they probably
won't if you enter a date. The server expects a date in UNIX timestamp
format (milliseconds since Jan 1, 1970). Also, any keywords you enter
won't be stored since the server expects an array of objects with the
attribute `keyword'.

We'll start by fixing the date issue. We don't really want the users to
manually enter a date in a specific format, so we'll use the standard
datepicker from jQuery UI. Go ahead and create a custom jQuery UI
download containing datepicker from
\href{http://jqueryui.com/download/}{here}. Add the css theme to
site/css/ and the JavaScript to site/js/lib. Link to them in index.html:

\begin{Shaded}
\begin{Highlighting}[]
\KeywordTok{<link}\OtherTok{ rel=}\StringTok{"stylesheet"}\OtherTok{ href=}\StringTok{"css/cupertino/jquery-ui-1.10.0.custom.css"}\KeywordTok{>}
\end{Highlighting}
\end{Shaded}

``cupertino'' is the name of the style I chose when downloading jQuery
UI.

The JavaScript file must be loaded after jQuery.

\begin{Shaded}
\begin{Highlighting}[]
\KeywordTok{<script}\OtherTok{ src=}\StringTok{"js/lib/jquery.min.js"}\KeywordTok{></script>}
\KeywordTok{<script}\OtherTok{ src=}\StringTok{"js/lib/jquery-ui-1.10.0.custom.min.js"}\KeywordTok{></script>}
\end{Highlighting}
\end{Shaded}

Now in app.js, bind a datepicker to our releaseDate field:

\begin{Shaded}
\begin{Highlighting}[]
\KeywordTok{var} \NormalTok{app = app || \{\};}

\FunctionTok{$}\NormalTok{(}\KeywordTok{function}\NormalTok{() \{}
    \FunctionTok{$}\NormalTok{( }\StringTok{'#releaseDate'} \NormalTok{).}\FunctionTok{datepicker}\NormalTok{();}
    \KeywordTok{new} \OtherTok{app}\NormalTok{.}\FunctionTok{LibraryView}\NormalTok{();}
\NormalTok{\});}
\end{Highlighting}
\end{Shaded}

You should now be able to pick a date when clicking in the releaseDate
field:

\begin{figure}[htbp]
\centering
\includegraphics{img/chapter5-11.png}
\end{figure}

Finally, we have to make sure that the form input is properly
transformed into our storage format. Change the addBook function in
LibraryView to:

\begin{Shaded}
\begin{Highlighting}[]
\NormalTok{addBook: }\KeywordTok{function}\NormalTok{( e ) \{}
    \OtherTok{e}\NormalTok{.}\FunctionTok{preventDefault}\NormalTok{();}

    \KeywordTok{var} \NormalTok{formData = \{\};}

    \FunctionTok{$}\NormalTok{( }\StringTok{'#addBook div'} \NormalTok{).}\FunctionTok{children}\NormalTok{( }\StringTok{'input'} \NormalTok{).}\FunctionTok{each}\NormalTok{( }\KeywordTok{function}\NormalTok{( i, el ) \{}
        \KeywordTok{if}\NormalTok{( }\FunctionTok{$}\NormalTok{( el ).}\FunctionTok{val}\NormalTok{() != }\StringTok{''} \NormalTok{)}
        \NormalTok{\{}
            \KeywordTok{if}\NormalTok{( }\OtherTok{el}\NormalTok{.}\FunctionTok{id} \NormalTok{=== }\StringTok{'keywords'} \NormalTok{) \{}
                \NormalTok{formData[ }\OtherTok{el}\NormalTok{.}\FunctionTok{id} \NormalTok{] = [];}
                \OtherTok{_}\NormalTok{.}\FunctionTok{each}\NormalTok{( }\FunctionTok{$}\NormalTok{( el ).}\FunctionTok{val}\NormalTok{().}\FunctionTok{split}\NormalTok{( }\StringTok{' '} \NormalTok{), }\KeywordTok{function}\NormalTok{( keyword ) \{}
                    \NormalTok{formData[ }\OtherTok{el}\NormalTok{.}\FunctionTok{id} \NormalTok{].}\FunctionTok{push}\NormalTok{(\{ }\StringTok{'keyword'}\NormalTok{: keyword \});}
                \NormalTok{\});}
            \NormalTok{\} }\KeywordTok{else} \KeywordTok{if}\NormalTok{( }\OtherTok{el}\NormalTok{.}\FunctionTok{id} \NormalTok{=== }\StringTok{'releaseDate'} \NormalTok{) \{}
                \NormalTok{formData[ }\OtherTok{el}\NormalTok{.}\FunctionTok{id} \NormalTok{] = }\FunctionTok{$}\NormalTok{( }\StringTok{'#releaseDate'} \NormalTok{).}\FunctionTok{datepicker}\NormalTok{( }\StringTok{'getDate'} \NormalTok{).}\FunctionTok{getTime}\NormalTok{();}
            \NormalTok{\} }\KeywordTok{else} \NormalTok{\{}
                \NormalTok{formData[ }\OtherTok{el}\NormalTok{.}\FunctionTok{id} \NormalTok{] = }\FunctionTok{$}\NormalTok{( el ).}\FunctionTok{val}\NormalTok{();}
            \NormalTok{\}}
        \NormalTok{\}}
        \CommentTok{// Clear input field value}
        \FunctionTok{$}\NormalTok{( el ).}\FunctionTok{val}\NormalTok{(}\StringTok{''}\NormalTok{);}
    \NormalTok{\});}

    \KeywordTok{this}\NormalTok{.}\OtherTok{collection}\NormalTok{.}\FunctionTok{create}\NormalTok{( formData );}
\NormalTok{\},}
\end{Highlighting}
\end{Shaded}

Our change adds two checks to the form input fields. First, we're
checking if the current element is the keywords input field, in which
case we're splitting the string on each space and creating an array of
keyword objects.

Then we're checking if the current element is the releaseDate input
field, in which case we're calling \texttt{datePicker(“getDate”)} which
returns a Date object. We then use the \texttt{getTime} function on that
to get the time in milliseconds.

Now you should be able to add new books with both a release date and
keywords!

\begin{figure}[htbp]
\centering
\includegraphics{img/chapter5-12.png}
\end{figure}

\subsubsection{Summary}\label{summary-3}

In this chapter we made our application persistent by binding it to a
server using a REST API. We also looked at some problems that might
occur when serializing and deserializing data and their solutions. We
looked at the dateFormat and the datepicker jQuery plugins and how to do
some more advanced things in our Underscore templates. The code is
available
\href{https://github.com/addyosmani/backbone-fundamentals/tree/gh-pages/practicals/exercise-2}{here}.

\section{Backbone Extensions}\label{backbone-extensions}

Backbone is flexible, simple, and powerful. However, you may find that
the complexity of the application you are working on requires more than
what it provides out of the box. There are certain concerns which it
just doesn't address directly as one of its goals is to be minimalist.

Take for example Views, which provide a default \texttt{render} method
which does nothing and produces no real results when called, despite
most implementations using it to generate the HTML that the view
manages. Also, Models and Collections have no built-in way of handling
nested hierarchies - if you require this functionality, you need to
write it yourself or use a plugin.

In these cases, there are many existing Backbone plugins which can
provide advanced solutions for large-scale Backbone apps. A fairly
complete list of plugins and frameworks available can be found on the
Backbone
\href{https://github.com/documentcloud/backbone/wiki/Extensions\%2C-Plugins\%2C-Resources}{wiki}.
Using these add-ons, there is enough for applications of most sizes to
be completed successfully.

In this section of the book we will look at two popular Backbone
add-ons: MarionetteJS and Thorax.

\hyperdef{}{marionettejs-backbone.marionette}{\subsection{MarionetteJS
(Backbone.Marionette)}\label{marionettejs-backbone.marionette}}

\emph{By Derick Bailey \& Addy Osmani}

As we've seen, Backbone provides a great set of building blocks for our
JavaScript applications. It gives us the core constructs that are needed
to build small to mid-sized apps, organize jQuery DOM events, or create
single page apps that support mobile devices and large scale enterprise
needs. But Backbone is not a complete framework. It's a set of building
blocks that leaves much of the application design, architecture, and
scalability to the developer, including memory management, view
management, and more.

\href{http://marionettejs.com}{MarionetteJS} (a.k.a.
Backbone.Marionette) provides many of the features that the non-trivial
application developer needs, above what Backbone itself provides. It is
a composite application library that aims to simplify the construction
of large scale applications. It does this by providing a collection of
common design and implementation patterns found in the applications that
the creator, \href{http://lostechies.com/derickbailey/}{Derick Bailey},
and many other
\href{https://github.com/marionettejs/backbone.marionette/graphs/contributors}{contributors}
have been using to build Backbone apps.

Marionette's key benefits include:

\begin{itemize}
\itemsep1pt\parskip0pt\parsep0pt
\item
  Scaling applications out with modular, event driven architecture
\item
  Sensible defaults, such as using Underscore templates for view
  rendering
\item
  Easy to modify to make it work with your application's specific needs
\item
  Reducing boilerplate for views, with specialized view types
\item
  Build on a modular architecture with an Application and modules that
  attach to it
\item
  Compose your application's visuals at runtime, with Region and Layout
\item
  Nested views and layouts within visual regions
\item
  Built-in memory management and zombie killing in views, regions, and
  layouts
\item
  Built-in event clean up with the EventBinder
\item
  Event-driven architecture with the EventAggregator
\item
  Flexible, ``as-needed'' architecture allowing you to pick and choose
  what you need
\item
  And much, much more
\end{itemize}

Marionette follows a similar philosophy to Backbone in that it provides
a suite of components that can be used independently of each other, or
used together to create significant advantages for us as developers. But
it steps above the structural components of Backbone and provides an
application layer, with more than a dozen components and building
blocks.

Marionette's components range greatly in the features they provide, but
they all work together to create a composite application layer that can
both reduce boilerplate code and provide a much needed application
structure. Its core components include various and specialized view
types that take the boilerplate out of rendering common Backbone.Model
and Backbone.Collection scenarios; an Application object and Module
architecture to scale applications across sub-applications, features and
files; integration of a command pattern, event aggregator, and
request/response mechanism; and many more object types that can be
extended in a myriad of ways to create an architecture that facilitates
an application's specific needs.

In spite of the large number of constructs that Marionette provides,
though, you're not required to use all of it just because you want to
use some of it. Much like Backbone itself, you can pick and choose which
features you want to use and when. This allows you to work with other
Backbone frameworks and plugins very easily. It also means that you are
not required to engage in an all-or-nothing migration to begin using
Marionette.

\subsubsection{Boilerplate Rendering
Code}\label{boilerplate-rendering-code}

Consider the code that it typically requires to render a view with
Backbone and Underscore template. We need a template to render, which
can be placed in the DOM directly, and we need the JavaScript that
defines a view that uses the template and populates it with data from a
model.

\begin{verbatim}
<script type="text/html" id="my-view-template">
  <div class="row">
    <label>First Name:</label>
    <span><%= firstName %></span>
  </div>
  <div class="row">
    <label>Last Name:</label>
    <span><%= lastName %></span>
  </div>
  <div class="row">
    <label>Email:</label>
    <span><%= email %></span>
  </div>
</script>
\end{verbatim}

\begin{Shaded}
\begin{Highlighting}[]
\KeywordTok{var} \NormalTok{MyView = }\OtherTok{Backbone}\NormalTok{.}\OtherTok{View}\NormalTok{.}\FunctionTok{extend}\NormalTok{(\{}
  \DataTypeTok{template}\NormalTok{: }\FunctionTok{$}\NormalTok{(}\StringTok{'#my-view-template'}\NormalTok{).}\FunctionTok{html}\NormalTok{(),}

  \DataTypeTok{render}\NormalTok{: }\KeywordTok{function}\NormalTok{() \{}

    \CommentTok{// compile the Underscore.js template}
    \KeywordTok{var} \NormalTok{compiledTemplate = }\OtherTok{_}\NormalTok{.}\FunctionTok{template}\NormalTok{(}\KeywordTok{this}\NormalTok{.}\FunctionTok{template}\NormalTok{);}

    \CommentTok{// render the template with the model data}
    \KeywordTok{var} \NormalTok{data = }\OtherTok{_}\NormalTok{.}\FunctionTok{clone}\NormalTok{(}\KeywordTok{this}\NormalTok{.}\OtherTok{model}\NormalTok{.}\FunctionTok{attributes}\NormalTok{);}
    \KeywordTok{var} \NormalTok{html = }\FunctionTok{compiledTemplate}\NormalTok{(data);}

    \CommentTok{// populate the view with the rendered html}
    \KeywordTok{this}\NormalTok{.}\OtherTok{$el}\NormalTok{.}\FunctionTok{html}\NormalTok{(html);}
  \NormalTok{\}}
\NormalTok{\});}
\end{Highlighting}
\end{Shaded}

Once this is in place, you need to create an instance of your view and
pass your model into it. Then you can take the view's \texttt{el} and
append it to the DOM in order to display the view.

\begin{Shaded}
\begin{Highlighting}[]
\KeywordTok{var} \NormalTok{Derick = }\KeywordTok{new} \FunctionTok{Person}\NormalTok{(\{}
  \DataTypeTok{firstName}\NormalTok{: }\StringTok{'Derick'}\NormalTok{,}
  \DataTypeTok{lastName}\NormalTok{: }\StringTok{'Bailey'}\NormalTok{,}
  \DataTypeTok{email}\NormalTok{: }\StringTok{'derickbailey@example.com'}
\NormalTok{\});}

\KeywordTok{var} \NormalTok{myView = }\KeywordTok{new} \FunctionTok{MyView}\NormalTok{(\{}
  \DataTypeTok{model}\NormalTok{: Derick}
\NormalTok{\});}

\OtherTok{myView}\NormalTok{.}\FunctionTok{setElement}\NormalTok{(}\StringTok{"#content"}\NormalTok{);}
\OtherTok{myView}\NormalTok{.}\FunctionTok{render}\NormalTok{();}
\end{Highlighting}
\end{Shaded}

This is a standard set up for defining, building, rendering, and
displaying a view with Backbone. This is also what we call ``boilerplate
code'' - code that is repeated over and over and over again, across
every project and every implementation with the same functionality. It
gets to be tedious and repetitious very quickly.

Enter Marionette's \texttt{ItemView} - a simple way to reduce the
boilerplate of defining a view.

\subsubsection{Reducing Boilerplate With
Marionette.ItemView}\label{reducing-boilerplate-with-marionette.itemview}

All of Marionette's view types - with the exception of
\texttt{Marionette.View} - include a built-in \texttt{render} method
that handles the core rendering logic for you. We can take advantage of
this by changing the \texttt{MyView} instance to inherit from one of
these rather than \texttt{Backbone.View}. Instead of having to provide
our own \texttt{render} method for the view, we can let Marionette
render it for us. We'll still use the same Underscore.js template and
rendering mechanism, but the implementation of this is hidden behind the
scenes. Thus, we can reduce the amount of code needed for this view.

\begin{Shaded}
\begin{Highlighting}[]
\KeywordTok{var} \NormalTok{MyView = }\OtherTok{Marionette}\NormalTok{.}\OtherTok{ItemView}\NormalTok{.}\FunctionTok{extend}\NormalTok{(\{}
  \DataTypeTok{template}\NormalTok{: }\StringTok{'#my-view-template'}
\NormalTok{\});}
\end{Highlighting}
\end{Shaded}

And that's it - that's all you need to get the exact same behaviour as
the previous view implementation. Just replace
\texttt{Backbone.View.extend} with \texttt{Marionette.ItemView.extend},
then get rid of the \texttt{render} method. You can still create the
view instance with a \texttt{model}, call the \texttt{render} method on
the view instance, and display the view in the DOM the same way that we
did before. But the view definition has been reduced to a single line of
configuration for the template.

\subsubsection{Memory Management}\label{memory-management}

In addition to the reduction of code needed to define a view, Marionette
includes some advanced memory management in all of its views, making the
job of cleaning up a view instance and its event handlers easy.

Consider the following view implementation:

\begin{Shaded}
\begin{Highlighting}[]
\KeywordTok{var} \NormalTok{ZombieView = }\OtherTok{Backbone}\NormalTok{.}\OtherTok{View}\NormalTok{.}\FunctionTok{extend}\NormalTok{(\{}
  \DataTypeTok{template}\NormalTok{: }\StringTok{'#my-view-template'}\NormalTok{,}

  \DataTypeTok{initialize}\NormalTok{: }\KeywordTok{function}\NormalTok{() \{}

    \CommentTok{// bind the model change to re-render this view}
    \KeywordTok{this}\NormalTok{.}\FunctionTok{listenTo}\NormalTok{(}\KeywordTok{this}\NormalTok{.}\FunctionTok{model}\NormalTok{, }\StringTok{'change'}\NormalTok{, }\KeywordTok{this}\NormalTok{.}\FunctionTok{render}\NormalTok{);}

  \NormalTok{\},}

  \DataTypeTok{render}\NormalTok{: }\KeywordTok{function}\NormalTok{() \{}

    \CommentTok{// This alert is going to demonstrate a problem}
    \FunctionTok{alert}\NormalTok{(}\StringTok{'We`re rendering the view'}\NormalTok{);}

  \NormalTok{\}}
\NormalTok{\});}
\end{Highlighting}
\end{Shaded}

If we create two instances of this view using the same variable name for
both instances, and then change a value in the model, how many times
will we see the alert box?

\begin{Shaded}
\begin{Highlighting}[]

\KeywordTok{var} \NormalTok{Person = }\OtherTok{Backbone}\NormalTok{.}\OtherTok{Model}\NormalTok{.}\FunctionTok{extend}\NormalTok{(\{}
  \DataTypeTok{defaults}\NormalTok{: \{}
    \StringTok{"firstName"}\NormalTok{: }\StringTok{"Jeremy"}\NormalTok{,}
    \StringTok{"lastName"}\NormalTok{: }\StringTok{"Ashkenas"}\NormalTok{,}
    \StringTok{"email"}\NormalTok{:    }\StringTok{"jeremy@example.com"}
  \NormalTok{\}}
\NormalTok{\});}

\KeywordTok{var} \NormalTok{Derick = }\KeywordTok{new} \FunctionTok{Person}\NormalTok{(\{}
  \DataTypeTok{firstName}\NormalTok{: }\StringTok{'Derick'}\NormalTok{,}
  \DataTypeTok{lastName}\NormalTok{: }\StringTok{'Bailey'}\NormalTok{,}
  \DataTypeTok{email}\NormalTok{: }\StringTok{'derick@example.com'}
\NormalTok{\});}


\CommentTok{// create the first view instance}
\KeywordTok{var} \NormalTok{zombieView = }\KeywordTok{new} \FunctionTok{ZombieView}\NormalTok{(\{}
  \DataTypeTok{model}\NormalTok{: Derick}
\NormalTok{\});}

\CommentTok{// create a second view instance, re-using}
\CommentTok{// the same variable name to store it}
\NormalTok{zombieView = }\KeywordTok{new} \FunctionTok{ZombieView}\NormalTok{(\{}
  \DataTypeTok{model}\NormalTok{: Derick}
\NormalTok{\});}

\OtherTok{Derick}\NormalTok{.}\FunctionTok{set}\NormalTok{(}\StringTok{'email'}\NormalTok{, }\StringTok{'derickbailey@example.com'}\NormalTok{);}
\end{Highlighting}
\end{Shaded}

Since we're re-using the same \texttt{zombieView} variable for both
instances, the first instance of the view will fall out of scope
immediately after the second is created. This allows the JavaScript
garbage collector to come along and clean it up, which should mean the
first view instance is no longer active and no longer going to respond
to the model's ``change'' event.

But when we run this code, we end up with the alert box showing up
twice!

The problem is caused by the model event binding in the view's
\texttt{initialize} method. Whenever we pass \texttt{this.render} as the
callback method to the model's \texttt{on} event binding, the model
itself is being given a direct reference to the view instance. Since the
model is now holding a reference to the view instance, replacing the
\texttt{zombieView} variable with a new view instance is not going to
let the original view fall out of scope. The model still has a
reference, therefore the view is still in scope.

Since the original view is still in scope, and the second view instance
is also in scope, changing data on the model will cause both view
instances to respond.

Fixing this is easy, though. You just need to call
\texttt{stopListening} when the view is done with its work and ready to
be closed. To do this, add a \texttt{close} method to the view.

\begin{Shaded}
\begin{Highlighting}[]
\KeywordTok{var} \NormalTok{ZombieView = }\OtherTok{Backbone}\NormalTok{.}\OtherTok{View}\NormalTok{.}\FunctionTok{extend}\NormalTok{(\{}
  \DataTypeTok{template}\NormalTok{: }\StringTok{'#my-view-template'}\NormalTok{,}

  \DataTypeTok{initialize}\NormalTok{: }\KeywordTok{function}\NormalTok{() \{}
    \CommentTok{// bind the model change to re-render this view}
    \KeywordTok{this}\NormalTok{.}\FunctionTok{listenTo}\NormalTok{(}\KeywordTok{this}\NormalTok{.}\FunctionTok{model}\NormalTok{, }\StringTok{'change'}\NormalTok{, }\KeywordTok{this}\NormalTok{.}\FunctionTok{render}\NormalTok{);}
  \NormalTok{\},}

  \DataTypeTok{close}\NormalTok{: }\KeywordTok{function}\NormalTok{() \{}
    \CommentTok{// unbind the events that this view is listening to}
    \KeywordTok{this}\NormalTok{.}\FunctionTok{stopListening}\NormalTok{();}
  \NormalTok{\},}

  \DataTypeTok{render}\NormalTok{: }\KeywordTok{function}\NormalTok{() \{}

    \CommentTok{// This alert is going to demonstrate a problem}
    \FunctionTok{alert}\NormalTok{(}\StringTok{'We`re rendering the view'}\NormalTok{);}

  \NormalTok{\}}
\NormalTok{\});}
\end{Highlighting}
\end{Shaded}

Then call \texttt{close} on the first instance when it is no longer
needed, and only one view instance will remain alive. For more
information about the \texttt{listenTo} and \texttt{stopListening}
functions, see the earlier Backbone Basics chapter and Derick's post on
\href{http://lostechies.com/derickbailey/2013/02/06/managing-events-as-relationships-not-just-references/}{Managing
Events As Relationships, Not Just Resources}.

\begin{Shaded}
\begin{Highlighting}[]
\KeywordTok{var} \NormalTok{Jeremy = }\KeywordTok{new} \FunctionTok{Person}\NormalTok{(\{}
  \DataTypeTok{firstName}\NormalTok{: }\StringTok{'Jeremy'}\NormalTok{,}
  \DataTypeTok{lastName}\NormalTok{: }\StringTok{'Ashkenas'}\NormalTok{,}
  \DataTypeTok{email}\NormalTok{: }\StringTok{'jeremy@example.com'}
\NormalTok{\});}

\CommentTok{// create the first view instance}
\KeywordTok{var} \NormalTok{zombieView = }\KeywordTok{new} \FunctionTok{ZombieView}\NormalTok{(\{}
  \DataTypeTok{model}\NormalTok{: Person}
\NormalTok{\})}
\OtherTok{zombieView}\NormalTok{.}\FunctionTok{close}\NormalTok{(); }\CommentTok{// double-tap the zombie}

\CommentTok{// create a second view instance, re-using}
\CommentTok{// the same variable name to store it}
\NormalTok{zombieView = }\KeywordTok{new} \FunctionTok{ZombieView}\NormalTok{(\{}
  \DataTypeTok{model}\NormalTok{: Person}
\NormalTok{\})}

\OtherTok{Person}\NormalTok{.}\FunctionTok{set}\NormalTok{(}\StringTok{'email'}\NormalTok{, }\StringTok{'jeremyashkenas@example.com'}\NormalTok{);}
\end{Highlighting}
\end{Shaded}

Now we only see one alert box when this code runs.

Rather than having to manually remove these event handlers, though, we
can let Marionette do it for us.

\begin{Shaded}
\begin{Highlighting}[]
\KeywordTok{var} \NormalTok{ZombieView = }\OtherTok{Marionette}\NormalTok{.}\OtherTok{ItemView}\NormalTok{.}\FunctionTok{extend}\NormalTok{(\{}
  \DataTypeTok{template}\NormalTok{: }\StringTok{'#my-view-template'}\NormalTok{,}

  \DataTypeTok{initialize}\NormalTok{: }\KeywordTok{function}\NormalTok{() \{}

    \CommentTok{// bind the model change to re-render this view}
    \KeywordTok{this}\NormalTok{.}\FunctionTok{listenTo}\NormalTok{(}\KeywordTok{this}\NormalTok{.}\FunctionTok{model}\NormalTok{, }\StringTok{'change'}\NormalTok{, }\KeywordTok{this}\NormalTok{.}\FunctionTok{render}\NormalTok{);}

  \NormalTok{\},}

  \DataTypeTok{render}\NormalTok{: }\KeywordTok{function}\NormalTok{() \{}

    \CommentTok{// This alert is going to demonstrate a problem}
    \FunctionTok{alert}\NormalTok{(}\StringTok{'We`re rendering the view'}\NormalTok{);}

  \NormalTok{\}}
\NormalTok{\});}
\end{Highlighting}
\end{Shaded}

Notice in this case we are using a method called \texttt{listenTo}. This
method comes from Backbone.Events, and is available in all objects that
mix in Backbone.Events - including most Marionette objects. The
\texttt{listenTo} method signature is similar to that of the \texttt{on}
method, with the exception of passing the object that triggers the event
as the first parameter.

Marionette's views also provide a \texttt{close} event, in which the
event bindings that are set up with the \texttt{listenTo} are
automatically removed. This means we no longer need to define a
\texttt{close} method directly, and when we use the \texttt{listenTo}
method, we know that our events will be removed and our views will not
turn into zombies.

But how do we automate the call to \texttt{close} on a view, in the real
application? When and where do we call that? Enter the
\texttt{Marionette.Region} - an object that manages the lifecycle of an
individual view.

\subsubsection{Region Management}\label{region-management}

After a view is created, it typically needs to be placed in the DOM so
that it becomes visible. This is usually done with a jQuery selector and
setting the \texttt{html()} of the resulting object:

\begin{Shaded}
\begin{Highlighting}[]
\KeywordTok{var} \NormalTok{Joe = }\KeywordTok{new} \FunctionTok{Person}\NormalTok{(\{}
  \DataTypeTok{firstName}\NormalTok{: }\StringTok{'Joe'}\NormalTok{,}
  \DataTypeTok{lastName}\NormalTok{: }\StringTok{'Bob'}\NormalTok{,}
  \DataTypeTok{email}\NormalTok{: }\StringTok{'joebob@example.com'}
\NormalTok{\});}

\KeywordTok{var} \NormalTok{myView = }\KeywordTok{new} \FunctionTok{MyView}\NormalTok{(\{}
  \DataTypeTok{model}\NormalTok{: Joe}
\NormalTok{\})}

\OtherTok{myView}\NormalTok{.}\FunctionTok{render}\NormalTok{();}

\CommentTok{// show the view in the DOM}
\FunctionTok{$}\NormalTok{(}\StringTok{'#content'}\NormalTok{).}\FunctionTok{html}\NormalTok{(}\OtherTok{myView}\NormalTok{.}\FunctionTok{el}\NormalTok{)}
\end{Highlighting}
\end{Shaded}

This, again, is boilerplate code. We shouldn't have to manually call
\texttt{render} and manually select the DOM elements to show the view.
Furthermore, this code doesn't lend itself to closing any previous view
instance that might be attached to the DOM element we want to populate.
And we've seen the danger of zombie views already.

To solve these problems, Marionette provides a \texttt{Region} object -
an object that manages the lifecycle of individual views, displayed in a
particular DOM element.

\begin{Shaded}
\begin{Highlighting}[]
\CommentTok{// create a region instance, telling it which DOM element to manage}
\KeywordTok{var} \NormalTok{myRegion = }\KeywordTok{new} \OtherTok{Marionette}\NormalTok{.}\FunctionTok{Region}\NormalTok{(\{}
  \DataTypeTok{el}\NormalTok{: }\StringTok{'#content'}
\NormalTok{\});}

\CommentTok{// show a view in the region}
\KeywordTok{var} \NormalTok{view1 = }\KeywordTok{new} \FunctionTok{MyView}\NormalTok{(\{ }\CommentTok{/* ... */} \NormalTok{\});}
\OtherTok{myRegion}\NormalTok{.}\FunctionTok{show}\NormalTok{(view1);}

\CommentTok{// somewhere else in the code,}
\CommentTok{// show a different view}
\KeywordTok{var} \NormalTok{view2 = }\KeywordTok{new} \FunctionTok{MyView}\NormalTok{(\{ }\CommentTok{/* ... */} \NormalTok{\});}
\OtherTok{myRegion}\NormalTok{.}\FunctionTok{show}\NormalTok{(view2);}
\end{Highlighting}
\end{Shaded}

There are several things to note, here. First, we're telling the region
what DOM element to manage by specifying an \texttt{el} in the region
instance. Second, we're no longer calling the \texttt{render} method on
our views. And lastly, we're not calling \texttt{close} on our view,
either, though this is getting called for us.

When we use a region to manage the lifecycle of our views, and display
the views in the DOM, the region itself handles these concerns. By
passing a view instance into the \texttt{show} method of the region, it
will call the render method on the view for us. It will then take the
resulting \texttt{el} of the view and populate the DOM element.

The next time we call the \texttt{show} method of the region, the region
remembers that it is currently displaying a view. The region calls the
\texttt{close} method on the view, removes it from the DOM, and then
proceeds to run the render \& display code for the new view that was
passed in.

Since the region handles calling \texttt{close} for us, and we're using
the \texttt{listenTo} event binder in our view instance, we no longer
have to worry about zombie views in our application.

Regions are not limited to just Marionette views, though. Any valid
Backbone.View can be managed by a Marionette.Region. If your view
happens to have a \texttt{close} method, it will be called when the view
is closed. If not, the Backbone.View built-in method, \texttt{remove},
will be called instead.

\subsubsection{Marionette Todo app}\label{marionette-todo-app}

Having learned about Marionette's high-level concepts, let's explore
refactoring the Todo application we created in our first exercise to use
it. The complete code for this application can be found in Derick's
TodoMVC
\href{https://github.com/derickbailey/todomvc/tree/marionette/labs/architecture-examples/backbone_marionette/js}{fork}.

Our final implementation will be visually and functionally equivalent to
the original app, as seen below.

\begin{figure}[htbp]
\centering
\includegraphics{img/marionette_todo0.png}
\end{figure}

First, we define an application object representing our base TodoMVC
app. This will contain initialization code and define the default layout
regions for our app.

\textbf{TodoMVC.js:}

\begin{Shaded}
\begin{Highlighting}[]
\KeywordTok{var} \NormalTok{TodoMVC = }\KeywordTok{new} \OtherTok{Backbone}\NormalTok{.}\OtherTok{Marionette}\NormalTok{.}\FunctionTok{Application}\NormalTok{();}

\OtherTok{TodoMVC}\NormalTok{.}\FunctionTok{addRegions}\NormalTok{(\{}
  \DataTypeTok{header}\NormalTok{: }\StringTok{'#header'}\NormalTok{,}
  \DataTypeTok{main}\NormalTok{: }\StringTok{'#main'}\NormalTok{,}
  \DataTypeTok{footer}\NormalTok{: }\StringTok{'#footer'}
\NormalTok{\});}

\OtherTok{TodoMVC}\NormalTok{.}\FunctionTok{on}\NormalTok{(}\StringTok{'initialize:after'}\NormalTok{, }\KeywordTok{function}\NormalTok{() \{}
  \OtherTok{Backbone}\NormalTok{.}\OtherTok{history}\NormalTok{.}\FunctionTok{start}\NormalTok{();}
\NormalTok{\});}
\end{Highlighting}
\end{Shaded}

Regions are used to manage the content that's displayed within specific
elements, and the \texttt{addRegions} method on the \texttt{TodoMVC}
object is just a shortcut for creating \texttt{Region} objects. We
supply a jQuery selector for each region to manage (e.g.,
\texttt{\#header}, \texttt{\#main}, and \texttt{\#footer}) and then tell
the region to show various Backbone views within that region.

Once the application object has been initialized, we call
\texttt{Backbone.history.start()} to route the initial URL.

Next, we define our Layouts. A layout is a specialized type of view that
directly extends \texttt{Marionette.ItemView}. This means it's intended
to render a single template and may or may not have a model (or
\texttt{item}) associated with the template.

One of the main differences between a Layout and an \texttt{ItemView} is
that the layout contains regions. When defining a Layout, we supply it
with both a \texttt{template} and the regions that the template
contains. After rendering the layout, we can display other views within
the layout using the regions that were defined.

In our TodoMVC Layout module below, we define Layouts for:

\begin{itemize}
\itemsep1pt\parskip0pt\parsep0pt
\item
  Header: where we can create new Todos
\item
  Footer: where we summarize how many Todos are remaining/have been
  completed
\end{itemize}

This captures some of the view logic that was previously in our
\texttt{AppView} and \texttt{TodoView}.

Note that Marionette modules (such as the below) offer a simple module
system which is used to create privacy and encapsulation in Marionette
apps. These certainly don't have to be used however, and later on in
this section we'll provide links to alternative implementations using
RequireJS + AMD instead.

\textbf{TodoMVC.Layout.js:}

\begin{Shaded}
\begin{Highlighting}[]
\OtherTok{TodoMVC}\NormalTok{.}\FunctionTok{module}\NormalTok{(}\StringTok{'Layout'}\NormalTok{, }\KeywordTok{function}\NormalTok{(Layout, App, Backbone, Marionette, $, _) \{}
  
  \CommentTok{// Layout Header View}
  \CommentTok{// ------------------}

  \OtherTok{Layout}\NormalTok{.}\FunctionTok{Header} \NormalTok{= }\OtherTok{Backbone}\NormalTok{.}\OtherTok{Marionette}\NormalTok{.}\OtherTok{ItemView}\NormalTok{.}\FunctionTok{extend}\NormalTok{(\{}
    \DataTypeTok{template}\NormalTok{: }\StringTok{'#template-header'}\NormalTok{,}

    \CommentTok{// UI Bindings create cached attributes that}
    \CommentTok{// point to jQuery selected objects.}
    \DataTypeTok{ui}\NormalTok{: \{}
      \DataTypeTok{input}\NormalTok{: }\StringTok{'#new-todo'}
    \NormalTok{\},}

    \DataTypeTok{events}\NormalTok{: \{}
      \StringTok{'keypress #new-todo'}\NormalTok{: }\StringTok{'onInputKeypress'}\NormalTok{,}
      \StringTok{'blur #new-todo'}\NormalTok{: }\StringTok{'onTodoBlur'}
    \NormalTok{\},}

    \DataTypeTok{onTodoBlur}\NormalTok{: }\KeywordTok{function}\NormalTok{()\{}
      \KeywordTok{var} \NormalTok{todoText = }\KeywordTok{this}\NormalTok{.}\OtherTok{ui}\NormalTok{.}\OtherTok{input}\NormalTok{.}\FunctionTok{val}\NormalTok{().}\FunctionTok{trim}\NormalTok{();}
      \KeywordTok{this}\NormalTok{.}\FunctionTok{createTodo}\NormalTok{(todoText);}
    \NormalTok{\},}

    \DataTypeTok{onInputKeypress}\NormalTok{: }\KeywordTok{function}\NormalTok{(e) \{}
      \KeywordTok{var} \NormalTok{ENTER_KEY = }\DecValTok{13}\NormalTok{;}
      \KeywordTok{var} \NormalTok{todoText = }\KeywordTok{this}\NormalTok{.}\OtherTok{ui}\NormalTok{.}\OtherTok{input}\NormalTok{.}\FunctionTok{val}\NormalTok{().}\FunctionTok{trim}\NormalTok{();}

      \KeywordTok{if} \NormalTok{( }\OtherTok{e}\NormalTok{.}\FunctionTok{which} \NormalTok{=== ENTER_KEY && todoText ) \{}
        \KeywordTok{this}\NormalTok{.}\FunctionTok{createTodo}\NormalTok{(todoText);}
        \NormalTok{\}}
      \NormalTok{\},}

    \DataTypeTok{completeAdd}\NormalTok{: }\KeywordTok{function}\NormalTok{() \{}
      \KeywordTok{this}\NormalTok{.}\OtherTok{ui}\NormalTok{.}\OtherTok{input}\NormalTok{.}\FunctionTok{val}\NormalTok{(}\StringTok{''}\NormalTok{);}
    \NormalTok{\},}

    \DataTypeTok{createTodo}\NormalTok{: }\KeywordTok{function}\NormalTok{(todoText) \{}
      \KeywordTok{if} \NormalTok{(}\OtherTok{todoText}\NormalTok{.}\FunctionTok{trim}\NormalTok{() === }\StringTok{""}\NormalTok{)\{ }\KeywordTok{return}\NormalTok{; \}}

      \KeywordTok{this}\NormalTok{.}\OtherTok{collection}\NormalTok{.}\FunctionTok{create}\NormalTok{(\{}
        \DataTypeTok{title}\NormalTok{: todoText}
      \NormalTok{\});}

      \KeywordTok{this}\NormalTok{.}\FunctionTok{completeAdd}\NormalTok{();}
    \NormalTok{\}}
  \NormalTok{\});}

  \CommentTok{// Layout Footer View}
  \CommentTok{// ------------------}

  \OtherTok{Layout}\NormalTok{.}\FunctionTok{Footer} \NormalTok{= }\OtherTok{Marionette}\NormalTok{.}\OtherTok{Layout}\NormalTok{.}\FunctionTok{extend}\NormalTok{(\{}
    \DataTypeTok{template}\NormalTok{: }\StringTok{'#template-footer'}\NormalTok{,}

    \CommentTok{// UI Bindings create cached attributes that}
    \CommentTok{// point to jQuery selected objects.}
    \DataTypeTok{ui}\NormalTok{: \{}
      \DataTypeTok{todoCount}\NormalTok{: }\StringTok{'#todo-count .count'}\NormalTok{,}
      \DataTypeTok{todoCountLabel}\NormalTok{: }\StringTok{'#todo-count .label'}\NormalTok{,}
      \DataTypeTok{clearCount}\NormalTok{: }\StringTok{'#clear-completed .count'}\NormalTok{,}
      \DataTypeTok{filters}\NormalTok{: }\StringTok{"#filters a"}
    \NormalTok{\},}

    \DataTypeTok{events}\NormalTok{: \{}
      \StringTok{"click #clear-completed"}\NormalTok{: }\StringTok{"onClearClick"}
    \NormalTok{\},}

    \DataTypeTok{initialize}\NormalTok{: }\KeywordTok{function}\NormalTok{() \{}
      \KeywordTok{this}\NormalTok{.}\FunctionTok{bindTo}\NormalTok{( }\OtherTok{App}\NormalTok{.}\FunctionTok{vent}\NormalTok{, }\StringTok{"todoList: filter"}\NormalTok{, }\KeywordTok{this}\NormalTok{.}\FunctionTok{updateFilterSelection}\NormalTok{, }\KeywordTok{this} \NormalTok{);}
      \KeywordTok{this}\NormalTok{.}\FunctionTok{bindTo}\NormalTok{( }\KeywordTok{this}\NormalTok{.}\FunctionTok{collection}\NormalTok{, }\StringTok{'all'}\NormalTok{, }\KeywordTok{this}\NormalTok{.}\FunctionTok{updateCount}\NormalTok{, }\KeywordTok{this} \NormalTok{);}
    \NormalTok{\},}

    \DataTypeTok{onRender}\NormalTok{: }\KeywordTok{function}\NormalTok{() \{}
      \KeywordTok{this}\NormalTok{.}\FunctionTok{updateCount}\NormalTok{();}
    \NormalTok{\},}

    \DataTypeTok{updateCount}\NormalTok{: }\KeywordTok{function}\NormalTok{() \{}
      \KeywordTok{var} \NormalTok{activeCount = }\KeywordTok{this}\NormalTok{.}\OtherTok{collection}\NormalTok{.}\FunctionTok{getActive}\NormalTok{().}\FunctionTok{length}\NormalTok{,}
      \NormalTok{completedCount = }\KeywordTok{this}\NormalTok{.}\OtherTok{collection}\NormalTok{.}\FunctionTok{getCompleted}\NormalTok{().}\FunctionTok{length}\NormalTok{;}
      \KeywordTok{this}\NormalTok{.}\OtherTok{ui}\NormalTok{.}\OtherTok{todoCount}\NormalTok{.}\FunctionTok{html}\NormalTok{(activeCount);}

      \KeywordTok{this}\NormalTok{.}\OtherTok{ui}\NormalTok{.}\OtherTok{todoCountLabel}\NormalTok{.}\FunctionTok{html}\NormalTok{(activeCount === }\DecValTok{1} \NormalTok{? }\StringTok{'item'} \NormalTok{: }\StringTok{'items'}\NormalTok{);}
      \KeywordTok{this}\NormalTok{.}\OtherTok{ui}\NormalTok{.}\OtherTok{clearCount}\NormalTok{.}\FunctionTok{html}\NormalTok{(completedCount === }\DecValTok{0} \NormalTok{? }\StringTok{''} \NormalTok{: }\StringTok{'('} \NormalTok{+ completedCount + }\StringTok{')'}\NormalTok{);}
    \NormalTok{\},}

    \DataTypeTok{updateFilterSelection}\NormalTok{: }\KeywordTok{function}\NormalTok{( filter ) \{}
      \KeywordTok{this}\NormalTok{.}\OtherTok{ui}\NormalTok{.}\FunctionTok{filters}
        \NormalTok{.}\FunctionTok{removeClass}\NormalTok{(}\StringTok{'selected'}\NormalTok{)}
        \NormalTok{.}\FunctionTok{filter}\NormalTok{( }\StringTok{'[href="#'} \NormalTok{+ filter + }\StringTok{'"]'}\NormalTok{)}
        \NormalTok{.}\FunctionTok{addClass}\NormalTok{( }\StringTok{'selected'} \NormalTok{);}
    \NormalTok{\},}

    \DataTypeTok{onClearClick}\NormalTok{: }\KeywordTok{function}\NormalTok{() \{}
      \KeywordTok{var} \NormalTok{completed = }\KeywordTok{this}\NormalTok{.}\OtherTok{collection}\NormalTok{.}\FunctionTok{getCompleted}\NormalTok{();}
      \OtherTok{completed}\NormalTok{.}\FunctionTok{forEach}\NormalTok{(}\KeywordTok{function} \FunctionTok{destroy}\NormalTok{(todo) \{}
        \OtherTok{todo}\NormalTok{.}\FunctionTok{destroy}\NormalTok{();}
      \NormalTok{\});}
    \NormalTok{\}}
  \NormalTok{\});}

\NormalTok{\});}
\end{Highlighting}
\end{Shaded}

Next, we tackle application routing and workflow, such as controlling
Layouts in the page which can be shown or hidden.

Recall how Backbone routes trigger methods within the Router as shown
below in our original Workspace router from our first exercise:

\begin{Shaded}
\begin{Highlighting}[]
  \KeywordTok{var} \NormalTok{Workspace = }\OtherTok{Backbone}\NormalTok{.}\OtherTok{Router}\NormalTok{.}\FunctionTok{extend}\NormalTok{(\{}
    \DataTypeTok{routes}\NormalTok{: \{}
      \StringTok{'*filter'}\NormalTok{: }\StringTok{'setFilter'}
    \NormalTok{\},}

    \DataTypeTok{setFilter}\NormalTok{: }\KeywordTok{function}\NormalTok{(param) \{}
      \CommentTok{// Set the current filter to be used}
      \OtherTok{window}\NormalTok{.}\OtherTok{app}\NormalTok{.}\FunctionTok{TodoFilter} \NormalTok{= }\OtherTok{param}\NormalTok{.}\FunctionTok{trim}\NormalTok{() || }\StringTok{''}\NormalTok{;}

      \CommentTok{// Trigger a collection filter event, causing hiding/unhiding}
      \CommentTok{// of Todo view items}
      \OtherTok{window}\NormalTok{.}\OtherTok{app}\NormalTok{.}\OtherTok{Todos}\NormalTok{.}\FunctionTok{trigger}\NormalTok{(}\StringTok{'filter'}\NormalTok{);}
    \NormalTok{\}}
  \NormalTok{\});}
\end{Highlighting}
\end{Shaded}

Marionette uses the concept of an AppRouter to simplify routing. This
reduces the boilerplate for handling route events and allows routers to
be configured to call methods on an object directly. We configure our
AppRouter using \texttt{appRoutes} which replaces the
\texttt{'*filter': 'setFilter'} route defined in our original router and
invokes a method on our Controller.

The TodoList Controller, also found in this next code block, handles
some of the remaining visibility logic originally found in
\texttt{AppView} and \texttt{TodoView}, albeit using very readable
Layouts.

\textbf{TodoMVC.TodoList.js:}

\begin{Shaded}
\begin{Highlighting}[]
\OtherTok{TodoMVC}\NormalTok{.}\FunctionTok{module}\NormalTok{(}\StringTok{'TodoList'}\NormalTok{, }\KeywordTok{function}\NormalTok{(TodoList, App, Backbone, Marionette, $, _) \{}

  \CommentTok{// TodoList Router}
  \CommentTok{// ---------------}
  \CommentTok{//}
  \CommentTok{// Handle routes to show the active vs complete todo items}

  \OtherTok{TodoList}\NormalTok{.}\FunctionTok{Router} \NormalTok{= }\OtherTok{Marionette}\NormalTok{.}\OtherTok{AppRouter}\NormalTok{.}\FunctionTok{extend}\NormalTok{(\{}
    \DataTypeTok{appRoutes}\NormalTok{: \{}
      \StringTok{'*filter'}\NormalTok{: }\StringTok{'filterItems'}
    \NormalTok{\}}
  \NormalTok{\});}

  \CommentTok{// TodoList Controller (Mediator)}
  \CommentTok{// ------------------------------}
  \CommentTok{//}
  \CommentTok{// Control the workflow and logic that exists at the application}
  \CommentTok{// level, above the implementation detail of views and models}
  
  \OtherTok{TodoList}\NormalTok{.}\FunctionTok{Controller} \NormalTok{= }\KeywordTok{function}\NormalTok{() \{}
    \KeywordTok{this}\NormalTok{.}\FunctionTok{todoList} \NormalTok{= }\KeywordTok{new} \OtherTok{App}\NormalTok{.}\OtherTok{Todos}\NormalTok{.}\FunctionTok{TodoList}\NormalTok{();}
  \NormalTok{\};}

  \OtherTok{_}\NormalTok{.}\FunctionTok{extend}\NormalTok{(}\OtherTok{TodoList}\NormalTok{.}\OtherTok{Controller}\NormalTok{.}\FunctionTok{prototype}\NormalTok{, \{}

    \CommentTok{// Start the app by showing the appropriate views}
    \CommentTok{// and fetching the list of todo items, if there are any}
    \DataTypeTok{start}\NormalTok{: }\KeywordTok{function}\NormalTok{() \{}
      \KeywordTok{this}\NormalTok{.}\FunctionTok{showHeader}\NormalTok{(}\KeywordTok{this}\NormalTok{.}\FunctionTok{todoList}\NormalTok{);}
      \KeywordTok{this}\NormalTok{.}\FunctionTok{showFooter}\NormalTok{(}\KeywordTok{this}\NormalTok{.}\FunctionTok{todoList}\NormalTok{);}
      \KeywordTok{this}\NormalTok{.}\FunctionTok{showTodoList}\NormalTok{(}\KeywordTok{this}\NormalTok{.}\FunctionTok{todoList}\NormalTok{);}
      
  \OtherTok{App}\NormalTok{.}\FunctionTok{bindTo}\NormalTok{(}\KeywordTok{this}\NormalTok{.}\FunctionTok{todoList}\NormalTok{, }\StringTok{'reset add remove'}\NormalTok{, }\KeywordTok{this}\NormalTok{.}\FunctionTok{toggleFooter}\NormalTok{, }\KeywordTok{this}\NormalTok{);}
      \KeywordTok{this}\NormalTok{.}\OtherTok{todoList}\NormalTok{.}\FunctionTok{fetch}\NormalTok{();}
    \NormalTok{\},}

    \DataTypeTok{showHeader}\NormalTok{: }\KeywordTok{function}\NormalTok{(todoList) \{}
      \KeywordTok{var} \NormalTok{header = }\KeywordTok{new} \OtherTok{App}\NormalTok{.}\OtherTok{Layout}\NormalTok{.}\FunctionTok{Header}\NormalTok{(\{}
        \DataTypeTok{collection}\NormalTok{: todoList}
      \NormalTok{\});}
      \OtherTok{App}\NormalTok{.}\OtherTok{header}\NormalTok{.}\FunctionTok{show}\NormalTok{(header);}
    \NormalTok{\},}

    \DataTypeTok{showFooter}\NormalTok{: }\KeywordTok{function}\NormalTok{(todoList) \{}
      \KeywordTok{var} \NormalTok{footer = }\KeywordTok{new} \OtherTok{App}\NormalTok{.}\OtherTok{Layout}\NormalTok{.}\FunctionTok{Footer}\NormalTok{(\{}
        \DataTypeTok{collection}\NormalTok{: todoList}
      \NormalTok{\});}
      \OtherTok{App}\NormalTok{.}\OtherTok{footer}\NormalTok{.}\FunctionTok{show}\NormalTok{(footer);}
    \NormalTok{\},}

    \DataTypeTok{showTodoList}\NormalTok{: }\KeywordTok{function}\NormalTok{(todoList) \{}
      \OtherTok{App}\NormalTok{.}\OtherTok{main}\NormalTok{.}\FunctionTok{show}\NormalTok{(}\KeywordTok{new} \OtherTok{TodoList}\NormalTok{.}\OtherTok{Views}\NormalTok{.}\FunctionTok{ListView}\NormalTok{(\{}
        \DataTypeTok{collection}\NormalTok{: todoList}
      \NormalTok{\}));}
    \NormalTok{\},}
    
    \DataTypeTok{toggleFooter}\NormalTok{: }\KeywordTok{function}\NormalTok{() \{}
      \OtherTok{App}\NormalTok{.}\OtherTok{footer}\NormalTok{.}\OtherTok{$el}\NormalTok{.}\FunctionTok{toggle}\NormalTok{(}\KeywordTok{this}\NormalTok{.}\OtherTok{todoList}\NormalTok{.}\FunctionTok{length}\NormalTok{);}
    \NormalTok{\},}

    \CommentTok{// Set the filter to show complete or all items}
    \DataTypeTok{filterItems}\NormalTok{: }\KeywordTok{function}\NormalTok{(filter) \{}
      \OtherTok{App}\NormalTok{.}\OtherTok{vent}\NormalTok{.}\FunctionTok{trigger}\NormalTok{(}\StringTok{'todoList:filter'}\NormalTok{, }\OtherTok{filter}\NormalTok{.}\FunctionTok{trim}\NormalTok{() || }\StringTok{''}\NormalTok{);}
    \NormalTok{\}}
  \NormalTok{\});}

  \CommentTok{// TodoList Initializer}
  \CommentTok{// --------------------}
  \CommentTok{//}
  \CommentTok{// Get the TodoList up and running by initializing the mediator}
  \CommentTok{// when the the application is started, pulling in all of the}
  \CommentTok{// existing Todo items and displaying them.}
  
  \OtherTok{TodoList}\NormalTok{.}\FunctionTok{addInitializer}\NormalTok{(}\KeywordTok{function}\NormalTok{() \{}

    \KeywordTok{var} \NormalTok{controller = }\KeywordTok{new} \OtherTok{TodoList}\NormalTok{.}\FunctionTok{Controller}\NormalTok{();}
    \KeywordTok{new} \OtherTok{TodoList}\NormalTok{.}\FunctionTok{Router}\NormalTok{(\{}
      \DataTypeTok{controller}\NormalTok{: controller}
    \NormalTok{\});}

    \OtherTok{controller}\NormalTok{.}\FunctionTok{start}\NormalTok{();}

  \NormalTok{\});}

\NormalTok{\});}
\end{Highlighting}
\end{Shaded}

\paragraph{Controllers}\label{controllers-1}

In this particular app, note that Controllers don't add a great deal to
the overall workflow. In general, Marionette's philosophy on routers is
that they should be an afterthought in the implementation of
applications. Quite often, we've seen developers abuse Backbone's
routing system by making it the sole controller of the entire
application workflow and logic.

This inevitably leads to mashing every possible combination of code into
the router methods - view creation, model loading, coordinating
different parts of the app, etc. Developers such as Derick view this as
a violation of the
\href{http://en.wikipedia.org/wiki/Single_responsibility_principle}{single-responsibility
principle} (SRP) and separation of concerns.

Backbone's router and history exist to deal with a specific aspect of
browsers - managing the forward and back buttons. Marionette's
philosophy is that it should be limited to that, with the code that gets
executed by the navigation being somewhere else. This allows the
application to be used with or without a router. We can call a
controller's ``show'' method from a button click, from an application
event handler, or from a router, and we will end up with the same
application state no matter how we called that method.

Derick has written extensively about his thoughts on this topic, which
you can read more about on his blog:

\begin{itemize}
\itemsep1pt\parskip0pt\parsep0pt
\item
  \url{http://lostechies.com/derickbailey/2011/12/27/the-responsibilities-of-the-various-pieces-of-backbone-js/}
\item
  \url{http://lostechies.com/derickbailey/2012/01/02/reducing-backbone-routers-to-nothing-more-than-configuration/}
\item
  \url{http://lostechies.com/derickbailey/2012/02/06/3-stages-of-a-backbone-applications-startup/}
\end{itemize}

\paragraph{CompositeView}\label{compositeview}

Our next task is defining the actual views for individual Todo items and
lists of items in our TodoMVC application. For this, we make use of
Marionette's \texttt{CompositeView}s. The idea behind a CompositeView is
that it represents a visualization of a composite or hierarchical
structure of leaves (or nodes) and branches.

Think of these views as being a hierarchy of parent-child models, and
recursive by default. The same CompositeView type will be used to render
each item in a collection that is handled by the composite view. For
non-recursive hierarchies, we are able to override the item view by
defining an \texttt{itemView} attribute.

For our Todo List Item View, we define it as an ItemView, then our Todo
List View is a CompositeView where we override the \texttt{itemView}
setting and tell it to use the Todo List Item View for each item in the
collection.

TodoMVC.TodoList.Views.js

\begin{Shaded}
\begin{Highlighting}[]
\OtherTok{TodoMVC}\NormalTok{.}\FunctionTok{module}\NormalTok{(}\StringTok{'TodoList.Views'}\NormalTok{, }\KeywordTok{function}\NormalTok{(Views, App, Backbone, Marionette, $, _) \{}

  \CommentTok{// Todo List Item View}
  \CommentTok{// -------------------}
  \CommentTok{//}
  \CommentTok{// Display an individual todo item, and respond to changes}
  \CommentTok{// that are made to the item, including marking completed.}

  \OtherTok{Views}\NormalTok{.}\FunctionTok{ItemView} \NormalTok{= }\OtherTok{Marionette}\NormalTok{.}\OtherTok{ItemView}\NormalTok{.}\FunctionTok{extend}\NormalTok{(\{}
      \DataTypeTok{tagName}\NormalTok{: }\StringTok{'li'}\NormalTok{,}
      \DataTypeTok{template}\NormalTok{: }\StringTok{'#template-todoItemView'}\NormalTok{,}

      \DataTypeTok{ui}\NormalTok{: \{}
        \DataTypeTok{edit}\NormalTok{: }\StringTok{'.edit'}
      \NormalTok{\},}

      \DataTypeTok{events}\NormalTok{: \{}
        \StringTok{'click .destroy'}\NormalTok{: }\StringTok{'destroy'}\NormalTok{,}
        \StringTok{'dblclick label'}\NormalTok{: }\StringTok{'onEditClick'}\NormalTok{,}
        \StringTok{'keypress .edit'}\NormalTok{: }\StringTok{'onEditKeypress'}\NormalTok{,}
        \StringTok{'blur .edit'}    \NormalTok{: }\StringTok{'onEditBlur'}\NormalTok{,}
        \StringTok{'click .toggle'} \NormalTok{: }\StringTok{'toggle'}
      \NormalTok{\},}

      \DataTypeTok{initialize}\NormalTok{: }\KeywordTok{function}\NormalTok{() \{}
        \KeywordTok{this}\NormalTok{.}\FunctionTok{bindTo}\NormalTok{(}\KeywordTok{this}\NormalTok{.}\FunctionTok{model}\NormalTok{, }\StringTok{'change'}\NormalTok{, }\KeywordTok{this}\NormalTok{.}\FunctionTok{render}\NormalTok{, }\KeywordTok{this}\NormalTok{);}
      \NormalTok{\},}

      \DataTypeTok{onRender}\NormalTok{: }\KeywordTok{function}\NormalTok{() \{}
        \KeywordTok{this}\NormalTok{.}\OtherTok{$el}\NormalTok{.}\FunctionTok{removeClass}\NormalTok{( }\StringTok{'active completed'} \NormalTok{);}
      
        \KeywordTok{if} \NormalTok{( }\KeywordTok{this}\NormalTok{.}\OtherTok{model}\NormalTok{.}\FunctionTok{get}\NormalTok{( }\StringTok{'completed'} \NormalTok{)) \{}
          \KeywordTok{this}\NormalTok{.}\OtherTok{$el}\NormalTok{.}\FunctionTok{addClass}\NormalTok{( }\StringTok{'completed'} \NormalTok{);}
        \NormalTok{\} }\KeywordTok{else} \NormalTok{\{ }
          \KeywordTok{this}\NormalTok{.}\OtherTok{$el}\NormalTok{.}\FunctionTok{addClass}\NormalTok{( }\StringTok{'active'} \NormalTok{);}
        \NormalTok{\}}
      \NormalTok{\},}

      \DataTypeTok{destroy}\NormalTok{: }\KeywordTok{function}\NormalTok{() \{}
        \KeywordTok{this}\NormalTok{.}\OtherTok{model}\NormalTok{.}\FunctionTok{destroy}\NormalTok{();}
      \NormalTok{\},}

      \DataTypeTok{toggle}\NormalTok{: }\KeywordTok{function}\NormalTok{() \{}
        \KeywordTok{this}\NormalTok{.}\OtherTok{model}\NormalTok{.}\FunctionTok{toggle}\NormalTok{().}\FunctionTok{save}\NormalTok{();}
      \NormalTok{\},}

      \DataTypeTok{onEditClick}\NormalTok{: }\KeywordTok{function}\NormalTok{() \{}
        \KeywordTok{this}\NormalTok{.}\OtherTok{$el}\NormalTok{.}\FunctionTok{addClass}\NormalTok{(}\StringTok{'editing'}\NormalTok{);}
        \KeywordTok{this}\NormalTok{.}\OtherTok{ui}\NormalTok{.}\OtherTok{edit}\NormalTok{.}\FunctionTok{focus}\NormalTok{();}
      \NormalTok{\},}
      
      \DataTypeTok{updateTodo }\NormalTok{: }\KeywordTok{function}\NormalTok{() \{}
        \KeywordTok{var} \NormalTok{todoText = }\KeywordTok{this}\NormalTok{.}\OtherTok{ui}\NormalTok{.}\OtherTok{edit}\NormalTok{.}\FunctionTok{val}\NormalTok{();}
        \KeywordTok{if} \NormalTok{(todoText === }\StringTok{''}\NormalTok{) \{}
          \KeywordTok{return} \KeywordTok{this}\NormalTok{.}\FunctionTok{destroy}\NormalTok{();}
        \NormalTok{\}}
        \KeywordTok{this}\NormalTok{.}\FunctionTok{setTodoText}\NormalTok{(todoText);}
        \KeywordTok{this}\NormalTok{.}\FunctionTok{completeEdit}\NormalTok{();}
      \NormalTok{\},}

      \DataTypeTok{onEditBlur}\NormalTok{: }\KeywordTok{function}\NormalTok{(e)\{}
        \KeywordTok{this}\NormalTok{.}\FunctionTok{updateTodo}\NormalTok{();}
      \NormalTok{\},}

      \DataTypeTok{onEditKeypress}\NormalTok{: }\KeywordTok{function}\NormalTok{(e) \{}
        \KeywordTok{var} \NormalTok{ENTER_KEY = }\DecValTok{13}\NormalTok{;}
        \KeywordTok{var} \NormalTok{todoText = }\KeywordTok{this}\NormalTok{.}\OtherTok{ui}\NormalTok{.}\OtherTok{edit}\NormalTok{.}\FunctionTok{val}\NormalTok{().}\FunctionTok{trim}\NormalTok{();}

        \KeywordTok{if} \NormalTok{( }\OtherTok{e}\NormalTok{.}\FunctionTok{which} \NormalTok{=== ENTER_KEY && todoText ) \{}
          \KeywordTok{this}\NormalTok{.}\OtherTok{model}\NormalTok{.}\FunctionTok{set}\NormalTok{(}\StringTok{'title'}\NormalTok{, todoText).}\FunctionTok{save}\NormalTok{();}
          \KeywordTok{this}\NormalTok{.}\OtherTok{$el}\NormalTok{.}\FunctionTok{removeClass}\NormalTok{(}\StringTok{'editing'}\NormalTok{);}
        \NormalTok{\}}
      \NormalTok{\},}
      
      \DataTypeTok{setTodoText}\NormalTok{: }\KeywordTok{function}\NormalTok{(todoText)\{}
        \KeywordTok{if} \NormalTok{(}\OtherTok{todoText}\NormalTok{.}\FunctionTok{trim}\NormalTok{() === }\StringTok{""}\NormalTok{)\{ }\KeywordTok{return}\NormalTok{; \}}
        \KeywordTok{this}\NormalTok{.}\OtherTok{model}\NormalTok{.}\FunctionTok{set}\NormalTok{(}\StringTok{'title'}\NormalTok{, todoText).}\FunctionTok{save}\NormalTok{();}
      \NormalTok{\},}

      \DataTypeTok{completeEdit}\NormalTok{: }\KeywordTok{function}\NormalTok{()\{}
        \KeywordTok{this}\NormalTok{.}\OtherTok{$el}\NormalTok{.}\FunctionTok{removeClass}\NormalTok{(}\StringTok{'editing'}\NormalTok{);}
      \NormalTok{\}}
  \NormalTok{\});}

  \CommentTok{// Item List View}
  \CommentTok{// --------------}
  \CommentTok{//}
  \CommentTok{// Controls the rendering of the list of items, including the}
  \CommentTok{// filtering of active vs completed items for display.}

  \OtherTok{Views}\NormalTok{.}\FunctionTok{ListView} \NormalTok{= }\OtherTok{Backbone}\NormalTok{.}\OtherTok{Marionette}\NormalTok{.}\OtherTok{CompositeView}\NormalTok{.}\FunctionTok{extend}\NormalTok{(\{}
      \DataTypeTok{template}\NormalTok{: }\StringTok{'#template-todoListCompositeView'}\NormalTok{,}
      \DataTypeTok{itemView}\NormalTok{: }\OtherTok{Views}\NormalTok{.}\FunctionTok{ItemView}\NormalTok{,}
      \DataTypeTok{itemViewContainer}\NormalTok{: }\StringTok{'#todo-list'}\NormalTok{,}

      \DataTypeTok{ui}\NormalTok{: \{}
        \DataTypeTok{toggle}\NormalTok{: }\StringTok{'#toggle-all'}
      \NormalTok{\},}

      \DataTypeTok{events}\NormalTok{: \{}
        \StringTok{'click #toggle-all'}\NormalTok{: }\StringTok{'onToggleAllClick'}
      \NormalTok{\},}

      \DataTypeTok{initialize}\NormalTok{: }\KeywordTok{function}\NormalTok{() \{}
        \KeywordTok{this}\NormalTok{.}\FunctionTok{bindTo}\NormalTok{(}\KeywordTok{this}\NormalTok{.}\FunctionTok{collection}\NormalTok{, }\StringTok{'all'}\NormalTok{, }\KeywordTok{this}\NormalTok{.}\FunctionTok{update}\NormalTok{, }\KeywordTok{this}\NormalTok{);}
      \NormalTok{\},}

      \DataTypeTok{onRender}\NormalTok{: }\KeywordTok{function}\NormalTok{() \{}
        \KeywordTok{this}\NormalTok{.}\FunctionTok{update}\NormalTok{();}
      \NormalTok{\},}

      \DataTypeTok{update}\NormalTok{: }\KeywordTok{function}\NormalTok{() \{}
        \KeywordTok{function} \FunctionTok{reduceCompleted}\NormalTok{(left, right) \{ }
          \KeywordTok{return} \NormalTok{left && }\OtherTok{right}\NormalTok{.}\FunctionTok{get}\NormalTok{(}\StringTok{'completed'}\NormalTok{); }
        \NormalTok{\}}
        
        \KeywordTok{var} \NormalTok{allCompleted = }\KeywordTok{this}\NormalTok{.}\OtherTok{collection}\NormalTok{.}\FunctionTok{reduce}\NormalTok{(reduceCompleted,}\KeywordTok{true}\NormalTok{);}
        \KeywordTok{this}\NormalTok{.}\OtherTok{ui}\NormalTok{.}\OtherTok{toggle}\NormalTok{.}\FunctionTok{prop}\NormalTok{(}\StringTok{'checked'}\NormalTok{, allCompleted);}
        \KeywordTok{this}\NormalTok{.}\OtherTok{$el}\NormalTok{.}\FunctionTok{parent}\NormalTok{().}\FunctionTok{toggle}\NormalTok{(!!}\KeywordTok{this}\NormalTok{.}\OtherTok{collection}\NormalTok{.}\FunctionTok{length}\NormalTok{);}
      \NormalTok{\},}

      \DataTypeTok{onToggleAllClick}\NormalTok{: }\KeywordTok{function}\NormalTok{(e) \{}
        \KeywordTok{var} \NormalTok{isChecked = }\OtherTok{e}\NormalTok{.}\OtherTok{currentTarget}\NormalTok{.}\FunctionTok{checked}\NormalTok{;}
        \KeywordTok{this}\NormalTok{.}\OtherTok{collection}\NormalTok{.}\FunctionTok{each}\NormalTok{(}\KeywordTok{function}\NormalTok{(todo) \{}
          \OtherTok{todo}\NormalTok{.}\FunctionTok{save}\NormalTok{(\{}\StringTok{'completed'}\NormalTok{: isChecked\});}
        \NormalTok{\});}
      \NormalTok{\}}
  \NormalTok{\});}

  \CommentTok{// Application Event Handlers}
  \CommentTok{// --------------------------}
  \CommentTok{//}
  \CommentTok{// Handler for filtering the list of items by showing and}
  \CommentTok{// hiding through the use of various CSS classes}
  
  \OtherTok{App}\NormalTok{.}\OtherTok{vent}\NormalTok{.}\FunctionTok{on}\NormalTok{(}\StringTok{'todoList:filter'}\NormalTok{,}\KeywordTok{function}\NormalTok{(filter) \{}
    \NormalTok{filter = filter || }\StringTok{'all'}\NormalTok{;}
    \FunctionTok{$}\NormalTok{(}\StringTok{'#todoapp'}\NormalTok{).}\FunctionTok{attr}\NormalTok{(}\StringTok{'class'}\NormalTok{, }\StringTok{'filter-'} \NormalTok{+ filter);}
  \NormalTok{\});}

\NormalTok{\});}
\end{Highlighting}
\end{Shaded}

At the end of the last code block, you will also notice an event handler
using \texttt{vent}. This is an event aggregator that allows us to
handle \texttt{filterItem} triggers from our TodoList controller.

Finally, we define the model and collection for representing our Todo
items. These are semantically not very different from the original
versions we used in our first exercise and have been re-written to
better fit in with Derick's preferred style of coding.

\textbf{TodoMVC.Todos.js:}

\begin{Shaded}
\begin{Highlighting}[]
\OtherTok{TodoMVC}\NormalTok{.}\FunctionTok{module}\NormalTok{(}\StringTok{'Todos'}\NormalTok{, }\KeywordTok{function}\NormalTok{(Todos, App, Backbone, Marionette, $, _) \{}

  \CommentTok{// Local Variables}
  \CommentTok{// ---------------}

  \KeywordTok{var} \NormalTok{localStorageKey = }\StringTok{'todos-backbone-marionettejs'}\NormalTok{;}

  \CommentTok{// Todo Model}
  \CommentTok{// ----------}
  
  \OtherTok{Todos}\NormalTok{.}\FunctionTok{Todo} \NormalTok{= }\OtherTok{Backbone}\NormalTok{.}\OtherTok{Model}\NormalTok{.}\FunctionTok{extend}\NormalTok{(\{}
    \DataTypeTok{localStorage}\NormalTok{: }\KeywordTok{new} \OtherTok{Backbone}\NormalTok{.}\FunctionTok{LocalStorage}\NormalTok{(localStorageKey),}

    \DataTypeTok{defaults}\NormalTok{: \{}
      \DataTypeTok{title}\NormalTok{: }\StringTok{''}\NormalTok{,}
      \DataTypeTok{completed}\NormalTok{: }\KeywordTok{false}\NormalTok{,}
      \DataTypeTok{created}\NormalTok{: }\DecValTok{0}
    \NormalTok{\},}

    \DataTypeTok{initialize}\NormalTok{: }\KeywordTok{function}\NormalTok{() \{}
      \KeywordTok{if} \NormalTok{(}\KeywordTok{this}\NormalTok{.}\FunctionTok{isNew}\NormalTok{()) \{}
        \KeywordTok{this}\NormalTok{.}\FunctionTok{set}\NormalTok{(}\StringTok{'created'}\NormalTok{, }\OtherTok{Date}\NormalTok{.}\FunctionTok{now}\NormalTok{());}
      \NormalTok{\}}
    \NormalTok{\},}

    \DataTypeTok{toggle}\NormalTok{: }\KeywordTok{function}\NormalTok{() \{}
      \KeywordTok{return} \KeywordTok{this}\NormalTok{.}\FunctionTok{set}\NormalTok{(}\StringTok{'completed'}\NormalTok{, !}\KeywordTok{this}\NormalTok{.}\FunctionTok{isCompleted}\NormalTok{());}
    \NormalTok{\},}

    \DataTypeTok{isCompleted}\NormalTok{: }\KeywordTok{function}\NormalTok{() \{ }
      \KeywordTok{return} \KeywordTok{this}\NormalTok{.}\FunctionTok{get}\NormalTok{(}\StringTok{'completed'}\NormalTok{); }
    \NormalTok{\}}
  \NormalTok{\});}

  \CommentTok{// Todo Collection}
  \CommentTok{// ---------------}

  \OtherTok{Todos}\NormalTok{.}\FunctionTok{TodoList} \NormalTok{= }\OtherTok{Backbone}\NormalTok{.}\OtherTok{Collection}\NormalTok{.}\FunctionTok{extend}\NormalTok{(\{}
    \DataTypeTok{model}\NormalTok{: }\OtherTok{Todos}\NormalTok{.}\FunctionTok{Todo}\NormalTok{,}

    \DataTypeTok{localStorage}\NormalTok{: }\KeywordTok{new} \OtherTok{Backbone}\NormalTok{.}\FunctionTok{LocalStorage}\NormalTok{(localStorageKey),}

    \DataTypeTok{getCompleted}\NormalTok{: }\KeywordTok{function}\NormalTok{() \{}
      \KeywordTok{return} \KeywordTok{this}\NormalTok{.}\FunctionTok{filter}\NormalTok{(}\KeywordTok{this}\NormalTok{.}\FunctionTok{_isCompleted}\NormalTok{);}
    \NormalTok{\},}

    \DataTypeTok{getActive}\NormalTok{: }\KeywordTok{function}\NormalTok{() \{}
      \KeywordTok{return} \KeywordTok{this}\NormalTok{.}\FunctionTok{reject}\NormalTok{(}\KeywordTok{this}\NormalTok{.}\FunctionTok{_isCompleted}\NormalTok{);}
    \NormalTok{\},}

    \DataTypeTok{comparator}\NormalTok{: }\KeywordTok{function}\NormalTok{(todo) \{}
      \KeywordTok{return} \OtherTok{todo}\NormalTok{.}\FunctionTok{get}\NormalTok{(}\StringTok{'created'}\NormalTok{);}
    \NormalTok{\},}

    \DataTypeTok{_isCompleted}\NormalTok{: }\KeywordTok{function}\NormalTok{(todo) \{}
      \KeywordTok{return} \OtherTok{todo}\NormalTok{.}\FunctionTok{isCompleted}\NormalTok{();}
    \NormalTok{\}}
  \NormalTok{\});}

\NormalTok{\});}
\end{Highlighting}
\end{Shaded}

We finally kick-start everything off in our application index file, by
calling \texttt{start} on our main application object:

Initialization:

\begin{Shaded}
\begin{Highlighting}[]
      \FunctionTok{$}\NormalTok{(}\KeywordTok{function}\NormalTok{() \{}
        \CommentTok{// Start the TodoMVC app (defined in js/TodoMVC.js)}
        \OtherTok{TodoMVC}\NormalTok{.}\FunctionTok{start}\NormalTok{();}
      \NormalTok{\});}
\end{Highlighting}
\end{Shaded}

And that's it!

\subsubsection{Is the Marionette implementation of the Todo app more
maintainable?}\label{is-the-marionette-implementation-of-the-todo-app-more-maintainable}

Derick feels that maintainability largely comes down to modularity,
separating responsibilities (Single Responsibility Principle and
Separation of Concerns) by using patterns to keep concerns from being
mixed together. It can, however, be difficult to simply extract things
into separate modules for the sake of extraction, abstraction, or
dividing the concept down into its simplest parts.

The Single Responsibility Principle (SRP) tells us quite the opposite -
that we need to understand the context in which things change. What
parts always change together, in \emph{this} system? What parts can
change independently? Without knowing this, we won't know what pieces
should be broken out into separate components and modules versus put
together into the same module or object.

The way Derick organizes his apps into modules is by creating a
breakdown of concepts at each level. A higher level module is a higher
level of concern - an aggregation of responsibilities. Each
responsibility is broken down into an expressive API set that is
implemented by lower level modules (Dependency Inversion Principle).
These are coordinated through a mediator - which he typically refers to
as the Controller in a module.

The way Derick organizes his files also plays directly into
maintainability and he has also written posts about the importance of
keeping a sane application folder structure that I recommend reading:

\begin{itemize}
\itemsep1pt\parskip0pt\parsep0pt
\item
  \url{http://lostechies.com/derickbailey/2012/02/02/javascript-file-folder-structures-just-pick-one/}
\item
  \url{http://hilojs.codeplex.com/discussions/362875\#post869640}
\end{itemize}

\subsubsection{Marionette And
Flexibility}\label{marionette-and-flexibility}

Marionette is a flexible framework, much like Backbone itself. It offers
a wide variety of tools to help create and organize an application
architecture on top of Backbone, but like Backbone itself, it doesn't
dictate that you have to use all of its pieces in order to use any of
them.

The flexibility and versatility in Marionette is easiest to understand
by examining three variations of TodoMVC implemented with it that have
been created for comparison purposes:

\begin{itemize}
\itemsep1pt\parskip0pt\parsep0pt
\item
  \href{https://github.com/jsoverson/todomvc/tree/master/labs/architecture-examples/backbone_marionette}{Simple}
  - by Jarrod Overson
\item
  \href{https://github.com/jsoverson/todomvc/tree/master/labs/dependency-examples/backbone_marionette_require}{RequireJS}
  - also by Jarrod
\item
  \href{https://github.com/derickbailey/todomvc/tree/marionette/labs/architecture-examples/backbone_marionette/js}{Marionette
  modules} - by Derick Bailey
\end{itemize}

\textbf{The simple version}: This version of TodoMVC shows some raw use
of Marionette's various view types, an application object, and the event
aggregator. The objects that are created are added directly to the
global namespace and are fairly straightforward. This is a great example
of how Marionette can be used to augment existing code without having to
re-write everything around Marionette.

\textbf{The RequireJS version}: Using Marionette with RequireJS helps to
create a modularized application architecture - a tremendously important
concept in scaling JavaScript applications. RequireJS provides a
powerful set of tools that can be leveraged to great advantage, making
Marionette even more flexible than it already is.

\textbf{The Marionette module version}: RequireJS isn't the only way to
create a modularized application architecture, though. For those that
wish to build applications in modules and namespaces, Marionette
provides a built-in module and namespacing structure. This example
application takes the simple version of the application and re-writes it
into a namespaced application architecture, with an application
controller (mediator / workflow object) that brings all of the pieces
together.

Marionette certainly provides its share of opinions on how a Backbone
application should be architected. The combination of modules, view
types, event aggregator, application objects, and more, can be used to
create a very powerful and flexible architecture based on these
opinions.

But as you can see, Marionette isn't a completely rigid, ``my way or the
highway'' framework. It provides many elements of an application
foundation that can be mixed and matched with other architectural
styles, such as AMD or namespacing, or provide simple augmentation to
existing projects by reducing boilerplate code for rendering views.

This flexibility creates a much greater opportunity for Marionette to
provide value to you and your projects, as it allows you to scale the
use of Marionette with your application's needs.

\subsubsection{And So Much More}\label{and-so-much-more}

This is just the tip of the proverbial iceberg for Marionette, even for
the \texttt{ItemView} and \texttt{Region} objects that we've explored.
There is far more functionality, more features, and more flexibility and
customizability that can be put to use in both of these objects. Then we
have the other dozen or so components that Marionette provides, each
with their own set of behaviors built in, customization and extension
points, and more.

To learn more about Marionette's components, the features they provide
and how to use them, check out the Marionette documentation, links to
the wiki, to the source code, the project core contributors, and much
more at \url{http://marionettejs.com}.

~

~

\hyperdef{}{thorax}{\subsection{Thorax}\label{thorax}}

\emph{By Ryan Eastridge \& Addy Osmani}

Part of Backbone's appeal is that it provides structure but is generally
un-opinionated, in particular when it comes to views. Thorax makes an
opinionated decision to use Handlebars as its templating solution. Some
of the patterns found in Marionette are found in Thorax as well.
Marionette exposes most of these patterns as JavaScript APIs while in
Thorax they are often exposed as template helpers. This chapter assumes
the reader has knowledge of Handlebars.

Thorax was created by Ryan Eastridge and Kevin Decker to create
Walmart's mobile web application. This chapter is limited to Thorax's
templating features and patterns implemented in Thorax that you can
utilize in your application regardless of whether you choose to adopt
Thorax. To learn more about other features implemented in Thorax and to
download boilerplate projects visit the
\href{http://thoraxjs.org}{Thorax website}.

\subsubsection{Hello World}\label{hello-world}

In Backbone, when creating a new view, options passed are merged into
any default options already present on a view and are exposed via
\texttt{this.options} for later reference.

\texttt{Thorax.View} differs from \texttt{Backbone.View} in that there
is no \texttt{options} object. All arguments passed to the constructor
become properties of the view, which in turn become available to the
\texttt{template}:

\begin{Shaded}
\begin{Highlighting}[]
    \KeywordTok{var} \NormalTok{view = }\KeywordTok{new} \OtherTok{Thorax}\NormalTok{.}\FunctionTok{View}\NormalTok{(\{}
        \DataTypeTok{greeting}\NormalTok{: }\StringTok{'Hello'}\NormalTok{,}
        \DataTypeTok{template}\NormalTok{: }\OtherTok{Handlebars}\NormalTok{.}\FunctionTok{compile}\NormalTok{(}\StringTok{'\{\{greeting\}\} World!'}\NormalTok{)}
    \NormalTok{\});}
    \OtherTok{view}\NormalTok{.}\FunctionTok{appendTo}\NormalTok{(}\StringTok{'body'}\NormalTok{);}
\end{Highlighting}
\end{Shaded}

In most examples in this chapter a \texttt{template} property will be
specified. In larger projects including the boilerplate projects
provided on the Thorax website a \texttt{name} property would instead be
used and a \texttt{template} of the same file name in your project would
automatically be assigned to the view.

If a \texttt{model} is set on a view, its attributes also become
available to the template:

\begin{verbatim}
var view = new Thorax.View({
    model: new Thorax.Model({key: 'value'}),
    template: Handlebars.compile('{{key}}')
});
\end{verbatim}

\subsubsection{Embedding child views}\label{embedding-child-views}

The view helper allows you to embed other views within a view. Child
views can be specified as properties of the view:

\begin{Shaded}
\begin{Highlighting}[]
    \KeywordTok{var} \NormalTok{parent = }\KeywordTok{new} \OtherTok{Thorax}\NormalTok{.}\FunctionTok{View}\NormalTok{(\{}
        \DataTypeTok{child}\NormalTok{: }\KeywordTok{new} \OtherTok{Thorax}\NormalTok{.}\FunctionTok{View}\NormalTok{(...),}
        \DataTypeTok{template}\NormalTok{: }\OtherTok{Handlebars}\NormalTok{.}\FunctionTok{compile}\NormalTok{(}\StringTok{'\{\{view child\}\}'}\NormalTok{)}
    \NormalTok{\});}
\end{Highlighting}
\end{Shaded}

Or the name of a child view to initialize as well as any optional
properties you wish to pass. In this case the child view must have
previously been created with \texttt{extend} and given a \texttt{name}
property:

\begin{Shaded}
\begin{Highlighting}[]
    \KeywordTok{var} \NormalTok{ChildView = }\OtherTok{Thorax}\NormalTok{.}\OtherTok{View}\NormalTok{.}\FunctionTok{extend}\NormalTok{(\{}
        \DataTypeTok{name}\NormalTok{: }\StringTok{'child'}\NormalTok{,}
        \DataTypeTok{template}\NormalTok{: ...}
    \NormalTok{\});}
  
    \KeywordTok{var} \NormalTok{parent = }\KeywordTok{new} \OtherTok{Thorax}\NormalTok{.}\FunctionTok{View}\NormalTok{(\{}
        \DataTypeTok{template}\NormalTok{: }\OtherTok{Handlebars}\NormalTok{.}\FunctionTok{compile}\NormalTok{(}\StringTok{'\{\{view "child" key="value"\}\}'}\NormalTok{)}
    \NormalTok{\});}
\end{Highlighting}
\end{Shaded}

The view helper may also be used as a block helper, in which case the
block will be assigned as the \texttt{template} property of the child
view:

\begin{verbatim}
    {{#view child}}
        child will have this block
        set as its template property
    {{/view}}
\end{verbatim}

Handlebars is string based, while \texttt{Backbone.View} instances have
a DOM \texttt{el}. Since we are mixing metaphors, the embedding of views
works via a placeholder mechanism where the \texttt{view} helper in this
case adds the view passed to the helper to a hash of \texttt{children},
then injects placeholder HTML into the template such as:

\begin{Shaded}
\begin{Highlighting}[]
    \KeywordTok{<div}\OtherTok{ data-view-placeholder-cid=}\StringTok{"view2"}\KeywordTok{></div>}
\end{Highlighting}
\end{Shaded}

Then once the parent view is rendered, we walk the DOM in search of all
the placeholders we created, replacing them with the child views'
\texttt{el}s:

\begin{Shaded}
\begin{Highlighting}[]
    \KeywordTok{this}\NormalTok{.}\OtherTok{$el}\NormalTok{.}\FunctionTok{find}\NormalTok{(}\StringTok{'[data-view-placeholder-cid]'}\NormalTok{).}\FunctionTok{forEach}\NormalTok{(}\KeywordTok{function}\NormalTok{(el) \{}
        \KeywordTok{var} \NormalTok{cid = }\OtherTok{el}\NormalTok{.}\FunctionTok{getAttribute}\NormalTok{(}\StringTok{'data-view-placeholder-cid'}\NormalTok{),}
            \NormalTok{view = }\KeywordTok{this}\NormalTok{.}\FunctionTok{children}\NormalTok{[cid];}
        \OtherTok{view}\NormalTok{.}\FunctionTok{render}\NormalTok{();}
        \FunctionTok{$}\NormalTok{(el).}\FunctionTok{replaceWith}\NormalTok{(}\OtherTok{view}\NormalTok{.}\FunctionTok{el}\NormalTok{);}
    \NormalTok{\}, }\KeywordTok{this}\NormalTok{);}
\end{Highlighting}
\end{Shaded}

\subsubsection{View helpers}\label{view-helpers}

One of the most useful constructs in Thorax is
\texttt{Handlebars.registerViewHelper} (not to be confused with
\texttt{Handlebars.registerHelper}). This method will register a new
block helper that will create and embed a \texttt{HelperView} instance
with its \texttt{template} set to the captured block. A
\texttt{HelperView} instance is different from that of a regular child
view in that its context will be that of the parent's in the template.
Like other child views it will have a \texttt{parent} property set to
that of the declaring view. Many of the built-in helpers in Thorax
including the \texttt{collection} helper are created in this manner.

A simple example would be an \texttt{on} helper that re-rendered the
generated \texttt{HelperView} instance each time an event was triggered
on the declaring / parent view:

\begin{verbatim}
Handlebars.registerViewHelper('on', function(eventName, helperView) {
    helperView.parent.on(eventName, function() {
        helperView.render();
    });
});
\end{verbatim}

An example use of this would be to have a counter that would increment
each time a button was clicked. This example makes use of the
\texttt{button} helper in Thorax which simply makes a button that
triggers a view event when clicked:

\begin{verbatim}
    {{#on "incremented"}}{{i}}{{/on}}
    {{#button trigger="incremented"}}Add{{/button}}
\end{verbatim}

And the corresponding view class:

\begin{Shaded}
\begin{Highlighting}[]
    \KeywordTok{new} \OtherTok{Thorax}\NormalTok{.}\FunctionTok{View}\NormalTok{(\{}
        \DataTypeTok{events}\NormalTok{: \{}
            \DataTypeTok{incremented}\NormalTok{: }\KeywordTok{function}\NormalTok{() \{}
                \NormalTok{++}\KeywordTok{this}\NormalTok{.}\FunctionTok{i}\NormalTok{;}
            \NormalTok{\}}
        \NormalTok{\},}
        \DataTypeTok{initialize}\NormalTok{: }\KeywordTok{function}\NormalTok{() \{}
            \KeywordTok{this}\NormalTok{.}\FunctionTok{i} \NormalTok{= }\DecValTok{0}\NormalTok{;}
        \NormalTok{\},}
        \DataTypeTok{template}\NormalTok{: ...}
    \NormalTok{\});}
\end{Highlighting}
\end{Shaded}

\subsubsection{collection helper}\label{collection-helper}

The \texttt{collection} helper creates and embeds a
\texttt{CollectionView} instance, creating a view for each item in a
collection, updating when items are added, removed, or changed in the
collection. The simplest usage of the helper would look like:

\begin{verbatim}
    {{#collection kittens}}
      <li>{{name}}</li>
    {{/collection}}
\end{verbatim}

And the corresponding view:

\begin{Shaded}
\begin{Highlighting}[]
    \KeywordTok{new} \OtherTok{Thorax}\NormalTok{.}\FunctionTok{View}\NormalTok{(\{}
      \DataTypeTok{kittens}\NormalTok{: }\KeywordTok{new} \OtherTok{Thorax}\NormalTok{.}\FunctionTok{Collection}\NormalTok{(...),}
      \DataTypeTok{template}\NormalTok{: ...}
    \NormalTok{\});}
\end{Highlighting}
\end{Shaded}

The block in this case will be assigned as the \texttt{template} for
each item view created, and the context will be the \texttt{attributes}
of the given model. This helper accepts options that can be arbitrary
HTML attributes, a \texttt{tag} option to specify the type of tag
containing the collection, or any of the following:

\begin{itemize}
\itemsep1pt\parskip0pt\parsep0pt
\item
  \texttt{item-template} - A template to display for each model. If a
  block is specified it will become the item-template
\item
  \texttt{item-view} - A view class to use when each item view is
  created
\item
  \texttt{empty-template} - A template to display when the collection is
  empty. If an inverse / else block is specified it will become the
  empty-template
\item
  \texttt{empty-view} - A view to display when the collection is empty
\end{itemize}

Options and blocks can be used in combination, in this case creating a
\texttt{KittenView} class with a \texttt{template} set to the captured
block for each kitten in the collection:

\begin{verbatim}
    {{#collection kittens item-view="KittenView" tag="ul"}}
      <li>{{name}}</li>
    {{else}}
      <li>No kittens!</li>
    {{/collection}}
\end{verbatim}

Note that multiple collections can be used per view, and collections can
be nested. This is useful when there are models that contain collections
that contain models that contain\ldots{}

\begin{verbatim}
    {{#collection kittens}}
      <h2>{{name}}</h2>
      <p>Kills:</p>
      {{#collection miceKilled tag="ul"}}
        <li>{{name}}</li>
      {{/collection}}
    {{/collection}}
\end{verbatim}

\subsubsection{Custom HTML data
attributes}\label{custom-html-data-attributes}

Thorax makes heavy use of custom HTML data attributes to operate. While
some make sense only within the context of Thorax, several are quite
useful to have in any Backbone project for writing other functions
against, or for general debugging. In order to add some to your views in
non-Thorax projects, override the \texttt{setElement} method in your
base view class:

\begin{Shaded}
\begin{Highlighting}[]
  \OtherTok{MyApplication}\NormalTok{.}\FunctionTok{View} \NormalTok{= }\OtherTok{Backbone}\NormalTok{.}\OtherTok{View}\NormalTok{.}\FunctionTok{extend}\NormalTok{(\{}
    \DataTypeTok{setElement}\NormalTok{: }\KeywordTok{function}\NormalTok{() \{}
        \KeywordTok{var} \NormalTok{response = }\OtherTok{Backbone}\NormalTok{.}\OtherTok{View}\NormalTok{.}\OtherTok{prototype}\NormalTok{.}\OtherTok{setElement}\NormalTok{.}\FunctionTok{apply}\NormalTok{(}\KeywordTok{this}\NormalTok{, arguments);}
        \KeywordTok{this}\NormalTok{.}\FunctionTok{name} \NormalTok{&& }\KeywordTok{this}\NormalTok{.}\OtherTok{$el}\NormalTok{.}\FunctionTok{attr}\NormalTok{(}\StringTok{'data-view-name'}\NormalTok{, }\KeywordTok{this}\NormalTok{.}\FunctionTok{name}\NormalTok{);}
        \KeywordTok{this}\NormalTok{.}\OtherTok{$el}\NormalTok{.}\FunctionTok{attr}\NormalTok{(}\StringTok{'data-view-cid'}\NormalTok{, }\KeywordTok{this}\NormalTok{.}\FunctionTok{cid}\NormalTok{);}
        \KeywordTok{this}\NormalTok{.}\FunctionTok{collection} \NormalTok{&& }\KeywordTok{this}\NormalTok{.}\OtherTok{$el}\NormalTok{.}\FunctionTok{attr}\NormalTok{(}\StringTok{'data-collection-cid'}\NormalTok{, }\KeywordTok{this}\NormalTok{.}\OtherTok{collection}\NormalTok{.}\FunctionTok{cid}\NormalTok{);}
        \KeywordTok{this}\NormalTok{.}\FunctionTok{model} \NormalTok{&& }\KeywordTok{this}\NormalTok{.}\OtherTok{$el}\NormalTok{.}\FunctionTok{attr}\NormalTok{(}\StringTok{'data-model-cid'}\NormalTok{, }\KeywordTok{this}\NormalTok{.}\OtherTok{model}\NormalTok{.}\FunctionTok{cid}\NormalTok{);}
        \KeywordTok{return} \NormalTok{response;}
    \NormalTok{\}}
  \NormalTok{\});}
\end{Highlighting}
\end{Shaded}

In addition to making your application more immediately comprehensible
in the inspector, it's now possible to extend jQuery / Zepto with
functions to lookup the closest view, model or collection to a given
element. In order to make it work you have to save references to each
view created in your base view class by overriding the
\texttt{\_configure} method:

\begin{Shaded}
\begin{Highlighting}[]
    \OtherTok{MyApplication}\NormalTok{.}\FunctionTok{View} \NormalTok{= }\OtherTok{Backbone}\NormalTok{.}\OtherTok{View}\NormalTok{.}\FunctionTok{extend}\NormalTok{(\{}
        \DataTypeTok{_configure}\NormalTok{: }\KeywordTok{function}\NormalTok{() \{}
            \OtherTok{Backbone}\NormalTok{.}\OtherTok{View}\NormalTok{.}\OtherTok{prototype}\NormalTok{.}\OtherTok{_configure}\NormalTok{.}\FunctionTok{apply}\NormalTok{(}\KeywordTok{this}\NormalTok{, arguments);}
            \OtherTok{Thorax}\NormalTok{.}\FunctionTok{_viewsIndexedByCid}\NormalTok{[}\KeywordTok{this}\NormalTok{.}\FunctionTok{cid}\NormalTok{] = }\KeywordTok{this}\NormalTok{;}
        \NormalTok{\},}
        \DataTypeTok{dispose}\NormalTok{: }\KeywordTok{function}\NormalTok{() \{}
            \OtherTok{Backbone}\NormalTok{.}\OtherTok{View}\NormalTok{.}\OtherTok{prototype}\NormalTok{.}\OtherTok{dispose}\NormalTok{.}\FunctionTok{apply}\NormalTok{(}\KeywordTok{this}\NormalTok{, arguments);}
            \KeywordTok{delete} \OtherTok{Thorax}\NormalTok{.}\FunctionTok{_viewsIndexedByCid}\NormalTok{[}\KeywordTok{this}\NormalTok{.}\FunctionTok{cid}\NormalTok{];}
        \NormalTok{\}}
    \NormalTok{\});}
\end{Highlighting}
\end{Shaded}

Then we can extend jQuery / Zepto:

\begin{Shaded}
\begin{Highlighting}[]
    \OtherTok{$}\NormalTok{.}\OtherTok{fn}\NormalTok{.}\FunctionTok{view} \NormalTok{= }\KeywordTok{function}\NormalTok{() \{}
        \KeywordTok{var} \NormalTok{el = }\FunctionTok{$}\NormalTok{(}\KeywordTok{this}\NormalTok{).}\FunctionTok{closest}\NormalTok{(}\StringTok{'[data-view-cid]'}\NormalTok{);}
        \KeywordTok{return} \NormalTok{el && }\OtherTok{Thorax}\NormalTok{.}\FunctionTok{_viewsIndexedByCid}\NormalTok{[}\OtherTok{el}\NormalTok{.}\FunctionTok{attr}\NormalTok{(}\StringTok{'data-view-cid'}\NormalTok{)];}
    \NormalTok{\};}

    \OtherTok{$}\NormalTok{.}\OtherTok{fn}\NormalTok{.}\FunctionTok{model} \NormalTok{= }\KeywordTok{function}\NormalTok{(view) \{}
        \KeywordTok{var} \NormalTok{$this = }\FunctionTok{$}\NormalTok{(}\KeywordTok{this}\NormalTok{),}
            \NormalTok{modelElement = }\OtherTok{$this}\NormalTok{.}\FunctionTok{closest}\NormalTok{(}\StringTok{'[data-model-cid]'}\NormalTok{),}
            \NormalTok{modelCid = modelElement && }\OtherTok{modelElement}\NormalTok{.}\FunctionTok{attr}\NormalTok{(}\StringTok{'[data-model-cid]'}\NormalTok{);}
        \KeywordTok{if} \NormalTok{(modelCid) \{}
            \KeywordTok{var} \NormalTok{view = }\OtherTok{$this}\NormalTok{.}\FunctionTok{view}\NormalTok{();}
            \KeywordTok{return} \NormalTok{view && }\OtherTok{view}\NormalTok{.}\FunctionTok{model}\NormalTok{;}
        \NormalTok{\}}
        \KeywordTok{return} \KeywordTok{false}\NormalTok{;}
    \NormalTok{\};}
\end{Highlighting}
\end{Shaded}

Now instead of storing references to models randomly throughout your
application to lookup when a given DOM event occurs you can use
\texttt{\$(element).model()}. In Thorax, this can particularly useful in
conjunction with the \texttt{collection} helper which generates a view
class (with a \texttt{model} property) for each \texttt{model} in the
collection. An example template:

\begin{verbatim}
    {{#collection kittens tag="ul"}}
      <li>{{name}}</li>
    {{/collection}}
\end{verbatim}

And the corresponding view class:

\begin{Shaded}
\begin{Highlighting}[]
    \OtherTok{Thorax}\NormalTok{.}\OtherTok{View}\NormalTok{.}\FunctionTok{extend}\NormalTok{(\{}
      \DataTypeTok{events}\NormalTok{: \{}
        \StringTok{'click li'}\NormalTok{: }\KeywordTok{function}\NormalTok{(event) \{}
          \KeywordTok{var} \NormalTok{kitten = }\FunctionTok{$}\NormalTok{(}\OtherTok{event}\NormalTok{.}\FunctionTok{target}\NormalTok{).}\FunctionTok{model}\NormalTok{();}
          \OtherTok{console}\NormalTok{.}\FunctionTok{log}\NormalTok{(}\StringTok{'Clicked on '} \NormalTok{+ }\OtherTok{kitten}\NormalTok{.}\FunctionTok{get}\NormalTok{(}\StringTok{'name'}\NormalTok{));}
        \NormalTok{\}}
      \NormalTok{\},}
      \DataTypeTok{kittens}\NormalTok{: }\KeywordTok{new} \OtherTok{Thorax}\NormalTok{.}\FunctionTok{Collection}\NormalTok{(...),}
      \DataTypeTok{template}\NormalTok{: ...}
    \NormalTok{\});  }
\end{Highlighting}
\end{Shaded}

A common anti-pattern in Backbone applications is to assign a
\texttt{className} to a single view class. Consider using the
\texttt{data-view-name} attribute as a CSS selector instead, saving CSS
classes for things that will be used multiple times:

\begin{Shaded}
\begin{Highlighting}[]
  \CharTok{[data-view-name=}\StringTok{"child"}\CharTok{]} \KeywordTok{\{}

  \KeywordTok{\}}
\end{Highlighting}
\end{Shaded}

\subsubsection{Thorax Resources}\label{thorax-resources}

No Backbone related tutorial would be complete without a todo
application. A
\href{http://todomvc.com/labs/architecture-examples/thorax/}{Thorax
implementation of TodoMVC} is available, in addition to this far simpler
example composed of this single Handlebars template:

\begin{verbatim}
  {{#collection todos tag="ul"}}
    <li{{#if done}} class="done"{{/if}}>
      <input type="checkbox" name="done"{{#if done}} checked="checked"{{/if}}>
      <span>{{item}}</span>
    </li>
  {{/collection}}
  <form>
    <input type="text">
    <input type="submit" value="Add">
  </form>
\end{verbatim}

and the corresponding JavaScript:

\begin{Shaded}
\begin{Highlighting}[]
  \KeywordTok{var} \NormalTok{todosView = }\OtherTok{Thorax}\NormalTok{.}\FunctionTok{View}\NormalTok{(\{}
      \DataTypeTok{todos}\NormalTok{: }\KeywordTok{new} \OtherTok{Thorax}\NormalTok{.}\FunctionTok{Collection}\NormalTok{(),}
      \DataTypeTok{events}\NormalTok{: \{}
          \StringTok{'change input[type="checkbox"]'}\NormalTok{: }\KeywordTok{function}\NormalTok{(event) \{}
              \KeywordTok{var} \NormalTok{target = }\FunctionTok{$}\NormalTok{(}\OtherTok{event}\NormalTok{.}\FunctionTok{target}\NormalTok{);}
              \OtherTok{target}\NormalTok{.}\FunctionTok{model}\NormalTok{().}\FunctionTok{set}\NormalTok{(\{}\DataTypeTok{done}\NormalTok{: !!}\OtherTok{target}\NormalTok{.}\FunctionTok{attr}\NormalTok{(}\StringTok{'checked'}\NormalTok{)\});}
          \NormalTok{\},}
          \StringTok{'submit form'}\NormalTok{: }\KeywordTok{function}\NormalTok{(event) \{}
              \OtherTok{event}\NormalTok{.}\FunctionTok{preventDefault}\NormalTok{();}
              \KeywordTok{var} \NormalTok{input = }\KeywordTok{this}\NormalTok{.}\FunctionTok{$}\NormalTok{(}\StringTok{'input[type="text"]'}\NormalTok{);}
              \KeywordTok{this}\NormalTok{.}\OtherTok{todos}\NormalTok{.}\FunctionTok{add}\NormalTok{(\{}\DataTypeTok{item}\NormalTok{: }\OtherTok{input}\NormalTok{.}\FunctionTok{val}\NormalTok{()\});}
              \OtherTok{input}\NormalTok{.}\FunctionTok{val}\NormalTok{(}\StringTok{''}\NormalTok{);}
          \NormalTok{\}}
      \NormalTok{\},}
      \DataTypeTok{template}\NormalTok{: }\StringTok{'...'}
  \NormalTok{\});}
  \OtherTok{todosView}\NormalTok{.}\FunctionTok{appendTo}\NormalTok{(}\StringTok{'body'}\NormalTok{);}
\end{Highlighting}
\end{Shaded}

To see Thorax in action on a large scale website visit walmart.com on
any Android or iOS device. For a complete list of resources visit the
\href{http://thoraxjs.org}{Thorax website}.

~

~

\section{Common Problems \& Solutions}\label{common-problems-solutions}

In this section, we will review a number of common problems developers
often experience once they've started to work on relatively non-trivial
projects using Backbone.js, as well as present potential solutions.

Perhaps the most frequent of these questions surround how to do more
with Views. If you are interested in discovering how to work with nested
Views, learn about view disposal and inheritance, this section will
hopefully have you covered.

\paragraph{Working With Nested Views}\label{working-with-nested-views}

\textbf{Problem}

What is the best approach for rendering and appending nested Views (or
Subviews) in Backbone.js?

\textbf{Solution 1}

Since pages are composed of nested elements and Backbone views
correspond to elements within the page, nesting views is an intuitive
approach to managing a hierarchy of elements.

The best way to combine views is simply using:

\begin{verbatim}
this.$('.someContainer').append(innerView.el);
\end{verbatim}

which just relies on jQuery. We could use this in a real example as
follows:

\begin{Shaded}
\begin{Highlighting}[]
\NormalTok{...}
\NormalTok{initialize : }\KeywordTok{function} \NormalTok{() \{}
    \CommentTok{//...}
\NormalTok{\},}

\NormalTok{render : }\KeywordTok{function} \NormalTok{() \{}

    \KeywordTok{this}\NormalTok{.}\OtherTok{$el}\NormalTok{.}\FunctionTok{empty}\NormalTok{();}

    \KeywordTok{this}\NormalTok{.}\FunctionTok{innerView1} \NormalTok{= }\KeywordTok{new} \FunctionTok{Subview}\NormalTok{(\{options\});}
    \KeywordTok{this}\NormalTok{.}\FunctionTok{innerView2} \NormalTok{= }\KeywordTok{new} \FunctionTok{Subview}\NormalTok{(\{options\});}

    \KeywordTok{this}\NormalTok{.}\FunctionTok{$}\NormalTok{(}\StringTok{'.inner-view-container'}\NormalTok{)}
        \NormalTok{.}\FunctionTok{append}\NormalTok{(}\KeywordTok{this}\NormalTok{.}\OtherTok{innerView1}\NormalTok{.}\FunctionTok{el}\NormalTok{)}
        \NormalTok{.}\FunctionTok{append}\NormalTok{(}\KeywordTok{this}\NormalTok{.}\OtherTok{innerView2}\NormalTok{.}\FunctionTok{el}\NormalTok{);}
\NormalTok{\}}
\end{Highlighting}
\end{Shaded}

\textbf{Solution 2}

Beginners sometimes also try using \texttt{setElement} to solve this
problem, however keep in mind that using this method is an easy way to
shoot yourself in the foot. Avoid using this approach when the first
solution is a viable option:

\begin{Shaded}
\begin{Highlighting}[]

\CommentTok{// Where we have previously defined a View, SubView}
\CommentTok{// in a parent View we could do:}

\NormalTok{...}
\NormalTok{initialize : }\KeywordTok{function} \NormalTok{() \{}

    \KeywordTok{this}\NormalTok{.}\FunctionTok{innerView1} \NormalTok{= }\KeywordTok{new} \FunctionTok{Subview}\NormalTok{(\{options\});}
    \KeywordTok{this}\NormalTok{.}\FunctionTok{innerView2} \NormalTok{= }\KeywordTok{new} \FunctionTok{Subview}\NormalTok{(\{options\});}
\NormalTok{\},}

\NormalTok{render : }\KeywordTok{function} \NormalTok{() \{}

    \KeywordTok{this}\NormalTok{.}\OtherTok{$el}\NormalTok{.}\FunctionTok{html}\NormalTok{(}\KeywordTok{this}\NormalTok{.}\FunctionTok{template}\NormalTok{());}

    \KeywordTok{this}\NormalTok{.}\OtherTok{innerView1}\NormalTok{.}\FunctionTok{setElement}\NormalTok{(}\StringTok{'.some-element1'}\NormalTok{).}\FunctionTok{render}\NormalTok{();}
    \KeywordTok{this}\NormalTok{.}\OtherTok{innerView2}\NormalTok{.}\FunctionTok{setElement}\NormalTok{(}\StringTok{'.some-element2'}\NormalTok{).}\FunctionTok{render}\NormalTok{();}
\NormalTok{\}}
\end{Highlighting}
\end{Shaded}

Here we are creating subviews in the parent view's \texttt{initialize()}
method and rendering the subviews in the parent's \texttt{render()}
method. The elements managed by the subviews exist in the parent's
template and the \texttt{View.setElement()} method is used to re-assign
the element associated with each subview.

\texttt{setElement()} changes a view's element, including re-delegating
event handlers by removing them from the old element and binding them to
the new element. Note that \texttt{setElement()} returns the view,
allowing us to chain the call to \texttt{render()}.

This works and has some positive qualities: you don't need to worry
about maintaining the order of your DOM elements when appending, views
are initialized early, and the render() method doesn't need to take on
too many responsibilities at once.

Unfortunately, downsides are that you can't set the \texttt{tagName}
property of subviews and events need to be re-delegated. The first
solution doesn't suffer from this problem.

\textbf{Solution 3}

One more possible solution to this problem could be written:

\begin{Shaded}
\begin{Highlighting}[]

\KeywordTok{var} \NormalTok{OuterView = }\OtherTok{Backbone}\NormalTok{.}\OtherTok{View}\NormalTok{.}\FunctionTok{extend}\NormalTok{(\{}
    \DataTypeTok{initialize}\NormalTok{: }\KeywordTok{function}\NormalTok{() \{}
        \KeywordTok{this}\NormalTok{.}\FunctionTok{inner} \NormalTok{= }\KeywordTok{new} \FunctionTok{InnerView}\NormalTok{();}
    \NormalTok{\},}

    \DataTypeTok{render}\NormalTok{: }\KeywordTok{function}\NormalTok{() \{}
        \CommentTok{// Detach InnerView before reseting OuterView's $el}
        \KeywordTok{this}\NormalTok{.}\OtherTok{inner}\NormalTok{.}\OtherTok{$el}\NormalTok{.}\FunctionTok{detach}\NormalTok{(); }

        \CommentTok{// or this.$el.empty() if you have no template}
        \CommentTok{// this.$el.html(template); }

        \KeywordTok{this}\NormalTok{.}\OtherTok{$el}\NormalTok{.}\FunctionTok{append}\NormalTok{(}\KeywordTok{this}\NormalTok{.}\OtherTok{inner}\NormalTok{.}\FunctionTok{$el}\NormalTok{);}
    \NormalTok{\}}
\NormalTok{\});}

\KeywordTok{var} \NormalTok{InnerView = }\OtherTok{Backbone}\NormalTok{.}\OtherTok{View}\NormalTok{.}\FunctionTok{extend}\NormalTok{(\{}
    \DataTypeTok{render}\NormalTok{: }\KeywordTok{function}\NormalTok{() \{}
        \KeywordTok{this}\NormalTok{.}\OtherTok{$el}\NormalTok{.}\FunctionTok{html}\NormalTok{(template);}
    \NormalTok{\}}
\NormalTok{\});}
\end{Highlighting}
\end{Shaded}

This tackles a few specific design decisions:

\begin{itemize}
\itemsep1pt\parskip0pt\parsep0pt
\item
  The order in which you append the sub-elements matters
\item
  The OuterView doesn't contain the HTML elements to be set in the
  InnerView(s), meaning that we can still specify tagName in the
  InnerView
\item
  render() is called after the InnerView element has been placed into
  the DOM. This is useful if your InnerView's render() method is sizing
  itself on the page based on the dimensions of another element. This is
  a common use case.
\end{itemize}

Note that InnerView needs to call \texttt{View.delegateEvents()} to bind
its event handlers to its new DOM since it is replacing the content of
its element.

\textbf{Solution 4}

A better solution, which is more clean but has the potential to affect
performance is:

\begin{Shaded}
\begin{Highlighting}[]

\KeywordTok{var} \NormalTok{OuterView = }\OtherTok{Backbone}\NormalTok{.}\OtherTok{View}\NormalTok{.}\FunctionTok{extend}\NormalTok{(\{}
    \DataTypeTok{initialize}\NormalTok{: }\KeywordTok{function}\NormalTok{() \{}
        \KeywordTok{this}\NormalTok{.}\FunctionTok{render}\NormalTok{();}
    \NormalTok{\},}

    \DataTypeTok{render}\NormalTok{: }\KeywordTok{function}\NormalTok{() \{}
        \KeywordTok{this}\NormalTok{.}\OtherTok{$el}\NormalTok{.}\FunctionTok{html}\NormalTok{(template); }\CommentTok{// or this.$el.empty() if you have no template}
        \KeywordTok{this}\NormalTok{.}\FunctionTok{inner} \NormalTok{= }\KeywordTok{new} \FunctionTok{InnerView}\NormalTok{();}
        \KeywordTok{this}\NormalTok{.}\OtherTok{$el}\NormalTok{.}\FunctionTok{append}\NormalTok{(}\KeywordTok{this}\NormalTok{.}\OtherTok{inner}\NormalTok{.}\FunctionTok{$el}\NormalTok{);}
    \NormalTok{\}}
\NormalTok{\});}

\KeywordTok{var} \NormalTok{InnerView = }\OtherTok{Backbone}\NormalTok{.}\OtherTok{View}\NormalTok{.}\FunctionTok{extend}\NormalTok{(\{}
    \DataTypeTok{initialize}\NormalTok{: }\KeywordTok{function}\NormalTok{() \{}
        \KeywordTok{this}\NormalTok{.}\FunctionTok{render}\NormalTok{();}
    \NormalTok{\},}

    \DataTypeTok{render}\NormalTok{: }\KeywordTok{function}\NormalTok{() \{}
        \KeywordTok{this}\NormalTok{.}\OtherTok{$el}\NormalTok{.}\FunctionTok{html}\NormalTok{(template);}
    \NormalTok{\}}
\NormalTok{\});}
\end{Highlighting}
\end{Shaded}

\textbf{Solution 5}

If multiple views need to be nested at particular locations in a
template, a hash of child views indexed by child view cids' should be
created. In the template, use a custom HTML attribute named
\texttt{data-view-cid} to create placeholder elements for each view to
embed. Once the template has been rendered and its output appended to
the parent view's \texttt{\$el}, each placeholder can be queried for and
replaced with the child view's \texttt{el}.

A sample implementation containing a single child view could be written:

\begin{Shaded}
\begin{Highlighting}[]

\KeywordTok{var} \NormalTok{OuterView = }\OtherTok{Backbone}\NormalTok{.}\OtherTok{View}\NormalTok{.}\FunctionTok{extend}\NormalTok{(\{}
    \DataTypeTok{initialize}\NormalTok{: }\KeywordTok{function}\NormalTok{() \{}
        \KeywordTok{this}\NormalTok{.}\FunctionTok{children} \NormalTok{= \{\};}
        \KeywordTok{this}\NormalTok{.}\FunctionTok{child} \NormalTok{= }\KeywordTok{new} \OtherTok{Backbone}\NormalTok{.}\FunctionTok{View}\NormalTok{();}
        \KeywordTok{this}\NormalTok{.}\FunctionTok{children}\NormalTok{[}\KeywordTok{this}\NormalTok{.}\OtherTok{child}\NormalTok{.}\FunctionTok{cid}\NormalTok{] = }\KeywordTok{this}\NormalTok{.}\FunctionTok{child}\NormalTok{;}
    \NormalTok{\},}

    \DataTypeTok{render}\NormalTok{: }\KeywordTok{function}\NormalTok{() \{}
        \KeywordTok{this}\NormalTok{.}\OtherTok{$el}\NormalTok{.}\FunctionTok{html}\NormalTok{(}\StringTok{'<div data-view-cid="'} \NormalTok{+ }\KeywordTok{this}\NormalTok{.}\OtherTok{child}\NormalTok{.}\FunctionTok{cid} \NormalTok{+ }\StringTok{'"></div>'}\NormalTok{);}
        \OtherTok{_}\NormalTok{.}\FunctionTok{each}\NormalTok{(}\KeywordTok{this}\NormalTok{.}\FunctionTok{children}\NormalTok{, }\KeywordTok{function}\NormalTok{(view, cid) \{}
            \KeywordTok{this}\NormalTok{.}\FunctionTok{$}\NormalTok{(}\StringTok{'[data-view-cid="'} \NormalTok{+ cid + }\StringTok{'"]'}\NormalTok{).}\FunctionTok{replaceWith}\NormalTok{(}\OtherTok{view}\NormalTok{.}\FunctionTok{el}\NormalTok{);}
        \NormalTok{\}, }\KeywordTok{this}\NormalTok{);}
    \NormalTok{\}}
\NormalTok{\};}
\end{Highlighting}
\end{Shaded}

The use of \texttt{cid}s (client ids) here is useful because it
illustrates separating a model and its views by having views referenced
by their instances and not their attributes. It's quite common to ask
for all views that satisfy an attribute on their models, but if you have
recursive subviews or repeated views (a common occurrence), you can't
simply ask for views by attributes. That is, unless you specify
additional attributes that separate duplicates. Using \texttt{cid}s
solves this problem as it allows for direct references to views.

Generally speaking, more developers opt for Solution 1 or 5 as:

\begin{itemize}
\itemsep1pt\parskip0pt\parsep0pt
\item
  The majority of their views may already rely on being in the DOM in
  their render() method
\item
  When the OuterView is re-rendered, views don't have to be
  re-initialized where re-initialization has the potential to cause
  memory leaks and issues with existing bindings
\end{itemize}

The Backbone extensions
\hyperref[marionettejs-backbone.marionette]{Marionette} and
\hyperref[thorax]{Thorax} provide logic for nesting views, and rendering
collections where each item has an associated view. Marionette provides
APIs in JavaScript while Thorax provides APIs via Handlebars template
helpers. We will examine both of these in an upcoming chapter.

(Thanks to \href{http://stackoverflow.com/users/299189/lukas}{Lukas} and
\href{http://stackoverflow.com/users/154765/ian-storm-taylor}{Ian
Taylor} for these tips).

\paragraph{Managing Models In Nested
Views}\label{managing-models-in-nested-views}

\textbf{Problem}

What is the best way to manage models in nested views?

\textbf{Solution}

In order to reach attributes on related models in a nested setup, models
require some prior knowledge of each other, something which Backbone
doesn't implicitly handle out of the box.

One approach is to make sure each child model has a `parent' attribute.
This way you can traverse the nesting first up to the parent and then
down to any siblings that you know of. So, assuming we have models
modelA, modelB and modelC:

\begin{Shaded}
\begin{Highlighting}[]

\CommentTok{// When initializing modelA, I would suggest setting a link to the parent}
\CommentTok{// model when doing this, like this:}

\NormalTok{ModelA = }\OtherTok{Backbone}\NormalTok{.}\OtherTok{Model}\NormalTok{.}\FunctionTok{extend}\NormalTok{(\{}

    \DataTypeTok{initialize}\NormalTok{: }\KeywordTok{function}\NormalTok{()\{}
        \KeywordTok{this}\NormalTok{.}\FunctionTok{modelB} \NormalTok{= }\KeywordTok{new} \FunctionTok{modelB}\NormalTok{();}
        \KeywordTok{this}\NormalTok{.}\OtherTok{modelB}\NormalTok{.}\FunctionTok{parent} \NormalTok{= }\KeywordTok{this}\NormalTok{;}
        \KeywordTok{this}\NormalTok{.}\FunctionTok{modelC} \NormalTok{= }\KeywordTok{new} \FunctionTok{modelC}\NormalTok{();}
        \KeywordTok{this}\NormalTok{.}\OtherTok{modelC}\NormalTok{.}\FunctionTok{parent} \NormalTok{= }\KeywordTok{this}\NormalTok{;}
    \NormalTok{\}}
\NormalTok{\}}
\end{Highlighting}
\end{Shaded}

This allows you to reach the parent model in any child model function
through \texttt{this.parent}.

Now, we have already discussed a few options for how to construct nested
Views using Backbone. For the sake of simplicity, let us imagine that we
are creating a new child view \texttt{ViewB} from within the
\texttt{initialize()} method of \texttt{ViewA} below. \texttt{ViewB} can
reach out over the \texttt{ViewA} model and listen out for changes on
any of its nested models.

See inline for comments on exactly what each step is enabling:

\begin{Shaded}
\begin{Highlighting}[]

\CommentTok{// Define View A}
\NormalTok{ViewA = }\OtherTok{Backbone}\NormalTok{.}\OtherTok{View}\NormalTok{.}\FunctionTok{extend}\NormalTok{(\{}

    \DataTypeTok{initialize}\NormalTok{: }\KeywordTok{function}\NormalTok{()\{}
       \CommentTok{// Create an instance of View B}
       \KeywordTok{this}\NormalTok{.}\FunctionTok{viewB} \NormalTok{= }\KeywordTok{new} \FunctionTok{ViewB}\NormalTok{();}

       \CommentTok{// Create a reference back to this (parent) view}
       \KeywordTok{this}\NormalTok{.}\OtherTok{viewB}\NormalTok{.}\FunctionTok{parentView} \NormalTok{= }\KeywordTok{this}\NormalTok{;}

       \CommentTok{// Append ViewB to ViewA}
       \FunctionTok{$}\NormalTok{(}\KeywordTok{this}\NormalTok{.}\FunctionTok{el}\NormalTok{).}\FunctionTok{append}\NormalTok{(}\KeywordTok{this}\NormalTok{.}\OtherTok{viewB}\NormalTok{.}\FunctionTok{el}\NormalTok{);}
    \NormalTok{\}}
\NormalTok{\});}

\CommentTok{// Define View B}
\NormalTok{ViewB = }\OtherTok{Backbone}\NormalTok{.}\OtherTok{View}\NormalTok{.}\FunctionTok{extend}\NormalTok{(\{}

    \CommentTok{//...,}

    \DataTypeTok{initialize}\NormalTok{: }\KeywordTok{function}\NormalTok{()\{}
        \CommentTok{// Listen for changes to the nested models in our parent ViewA}
        \KeywordTok{this}\NormalTok{.}\FunctionTok{listenTo}\NormalTok{(}\KeywordTok{this}\NormalTok{.}\OtherTok{model}\NormalTok{.}\OtherTok{parent}\NormalTok{.}\FunctionTok{modelB}\NormalTok{, }\StringTok{"change"}\NormalTok{, }\KeywordTok{this}\NormalTok{.}\FunctionTok{render}\NormalTok{);}
        \KeywordTok{this}\NormalTok{.}\FunctionTok{listenTo}\NormalTok{(}\KeywordTok{this}\NormalTok{.}\OtherTok{model}\NormalTok{.}\OtherTok{parent}\NormalTok{.}\FunctionTok{modelC}\NormalTok{, }\StringTok{"change"}\NormalTok{, }\KeywordTok{this}\NormalTok{.}\FunctionTok{render}\NormalTok{);}

        \CommentTok{// We can also call any method on our parent view if it is defined}
        \CommentTok{// $(this.parentView.el).shake();}
    \NormalTok{\}}

\NormalTok{\});}

\CommentTok{// Create an instance of ViewA with ModelA}
\CommentTok{// viewA will create its own instance of ViewB}
\CommentTok{// from inside the initialize() method}
\KeywordTok{var} \NormalTok{viewA = }\KeywordTok{new} \FunctionTok{ViewA}\NormalTok{(\{ }\DataTypeTok{model}\NormalTok{: ModelA \});}
\end{Highlighting}
\end{Shaded}

\paragraph{Rendering A Parent View From A Child
View}\label{rendering-a-parent-view-from-a-child-view}

\textbf{Problem}

How would one render a Parent View from one of its Children?

\textbf{Solution}

In a scenario where you have a view containing another view, such as a
photo gallery containing a larger view modal, you may find that you need
to render or re-render the parent view from the child. The good news is
that solving this problem is quite straight-forward.

The simplest solution is to just use \texttt{this.parentView.render();}.

If however inversion of control is desired, events may be used to
provide an equally valid solution.

Say we wish to begin rendering when a particular event has occurred. For
the sake of example, let us call this event `somethingHappened'. The
parent view can bind notifications on the child view to know when the
event has occurred. It can then render itself.

In the parent view:

\begin{Shaded}
\begin{Highlighting}[]
\CommentTok{// Parent initialize}
\KeywordTok{this}\NormalTok{.}\FunctionTok{listenTo}\NormalTok{(}\KeywordTok{this}\NormalTok{.}\FunctionTok{childView}\NormalTok{, }\StringTok{'somethingHappened'}\NormalTok{, }\KeywordTok{this}\NormalTok{.}\FunctionTok{render}\NormalTok{);}

\CommentTok{// Parent removal}
\KeywordTok{this}\NormalTok{.}\FunctionTok{stopListening}\NormalTok{(}\KeywordTok{this}\NormalTok{.}\FunctionTok{childView}\NormalTok{, }\StringTok{'somethingHappened'}\NormalTok{);}
\end{Highlighting}
\end{Shaded}

In the child view:

\begin{Shaded}
\begin{Highlighting}[]

\CommentTok{// After the event has occurred}
\KeywordTok{this}\NormalTok{.}\FunctionTok{trigger}\NormalTok{(}\StringTok{'somethingHappened'}\NormalTok{);}
\end{Highlighting}
\end{Shaded}

The child will trigger a ``somethingHappened'' event and the parent's
render function will be called.

(Thanks to Tal
\href{http://stackoverflow.com/users/269666/tal-bereznitskey}{Bereznitskey}
for this tip)

\paragraph{Disposing View Hierarchies}\label{disposing-view-hierarchies}

\textbf{Problem}

Where your application is setup with multiple Parent and Child Views, it
is also common to desire removing any DOM elements associated with such
views as well as unbinding any event handlers tied to child elements
when you no longer require them.

\textbf{Solution}

The solution in the last question should be enough to handle this use
case, but if you require a more explicit example that handles children,
we can see one below:

\begin{Shaded}
\begin{Highlighting}[]
\OtherTok{Backbone}\NormalTok{.}\OtherTok{View}\NormalTok{.}\OtherTok{prototype}\NormalTok{.}\FunctionTok{close} \NormalTok{= }\KeywordTok{function}\NormalTok{() \{}
    \KeywordTok{if} \NormalTok{(}\KeywordTok{this}\NormalTok{.}\FunctionTok{onClose}\NormalTok{) \{}
        \KeywordTok{this}\NormalTok{.}\FunctionTok{onClose}\NormalTok{();}
    \NormalTok{\}}
    \KeywordTok{this}\NormalTok{.}\FunctionTok{remove}\NormalTok{();}
\NormalTok{\};}

\NormalTok{NewView = }\OtherTok{Backbone}\NormalTok{.}\OtherTok{View}\NormalTok{.}\FunctionTok{extend}\NormalTok{(\{}
    \DataTypeTok{initialize}\NormalTok{: }\KeywordTok{function}\NormalTok{() \{}
       \KeywordTok{this}\NormalTok{.}\FunctionTok{childViews} \NormalTok{= [];}
    \NormalTok{\},}
    \DataTypeTok{renderChildren}\NormalTok{: }\KeywordTok{function}\NormalTok{(item) \{}
        \KeywordTok{var} \NormalTok{itemView = }\KeywordTok{new} \FunctionTok{NewChildView}\NormalTok{(\{ }\DataTypeTok{model}\NormalTok{: item \});}
        \FunctionTok{$}\NormalTok{(}\KeywordTok{this}\NormalTok{.}\FunctionTok{el}\NormalTok{).}\FunctionTok{prepend}\NormalTok{(}\OtherTok{itemView}\NormalTok{.}\FunctionTok{render}\NormalTok{());}
        \KeywordTok{this}\NormalTok{.}\OtherTok{childViews}\NormalTok{.}\FunctionTok{push}\NormalTok{(itemView);}
    \NormalTok{\},}
    \DataTypeTok{onClose}\NormalTok{: }\KeywordTok{function}\NormalTok{() \{}
      \FunctionTok{_}\NormalTok{(}\KeywordTok{this}\NormalTok{.}\FunctionTok{childViews}\NormalTok{).}\FunctionTok{each}\NormalTok{(}\KeywordTok{function}\NormalTok{(view) \{}
        \OtherTok{view}\NormalTok{.}\FunctionTok{close}\NormalTok{();}
      \NormalTok{\});}
    \NormalTok{\}}
\NormalTok{\});}

\NormalTok{NewChildView = }\OtherTok{Backbone}\NormalTok{.}\OtherTok{View}\NormalTok{.}\FunctionTok{extend}\NormalTok{(\{}
    \DataTypeTok{tagName}\NormalTok{: }\StringTok{'li'}\NormalTok{,}
    \DataTypeTok{render}\NormalTok{: }\KeywordTok{function}\NormalTok{() \{}
    \NormalTok{\}}
\NormalTok{\});}
\end{Highlighting}
\end{Shaded}

Here, a close() method for views is implemented which disposes of a view
when it is no longer needed or needs to be reset.

In most cases, the view removal should not affect any associated models.
For example, if you are working on a blogging application and you remove
a view with comments, perhaps another view in your app shows a selection
of comments and resetting the collection would affect those views as
well.

(Thanks to \href{http://stackoverflow.com/users/906136/dira}{dira} for
this tip)

Note: You may also be interested in reading about the Marionette
Composite Views in the Extensions part of the book.

\paragraph{Rendering View Hierarchies}\label{rendering-view-hierarchies}

\textbf{Problem}

Let us say you have a Collection, where each item in the Collection
could itself be a Collection. You can render each item in the
Collection, and indeed can render any items which themselves are
Collections. The problem you might have is how to render HTML that
reflects the hierarchical nature of the data structure.

\textbf{Solution}

The most straight-forward way to approach this problem is to use a
framework like Derick Bailey's
\href{https://github.com/marionettejs/backbone.marionette}{Backbone.Marionette}.
In this framework is a type of view called a CompositeView.

The basic idea of a CompositeView is that it can render a model and a
collection within the same view.

It can render a single model with a template. It can also take a
collection from that model and for each model in that collection, render
a view. By default it uses the same composite view type that you've
defined to render each of the models in the collection. All you have to
do is tell the view instance where the collection is, via the initialize
method, and you'll get a recursive hierarchy rendered.

There is a working demo of this in action available
\href{http://jsfiddle.net/derickbailey/AdWjU/}{online}.

And you can get the source code and documentation for
\href{https://github.com/marionettejs/backbone.marionette}{Marionette}
too.

\paragraph{Working With Nested Models Or
Collections}\label{working-with-nested-models-or-collections}

\textbf{Problem}

Backbone doesn't include support for nested models or collections out of
the box, favoring the use of good patterns for modeling your structured
data on the client side. How do I work around this?

\textbf{Solution}

As we've seen, it's common to create collections representing groups of
models using Backbone. It's also however common to wish to nest
collections within models, depending on the type of application you are
working on.

Take for example a Building model that contains many Room models which
could sit in a Rooms collection.

You could expose a \texttt{this.rooms} collection for each building,
allowing you to lazy-load rooms once a building has been opened.

\begin{Shaded}
\begin{Highlighting}[]
\KeywordTok{var} \NormalTok{Building = }\OtherTok{Backbone}\NormalTok{.}\OtherTok{Model}\NormalTok{.}\FunctionTok{extend}\NormalTok{(\{}

    \DataTypeTok{initialize}\NormalTok{: }\KeywordTok{function}\NormalTok{()\{}
        \KeywordTok{this}\NormalTok{.}\FunctionTok{rooms} \NormalTok{= }\KeywordTok{new} \NormalTok{Rooms;}
        \KeywordTok{this}\NormalTok{.}\OtherTok{rooms}\NormalTok{.}\FunctionTok{url} \NormalTok{= }\StringTok{'/building/'} \NormalTok{+ }\KeywordTok{this}\NormalTok{.}\FunctionTok{id} \NormalTok{+ }\StringTok{'/rooms'}\NormalTok{;}
        \KeywordTok{this}\NormalTok{.}\OtherTok{rooms}\NormalTok{.}\FunctionTok{on}\NormalTok{(}\StringTok{"reset"}\NormalTok{, }\KeywordTok{this}\NormalTok{.}\FunctionTok{updateCounts}\NormalTok{);}
    \NormalTok{\},}

    \CommentTok{// ...}

\NormalTok{\});}

\CommentTok{// Create a new building model}
\KeywordTok{var} \NormalTok{townHall = }\KeywordTok{new} \NormalTok{Building;}

\CommentTok{// once opened, lazy-load the rooms}
\OtherTok{townHall}\NormalTok{.}\OtherTok{rooms}\NormalTok{.}\FunctionTok{fetch}\NormalTok{(\{}\DataTypeTok{reset}\NormalTok{: }\KeywordTok{true}\NormalTok{\});}
\end{Highlighting}
\end{Shaded}

There are also a number of Backbone plugins which can help with nested
data structures, such as
\href{https://github.com/PaulUithol/Backbone-relational}{Backbone
Relational}. This plugin handles one-to-one, one-to-many and many-to-one
relations between models for Backbone and has some excellent
\href{http://backbonerelational.org/}{documentation}.

\paragraph{Better Model Property
Validation}\label{better-model-property-validation}

\textbf{Problem}

As we learned earlier in the book, the \texttt{validate} method on a
Model is called by \texttt{set} (when the validate option is set) and
\texttt{save}. It is passed the model attributes updated with the values
passed to these methods.

By default, when we define a custom \texttt{validate} method, Backbone
passes all of a model's attributes through this validation each time,
regardless of which model attributes are being set.

This means that it can be a challenge to determine which specific fields
are being set or validated without being concerned about the others that
aren't being set at the same time.

\textbf{Solution}

To illustrate this problem better, let us look at a typical registration
form use case that:

\begin{itemize}
\itemsep1pt\parskip0pt\parsep0pt
\item
  Validates form fields using the blur event
\item
  Validates each field regardless of whether other model attributes
  (i.e., other form data) are valid or not.
\end{itemize}

Here is one example of a desired use case:

We have a form where a user focuses and blurs first name, last name, and
email HTML input boxes without entering any data. A ``this field is
required'' message should be presented next to each form field.

HTML:

\begin{verbatim}
<!doctype html>
<html>
<head>
  <meta charset=utf-8>
  <title>Form Validation - Model#validate</title>
  <script src='http://code.jquery.com/jquery.js'></script>
  <script src='http://underscorejs.org/underscore.js'></script>
  <script src='http://backbonejs.org/backbone.js'></script>
</head>
<body>
  <form>
    <label>First Name</label>
    <input name='firstname'>
    <span data-msg='firstname'></span>
    <br>
    <label>Last Name</label>
    <input name='lastname'>
    <span data-msg='lastname'></span>
    <br>
    <label>Email</label>
    <input name='email'>
    <span data-msg='email'></span>
  </form>
</body>
</html>
\end{verbatim}

Basic validation that could be written using the current Backbone
\texttt{validate} method to work with this form could be implemented
using something like:

\begin{Shaded}
\begin{Highlighting}[]
\NormalTok{validate: }\KeywordTok{function}\NormalTok{(attrs) \{}

    \KeywordTok{if}\NormalTok{(!}\OtherTok{attrs}\NormalTok{.}\FunctionTok{firstname}\NormalTok{) }\KeywordTok{return} \StringTok{'first name is empty'}\NormalTok{;}
    \KeywordTok{if}\NormalTok{(!}\OtherTok{attrs}\NormalTok{.}\FunctionTok{lastname}\NormalTok{) }\KeywordTok{return} \StringTok{'last name is empty'}\NormalTok{;}
    \KeywordTok{if}\NormalTok{(!}\OtherTok{attrs}\NormalTok{.}\FunctionTok{email}\NormalTok{) }\KeywordTok{return} \StringTok{'email is empty'}\NormalTok{;}

\NormalTok{\}}
\end{Highlighting}
\end{Shaded}

Unfortunately, this method would trigger a \texttt{firstname} error each
time any of the fields were blurred and only an error message next to
the first name field would be presented.

One potential solution to the problem is to validate all fields and
return all of the errors:

\begin{Shaded}
\begin{Highlighting}[]
\NormalTok{validate: }\KeywordTok{function}\NormalTok{(attrs) \{}
  \KeywordTok{var} \NormalTok{errors = \{\};}

  \KeywordTok{if} \NormalTok{(!}\OtherTok{attrs}\NormalTok{.}\FunctionTok{firstname}\NormalTok{) }\OtherTok{errors}\NormalTok{.}\FunctionTok{firstname} \NormalTok{= }\StringTok{'first name is empty'}\NormalTok{;}
  \KeywordTok{if} \NormalTok{(!}\OtherTok{attrs}\NormalTok{.}\FunctionTok{lastname}\NormalTok{) }\OtherTok{errors}\NormalTok{.}\FunctionTok{lastname} \NormalTok{= }\StringTok{'last name is empty'}\NormalTok{;}
  \KeywordTok{if} \NormalTok{(!}\OtherTok{attrs}\NormalTok{.}\FunctionTok{email}\NormalTok{) }\OtherTok{errors}\NormalTok{.}\FunctionTok{email} \NormalTok{= }\StringTok{'email is empty'}\NormalTok{;}

  \KeywordTok{if} \NormalTok{(!}\OtherTok{_}\NormalTok{.}\FunctionTok{isEmpty}\NormalTok{(errors)) }\KeywordTok{return} \NormalTok{errors;}
\NormalTok{\}}
\end{Highlighting}
\end{Shaded}

This can be adapted into a solution that defines a Field model for each
input in our form and works within the parameters of our use case as
follows:

\begin{Shaded}
\begin{Highlighting}[]

\FunctionTok{$}\NormalTok{(}\KeywordTok{function}\NormalTok{($) \{}

  \KeywordTok{var} \NormalTok{User = }\OtherTok{Backbone}\NormalTok{.}\OtherTok{Model}\NormalTok{.}\FunctionTok{extend}\NormalTok{(\{}
    \DataTypeTok{validate}\NormalTok{: }\KeywordTok{function}\NormalTok{(attrs) \{}
      \KeywordTok{var} \NormalTok{errors = }\KeywordTok{this}\NormalTok{.}\FunctionTok{errors} \NormalTok{= \{\};}

      \KeywordTok{if} \NormalTok{(!}\OtherTok{attrs}\NormalTok{.}\FunctionTok{firstname}\NormalTok{) }\OtherTok{errors}\NormalTok{.}\FunctionTok{firstname} \NormalTok{= }\StringTok{'firstname is required'}\NormalTok{;}
      \KeywordTok{if} \NormalTok{(!}\OtherTok{attrs}\NormalTok{.}\FunctionTok{lastname}\NormalTok{) }\OtherTok{errors}\NormalTok{.}\FunctionTok{lastname} \NormalTok{= }\StringTok{'lastname is required'}\NormalTok{;}
      \KeywordTok{if} \NormalTok{(!}\OtherTok{attrs}\NormalTok{.}\FunctionTok{email}\NormalTok{) }\OtherTok{errors}\NormalTok{.}\FunctionTok{email} \NormalTok{= }\StringTok{'email is required'}\NormalTok{;}

      \KeywordTok{if} \NormalTok{(!}\OtherTok{_}\NormalTok{.}\FunctionTok{isEmpty}\NormalTok{(errors)) }\KeywordTok{return} \NormalTok{errors;}
    \NormalTok{\}}
  \NormalTok{\});}

  \KeywordTok{var} \NormalTok{Field = }\OtherTok{Backbone}\NormalTok{.}\OtherTok{View}\NormalTok{.}\FunctionTok{extend}\NormalTok{(\{}
    \DataTypeTok{events}\NormalTok{: \{}\DataTypeTok{blur}\NormalTok{: }\StringTok{'validate'}\NormalTok{\},}
    \DataTypeTok{initialize}\NormalTok{: }\KeywordTok{function}\NormalTok{() \{}
      \KeywordTok{this}\NormalTok{.}\FunctionTok{name} \NormalTok{= }\KeywordTok{this}\NormalTok{.}\OtherTok{$el}\NormalTok{.}\FunctionTok{attr}\NormalTok{(}\StringTok{'name'}\NormalTok{);}
      \KeywordTok{this}\NormalTok{.}\FunctionTok{$msg} \NormalTok{= }\FunctionTok{$}\NormalTok{(}\StringTok{'[data-msg='} \NormalTok{+ }\KeywordTok{this}\NormalTok{.}\FunctionTok{name} \NormalTok{+ }\StringTok{']'}\NormalTok{);}
    \NormalTok{\},}
    \DataTypeTok{validate}\NormalTok{: }\KeywordTok{function}\NormalTok{() \{}
      \KeywordTok{this}\NormalTok{.}\OtherTok{model}\NormalTok{.}\FunctionTok{set}\NormalTok{(}\KeywordTok{this}\NormalTok{.}\FunctionTok{name}\NormalTok{, }\KeywordTok{this}\NormalTok{.}\OtherTok{$el}\NormalTok{.}\FunctionTok{val}\NormalTok{(), \{}\DataTypeTok{validate}\NormalTok{: }\KeywordTok{true}\NormalTok{\});}
      \KeywordTok{this}\NormalTok{.}\OtherTok{$msg}\NormalTok{.}\FunctionTok{text}\NormalTok{(}\KeywordTok{this}\NormalTok{.}\OtherTok{model}\NormalTok{.}\FunctionTok{errors}\NormalTok{[}\KeywordTok{this}\NormalTok{.}\FunctionTok{name}\NormalTok{] || }\StringTok{''}\NormalTok{);}
    \NormalTok{\}}
  \NormalTok{\});}

  \KeywordTok{var} \NormalTok{user = }\KeywordTok{new} \NormalTok{User;}

  \FunctionTok{$}\NormalTok{(}\StringTok{'input'}\NormalTok{).}\FunctionTok{each}\NormalTok{(}\KeywordTok{function}\NormalTok{() \{}
    \KeywordTok{new} \FunctionTok{Field}\NormalTok{(\{}\DataTypeTok{el}\NormalTok{: }\KeywordTok{this}\NormalTok{, }\DataTypeTok{model}\NormalTok{: user\});}
  \NormalTok{\});}

\NormalTok{\});}
\end{Highlighting}
\end{Shaded}

This works fine as the solution checks the validation for each attribute
individually and sets the message for the correct blurred field. A
\href{http://jsbin.com/afetez/2/edit}{demo} of the above by
{[}@braddunbar{]}(http://github.com/braddunbar) is also available.

Unfortunately, this solution does perform validation on all fields every
time, even though we are only displaying errors for the field that has
changed. If we have multiple client-side validation methods, we may not
want to have to call each validation method on every attribute every
time, so this solution might not be ideal for everyone.

\subparagraph{Backbone.validateAll}\label{backbone.validateall}

A potentially better alternative to the above is to use
{[}@gfranko{]}(http://github.com/@franko)'s
\href{https://github.com/gfranko/Backbone.validateAll}{Backbone.validateAll}
plugin, specifically created to validate specific Model properties (or
form fields) without worrying about the validation of any other Model
properties (or form fields).

Here is how we would setup a partial User Model and validate method
using this plugin for our use case:

\begin{Shaded}
\begin{Highlighting}[]

\CommentTok{// Create a new User Model}
\KeywordTok{var} \NormalTok{User = }\OtherTok{Backbone}\NormalTok{.}\OtherTok{Model}\NormalTok{.}\FunctionTok{extend}\NormalTok{(\{}

      \CommentTok{// RegEx Patterns}
      \DataTypeTok{patterns}\NormalTok{: \{}

          \DataTypeTok{specialCharacters}\NormalTok{: }\StringTok{'[^a-zA-Z 0-9]+'}\NormalTok{,}

          \DataTypeTok{digits}\NormalTok{: }\StringTok{'[0-9]'}\NormalTok{,}

          \DataTypeTok{email}\NormalTok{: }\StringTok{'^[a-zA-Z0-9._-]+@[a-zA-Z0-9][a-zA-Z0-9.-]*[.]\{1\}[a-zA-Z]\{2,6\}$'}
      \NormalTok{\},}

    \CommentTok{// Validators}
      \DataTypeTok{validators}\NormalTok{: \{}

          \DataTypeTok{minLength}\NormalTok{: }\KeywordTok{function}\NormalTok{(value, minLength) \{}
            \KeywordTok{return} \OtherTok{value}\NormalTok{.}\FunctionTok{length} \NormalTok{>= minLength;}

          \NormalTok{\},}

          \DataTypeTok{maxLength}\NormalTok{: }\KeywordTok{function}\NormalTok{(value, maxLength) \{}
            \KeywordTok{return} \OtherTok{value}\NormalTok{.}\FunctionTok{length} \NormalTok{<= maxLength;}

          \NormalTok{\},}

          \DataTypeTok{isEmail}\NormalTok{: }\KeywordTok{function}\NormalTok{(value) \{}
            \KeywordTok{return} \OtherTok{User}\NormalTok{.}\OtherTok{prototype}\NormalTok{.}\OtherTok{validators}\NormalTok{.}\FunctionTok{pattern}\NormalTok{(value, }\OtherTok{User}\NormalTok{.}\OtherTok{prototype}\NormalTok{.}\OtherTok{patterns}\NormalTok{.}\FunctionTok{email}\NormalTok{);}

          \NormalTok{\},}

          \DataTypeTok{hasSpecialCharacter}\NormalTok{: }\KeywordTok{function}\NormalTok{(value) \{}
            \KeywordTok{return} \OtherTok{User}\NormalTok{.}\OtherTok{prototype}\NormalTok{.}\OtherTok{validators}\NormalTok{.}\FunctionTok{pattern}\NormalTok{(value, }\OtherTok{User}\NormalTok{.}\OtherTok{prototype}\NormalTok{.}\OtherTok{patterns}\NormalTok{.}\FunctionTok{specialCharacters}\NormalTok{);}

          \NormalTok{\},}
         \NormalTok{...}

    \CommentTok{// We can determine which properties are getting validated by}
    \CommentTok{// checking to see if properties are equal to null}

        \DataTypeTok{validate}\NormalTok{: }\KeywordTok{function}\NormalTok{(attrs) \{}

          \KeywordTok{var} \NormalTok{errors = }\KeywordTok{this}\NormalTok{.}\FunctionTok{errors} \NormalTok{= \{\};}

          \KeywordTok{if}\NormalTok{(}\OtherTok{attrs}\NormalTok{.}\FunctionTok{firstname} \NormalTok{!= }\KeywordTok{null}\NormalTok{) \{}
              \KeywordTok{if} \NormalTok{(!}\OtherTok{attrs}\NormalTok{.}\FunctionTok{firstname}\NormalTok{) \{}
                  \OtherTok{errors}\NormalTok{.}\FunctionTok{firstname} \NormalTok{= }\StringTok{'firstname is required'}\NormalTok{;}
                  \OtherTok{console}\NormalTok{.}\FunctionTok{log}\NormalTok{(}\StringTok{'first name isEmpty validation called'}\NormalTok{);}
              \NormalTok{\}}

              \KeywordTok{else} \KeywordTok{if}\NormalTok{(!}\KeywordTok{this}\NormalTok{.}\OtherTok{validators}\NormalTok{.}\FunctionTok{minLength}\NormalTok{(}\OtherTok{attrs}\NormalTok{.}\FunctionTok{firstname}\NormalTok{, }\DecValTok{2}\NormalTok{))}
                \OtherTok{errors}\NormalTok{.}\FunctionTok{firstname} \NormalTok{= }\StringTok{'firstname is too short'}\NormalTok{;}
              \KeywordTok{else} \KeywordTok{if}\NormalTok{(!}\KeywordTok{this}\NormalTok{.}\OtherTok{validators}\NormalTok{.}\FunctionTok{maxLength}\NormalTok{(}\OtherTok{attrs}\NormalTok{.}\FunctionTok{firstname}\NormalTok{, }\DecValTok{15}\NormalTok{))}
                \OtherTok{errors}\NormalTok{.}\FunctionTok{firstname} \NormalTok{= }\StringTok{'firstname is too large'}\NormalTok{;}
              \KeywordTok{else} \KeywordTok{if}\NormalTok{(}\KeywordTok{this}\NormalTok{.}\OtherTok{validators}\NormalTok{.}\FunctionTok{hasSpecialCharacter}\NormalTok{(}\OtherTok{attrs}\NormalTok{.}\FunctionTok{firstname}\NormalTok{)) }\OtherTok{errors}\NormalTok{.}\FunctionTok{firstname} \NormalTok{= }\StringTok{'firstname cannot contain special characters'}\NormalTok{;}
          \NormalTok{\}}

          \KeywordTok{if}\NormalTok{(}\OtherTok{attrs}\NormalTok{.}\FunctionTok{lastname} \NormalTok{!= }\KeywordTok{null}\NormalTok{) \{}

              \KeywordTok{if} \NormalTok{(!}\OtherTok{attrs}\NormalTok{.}\FunctionTok{lastname}\NormalTok{) \{}
                  \OtherTok{errors}\NormalTok{.}\FunctionTok{lastname} \NormalTok{= }\StringTok{'lastname is required'}\NormalTok{;}
                  \OtherTok{console}\NormalTok{.}\FunctionTok{log}\NormalTok{(}\StringTok{'last name isEmpty validation called'}\NormalTok{);}
              \NormalTok{\}}

              \KeywordTok{else} \KeywordTok{if}\NormalTok{(!}\KeywordTok{this}\NormalTok{.}\OtherTok{validators}\NormalTok{.}\FunctionTok{minLength}\NormalTok{(}\OtherTok{attrs}\NormalTok{.}\FunctionTok{lastname}\NormalTok{, }\DecValTok{2}\NormalTok{))}
                \OtherTok{errors}\NormalTok{.}\FunctionTok{lastname} \NormalTok{= }\StringTok{'lastname is too short'}\NormalTok{;}
              \KeywordTok{else} \KeywordTok{if}\NormalTok{(!}\KeywordTok{this}\NormalTok{.}\OtherTok{validators}\NormalTok{.}\FunctionTok{maxLength}\NormalTok{(}\OtherTok{attrs}\NormalTok{.}\FunctionTok{lastname}\NormalTok{, }\DecValTok{15}\NormalTok{))}
                \OtherTok{errors}\NormalTok{.}\FunctionTok{lastname} \NormalTok{= }\StringTok{'lastname is too large'}\NormalTok{;}
              \KeywordTok{else} \KeywordTok{if}\NormalTok{(}\KeywordTok{this}\NormalTok{.}\OtherTok{validators}\NormalTok{.}\FunctionTok{hasSpecialCharacter}\NormalTok{(}\OtherTok{attrs}\NormalTok{.}\FunctionTok{lastname}\NormalTok{)) }\OtherTok{errors}\NormalTok{.}\FunctionTok{lastname} \NormalTok{= }\StringTok{'lastname cannot contain special characters'}\NormalTok{;}

          \NormalTok{\}}
\end{Highlighting}
\end{Shaded}

This allows the logic inside of our validate methods to determine which
form fields are currently being set/validated, and ignore the model
properties that are not being set.

It's fairly straight-forward to use as well. We can simply define a new
Model instance and then set the data on our model using the
\texttt{validateAll} option to use the behavior defined by the plugin:

\begin{Shaded}
\begin{Highlighting}[]
\KeywordTok{var} \NormalTok{user = }\KeywordTok{new} \FunctionTok{User}\NormalTok{();}
\OtherTok{user}\NormalTok{.}\FunctionTok{set}\NormalTok{(\{ }\StringTok{'firstname'}\NormalTok{: }\StringTok{'Greg'} \NormalTok{\}, \{}\DataTypeTok{validate}\NormalTok{: }\KeywordTok{true}\NormalTok{, }\DataTypeTok{validateAll}\NormalTok{: }\KeywordTok{false}\NormalTok{\});}
\end{Highlighting}
\end{Shaded}

That's it. The Backbone.validateAll logic doesn't override the default
Backbone logic by default and so it's perfectly capable of being used
for scenarios where you might care more about field-validation
\href{http://jsperf.com/backbone-validateall}{performance} as well as
those where you don't. Both solutions presented in this section should
work fine however.

\subparagraph{Backbone.Validation}\label{backbone.validation}

As we've seen, the \texttt{validate} method Backbone offers is
\texttt{undefined} by default and you need to override it with your own
custom validation logic to get model validation in place. Often
developers run into the issue of implementing this validation as nested
ifs and elses, which can become unmaintainable when things get
complicated.

Another helpful plugin for Backbone called
\href{https://github.com/thedersen/backbone.validation}{Backbone.Validation}
attempts to solve this problem by offering an extensible way to declare
validation rules on the model and overrides the \texttt{validate} method
behind the scenes.

One of the useful methods this plugin includes is (pseudo) live
validation via a \texttt{preValidate} method. This can be used to check
on key-press if the input for a model is valid without changing the
model itself. You can run any validators for a model attribute by
calling the \texttt{preValidate} method, passing it the name of the
attribute along with the value you would like validated.

\begin{Shaded}
\begin{Highlighting}[]
\CommentTok{// If the value of the attribute is invalid, a truthy error message is returned}
\CommentTok{// if not, it returns a falsy value}

\KeywordTok{var} \NormalTok{errorMsg = }\OtherTok{user}\NormalTok{.}\FunctionTok{preValidate}\NormalTok{(}\StringTok{'firstname'}\NormalTok{, }\StringTok{'Greg'}\NormalTok{);}
\end{Highlighting}
\end{Shaded}

\subparagraph{Form-specific validation
classes}\label{form-specific-validation-classes}

That said, the most optimal solution to this problem may not be to stick
validation in your model attributes. Instead, you could have a function
specifically designed for validating a specific form and there are many
good JavaScript form validation libraries out there that can help with
this.

If you want to stick it on your model, you can also make it a class
function:

\begin{Shaded}
\begin{Highlighting}[]
\OtherTok{User}\NormalTok{.}\FunctionTok{validate} \NormalTok{= }\KeywordTok{function}\NormalTok{(formElement) \{}
  \CommentTok{//...}
\NormalTok{\};}
\end{Highlighting}
\end{Shaded}

For more information on validation plugins available for Backbone, see
the
\href{https://github.com/documentcloud/backbone/wiki/Extensions\%2C-Plugins\%2C-Resources\#model}{Backbone
wiki}.

\paragraph{Avoiding Conflicts With Multiple Backbone
Versions}\label{avoiding-conflicts-with-multiple-backbone-versions}

\textbf{Problem}

In instances out of your control, you may have to work around having
more than one version of Backbone in the same page. How do you work
around this without causing conflicts?

\textbf{Solution}

Like most client-side projects, Backbone's code is wrapped in an
immediately-invoked function expression:

\begin{Shaded}
\begin{Highlighting}[]
\NormalTok{(}\KeywordTok{function}\NormalTok{()\{}
  \CommentTok{// Backbone.js}
\NormalTok{\}).}\FunctionTok{call}\NormalTok{(}\KeywordTok{this}\NormalTok{);}
\end{Highlighting}
\end{Shaded}

Several things happen during this configuration stage. A Backbone
\texttt{namespace} is created, and multiple versions of Backbone on the
same page are supported through the noConflict mode:

\begin{Shaded}
\begin{Highlighting}[]
\KeywordTok{var} \NormalTok{root = }\KeywordTok{this}\NormalTok{;}
\KeywordTok{var} \NormalTok{previousBackbone = }\OtherTok{root}\NormalTok{.}\FunctionTok{Backbone}\NormalTok{;}

\OtherTok{Backbone}\NormalTok{.}\FunctionTok{noConflict} \NormalTok{= }\KeywordTok{function}\NormalTok{() \{}
  \OtherTok{root}\NormalTok{.}\FunctionTok{Backbone} \NormalTok{= previousBackbone;}
  \KeywordTok{return} \KeywordTok{this}\NormalTok{;}
\NormalTok{\};}
\end{Highlighting}
\end{Shaded}

Multiple versions of Backbone can be used on the same page by calling
\texttt{noConflict} like this:

\begin{Shaded}
\begin{Highlighting}[]
\KeywordTok{var} \NormalTok{Backbone19 = }\OtherTok{Backbone}\NormalTok{.}\FunctionTok{noConflict}\NormalTok{();}
\CommentTok{// Backbone19 refers to the most recently loaded version,}
\CommentTok{// and `window.Backbone` will be restored to the previously}
\CommentTok{// loaded version}
\end{Highlighting}
\end{Shaded}

\paragraph{Building Model And View
Hierarchies}\label{building-model-and-view-hierarchies}

\textbf{Problem}

How does inheritance work with Backbone? How can I share code between
similar models and views? How can I call methods that have been
overridden?

\textbf{Solution}

For its inheritance, Backbone internally uses an \texttt{inherits}
function inspired by \texttt{goog.inherits}, Google's implementation
from the Closure Library. It's basically a function to correctly setup
the prototype chain.

\begin{Shaded}
\begin{Highlighting}[]
 \KeywordTok{var} \NormalTok{inherits = }\KeywordTok{function}\NormalTok{(parent, protoProps, staticProps) \{}
      \NormalTok{...}
\end{Highlighting}
\end{Shaded}

The only major difference here is that Backbone's API accepts two
objects containing \texttt{instance} and \texttt{static} methods.

Following on from this, for inheritance purposes all of Backbone's
objects contain an \texttt{extend} method as follows:

\begin{Shaded}
\begin{Highlighting}[]
\OtherTok{Model}\NormalTok{.}\FunctionTok{extend} \NormalTok{= }\OtherTok{Collection}\NormalTok{.}\FunctionTok{extend} \NormalTok{= }\OtherTok{Router}\NormalTok{.}\FunctionTok{extend} \NormalTok{= }\OtherTok{View}\NormalTok{.}\FunctionTok{extend} \NormalTok{= extend;}
\end{Highlighting}
\end{Shaded}

Most development with Backbone is based around inheriting from these
objects, and they're designed to mimic a classical object-oriented
implementation.

The above isn't quite the same as ECMAScript 5's \texttt{Object.create},
as it's actually copying properties (methods and values) from one object
to another. As this isn't enough to support Backbone's inheritance and
class model, the following steps are performed:

\begin{itemize}
\itemsep1pt\parskip0pt\parsep0pt
\item
  The instance methods are checked to see if there's a constructor
  property. If so, the class's constructor is used, otherwise the
  parent's constructor is used (i.e., Backbone.Model)
\item
  Underscore's extend method is called to add the parent class's methods
  to the new child class
\item
  The \texttt{prototype} property of a blank constructor function is
  assigned with the parent's prototype, and a new instance of this is
  set to the child's \texttt{prototype} property
\item
  Underscore's extend method is called twice to add the static and
  instance methods to the child class
\item
  The child's prototype's constructor and a \texttt{\_\_super\_\_}
  property are assigned
\item
  This pattern is also used for classes in CoffeeScript, so Backbone
  classes are compatible with CoffeeScript classes.
\end{itemize}

\texttt{extend} can be used for a great deal more and developers who are
fans of mixins will like that it can be used for this too. You can
define functionality on any custom object, and then quite literally copy
\& paste all of the methods and attributes from that object to a
Backbone one:

For example:

\begin{Shaded}
\begin{Highlighting}[]
\KeywordTok{var} \NormalTok{MyMixin = \{}
  \DataTypeTok{foo}\NormalTok{: }\StringTok{'bar'}\NormalTok{,}
  \DataTypeTok{sayFoo}\NormalTok{: }\KeywordTok{function}\NormalTok{()\{}\FunctionTok{alert}\NormalTok{(}\KeywordTok{this}\NormalTok{.}\FunctionTok{foo}\NormalTok{);\}}
\NormalTok{\};}

\KeywordTok{var} \NormalTok{MyView = }\OtherTok{Backbone}\NormalTok{.}\OtherTok{View}\NormalTok{.}\FunctionTok{extend}\NormalTok{(\{}
 \CommentTok{// ...}
\NormalTok{\});}

\OtherTok{_}\NormalTok{.}\FunctionTok{extend}\NormalTok{(}\OtherTok{MyView}\NormalTok{.}\FunctionTok{prototype}\NormalTok{, MyMixin);}

\KeywordTok{var} \NormalTok{myView = }\KeywordTok{new} \FunctionTok{MyView}\NormalTok{();}
\OtherTok{myView}\NormalTok{.}\FunctionTok{sayFoo}\NormalTok{(); }\CommentTok{//=> 'bar'}
\end{Highlighting}
\end{Shaded}

We can take this further and also apply it to View inheritance. The
following is an example of how to extend one View using another:

\begin{Shaded}
\begin{Highlighting}[]
\KeywordTok{var} \NormalTok{Panel = }\OtherTok{Backbone}\NormalTok{.}\OtherTok{View}\NormalTok{.}\FunctionTok{extend}\NormalTok{(\{}
\NormalTok{\});}

\KeywordTok{var} \NormalTok{PanelAdvanced = }\OtherTok{Panel}\NormalTok{.}\FunctionTok{extend}\NormalTok{(\{}
\NormalTok{\});}
\end{Highlighting}
\end{Shaded}

\textbf{Calling Overridden Methods}

However, if you have an \texttt{initialize()} method in Panel, then it
won't be called if you also have an \texttt{initialize()} method in
PanelAdvanced, so you would have to call Panel's initialize method
explicitly:

\begin{Shaded}
\begin{Highlighting}[]
\KeywordTok{var} \NormalTok{Panel = }\OtherTok{Backbone}\NormalTok{.}\OtherTok{View}\NormalTok{.}\FunctionTok{extend}\NormalTok{(\{}
  \DataTypeTok{initialize}\NormalTok{: }\KeywordTok{function}\NormalTok{(options)\{}
    \OtherTok{console}\NormalTok{.}\FunctionTok{log}\NormalTok{(}\StringTok{'Panel initialized'}\NormalTok{);}
    \KeywordTok{this}\NormalTok{.}\FunctionTok{foo} \NormalTok{= }\StringTok{'bar'}\NormalTok{;}
  \NormalTok{\}}
\NormalTok{\});}

\KeywordTok{var} \NormalTok{PanelAdvanced = }\OtherTok{Panel}\NormalTok{.}\FunctionTok{extend}\NormalTok{(\{}
  \DataTypeTok{initialize}\NormalTok{: }\KeywordTok{function}\NormalTok{(options)\{}
    \OtherTok{Panel}\NormalTok{.}\OtherTok{prototype}\NormalTok{.}\OtherTok{initialize}\NormalTok{.}\FunctionTok{call}\NormalTok{(}\KeywordTok{this}\NormalTok{, [options]);}
    \OtherTok{console}\NormalTok{.}\FunctionTok{log}\NormalTok{(}\StringTok{'PanelAdvanced initialized'}\NormalTok{);}
    \OtherTok{console}\NormalTok{.}\FunctionTok{log}\NormalTok{(}\KeywordTok{this}\NormalTok{.}\FunctionTok{foo}\NormalTok{); }\CommentTok{// Log: bar}
  \NormalTok{\}}
\NormalTok{\});}

\CommentTok{// We can also inherit PanelAdvaned if needed}
\KeywordTok{var} \NormalTok{PanelAdvancedExtra = }\OtherTok{PanelAdvanced}\NormalTok{.}\FunctionTok{extend}\NormalTok{(\{}
  \DataTypeTok{initialize}\NormalTok{: }\KeywordTok{function}\NormalTok{(options)\{}
    \OtherTok{PanelAdvanced}\NormalTok{.}\OtherTok{prototype}\NormalTok{.}\OtherTok{initialize}\NormalTok{.}\FunctionTok{call}\NormalTok{(}\KeywordTok{this}\NormalTok{, [options]);}
    \OtherTok{console}\NormalTok{.}\FunctionTok{log}\NormalTok{(}\StringTok{'PanelAdvancedExtra initialized'}\NormalTok{);}
  \NormalTok{\}}
\NormalTok{\});}

\KeywordTok{new} \FunctionTok{Panel}\NormalTok{();}
\KeywordTok{new} \FunctionTok{PanelAdvanced}\NormalTok{();}
\KeywordTok{new} \FunctionTok{PanelAdvancedExtra}\NormalTok{();}
\end{Highlighting}
\end{Shaded}

This isn't the most elegant of solutions because if you have a lot of
Views that inherit from Panel, then you'll have to remember to call
Panel's initialize from all of them.

It's worth noting that if Panel doesn't have an initialize method now
but you choose to add it in the future, then you'll need to go to all of
the inherited classes in the future and make sure they call Panel's
initialize.

So here's an alternative way to define Panel so that your inherited
views don't need to call Panel's initialize method:

\begin{Shaded}
\begin{Highlighting}[]
\KeywordTok{var} \NormalTok{Panel = }\KeywordTok{function} \NormalTok{(options) \{}
  \CommentTok{// put all of Panel's initialization code here}
  \OtherTok{console}\NormalTok{.}\FunctionTok{log}\NormalTok{(}\StringTok{'Panel initialized'}\NormalTok{);}
  \KeywordTok{this}\NormalTok{.}\FunctionTok{foo} \NormalTok{= }\StringTok{'bar'}\NormalTok{;}

  \OtherTok{Backbone}\NormalTok{.}\OtherTok{View}\NormalTok{.}\FunctionTok{apply}\NormalTok{(}\KeywordTok{this}\NormalTok{, [options]);}
\NormalTok{\};}

\OtherTok{_}\NormalTok{.}\FunctionTok{extend}\NormalTok{(}\OtherTok{Panel}\NormalTok{.}\FunctionTok{prototype}\NormalTok{, }\OtherTok{Backbone}\NormalTok{.}\OtherTok{View}\NormalTok{.}\FunctionTok{prototype}\NormalTok{, \{}
  \CommentTok{// put all of Panel's methods here. For example:}
  \DataTypeTok{sayHi}\NormalTok{: }\KeywordTok{function} \NormalTok{() \{}
    \OtherTok{console}\NormalTok{.}\FunctionTok{log}\NormalTok{(}\StringTok{'hello from Panel'}\NormalTok{);}
  \NormalTok{\}}
\NormalTok{\});}

\OtherTok{Panel}\NormalTok{.}\FunctionTok{extend} \NormalTok{= }\OtherTok{Backbone}\NormalTok{.}\OtherTok{View}\NormalTok{.}\FunctionTok{extend}\NormalTok{;}

\CommentTok{// other classes then inherit from Panel like this:}
\KeywordTok{var} \NormalTok{PanelAdvanced = }\OtherTok{Panel}\NormalTok{.}\FunctionTok{extend}\NormalTok{(\{}
  \DataTypeTok{initialize}\NormalTok{: }\KeywordTok{function} \NormalTok{(options) \{}
    \OtherTok{console}\NormalTok{.}\FunctionTok{log}\NormalTok{(}\StringTok{'PanelAdvanced initialized'}\NormalTok{);}
    \OtherTok{console}\NormalTok{.}\FunctionTok{log}\NormalTok{(}\KeywordTok{this}\NormalTok{.}\FunctionTok{foo}\NormalTok{);}
  \NormalTok{\}}
\NormalTok{\});}

\KeywordTok{var} \NormalTok{panelAdvanced = }\KeywordTok{new} \FunctionTok{PanelAdvanced}\NormalTok{(); }\CommentTok{//Logs: Panel initialized, PanelAdvanced initialized, bar}
\OtherTok{panelAdvanced}\NormalTok{.}\FunctionTok{sayHi}\NormalTok{(); }\CommentTok{// Logs: hello from Panel}
\end{Highlighting}
\end{Shaded}

When used appropriately, Underscore's \texttt{extend} method can save a
great deal of time and effort writing redundant code.

(Thanks to \href{http://dailyjs.com}{Alex Young},
\href{http://stackoverflow.com/users/93448/derick-bailey}{Derick Bailey}
and \href{http://stackoverflow.com/users/188740/johnnyo}{JohnnyO} for
the heads up about these tips).

\textbf{Backbone-Super}

\href{https://github.com/lukasolson/Backbone-Super}{Backbone-Super} by
Lukas Olson adds a *\emph{super* method to \emph{Backbone.Model} using
\href{http://ejohn.org/blog/simple-javascript-inheritance/}{John Resig's
Inheritance script}. Rather than using Backbone.Model.prototype.set.call
as per the Backbone.js documentation,}super can be called instead:

\begin{Shaded}
\begin{Highlighting}[]
\CommentTok{// This is how we normally do it}
\KeywordTok{var} \NormalTok{OldFashionedNote = }\OtherTok{Backbone}\NormalTok{.}\OtherTok{Model}\NormalTok{.}\FunctionTok{extend}\NormalTok{(\{}
  \DataTypeTok{set}\NormalTok{: }\KeywordTok{function}\NormalTok{(attributes, options) \{}
    \CommentTok{// Call parent's method}
    \OtherTok{Backbone}\NormalTok{.}\OtherTok{Model}\NormalTok{.}\OtherTok{prototype}\NormalTok{.}\OtherTok{set}\NormalTok{.}\FunctionTok{call}\NormalTok{(}\KeywordTok{this}\NormalTok{, attributes, options);}
    \CommentTok{// some custom code here}
    \CommentTok{// ...}
  \NormalTok{\}}
\NormalTok{\});}
\end{Highlighting}
\end{Shaded}

After including this plugin, you can do the same thing with the
following syntax:

\begin{Shaded}
\begin{Highlighting}[]
\CommentTok{// This is how we can do it after using the Backbone-super plugin}
\KeywordTok{var} \NormalTok{Note = }\OtherTok{Backbone}\NormalTok{.}\OtherTok{Model}\NormalTok{.}\FunctionTok{extend}\NormalTok{(\{}
  \DataTypeTok{set}\NormalTok{: }\KeywordTok{function}\NormalTok{(attributes, options) \{}
    \CommentTok{// Call parent's method}
    \KeywordTok{this}\NormalTok{.}\FunctionTok{_super}\NormalTok{(attributes, options);}
    \CommentTok{// some custom code here}
    \CommentTok{// ...}
  \NormalTok{\}}
\NormalTok{\});}
\end{Highlighting}
\end{Shaded}

\paragraph{Event Aggregators And
Mediators}\label{event-aggregators-and-mediators}

\textbf{Problem}

How do I channel multiple event sources through a single object?

\textbf{Solution}

Using an Event Aggregator. It's common for developers to think of
Mediators when faced with this problem, so let's explore what an Event
Aggregator is, what the Mediator pattern is and how they differ.

Design patterns often differ only in semantics and intent. That is, the
language used to describe the pattern is what sets it apart, more than
an implementation of that specific pattern. It often comes down to
squares vs rectangles vs polygons. You can create the same end result
with all three, given the constraints of a square are still met -- or
you can use polygons to create an infinitely larger and more complex set
of things.

When it comes to the Mediator and Event Aggregator patterns, there are
some times where it may look like the patterns are interchangeable due
to implementation similarities. However, the semantics and intent of
these patterns are very different. And even if the implementations both
use some of the same core constructs, I believe there is a distinct
difference between them. I also believe they should not be interchanged
or confused in communication because of the differences.

\subparagraph{Event Aggregator}\label{event-aggregator}

The core idea of the Event Aggregator, according to Martin Fowler, is to
channel multiple event sources through a single object so that other
objects needing to subscribe to the events don't need to know about
every event source.

Backbone's Event Aggregator

The easiest event aggregator to show is that of Backbone.js -- it's
built into the Backbone object directly.

\begin{Shaded}
\begin{Highlighting}[]
\KeywordTok{var} \NormalTok{View1 = }\OtherTok{Backbone}\NormalTok{.}\OtherTok{View}\NormalTok{.}\FunctionTok{extend}\NormalTok{(\{}
  \CommentTok{// ...}

  \DataTypeTok{events}\NormalTok{: \{}
    \StringTok{"click .foo"}\NormalTok{: }\StringTok{"doIt"}
  \NormalTok{\},}

  \DataTypeTok{doIt}\NormalTok{: }\KeywordTok{function}\NormalTok{()\{}
    \CommentTok{// trigger an event through the event aggregator}
    \OtherTok{Backbone}\NormalTok{.}\FunctionTok{trigger}\NormalTok{(}\StringTok{"some:event"}\NormalTok{);}
  \NormalTok{\}}
\NormalTok{\});}

\KeywordTok{var} \NormalTok{View2 = }\OtherTok{Backbone}\NormalTok{.}\OtherTok{View}\NormalTok{.}\FunctionTok{extend}\NormalTok{(\{}
  \CommentTok{// ...}

  \DataTypeTok{initialize}\NormalTok{: }\KeywordTok{function}\NormalTok{()\{}
    \CommentTok{// subscribe to the event aggregator's event}
    \OtherTok{Backbone}\NormalTok{.}\FunctionTok{on}\NormalTok{(}\StringTok{"some:event"}\NormalTok{, }\KeywordTok{this}\NormalTok{.}\FunctionTok{doStuff}\NormalTok{, }\KeywordTok{this}\NormalTok{);}
  \NormalTok{\},}

  \DataTypeTok{doStuff}\NormalTok{: }\KeywordTok{function}\NormalTok{()\{}
    \CommentTok{// ...}
  \NormalTok{\}}
\NormalTok{\})}
\end{Highlighting}
\end{Shaded}

In this example, the first view is triggering an event when a DOM
element is clicked. The event is triggered through Backbone's built-in
event aggregator -- the Backbone object. Of course, it's trivial to
create your own event aggregator in Backbone, and there are some key
things that we need to keep in mind when using an event aggregator, to
keep our code simple.

jQuery's Event Aggregator

Did you know that jQuery has a built-in event aggregator? They don't
call it this, but it's in there and it's scoped to DOM events. It also
happens to look like Backbone's event aggregator:

\begin{verbatim}
$("#mainArticle").on("click", function(e){

  // handle click event on any element underneath our #mainArticle element

});
\end{verbatim}

This code sets up an event handler function that waits for an unknown
number of event sources to trigger a ``click'' event, and it allows any
number of listeners to attach to the events of those event publishers.
jQuery just happens to scope this event aggregator to the DOM.

\subparagraph{Mediator}\label{mediator}

A Mediator is an object that coordinates interactions (logic and
behavior) between multiple objects. It makes decisions on when to call
which objects, based on the actions (or inaction) of other objects and
input.

\textbf{A Mediator For Backbone}

Backbone doesn't have the idea of a mediator built into it like a lot of
MV* frameworks do. But that doesn't mean you can't write one using a
single line of code:

\begin{Shaded}
\begin{Highlighting}[]
\KeywordTok{var} \NormalTok{mediator = \{\};}
\end{Highlighting}
\end{Shaded}

Yes, of course this is just an object literal in JavaScript. Once again,
we're talking about semantics here. The purpose of the mediator is to
control the workflow between objects and we really don't need anything
more than an object literal to do this.

\begin{Shaded}
\begin{Highlighting}[]
\KeywordTok{var} \NormalTok{orgChart = \{}

  \DataTypeTok{addNewEmployee}\NormalTok{: }\KeywordTok{function}\NormalTok{()\{}

    \CommentTok{// getEmployeeDetail provides a view that users interact with}
    \KeywordTok{var} \NormalTok{employeeDetail = }\KeywordTok{this}\NormalTok{.}\FunctionTok{getEmployeeDetail}\NormalTok{();}

    \CommentTok{// when the employee detail is complete, the mediator (the 'orgchart' object)}
    \CommentTok{// decides what should happen next}
    \OtherTok{employeeDetail}\NormalTok{.}\FunctionTok{on}\NormalTok{(}\StringTok{"complete"}\NormalTok{, }\KeywordTok{function}\NormalTok{(employee)\{}

      \CommentTok{// set up additional objects that have additional events, which are used}
      \CommentTok{// by the mediator to do additional things}
      \KeywordTok{var} \NormalTok{managerSelector = }\KeywordTok{this}\NormalTok{.}\FunctionTok{selectManager}\NormalTok{(employee);}
      \OtherTok{managerSelector}\NormalTok{.}\FunctionTok{on}\NormalTok{(}\StringTok{"save"}\NormalTok{, }\KeywordTok{function}\NormalTok{(employee)\{}
        \OtherTok{employee}\NormalTok{.}\FunctionTok{save}\NormalTok{();}
      \NormalTok{\});}

    \NormalTok{\});}
  \NormalTok{\},}

  \CommentTok{// ...}
\NormalTok{\}}
\end{Highlighting}
\end{Shaded}

This example shows a very basic implementation of a mediator object with
Backbone-based objects that can trigger and subscribe to events. I've
often referred to this type of object as a ``workflow'' object in the
past, but the truth is that it is a mediator. It is an object that
handles the workflow between many other objects, aggregating the
responsibility of that workflow knowledge into a single object. The
result is workflow that is easier to understand and maintain.

\subparagraph{Similarities And
Differences}\label{similarities-and-differences}

There are, without a doubt, similarities between the event aggregator
and mediator examples that I've shown here. The similarities boil down
to two primary items: events and third-party objects. These differences
are superficial at best, though. When we dig into the intent of the
pattern and see that the implementations can be dramatically different,
the nature of the patterns become more apparent.

Events

Both the event aggregator and mediator use events, in the above
examples. An event aggregator obviously deals with events -- it's in the
name after all. The mediator only uses events because it makes life easy
when dealing with Backbone, though. There is nothing that says a
mediator must be built with events. You can build a mediator with
callback methods, by handing the mediator reference to the child object,
or by any of a number of other means.

The difference, then, is why these two patterns are both using events.
The event aggregator, as a pattern, is designed to deal with events. The
mediator, though, only uses them because it's convenient.

Third-Party Objects

Both the event aggregator and mediator, by design, use a third-party
object to facilitate things. The event aggregator itself is a
third-party to the event publisher and the event subscriber. It acts as
a central hub for events to pass through. The mediator is also a third
party to other objects, though. So where is the difference? Why don't we
call an event aggregator a mediator? The answer largely comes down to
where the application logic and workflow is coded.

In the case of an event aggregator, the third party object is there only
to facilitate the pass-through of events from an unknown number of
sources to an unknown number of handlers. All workflow and business
logic that needs to be kicked off is put directly into the object that
triggers the events and the objects that handle the events.

In the case of the mediator, though, the business logic and workflow is
aggregated into the mediator itself. The mediator decides when an object
should have its methods called and attributes updated based on factors
that the mediator knows about. It encapsulates the workflow and process,
coordinating multiple objects to produce the desired system behaviour.
The individual objects involved in this workflow each know how to
perform their own task. But it's the mediator that tells the objects
when to perform the tasks by making decisions at a higher level than the
individual objects.

An event aggregator facilitates a ``fire and forget'' model of
communication. The object triggering the event doesn't care if there are
any subscribers. It just fires the event and moves on. A mediator,
though, might use events to make decisions, but it is definitely not
``fire and forget''. A mediator pays attention to a known set of input
or activities so that it can facilitate and coordinate additional
behavior with a known set of actors (objects).

\subparagraph{Relationships: When To Use
Which}\label{relationships-when-to-use-which}

Understanding the similarities and differences between an event
aggregator and mediator is important for semantic reasons. It's equally
as important to understand when to use which pattern, though. The basic
semantics and intent of the patterns does inform the question of when,
but actual experience in using the patterns will help you understand the
more subtle points and nuanced decisions that have to be made.

Event Aggregator Use

In general, an event aggregator is used when you either have too many
objects to listen to directly, or you have objects that are entirely
unrelated.

When two objects have a direct relationship already -- say, a parent
view and child view -- then there might be little benefit in using an
event aggregator. Have the child view trigger an event and the parent
view can handle the event. This is most commonly seen in Backbone's
Collection and Model, where all Model events are bubbled up to and
through its parent Collection. A Collection often uses model events to
modify the state of itself or other models. Handling ``selected'' items
in a collection is a good example of this.

jQuery's on method as an event aggregator is a great example of too many
objects to listen to. If you have 10, 20 or 200 DOM elements that can
trigger a ``click'' event, it might be a bad idea to set up a listener
on all of them individually. This could quickly deteriorate performance
of the application and user experience. Instead, using jQuery's on
method allows us to aggregate all of the events and reduce the overhead
of 10, 20, or 200 event handlers down to 1.

Indirect relationships are also a great time to use event aggregators.
In Backbone applications, it is very common to have multiple view
objects that need to communicate, but have no direct relationship. For
example, a menu system might have a view that handles the menu item
clicks. But we don't want the menu to be directly tied to the content
views that show all of the details and information when a menu item is
clicked. Having the content and menu coupled together would make the
code very difficult to maintain, in the long run. Instead, we can use an
event aggregator to trigger ``menu:click:foo'' events, and have a
``foo'' object handle the click event to show its content on the screen.

Mediator Use

A mediator is best applied when two or more objects have an indirect
working relationship, and business logic or workflow needs to dictate
the interactions and coordination of these objects.

A wizard interface is a good example of this, as shown with the
``orgChart'' example, above. There are multiple views that facilitate
the entire workflow of the wizard. Rather than tightly coupling the view
together by having them reference each other directly, we can decouple
them and more explicitly model the workflow between them by introducing
a mediator.

The mediator extracts the workflow from the implementation details and
creates a more natural abstraction at a higher level, showing us at a
much faster glance what that workflow is. We no longer have to dig into
the details of each view in the workflow, to see what the workflow
actually is.

\subparagraph{Event Aggregator And Mediator
Together}\label{event-aggregator-and-mediator-together}

The crux of the difference between an event aggregator and a mediator,
and why these pattern names should not be interchanged with each other,
is illustrated best by showing how they can be used together. The menu
example for an event aggregator is the perfect place to introduce a
mediator as well.

Clicking a menu item may trigger a series of changes throughout an
application. Some of these changes will be independent of others, and
using an event aggregator for this makes sense. Some of these changes
may be internally related to each other, though, and may use a mediator
to enact those changes. A mediator, then, could be set up to listen to
the event aggregator. It could run its logic and process to facilitate
and coordinate many objects that are related to each other, but
unrelated to the original event source.

\begin{Shaded}
\begin{Highlighting}[]
\KeywordTok{var} \NormalTok{MenuItem = }\OtherTok{Backbone}\NormalTok{.}\OtherTok{View}\NormalTok{.}\FunctionTok{extend}\NormalTok{(\{}

  \DataTypeTok{events}\NormalTok{: \{}
    \StringTok{"click .thatThing"}\NormalTok{: }\StringTok{"clickedIt"}
  \NormalTok{\},}

  \DataTypeTok{clickedIt}\NormalTok{: }\KeywordTok{function}\NormalTok{(e)\{}
    \OtherTok{e}\NormalTok{.}\FunctionTok{preventDefault}\NormalTok{();}

    \CommentTok{// assume this triggers "menu:click:foo"}
    \OtherTok{Backbone}\NormalTok{.}\FunctionTok{trigger}\NormalTok{(}\StringTok{"menu:click:"} \NormalTok{+ }\KeywordTok{this}\NormalTok{.}\OtherTok{model}\NormalTok{.}\FunctionTok{get}\NormalTok{(}\StringTok{"name"}\NormalTok{));}
  \NormalTok{\}}

\NormalTok{\});}

\CommentTok{// ... somewhere else in the app}

\KeywordTok{var} \NormalTok{MyWorkflow = }\KeywordTok{function}\NormalTok{()\{}
  \OtherTok{Backbone}\NormalTok{.}\FunctionTok{on}\NormalTok{(}\StringTok{"menu:click:foo"}\NormalTok{, }\KeywordTok{this}\NormalTok{.}\FunctionTok{doStuff}\NormalTok{, }\KeywordTok{this}\NormalTok{);}
\NormalTok{\};}

\OtherTok{MyWorkflow}\NormalTok{.}\OtherTok{prototype}\NormalTok{.}\FunctionTok{doStuff} \NormalTok{= }\KeywordTok{function}\NormalTok{()\{}
  \CommentTok{// instantiate multiple objects here.}
  \CommentTok{// set up event handlers for those objects.}
  \CommentTok{// coordinate all of the objects into a meaningful workflow.}
\NormalTok{\};}
\end{Highlighting}
\end{Shaded}

In this example, when the MenuItem with the right model is clicked, the
\texttt{“menu:click:foo”} event will be triggered. An instance of the
``MyWorkflow'' object, assuming one is already instantiated, will handle
this specific event and will coordinate all of the objects that it knows
about, to create the desired user experience and workflow.

An event aggregator and a mediator have been combined to create a much
more meaningful experience in both the code and the application itself.
We now have a clean separation between the menu and the workflow through
an event aggregator and we are still keeping the workflow itself clean
and maintainable through the use of a mediator.

\subparagraph{Pattern Language:
Semantics}\label{pattern-language-semantics}

There is one overriding point to make in all of this discussion:
semantics. Communicating intent and semantics through the use of named
patterns is only viable and only valid when all parties in a
communication medium understand the language in the same way.

If I say ``apple'', what am I talking about? Am I talking about a fruit?
Or am I talking about a technology and consumer products company? As
Sharon Cichelli says: ``semantics will continue to be important, until
we learn how to communicate in something other than language''.

\section{Modular Development}\label{modular-development}

\subsection{Introduction}\label{introduction-1}

When we say an application is modular, we generally mean it's composed
of a set of highly decoupled, distinct pieces of functionality stored in
modules. As you probably know, loose coupling facilitates easier
maintainability of apps by removing dependencies where possible. When
this is implemented efficiently, it's quite easy to see how changes to
one part of a system may affect another.

Unlike some more traditional programming languages, the current
iteration of JavaScript (ECMA-262) doesn't provide developers with the
means to import such modules of code in a clean, organized manner.

Instead, developers are left to fall back on variations of the module or
object literal patterns combined with script tags or a script loader.
With many of these, module scripts are strung together in the DOM with
namespaces being described by a single global object where it's still
possible to have name collisions. There's also no clean way to handle
dependency management without some manual effort or third party tools.

Whilst native solutions to these problems may be arriving via
\href{http://wiki.ecmascript.org/doku.php?id=harmony:specification_drafts}{ES6}
(the next version of the official JavaScript specification)
\href{http://wiki.ecmascript.org/doku.php?id=harmony:modules}{modules
proposal}, the good news is that writing modular JavaScript has never
been easier and you can start doing it today.

In this next part of the book, we're going to look at how to use AMD
modules and RequireJS to cleanly wrap units of code in your application
into manageable modules. We'll also cover an alternate approach called
Lumbar which uses routes to determine when modules are loaded.

\subsection{Organizing modules with RequireJS and
AMD}\label{organizing-modules-with-requirejs-and-amd}

\emph{Partly Contributed by \href{https://github.com/jackfranklin}{Jack
Franklin}}

\href{http://requirejs.org}{RequireJS} is a popular script loader
written by James Burke - a developer who has been quite instrumental in
helping shape the AMD module format, which we'll discuss shortly.
Amongst other things RequireJS helps you to load multiple script files,
define modules with or without dependencies, and load in non-script
dependencies such as text files.

\subsubsection{Maintainability problems with multiple script
files}\label{maintainability-problems-with-multiple-script-files}

You might be thinking that there is little benefit to RequireJS. After
all, you can simply load in your JavaScript files through multiple
\texttt{\textless{}script\textgreater{}} tags, which is very
straightforward. However, doing it that way has a lot of drawbacks,
including increasing the HTTP overhead.

Every time the browser loads in a file you've referenced in a
\texttt{\textless{}script\textgreater{}} tag, it makes an HTTP request
to load the file's contents. It has to make a new HTTP request for each
file you want to load, which causes problems.

\begin{itemize}
\itemsep1pt\parskip0pt\parsep0pt
\item
  Browsers are limited in how many parallel requests they can make, so
  often it's slow to load multiple files, as it can only do a certain
  number at a time. This number depends on the user's settings and
  browser, but is usually around 4-8. When working on Backbone
  applications it's good to split your app into multiple JS files, so
  it's easy to hit that limit quickly. This can be negated by minifying
  your code into one file as part of a build process, but does not help
  with the next point.
\item
  Scripts are loaded synchronously. This means that the browser cannot
  continue page rendering while the script is loading, .
\end{itemize}

What tools like RequireJS do is load scripts asynchronously. This means
we have to adjust our code slightly, you can't just swap out
\texttt{\textless{}script\textgreater{}} elements for a small piece of
RequireJS code, but the benefits are very worthwhile:

\begin{itemize}
\itemsep1pt\parskip0pt\parsep0pt
\item
  Loading the scripts asynchronously means the load process is
  non-blocking. The browser can continue to render the rest of the page
  as the scripts are being loaded, speeding up the initial load time.
\item
  We can load modules in more intelligently, having more control over
  when they are loaded and ensuring that modules which have dependencies
  are loaded in the right order.
\end{itemize}

\subsubsection{Need for better dependency
management}\label{need-for-better-dependency-management}

Dependency management is a challenging subject, in particular when
writing JavaScript in the browser. The closest thing we have to
dependency management by default is simply making sure we order our
\texttt{\textless{}script\textgreater{}} tags such that code that
depends on code in another file is loaded after the file it depends on.
This is not a good approach. As I've already discussed, loading multiple
files in that way is bad for performance; needing them to be loaded in a
certain order is very brittle.

Being able to load code on an as-needed basis is something RequireJS is
very good at. Rather than load all our JavaScript code in during initial
page load, a better approach is to dynamically load modules when that
code is required. This avoids loading all the code when the user first
hits your application, consequently speeding up initial load times.

Think about the GMail web client for a moment. When a user initially
loads the page on their first visit, Google can simply hide widgets such
as the chat module until the user has indicated (by clicking `expand')
that they wish to use it. Through dynamic dependency loading, Google
could load up the chat module at that time, rather than forcing all
users to load it when the page first initializes. This can improve
performance and load times and can definitely prove useful when building
larger applications. As the codebase for an application grows this
becomes even more important.

The important thing to note here is that while it's absolutely fine to
develop applications without a script loader, there are significant
benefits to utilizing tools like RequireJS in your application.

\subsubsection{Asynchronous Module Definition
(AMD)}\label{asynchronous-module-definition-amd}

RequireJS implements the
\href{https://github.com/amdjs/amdjs-api/wiki/AMD}{AMD Specification}
which defines a method for writing modular code and managing
dependencies. The RequireJS website also has a section
\href{http://requirejs.org/docs/whyamd.html}{documenting the reasons
behind implementing AMD}:

\begin{quote}
The AMD format comes from wanting a module format that was better than
today's ``write a bunch of script tags with implicit dependencies that
you have to manually order'' and something that was easy to use directly
in the browser. Something with good debugging characteristics that did
not require server-specific tooling to get started.
\end{quote}

\subsubsection{Writing AMD modules with
RequireJS}\label{writing-amd-modules-with-requirejs}

As discussed above, the overall goal for the AMD format is to provide a
solution for modular JavaScript that developers can use today. The two
key concepts you need to be aware of when using it with a script-loader
are the \texttt{define()} method for defining modules and the
\texttt{require()} method for loading dependencies. \texttt{define()} is
used to define named or unnamed modules using the following signature:

\begin{Shaded}
\begin{Highlighting}[]
\FunctionTok{define}\NormalTok{(}
    \NormalTok{module_id }\CommentTok{/*optional*/}\NormalTok{,}
    \NormalTok{[dependencies] }\CommentTok{/*optional*/}\NormalTok{,}
    \NormalTok{definition }\KeywordTok{function} \CommentTok{/*function for instantiating the module or object*/}
\NormalTok{);}
\end{Highlighting}
\end{Shaded}

As you can tell by the inline comments, the \texttt{module\_id} is an
optional argument which is typically only required when non-AMD
concatenation tools are being used (there may be some other edge cases
where it's useful too). When this argument is left out, we call the
module `anonymous'. When working with anonymous modules, RequireJS will
use a module's file path as its module id, so the adage Don't Repeat
Yourself (DRY) should be applied by omitting the module id in the
\texttt{define()} invocation.

The dependencies argument is an array representing all of the other
modules that this module depends on and the third argument is a factory
that can either be a function that should be executed to instantiate the
module or an object.

A barebones module (compatible with RequireJS) could be defined using
\texttt{define()} as follows:

\begin{Shaded}
\begin{Highlighting}[]
\CommentTok{// A module ID has been omitted here to make the module anonymous}

\FunctionTok{define}\NormalTok{([}\StringTok{'foo'}\NormalTok{, }\StringTok{'bar'}\NormalTok{],}
    \CommentTok{// module definition function}
    \CommentTok{// dependencies (foo and bar) are mapped to function parameters}
    \KeywordTok{function} \NormalTok{( foo, bar ) \{}
        \CommentTok{// return a value that defines the module export}
        \CommentTok{// (i.e the functionality we want to expose for consumption)}

        \CommentTok{// create your module here}
        \KeywordTok{var} \NormalTok{myModule = \{}
            \DataTypeTok{doStuff}\NormalTok{:}\KeywordTok{function}\NormalTok{()\{}
                \OtherTok{console}\NormalTok{.}\FunctionTok{log}\NormalTok{(}\StringTok{'Yay! Stuff'}\NormalTok{);}
            \NormalTok{\}}
        \NormalTok{\}}

        \KeywordTok{return} \NormalTok{myModule;}
\NormalTok{\});}
\end{Highlighting}
\end{Shaded}

\emph{Note: RequireJS is intelligent enough to automatically infer the
`.js' extension to your script file names. As such, this extension is
generally omitted when specifying dependencies.}

\paragraph{Alternate syntax}\label{alternate-syntax}

There is also a
\href{http://requirejs.org/docs/whyamd.html\#sugar}{sugared version} of
\texttt{define()} available that allows you to declare your dependencies
as local variables using \texttt{require()}. This will feel familiar to
anyone who's used node, and can be easier to add or remove dependencies.
Here is the previous snippet using the alternate syntax:

\begin{Shaded}
\begin{Highlighting}[]
\CommentTok{// A module ID has been omitted here to make the module anonymous}

\FunctionTok{define}\NormalTok{(}\KeywordTok{function}\NormalTok{(require)\{}
        \CommentTok{// module definition function}
    \CommentTok{// dependencies (foo and bar) are defined as local vars}
    \KeywordTok{var} \NormalTok{foo = }\FunctionTok{require}\NormalTok{(}\StringTok{'foo'}\NormalTok{),}
        \NormalTok{bar = }\FunctionTok{require}\NormalTok{(}\StringTok{'bar'}\NormalTok{);}

    \CommentTok{// return a value that defines the module export}
    \CommentTok{// (i.e the functionality we want to expose for consumption)}

    \CommentTok{// create your module here}
    \KeywordTok{var} \NormalTok{myModule = \{}
        \DataTypeTok{doStuff}\NormalTok{:}\KeywordTok{function}\NormalTok{()\{}
            \OtherTok{console}\NormalTok{.}\FunctionTok{log}\NormalTok{(}\StringTok{'Yay! Stuff'}\NormalTok{);}
        \NormalTok{\}}
    \NormalTok{\}}

    \KeywordTok{return} \NormalTok{myModule;}
\NormalTok{\});}
\end{Highlighting}
\end{Shaded}

The \texttt{require()} method is typically used to load code in a
top-level JavaScript file or within a module should you wish to
dynamically fetch dependencies. An example of its usage is:

\begin{Shaded}
\begin{Highlighting}[]
\CommentTok{// Consider 'foo' and 'bar' are two external modules}
\CommentTok{// In this example, the 'exports' from the two modules loaded are passed as}
\CommentTok{// function arguments to the callback (foo and bar)}
\CommentTok{// so that they can similarly be accessed}

\FunctionTok{require}\NormalTok{( [}\StringTok{'foo'}\NormalTok{, }\StringTok{'bar'}\NormalTok{], }\KeywordTok{function} \NormalTok{( foo, bar ) \{}
    \CommentTok{// rest of your code here}
    \OtherTok{foo}\NormalTok{.}\FunctionTok{doSomething}\NormalTok{();}
\NormalTok{\});}
\end{Highlighting}
\end{Shaded}

Addy's post on \href{http://addyosmani.com/writing-modular-js/}{Writing
Modular JS} covers the AMD specification in much more detail. Defining
and using modules will be covered in this book shortly when we look at
more structured examples of using RequireJS.

\subsubsection{Getting Started with
RequireJS}\label{getting-started-with-requirejs}

Before using RequireJS and Backbone we will first set up a very basic
RequireJS project to demonstrate how it works. The first thing to do is
to \href{http://requirejs.org/docs/download.html\#requirejs}{Download
RequireJS}. When you load in the RequireJS script in your HTML file, you
need to also tell it where your main JavaScript file is located.
Typically this will be called something like ``app.js'', and is the main
entry point for your application. You do this by adding in a
\texttt{data-main} attribute to the \texttt{script} tag:

\begin{Shaded}
\begin{Highlighting}[]
\KeywordTok{<script}\OtherTok{ data-main=}\StringTok{"app"}\OtherTok{ src=}\StringTok{"lib/require.js"}\KeywordTok{></script>}
\end{Highlighting}
\end{Shaded}

Now, RequireJS will automatically load \texttt{app.js} for you.

\paragraph{RequireJS Configuration}\label{requirejs-configuration}

In the main JavaScript file that you load with the \texttt{data-main}
attribute you can configure how RequireJS loads the rest of your
application. This is done by calling \texttt{require.config}, and
passing in an object:

\begin{Shaded}
\begin{Highlighting}[]
\OtherTok{require}\NormalTok{.}\FunctionTok{config}\NormalTok{(\{}
    \CommentTok{// your configuration key/values here}
    \DataTypeTok{baseUrl}\NormalTok{: }\StringTok{"app"}\NormalTok{, }\CommentTok{// generally the same directory as the script used in a data-main attribute for the top level script}
    \DataTypeTok{paths}\NormalTok{: \{\}, }\CommentTok{// set up custom paths to libraries, or paths to RequireJS plugins}
    \DataTypeTok{shim}\NormalTok{: \{\}, }\CommentTok{// used for setting up all Shims (see below for more detail)}
\NormalTok{\});}
\end{Highlighting}
\end{Shaded}

The main reason you'd want to configure RequireJS is to add shims, which
we'll cover next. To see other configuration options available to you, I
recommend checking out the
\href{http://requirejs.org/docs/api.html\#config}{RequireJS
documentation}.

\subparagraph{RequireJS Shims}\label{requirejs-shims}

Ideally, each library that we use with RequireJS will come with AMD
support. That is, it uses the \texttt{define} method to define the
library as a module. However, some libraries - including Backbone and
one of its dependencies, Underscore - don't do this. Fortunately
RequireJS comes with a way to work around this.

To demonstrate this, first let's shim Underscore, and then we'll shim
Backbone too. Shims are very simple to implement:

\begin{Shaded}
\begin{Highlighting}[]
\OtherTok{require}\NormalTok{.}\FunctionTok{config}\NormalTok{(\{}
    \DataTypeTok{shim}\NormalTok{: \{}
        \StringTok{'lib/underscore'}\NormalTok{: \{}
            \DataTypeTok{exports}\NormalTok{: }\StringTok{'_'}
        \NormalTok{\}}
    \NormalTok{\}}
\NormalTok{\});}
\end{Highlighting}
\end{Shaded}

Note that when specifying paths for RequireJS you should omit the
\texttt{.js} from the end of script names.

The important line here is \texttt{exports: '\_'}. This line tells
RequireJS that the script in \texttt{'lib/underscore.js'} creates a
global variable called \texttt{\_} instead of defining a module. Now
when we list Underscore as a dependency RequireJS will know to give us
the \texttt{\_} global variable as though it was the module defined by
that script. We can set up a shim for Backbone too:

\begin{Shaded}
\begin{Highlighting}[]
\OtherTok{require}\NormalTok{.}\FunctionTok{config}\NormalTok{(\{}
    \DataTypeTok{shim}\NormalTok{: \{}
        \StringTok{'lib/underscore'}\NormalTok{: \{}
          \DataTypeTok{exports}\NormalTok{: }\StringTok{'_'}
        \NormalTok{\},}
        \StringTok{'lib/backbone'}\NormalTok{: \{}
            \DataTypeTok{deps}\NormalTok{: [}\StringTok{'lib/underscore'}\NormalTok{, }\StringTok{'jquery'}\NormalTok{],}
            \DataTypeTok{exports}\NormalTok{: }\StringTok{'Backbone'}
        \NormalTok{\}}
    \NormalTok{\}}
\NormalTok{\});}
\end{Highlighting}
\end{Shaded}

Again, that configuration tells RequireJS to return the global
\texttt{Backbone} variable that Backbone exports, but this time you'll
notice that Backbone's dependencies are defined. This means whenever
this:

\begin{Shaded}
\begin{Highlighting}[]
\FunctionTok{require}\NormalTok{( }\StringTok{'lib/backbone'}\NormalTok{, }\KeywordTok{function}\NormalTok{( Backbone ) \{...\} );}
\end{Highlighting}
\end{Shaded}

Is run, it will first make sure the dependencies are met, and then pass
the global \texttt{Backbone} object into the callback function. You
don't need to do this with every library, only the ones that don't
support AMD. For example, jQuery does support it, as of jQuery 1.7.

If you'd like to read more about general RequireJS usage, the
\href{http://requirejs.org/docs/api.html}{RequireJS API docs} are
incredibly thorough and easy to read.

\paragraph{Custom Paths}\label{custom-paths}

Typing long paths to file names like \texttt{lib/backbone} can get
tedious. RequireJS lets us set up custom paths in our configuration
object. Here, whenever I refer to ``underscore'', RequireJS will look
for the file \texttt{lib/underscore.js}:

\begin{Shaded}
\begin{Highlighting}[]
\OtherTok{require}\NormalTok{.}\FunctionTok{config}\NormalTok{(\{}
    \DataTypeTok{paths}\NormalTok{: \{}
        \StringTok{'underscore'}\NormalTok{: }\StringTok{'lib/underscore'}
    \NormalTok{\}}
\NormalTok{\});}
\end{Highlighting}
\end{Shaded}

Of course, this can be combined with a shim:

\begin{Shaded}
\begin{Highlighting}[]
\OtherTok{require}\NormalTok{.}\FunctionTok{config}\NormalTok{(\{}
    \DataTypeTok{paths}\NormalTok{: \{}
        \StringTok{'underscore'}\NormalTok{: }\StringTok{'lib/underscore'}
    \NormalTok{\},}
    \DataTypeTok{shim}\NormalTok{: \{}
        \StringTok{'underscore'}\NormalTok{: \{}
          \DataTypeTok{exports}\NormalTok{: }\StringTok{'_'}
        \NormalTok{\}}
    \NormalTok{\}}
\NormalTok{\});}
\end{Highlighting}
\end{Shaded}

Just make sure that you refer to the custom path in your shim settings,
too. Now you can do

\begin{Shaded}
\begin{Highlighting}[]
\FunctionTok{require}\NormalTok{( [}\StringTok{'underscore'}\NormalTok{], }\KeywordTok{function}\NormalTok{(_) \{}
\CommentTok{// code here}
\NormalTok{\});}
\end{Highlighting}
\end{Shaded}

to shim Underscore but still use a custom path.

\subsubsection{Require.js and Backbone
Examples}\label{require.js-and-backbone-examples}

Now that we've taken a look at how to define AMD modules, let's review
how to go about wrapping components like views and collections so that
they can also be easily loaded as dependencies for any parts of your
application that require them. At its simplest, a Backbone model may
just require Backbone and Underscore.js. These are dependencies, so we
can define those when defining the new modules. Note that the following
examples presume you have configured RequireJS to shim Backbone and
Underscore, as discussed previously.

\paragraph{Wrapping models, views, and other components with
AMD}\label{wrapping-models-views-and-other-components-with-amd}

For example, here is how a model is defined.

\begin{Shaded}
\begin{Highlighting}[]
\FunctionTok{define}\NormalTok{([}\StringTok{'underscore'}\NormalTok{, }\StringTok{'backbone'}\NormalTok{], }\KeywordTok{function}\NormalTok{(_, Backbone) \{}
  \KeywordTok{var} \NormalTok{myModel = }\OtherTok{Backbone}\NormalTok{.}\OtherTok{Model}\NormalTok{.}\FunctionTok{extend}\NormalTok{(\{}

    \CommentTok{// Default attributes}
    \DataTypeTok{defaults}\NormalTok{: \{}
      \DataTypeTok{content}\NormalTok{: }\StringTok{'hello world'}\NormalTok{,}
    \NormalTok{\},}

    \CommentTok{// A dummy initialization method}
    \DataTypeTok{initialize}\NormalTok{: }\KeywordTok{function}\NormalTok{() \{}
    \NormalTok{\},}

    \DataTypeTok{clear}\NormalTok{: }\KeywordTok{function}\NormalTok{() \{}
      \KeywordTok{this}\NormalTok{.}\FunctionTok{destroy}\NormalTok{();}
      \KeywordTok{this}\NormalTok{.}\OtherTok{view}\NormalTok{.}\FunctionTok{remove}\NormalTok{();}
    \NormalTok{\}}

  \NormalTok{\});}
  \KeywordTok{return} \NormalTok{myModel;}
\NormalTok{\});}
\end{Highlighting}
\end{Shaded}

Note how we alias Underscore.js's instance to \texttt{\_} and Backbone
to just \texttt{Backbone}, making it very trivial to convert non-AMD
code over to using this module format. For a view which might require
other dependencies such as jQuery, this can similarly be done as
follows:

\begin{Shaded}
\begin{Highlighting}[]
\FunctionTok{define}\NormalTok{([}
  \StringTok{'jquery'}\NormalTok{,}
  \StringTok{'underscore'}\NormalTok{,}
  \StringTok{'backbone'}\NormalTok{,}
  \StringTok{'collections/mycollection'}\NormalTok{,}
  \StringTok{'views/myview'}
  \NormalTok{], }\KeywordTok{function}\NormalTok{($, _, Backbone, myCollection, myView)\{}

  \KeywordTok{var} \NormalTok{AppView = }\OtherTok{Backbone}\NormalTok{.}\OtherTok{View}\NormalTok{.}\FunctionTok{extend}\NormalTok{(\{}
  \NormalTok{...}
\end{Highlighting}
\end{Shaded}

Aliasing to the dollar-sign (\texttt{\$}) once again makes it very easy
to encapsulate any part of an application you wish using AMD.

Doing it this way makes it easy to organize your Backbone application as
you like. It's recommended to separate modules into folders. For
example, individual folders for models, collections, views and so on.
RequireJS doesn't care about what folder structure you use; as long as
you use the correct path when using \texttt{require}, it will happily
pull in the file.

As part of this chapter I've made a very simple
\href{https://github.com/javascript-playground/backbone-require-example}{Backbone
application with RequireJS that you can find on Github}. It is a stock
application for a manager of a shop. They can add new items and filter
down the items based on price, but nothing more. Because it's so simple
it's easier to focus purely on the RequireJS part of the implementation,
rather than deal with complex JavaScript and Backbone logic too.

At the base of this application is the \texttt{Item} model, which
describes a single item in the stock. Its implementation is very
straight forward:

\begin{Shaded}
\begin{Highlighting}[]
\FunctionTok{define}\NormalTok{( [}\StringTok{"lib/backbone"}\NormalTok{], }\KeywordTok{function} \NormalTok{( Backbone ) \{}
  \KeywordTok{var} \NormalTok{Item = }\OtherTok{Backbone}\NormalTok{.}\OtherTok{Model}\NormalTok{.}\FunctionTok{extend}\NormalTok{(\{}
    \DataTypeTok{defaults}\NormalTok{: \{}
      \DataTypeTok{price}\NormalTok{: }\DecValTok{35}\NormalTok{,}
      \DataTypeTok{photo}\NormalTok{: }\StringTok{"http://www.placedog.com/100/100"}
    \NormalTok{\}}
  \NormalTok{\});}
  \KeywordTok{return} \NormalTok{Item;}
\NormalTok{\});}
\end{Highlighting}
\end{Shaded}

Converting an individual model, collection, view or similar into an AMD,
RequireJS compliant one is typically very straight forward. Usually all
that's needed is the first line, calling \texttt{define}, and to make
sure that once you've defined your object - in this case, the
\texttt{Item} model, to return it.

Let's now set up a view for that individual item:

\begin{Shaded}
\begin{Highlighting}[]
\FunctionTok{define}\NormalTok{( [}\StringTok{"lib/backbone"}\NormalTok{], }\KeywordTok{function} \NormalTok{( Backbone ) \{}
  \KeywordTok{var} \NormalTok{ItemView = }\OtherTok{Backbone}\NormalTok{.}\OtherTok{View}\NormalTok{.}\FunctionTok{extend}\NormalTok{(\{}
    \DataTypeTok{tagName}\NormalTok{: }\StringTok{"div"}\NormalTok{,}
    \DataTypeTok{className}\NormalTok{: }\StringTok{"item-wrap"}\NormalTok{,}
    \DataTypeTok{template}\NormalTok{: }\OtherTok{_}\NormalTok{.}\FunctionTok{template}\NormalTok{(}\FunctionTok{$}\NormalTok{(}\StringTok{"#itemTemplate"}\NormalTok{).}\FunctionTok{html}\NormalTok{()),}

    \DataTypeTok{render}\NormalTok{: }\KeywordTok{function}\NormalTok{() \{}
      \KeywordTok{this}\NormalTok{.}\OtherTok{$el}\NormalTok{.}\FunctionTok{html}\NormalTok{(}\KeywordTok{this}\NormalTok{.}\FunctionTok{template}\NormalTok{(}\KeywordTok{this}\NormalTok{.}\OtherTok{model}\NormalTok{.}\FunctionTok{attributes}\NormalTok{));}
      \KeywordTok{return} \KeywordTok{this}\NormalTok{;}
    \NormalTok{\}}
  \NormalTok{\});}
  \KeywordTok{return} \NormalTok{ItemView;}
\NormalTok{\});}
\end{Highlighting}
\end{Shaded}

This view doesn't actually depend on the model it will be used with, so
again the only dependency is Backbone. Other than that it's just a
regular Backbone view. There's nothing special going on here, other than
returning the object and using \texttt{define} so RequireJS can pick it
up. Now let's make a collection to view a list of items. This time we
will need to reference the \texttt{Item} model, so we add it as a
dependency:

\begin{Shaded}
\begin{Highlighting}[]
\FunctionTok{define}\NormalTok{([}\StringTok{"lib/backbone"}\NormalTok{, }\StringTok{"models/item"}\NormalTok{], }\KeywordTok{function}\NormalTok{(Backbone, Item) \{}
  \KeywordTok{var} \NormalTok{Cart = }\OtherTok{Backbone}\NormalTok{.}\OtherTok{Collection}\NormalTok{.}\FunctionTok{extend}\NormalTok{(\{}
    \DataTypeTok{model}\NormalTok{: Item,}
    \DataTypeTok{initialize}\NormalTok{: }\KeywordTok{function}\NormalTok{() \{}
      \KeywordTok{this}\NormalTok{.}\FunctionTok{on}\NormalTok{(}\StringTok{"add"}\NormalTok{, }\KeywordTok{this}\NormalTok{.}\FunctionTok{updateSet}\NormalTok{, }\KeywordTok{this}\NormalTok{);}
    \NormalTok{\},}
    \DataTypeTok{updateSet}\NormalTok{: }\KeywordTok{function}\NormalTok{() \{}
      \NormalTok{items = }\KeywordTok{this}\NormalTok{.}\FunctionTok{models}\NormalTok{;}
    \NormalTok{\}}
  \NormalTok{\});}
  \KeywordTok{return} \NormalTok{Cart;}
\NormalTok{\});}
\end{Highlighting}
\end{Shaded}

I've called this collection \texttt{Cart}, as it's a group of items. As
the \texttt{Item} model is the second dependency, I can bind the
variable \texttt{Item} to it by declaring it as the second argument to
the callback function. I can then refer to this within my collection
implementation.

Finally, let's have a look at the view for this collection. (This file
is much bigger in the application, but I've taken some bits out so it's
easier to examine).

\begin{Shaded}
\begin{Highlighting}[]
\FunctionTok{define}\NormalTok{([}\StringTok{"lib/backbone"}\NormalTok{, }\StringTok{"views/itemview"}\NormalTok{], }\KeywordTok{function}\NormalTok{(Backbone, ItemView) \{}
  \KeywordTok{var} \NormalTok{ItemCollectionView = }\OtherTok{Backbone}\NormalTok{.}\OtherTok{View}\NormalTok{.}\FunctionTok{extend}\NormalTok{(\{}
    \DataTypeTok{el}\NormalTok{: }\StringTok{'#yourcart'}\NormalTok{,}
    \DataTypeTok{initialize}\NormalTok{: }\KeywordTok{function}\NormalTok{(collection) \{}
      \KeywordTok{this}\NormalTok{.}\FunctionTok{collection} \NormalTok{= collection;}
      \KeywordTok{this}\NormalTok{.}\FunctionTok{render}\NormalTok{();}
      \KeywordTok{this}\NormalTok{.}\OtherTok{collection}\NormalTok{.}\FunctionTok{on}\NormalTok{(}\StringTok{"reset"}\NormalTok{, }\KeywordTok{this}\NormalTok{.}\FunctionTok{render}\NormalTok{, }\KeywordTok{this}\NormalTok{);}
    \NormalTok{\},}
    \DataTypeTok{render}\NormalTok{: }\KeywordTok{function}\NormalTok{() \{}
      \KeywordTok{this}\NormalTok{.}\OtherTok{$el}\NormalTok{.}\FunctionTok{html}\NormalTok{(}\StringTok{""}\NormalTok{);}
      \KeywordTok{this}\NormalTok{.}\OtherTok{collection}\NormalTok{.}\FunctionTok{each}\NormalTok{(}\KeywordTok{function}\NormalTok{(item) \{}
        \KeywordTok{this}\NormalTok{.}\FunctionTok{renderItem}\NormalTok{(item);}
      \NormalTok{\}, }\KeywordTok{this}\NormalTok{);}
    \NormalTok{\},}
    \DataTypeTok{renderItem}\NormalTok{: }\KeywordTok{function}\NormalTok{(item) \{}
      \KeywordTok{var} \NormalTok{itemView = }\KeywordTok{new} \FunctionTok{ItemView}\NormalTok{(\{}\DataTypeTok{model}\NormalTok{: item\});}
      \KeywordTok{this}\NormalTok{.}\OtherTok{$el}\NormalTok{.}\FunctionTok{append}\NormalTok{(}\OtherTok{itemView}\NormalTok{.}\FunctionTok{render}\NormalTok{().}\FunctionTok{el}\NormalTok{);}
    \NormalTok{\},}
    \CommentTok{// more methods here removed}
  \NormalTok{\});}
  \KeywordTok{return} \NormalTok{ItemCollectionView;}
\NormalTok{\});}
\end{Highlighting}
\end{Shaded}

There really is nothing to it once you've got the general pattern.
Define each ``object'' (a model, view, collection, router or otherwise)
through RequireJS, and then specify them as dependencies to other
objects that need them. Again, you can find this entire application
\href{https://github.com/javascript-playground/backbone-require-example}{on
Github}.

If you'd like to take a look at how others do it,
\href{https://github.com/phawk/Backbone-Stack}{Pete Hawkins' Backbone
Stack repository} is a good example of structuring a Backbone
application using RequireJS. Greg Franko has also written
\href{http://gregfranko.com/blog/using-backbone-dot-js-with-require-dot-js/}{an
overview of how he uses Backbone and Require}, and
\href{http://jeremyckahn.github.com/blog/2012/08/18/keeping-it-sane-backbone-views-and-require-dot-js/}{Jeremy
Kahn's post} neatly describes his approach. For a look at a full sample
application, the
\href{https://github.com/addyosmani/todomvc/tree/gh-pages/dependency-examples/backbone_require}{Backbone
and Require version} of the TodoMVC application is a good starting
point.

\subsubsection{Keeping Your Templates External Using RequireJS And The
Text
Plugin}\label{keeping-your-templates-external-using-requirejs-and-the-text-plugin}

Moving your templates to external files is actually quite
straight-forward, whether they are Underscore, Mustache, Handlebars or
any other text-based template format. Let's look at how we do that with
RequireJS.

RequireJS has a special plugin called text.js which is used to load in
text file dependencies. To use the text plugin, follow these steps:

\begin{enumerate}
\def\labelenumi{\arabic{enumi}.}
\item
  Download the plugin from http://requirejs.org/docs/download.html\#text
  and place it in either the same directory as your application's main
  JS file or a suitable sub-directory.
\item
  Next, include the text.js plugin in your initial RequireJS
  configuration options. In the code snippet below, we assume that
  RequireJS is being included in our page prior to this code snippet
  being executed.
\end{enumerate}

\begin{Shaded}
\begin{Highlighting}[]
\OtherTok{require}\NormalTok{.}\FunctionTok{config}\NormalTok{( \{}
    \DataTypeTok{paths}\NormalTok{: \{}
        \StringTok{'text'}\NormalTok{: }\StringTok{'libs/require/text'}\NormalTok{,}
    \NormalTok{\},}
    \DataTypeTok{baseUrl}\NormalTok{: }\StringTok{'app'}
\NormalTok{\} );}
\end{Highlighting}
\end{Shaded}

\begin{enumerate}
\def\labelenumi{\arabic{enumi}.}
\setcounter{enumi}{2}
\itemsep1pt\parskip0pt\parsep0pt
\item
  When the \texttt{text!} prefix is used for a dependency, RequireJS
  will automatically load the text plugin and treat the dependency as a
  text resource. A typical example of this in action may look like:
\end{enumerate}

\begin{Shaded}
\begin{Highlighting}[]
\FunctionTok{require}\NormalTok{([}\StringTok{'js/app'}\NormalTok{, }\StringTok{'text!templates/mainView.html'}\NormalTok{],}
    \KeywordTok{function}\NormalTok{( app, mainView ) \{}
        \CommentTok{// the contents of the mainView file will be}
        \CommentTok{// loaded into mainView for usage.}
    \NormalTok{\}}
\NormalTok{);}
\end{Highlighting}
\end{Shaded}

\begin{enumerate}
\def\labelenumi{\arabic{enumi}.}
\setcounter{enumi}{3}
\itemsep1pt\parskip0pt\parsep0pt
\item
  Finally we can use the text resource that's been loaded for templating
  purposes. You're probably used to storing your HTML templates inline
  using a script with a specific identifier.
\end{enumerate}

With Underscore.js's micro-templating (and jQuery) this would typically
be:

HTML:

\begin{Shaded}
\begin{Highlighting}[]
\KeywordTok{<script}\OtherTok{ type=}\StringTok{"text/template"}\OtherTok{ id=}\StringTok{"mainViewTemplate"}\KeywordTok{>}
\ErrorTok{    <% _.each( person, function( person_item )\{ %>}
\ErrorTok{        <li><%= person_item.get('name') %></li>}
    \NormalTok{<% \}); %>}
\KeywordTok{</script>}
\end{Highlighting}
\end{Shaded}

JS:

\begin{Shaded}
\begin{Highlighting}[]
\KeywordTok{var} \NormalTok{compiled_template = }\OtherTok{_}\NormalTok{.}\FunctionTok{template}\NormalTok{( }\FunctionTok{$}\NormalTok{(}\StringTok{'#mainViewTemplate'}\NormalTok{).}\FunctionTok{html}\NormalTok{() );}
\end{Highlighting}
\end{Shaded}

With RequireJS and the text plugin however, it's as simple as saving the
same template into an external text file (say, \texttt{mainView.html})
and doing the following:

\begin{Shaded}
\begin{Highlighting}[]
\FunctionTok{require}\NormalTok{([}\StringTok{'js/app'}\NormalTok{, }\StringTok{'text!templates/mainView.html'}\NormalTok{],}
    \KeywordTok{function}\NormalTok{(app, mainView)\{}
        \KeywordTok{var} \NormalTok{compiled_template = }\OtherTok{_}\NormalTok{.}\FunctionTok{template}\NormalTok{( mainView );}
    \NormalTok{\}}
\NormalTok{);}
\end{Highlighting}
\end{Shaded}

That's it! Now you can apply your template to a view in Backbone with
something like:

\begin{Shaded}
\begin{Highlighting}[]
\OtherTok{collection}\NormalTok{.}\OtherTok{someview}\NormalTok{.}\OtherTok{$el}\NormalTok{.}\FunctionTok{html}\NormalTok{( }\FunctionTok{compiled_template}\NormalTok{( \{ }\DataTypeTok{results}\NormalTok{: }\OtherTok{collection}\NormalTok{.}\FunctionTok{models} \NormalTok{\} ) );}
\end{Highlighting}
\end{Shaded}

All templating solutions will have their own custom methods for handling
template compilation, but if you understand the above, substituting
Underscore's micro-templating for any other solution should be fairly
trivial.

\subsubsection{Optimizing Backbone apps for production with the
RequireJS
Optimizer}\label{optimizing-backbone-apps-for-production-with-the-requirejs-optimizer}

Once you're written your application, the next important step is to
prepare it for deployment to production. The majority of non-trivial
apps are likely to consist of several scripts and so optimizing,
minimizing, and concatenating your scripts prior to pushing can reduce
the number of scripts your users need to download.

A command-line optimization tool for RequireJS projects called r.js is
available to help with this workflow. It offers a number of
capabilities, including:

\begin{itemize}
\itemsep1pt\parskip0pt\parsep0pt
\item
  Concatenating specific scripts and minifying them using external tools
  such as UglifyJS (which is used by default) or Google's Closure
  Compiler for optimal browser delivery, whilst preserving the ability
  to dynamically load modules
\item
  Optimizing CSS and stylesheets by inlining CSS files imported using
  @import, stripping out comments, etc.
\item
  The ability to run AMD projects in both Node and Rhino (more on this
  later)
\end{itemize}

If you find yourself wanting to ship a single file with all dependencies
included, r.js can help with this too. Whilst RequireJS does support
lazy-loading, your application may be small enough that reducing HTTP
requests to a single script file is feasible.

You'll notice that I mentioned the word `specific' in the first bullet
point. The RequireJS optimizer only concatenates module scripts that
have been specified as string literals in \texttt{require} and
\texttt{define} calls (which you've probably used). As clarified by the
\href{http://requirejs.org/docs/optimization.html}{optimizer docs} this
means that Backbone modules defined like this:

\begin{Shaded}
\begin{Highlighting}[]
\FunctionTok{define}\NormalTok{([}\StringTok{'jquery'}\NormalTok{, }\StringTok{'backbone'}\NormalTok{, }\StringTok{'underscore'}\NormalTok{, }\StringTok{'collections/sample'}\NormalTok{, }\StringTok{'views/test'}\NormalTok{],}
    \KeywordTok{function}\NormalTok{($, Backbone, _, Sample, Test)\{}
        \CommentTok{//...}
    \NormalTok{\});}
\end{Highlighting}
\end{Shaded}

will combine fine, however dynamic dependencies such as:

\begin{Shaded}
\begin{Highlighting}[]
\KeywordTok{var} \NormalTok{models = someCondition ? [}\StringTok{'models/ab'}\NormalTok{, }\StringTok{'models/ac'}\NormalTok{] : [}\StringTok{'models/ba'}\NormalTok{, }\StringTok{'models/bc'}\NormalTok{];}
\FunctionTok{define}\NormalTok{([}\StringTok{'jquery'}\NormalTok{, }\StringTok{'backbone'}\NormalTok{, }\StringTok{'underscore'}\NormalTok{].}\FunctionTok{concat}\NormalTok{(models),}
    \KeywordTok{function}\NormalTok{($, Backbone, _, firstModel, secondModel)\{}
        \CommentTok{//...}
    \NormalTok{\});}
\end{Highlighting}
\end{Shaded}

will be ignored. This is by design as it ensures that dynamic
dependency/module loading can still take place even after optimization.

Although the RequireJS optimizer works fine in both Node and Java
environments, it's strongly recommended to run it under Node as it
executes significantly faster there.

To get started with r.js, grab it from the
\href{http://requirejs.org/docs/download.html\#rjs}{RequireJS download
page} or
\href{http://requirejs.org/docs/optimization.html\#download}{through
NPM}. To begin getting our project to build with r.js, we will need to
create a new build profile.

Assuming the code for our application and external dependencies are in
\texttt{app/libs}, our build.js build profile could simply be:

\begin{verbatim}
({
  baseUrl: 'app',
  out: 'dist/main.js',
\end{verbatim}

The paths above are relative to the \texttt{baseUrl} for our project and
in our case it would make sense to make this the \texttt{app} folder.
The \texttt{out} parameter informs r.js that we want to concatenate
everything into a single file called \texttt{main.js} under the
\texttt{dist/} directory. Note that here we do need to add the
\texttt{.js} extension to the filename. Earlier, we saw that when
referencing modules by filenames, you don't need to use the \texttt{.js}
extension, however this is one case in which you do.

Alternatively, we can specify \texttt{dir}, which will ensure the
contents of our \texttt{app} directory are copied into this directory.
e.g:

\begin{verbatim}
({
  baseUrl: 'app',
  dir: 'release',
  out: 'dist/main.js'
\end{verbatim}

Additional options that can be specified such as \texttt{modules} and
\texttt{appDir} are not compatible with \texttt{out}, however let's
briefly discuss them in case you do wish to use them.

\texttt{modules} is an array where we can explicitly specify the module
names we would like to have optimized.

\begin{verbatim}
    modules: [
        {
            name: 'app',
            exclude: [
                // If you prefer not to include certain 
                // libs exclude them here
            ]
        }
\end{verbatim}

\texttt{appDir} - when specified, our\texttt{baseUrl} is relative to
this parameter. If \texttt{appDir} is not defined, \texttt{baseUrl} is
simply relative to the \texttt{build.js} file.

\begin{verbatim}
    appDir: './',
\end{verbatim}

Back to our build profile, the \texttt{main} parameter is used to
specify our main module - we are making use of \texttt{include} here as
we're going to take advantage of
\href{https://github.com/jrburke/almond}{Almond} - a stripped down
loader for RequireJS modules which is useful should you not need to load
modules in dynamically.

\begin{verbatim}
  include: ['libs/almond', 'main'],
  wrap: true,
\end{verbatim}

\texttt{include} is another array which specifies the modules we want to
include in the build. By specifying ``main'', r.js will trace over all
modules main depends on and will include them. \texttt{wrap} wraps
modules which RequireJS needs into a closure so that only what we export
is included in the global environment.

\begin{verbatim}
  paths: {
    backbone: 'libs/backbone',
    underscore: 'libs/underscore',
    jquery: 'libs/jquery',
    text: 'libs/text'
  }
})
\end{verbatim}

The remainder of the build.js file would be a regular paths
configuration object. We can compile our project into a target file by
running:

\begin{verbatim}
node r.js -o build.js
\end{verbatim}

which should place our compiled project into dist/main.js.

The build profile is usually placed inside the `scripts' or `js'
directory of your project. As per the docs, this file can however exist
anywhere you wish, but you'll need to edit the contents of your build
profile accordingly.

That's it. As long as you have UglifyJS/Closure tools setup correctly,
r.js should be able to easily optimize your entire Backbone project in
just a few key-strokes.

If you would like to learn more about build profiles, James Burke has a
\href{https://github.com/jrburke/r.js/blob/master/build/example.build.js}{heavily
commented sample file} with all the possible options available.

\section{Exercise 3: Your First Modular Backbone + RequireJS
App}\label{exercise-3-your-first-modular-backbone-requirejs-app}

In this chapter, we'll look at our first practical Backbone \& RequireJS
project - how to build a modular Todo application. Similar to exercise
1, the application will allow us to add new todos, edit new todos and
clear todo items that have been marked as completed. For a more advanced
practical, see the section on mobile Backbone development.

The complete code for the application can can be found in the
\texttt{practicals/modular-todo-app} folder of this repo (thanks to
Thomas Davis and Jérôme Gravel-Niquet). Alternatively grab a copy of my
side-project \href{https://github.com/addyosmani/todomvc}{TodoMVC} which
contains the sources to both AMD and non-AMD versions.

\subsubsection{Overview}\label{overview}

Writing a modular Backbone application can be a straight-forward
process. There are however, some key conceptual differences to be aware
of if opting to use AMD as your module format of choice:

\begin{itemize}
\itemsep1pt\parskip0pt\parsep0pt
\item
  As AMD isn't a standard native to JavaScript or the browser, it's
  necessary to use a script loader (such as RequireJS or curl.js) in
  order to support defining components and modules using this module
  format. As we've already reviewed, there are a number of advantages to
  using the AMD as well as RequireJS to assist here.
\item
  Models, views, controllers and routers need to be encapsulated
  \emph{using} the AMD-format. This allows each component of our
  Backbone application to cleanly manage dependencies (e.g collections
  required by a view) in the same way that AMD allows non-Backbone
  modules to.
\item
  Non-Backbone components/modules (such as utilities or application
  helpers) can also be encapsulated using AMD. I encourage you to try
  developing these modules in such a way that they can both be used and
  tested independent of your Backbone code as this will increase their
  ability to be re-used elsewhere.
\end{itemize}

Now that we've reviewed the basics, let's take a look at developing our
application. For reference, the structure of our app is as follows:

\begin{verbatim}
index.html
...js/
    main.js
    .../models
            todo.js
    .../views
            app.js
            todos.js
    .../collections
            todos.js
    .../templates
            stats.html
            todos.html
    ../libs
        .../backbone
        .../jquery
        .../underscore
        .../require
                require.js
                text.js
...css/
\end{verbatim}

\subsubsection{Markup}\label{markup}

The markup for the application is relatively simple and consists of
three primary parts: an input section for entering new todo items
(\texttt{create-todo}), a list section to display existing items (which
can also be edited in-place) (\texttt{todo-list}) and finally a section
summarizing how many items are left to be completed
(\texttt{todo-stats}).

\begin{verbatim}
<div id="todoapp">

      <div class="content">

        <div id="create-todo">
          <input id="new-todo" placeholder="What needs to be done?" type="text" />
          <span class="ui-tooltip-top">Press Enter to save this task</span>
        </div>

        <div id="todos">
          <ul id="todo-list"></ul>
        </div>

        <div id="todo-stats"></div>

      </div>

</div>
\end{verbatim}

The rest of the tutorial will now focus on the JavaScript side of the
practical.

\subsubsection{Configuration options}\label{configuration-options}

If you've read the earlier chapter on AMD, you may have noticed that
explicitly needing to define each dependency a Backbone module (view,
collection or other module) may require with it can get a little
tedious. This can however be improved.

In order to simplify referencing common paths the modules in our
application may use, we use a RequireJS
\href{http://requirejs.org/docs/api.html\#config}{configuration object},
which is typically defined as a top-level script file. Configuration
objects have a number of useful capabilities, the most useful being mode
name-mapping. Name-maps are basically a key:value pair, where the key
defines the alias you wish to use for a path and the value represents
the true location of the path.

In the code-sample below, you can see some typical examples of common
name-maps which include: \texttt{backbone}, \texttt{underscore},
\texttt{jquery} and depending on your choice, the RequireJS
\texttt{text} plugin, which assists with loading text assets like
templates.

\textbf{main.js}

\begin{Shaded}
\begin{Highlighting}[]
\OtherTok{require}\NormalTok{.}\FunctionTok{config}\NormalTok{(\{}
  \DataTypeTok{baseUrl}\NormalTok{:}\StringTok{'../'}\NormalTok{,}
  \DataTypeTok{paths}\NormalTok{: \{}
    \DataTypeTok{jquery}\NormalTok{: }\StringTok{'libs/jquery/jquery-min'}\NormalTok{,}
    \DataTypeTok{underscore}\NormalTok{: }\StringTok{'libs/underscore/underscore-min'}\NormalTok{,}
    \DataTypeTok{backbone}\NormalTok{: }\StringTok{'libs/backbone/backbone-optamd3-min'}\NormalTok{,}
    \DataTypeTok{text}\NormalTok{: }\StringTok{'libs/require/text'}
  \NormalTok{\}}
\NormalTok{\});}

\FunctionTok{require}\NormalTok{([}\StringTok{'views/app'}\NormalTok{], }\KeywordTok{function}\NormalTok{(AppView)\{}
  \KeywordTok{var} \NormalTok{app_view = }\KeywordTok{new} \NormalTok{AppView;}
\NormalTok{\});}
\end{Highlighting}
\end{Shaded}

The \texttt{require()} at the end of our main.js file is simply there so
we can load and instantiate the primary view for our application
(\texttt{views/app.js}). You'll commonly see both this and the
configuration object included in most top-level script files for a
project.

In addition to offering name-mapping, the configuration object can be
used to define additional properties such as \texttt{waitSeconds} - the
number of seconds to wait before script loading times out and
\texttt{locale}, should you wish to load up i18n bundles for custom
languages. The \texttt{baseUrl} is simply the path to use for module
lookups.

For more information on configuration objects, please feel free to check
out the excellent guide to them in the
\href{http://requirejs.org/docs/api.html\#config}{RequireJS docs}.

\subsubsection{Modularizing our models, views and
collections}\label{modularizing-our-models-views-and-collections}

Before we dive into AMD-wrapped versions of our Backbone components,
let's review a sample of a non-AMD view. The following view listens for
changes to its model (a Todo item) and re-renders if a user edits the
value of the item.

\begin{Shaded}
\begin{Highlighting}[]
\KeywordTok{var} \NormalTok{TodoView = }\OtherTok{Backbone}\NormalTok{.}\OtherTok{View}\NormalTok{.}\FunctionTok{extend}\NormalTok{(\{}

    \CommentTok{//... is a list tag.}
    \DataTypeTok{tagName}\NormalTok{:  }\StringTok{'li'}\NormalTok{,}

    \CommentTok{// Cache the template function for a single item.}
    \DataTypeTok{template}\NormalTok{: }\OtherTok{_}\NormalTok{.}\FunctionTok{template}\NormalTok{(}\FunctionTok{$}\NormalTok{(}\StringTok{'#item-template'}\NormalTok{).}\FunctionTok{html}\NormalTok{()),}

    \CommentTok{// The DOM events specific to an item.}
    \DataTypeTok{events}\NormalTok{: \{}
      \StringTok{'click .check'}              \NormalTok{: }\StringTok{'toggleDone'}\NormalTok{,}
      \StringTok{'dblclick div.todo-content'} \NormalTok{: }\StringTok{'edit'}\NormalTok{,}
      \StringTok{'click span.todo-destroy'}   \NormalTok{: }\StringTok{'clear'}\NormalTok{,}
      \StringTok{'keypress .todo-input'}      \NormalTok{: }\StringTok{'updateOnEnter'}
    \NormalTok{\},}

    \CommentTok{// The TodoView listens for changes to its model, re-rendering. Since there's}
    \CommentTok{// a one-to-one correspondence between a **Todo** and a **TodoView** in this}
    \CommentTok{// app, we set a direct reference on the model for convenience.}
    \DataTypeTok{initialize}\NormalTok{: }\KeywordTok{function}\NormalTok{() \{}
      \KeywordTok{this}\NormalTok{.}\FunctionTok{listenTo}\NormalTok{(}\KeywordTok{this}\NormalTok{.}\FunctionTok{model}\NormalTok{, }\StringTok{'change'}\NormalTok{, }\KeywordTok{this}\NormalTok{.}\FunctionTok{render}\NormalTok{);}
      \KeywordTok{this}\NormalTok{.}\OtherTok{model}\NormalTok{.}\FunctionTok{view} \NormalTok{= }\KeywordTok{this}\NormalTok{;}
    \NormalTok{\},}
    \NormalTok{...}
\end{Highlighting}
\end{Shaded}

Note how for templating the common practice of referencing a script by
an ID (or other selector) and obtaining its value is used. This of
course requires that the template being accessed is implicitly defined
in our markup. The following is the `embedded' version of our template
being referenced above:

\begin{verbatim}
<script type="text/template" id="item-template">
      <div class="todo <%= done ? 'done' : '' %>">
        <div class="display">
          <input class="check" type="checkbox" <%= done ? 'checked="checked"' : '' %> />
          <div class="todo-content"></div>
          <span class="todo-destroy"></span>
        </div>
        <div class="edit">
          <input class="todo-input" type="text" value="" />
        </div>
      </div>
</script>
\end{verbatim}

Whilst there is nothing wrong with the template itself, once we begin to
develop larger applications requiring multiple templates, including them
all in our markup on page-load can quickly become both unmanageable and
come with performance costs. We'll look at solving this problem in a
minute.

Let's now take a look at the AMD-version of our view. As discussed
earlier, the `module' is wrapped using AMD's \texttt{define()} which
allows us to specify the dependencies our view requires. Using the
mapped paths to `jquery' etc. simplifies referencing common dependencies
and instances of dependencies are themselves mapped to local variables
that we can access (e.g `jquery' is mapped to \texttt{\$}).

\textbf{views/todo.js}

\begin{Shaded}
\begin{Highlighting}[]
\FunctionTok{define}\NormalTok{([}
  \StringTok{'jquery'}\NormalTok{,}
  \StringTok{'underscore'}\NormalTok{,}
  \StringTok{'backbone'}\NormalTok{,}
  \StringTok{'text!templates/todos.html'}
  \NormalTok{], }\KeywordTok{function}\NormalTok{($, _, Backbone, todosTemplate)\{}
  \KeywordTok{var} \NormalTok{TodoView = }\OtherTok{Backbone}\NormalTok{.}\OtherTok{View}\NormalTok{.}\FunctionTok{extend}\NormalTok{(\{}

    \CommentTok{//... is a list tag.}
    \DataTypeTok{tagName}\NormalTok{:  }\StringTok{'li'}\NormalTok{,}

    \CommentTok{// Cache the template function for a single item.}
    \DataTypeTok{template}\NormalTok{: }\OtherTok{_}\NormalTok{.}\FunctionTok{template}\NormalTok{(todosTemplate),}

    \CommentTok{// The DOM events specific to an item.}
    \DataTypeTok{events}\NormalTok{: \{}
      \StringTok{'click .check'}              \NormalTok{: }\StringTok{'toggleDone'}\NormalTok{,}
      \StringTok{'dblclick div.todo-content'} \NormalTok{: }\StringTok{'edit'}\NormalTok{,}
      \StringTok{'click span.todo-destroy'}   \NormalTok{: }\StringTok{'clear'}\NormalTok{,}
      \StringTok{'keypress .todo-input'}      \NormalTok{: }\StringTok{'updateOnEnter'}
    \NormalTok{\},}

    \CommentTok{// The TodoView listens for changes to its model, re-rendering. Since there's}
    \CommentTok{// a one-to-one correspondence between a **Todo** and a **TodoView** in this}
    \CommentTok{// app, we set a direct reference on the model for convenience.}
    \DataTypeTok{initialize}\NormalTok{: }\KeywordTok{function}\NormalTok{() \{}
      \KeywordTok{this}\NormalTok{.}\FunctionTok{listenTo}\NormalTok{(}\KeywordTok{this}\NormalTok{.}\FunctionTok{model}\NormalTok{, }\StringTok{'change'}\NormalTok{, }\KeywordTok{this}\NormalTok{.}\FunctionTok{render}\NormalTok{);}
      \KeywordTok{this}\NormalTok{.}\OtherTok{model}\NormalTok{.}\FunctionTok{view} \NormalTok{= }\KeywordTok{this}\NormalTok{;}
    \NormalTok{\},}

    \CommentTok{// Re-render the contents of the todo item.}
    \DataTypeTok{render}\NormalTok{: }\KeywordTok{function}\NormalTok{() \{}
      \KeywordTok{this}\NormalTok{.}\OtherTok{$el}\NormalTok{.}\FunctionTok{html}\NormalTok{(}\KeywordTok{this}\NormalTok{.}\FunctionTok{template}\NormalTok{(}\KeywordTok{this}\NormalTok{.}\OtherTok{model}\NormalTok{.}\FunctionTok{attributes}\NormalTok{));}
      \KeywordTok{this}\NormalTok{.}\FunctionTok{setContent}\NormalTok{();}
      \KeywordTok{return} \KeywordTok{this}\NormalTok{;}
    \NormalTok{\},}

    \CommentTok{// Use `jQuery.text` to set the contents of the todo item.}
    \DataTypeTok{setContent}\NormalTok{: }\KeywordTok{function}\NormalTok{() \{}
      \KeywordTok{var} \NormalTok{content = }\KeywordTok{this}\NormalTok{.}\OtherTok{model}\NormalTok{.}\FunctionTok{get}\NormalTok{(}\StringTok{'content'}\NormalTok{);}
      \KeywordTok{this}\NormalTok{.}\FunctionTok{$}\NormalTok{(}\StringTok{'.todo-content'}\NormalTok{).}\FunctionTok{text}\NormalTok{(content);}
      \KeywordTok{this}\NormalTok{.}\FunctionTok{input} \NormalTok{= }\KeywordTok{this}\NormalTok{.}\FunctionTok{$}\NormalTok{(}\StringTok{'.todo-input'}\NormalTok{);}
      \KeywordTok{this}\NormalTok{.}\OtherTok{input}\NormalTok{.}\FunctionTok{on}\NormalTok{(}\StringTok{'blur'}\NormalTok{, }\KeywordTok{this}\NormalTok{.}\FunctionTok{close}\NormalTok{);}
      \KeywordTok{this}\NormalTok{.}\OtherTok{input}\NormalTok{.}\FunctionTok{val}\NormalTok{(content);}
    \NormalTok{\},}
    \NormalTok{...}
\end{Highlighting}
\end{Shaded}

From a maintenance perspective, there's nothing logically different in
this version of our view, except for how we approach templating.

Using the RequireJS text plugin (the dependency marked \texttt{text}),
we can actually store all of the contents for the template we looked at
earlier in an external file (todos.html).

\textbf{templates/todos.html}

\begin{Shaded}
\begin{Highlighting}[]
\KeywordTok{<div}\OtherTok{ class=}\StringTok{"todo }\ErrorTok{<}\StringTok{%= done ? 'done' : '' %>"}\KeywordTok{>}
    \KeywordTok{<div}\OtherTok{ class=}\StringTok{"display"}\KeywordTok{>}
      \KeywordTok{<input}\OtherTok{ class=}\StringTok{"check"}\OtherTok{ type=}\StringTok{"checkbox"} \ErrorTok{<%}\OtherTok{=} \StringTok{done} \ErrorTok{?} \ErrorTok{'checked}\OtherTok{=}\StringTok{"checked"}\ErrorTok{'}\OtherTok{ :} \ErrorTok{''} \ErrorTok{%}\KeywordTok{>} \NormalTok{/>}
      \KeywordTok{<div}\OtherTok{ class=}\StringTok{"todo-content"}\KeywordTok{></div>}
      \KeywordTok{<span}\OtherTok{ class=}\StringTok{"todo-destroy"}\KeywordTok{></span>}
    \KeywordTok{</div>}
    \KeywordTok{<div}\OtherTok{ class=}\StringTok{"edit"}\KeywordTok{>}
      \KeywordTok{<input}\OtherTok{ class=}\StringTok{"todo-input"}\OtherTok{ type=}\StringTok{"text"}\OtherTok{ value=}\StringTok{""} \KeywordTok{/>}
    \KeywordTok{</div>}
\KeywordTok{</div>}
\end{Highlighting}
\end{Shaded}

There's no longer a need to be concerned with IDs for the template as we
can map its contents to a local variable (in this case
\texttt{todosTemplate}). We then simply pass this to the Underscore.js
templating function \texttt{\_.template()} the same way we normally
would have the value of our template script.

Next, let's look at how to define models as dependencies which can be
pulled into collections. Here's an AMD-compatible model module, which
has two default values: a \texttt{content} attribute for the content of
a Todo item and a boolean \texttt{done} state, allowing us to trigger
whether the item has been completed or not.

\textbf{models/todo.js}

\begin{Shaded}
\begin{Highlighting}[]
\FunctionTok{define}\NormalTok{([}\StringTok{'underscore'}\NormalTok{, }\StringTok{'backbone'}\NormalTok{], }\KeywordTok{function}\NormalTok{(_, Backbone) \{}
  \KeywordTok{var} \NormalTok{TodoModel = }\OtherTok{Backbone}\NormalTok{.}\OtherTok{Model}\NormalTok{.}\FunctionTok{extend}\NormalTok{(\{}

    \CommentTok{// Default attributes for the todo.}
    \DataTypeTok{defaults}\NormalTok{: \{}
      \CommentTok{// Ensure that each todo created has `content`.}
      \DataTypeTok{content}\NormalTok{: }\StringTok{'empty todo...'}\NormalTok{,}
      \DataTypeTok{done}\NormalTok{: }\KeywordTok{false}
    \NormalTok{\},}

    \DataTypeTok{initialize}\NormalTok{: }\KeywordTok{function}\NormalTok{() \{}
    \NormalTok{\},}

    \CommentTok{// Toggle the `done` state of this todo item.}
    \DataTypeTok{toggle}\NormalTok{: }\KeywordTok{function}\NormalTok{() \{}
      \KeywordTok{this}\NormalTok{.}\FunctionTok{save}\NormalTok{(\{}\DataTypeTok{done}\NormalTok{: !}\KeywordTok{this}\NormalTok{.}\FunctionTok{get}\NormalTok{(}\StringTok{'done'}\NormalTok{)\});}
    \NormalTok{\},}

    \CommentTok{// Remove this Todo from *localStorage* and delete its view.}
    \DataTypeTok{clear}\NormalTok{: }\KeywordTok{function}\NormalTok{() \{}
      \KeywordTok{this}\NormalTok{.}\FunctionTok{destroy}\NormalTok{();}
      \KeywordTok{this}\NormalTok{.}\OtherTok{view}\NormalTok{.}\FunctionTok{remove}\NormalTok{();}
    \NormalTok{\}}

  \NormalTok{\});}
  \KeywordTok{return} \NormalTok{TodoModel;}
\NormalTok{\});}
\end{Highlighting}
\end{Shaded}

As per other types of dependencies, we can easily map our model module
to a local variable (in this case \texttt{Todo}) so it can be referenced
as the model to use for our \texttt{TodosCollection}. This collection
also supports a simple \texttt{done()} filter for narrowing down Todo
items that have been completed and a \texttt{remaining()} filter for
those that are still outstanding.

\textbf{collections/todos.js}

\begin{Shaded}
\begin{Highlighting}[]
\FunctionTok{define}\NormalTok{([}
  \StringTok{'underscore'}\NormalTok{,}
  \StringTok{'backbone'}\NormalTok{,}
  \StringTok{'libs/backbone/localstorage'}\NormalTok{,}
  \StringTok{'models/todo'}
  \NormalTok{], }\KeywordTok{function}\NormalTok{(_, Backbone, Store, Todo)\{}

    \KeywordTok{var} \NormalTok{TodosCollection = }\OtherTok{Backbone}\NormalTok{.}\OtherTok{Collection}\NormalTok{.}\FunctionTok{extend}\NormalTok{(\{}

    \CommentTok{// Reference to this collection's model.}
    \DataTypeTok{model}\NormalTok{: Todo,}

    \CommentTok{// Save all of the todo items under the `todos` namespace.}
    \DataTypeTok{localStorage}\NormalTok{: }\KeywordTok{new} \FunctionTok{Store}\NormalTok{(}\StringTok{'todos'}\NormalTok{),}

    \CommentTok{// Filter down the list of all todo items that are finished.}
    \DataTypeTok{done}\NormalTok{: }\KeywordTok{function}\NormalTok{() \{}
      \KeywordTok{return} \KeywordTok{this}\NormalTok{.}\FunctionTok{filter}\NormalTok{(}\KeywordTok{function}\NormalTok{(todo)\{ }\KeywordTok{return} \OtherTok{todo}\NormalTok{.}\FunctionTok{get}\NormalTok{(}\StringTok{'done'}\NormalTok{); \});}
    \NormalTok{\},}

    \CommentTok{// Filter down the list to only todo items that are still not finished.}
    \DataTypeTok{remaining}\NormalTok{: }\KeywordTok{function}\NormalTok{() \{}
      \KeywordTok{return} \KeywordTok{this}\NormalTok{.}\OtherTok{without}\NormalTok{.}\FunctionTok{apply}\NormalTok{(}\KeywordTok{this}\NormalTok{, }\KeywordTok{this}\NormalTok{.}\FunctionTok{done}\NormalTok{());}
    \NormalTok{\},}
    \NormalTok{...}
\end{Highlighting}
\end{Shaded}

In addition to allowing users to add new Todo items from views (which we
then insert as models in a collection), we ideally also want to be able
to display how many items have been completed and how many are
remaining. We've already defined filters that can provide us this
information in the above collection, so let's use them in our main
application view.

\textbf{views/app.js}

\begin{Shaded}
\begin{Highlighting}[]
\FunctionTok{define}\NormalTok{([}
  \StringTok{'jquery'}\NormalTok{,}
  \StringTok{'underscore'}\NormalTok{,}
  \StringTok{'backbone'}\NormalTok{,}
  \StringTok{'collections/todos'}\NormalTok{,}
  \StringTok{'views/todo'}\NormalTok{,}
  \StringTok{'text!templates/stats.html'}
  \NormalTok{], }\KeywordTok{function}\NormalTok{($, _, Backbone, Todos, TodoView, statsTemplate)\{}

  \KeywordTok{var} \NormalTok{AppView = }\OtherTok{Backbone}\NormalTok{.}\OtherTok{View}\NormalTok{.}\FunctionTok{extend}\NormalTok{(\{}

    \CommentTok{// Instead of generating a new element, bind to the existing skeleton of}
    \CommentTok{// the App already present in the HTML.}
    \DataTypeTok{el}\NormalTok{: }\FunctionTok{$}\NormalTok{(}\StringTok{'#todoapp'}\NormalTok{),}

    \CommentTok{// Our template for the line of statistics at the bottom of the app.}
    \DataTypeTok{statsTemplate}\NormalTok{: }\OtherTok{_}\NormalTok{.}\FunctionTok{template}\NormalTok{(statsTemplate),}

    \CommentTok{// ...events, initialize() etc. can be seen in the complete file}

    \CommentTok{// Re-rendering the App just means refreshing the statistics -- the rest}
    \CommentTok{// of the app doesn't change.}
    \DataTypeTok{render}\NormalTok{: }\KeywordTok{function}\NormalTok{() \{}
      \KeywordTok{var} \NormalTok{done = }\OtherTok{Todos}\NormalTok{.}\FunctionTok{done}\NormalTok{().}\FunctionTok{length}\NormalTok{;}
      \KeywordTok{this}\NormalTok{.}\FunctionTok{$}\NormalTok{(}\StringTok{'#todo-stats'}\NormalTok{).}\FunctionTok{html}\NormalTok{(}\KeywordTok{this}\NormalTok{.}\FunctionTok{statsTemplate}\NormalTok{(\{}
        \DataTypeTok{total}\NormalTok{:      }\OtherTok{Todos}\NormalTok{.}\FunctionTok{length}\NormalTok{,}
        \DataTypeTok{done}\NormalTok{:       }\OtherTok{Todos}\NormalTok{.}\FunctionTok{done}\NormalTok{().}\FunctionTok{length}\NormalTok{,}
        \DataTypeTok{remaining}\NormalTok{:  }\OtherTok{Todos}\NormalTok{.}\FunctionTok{remaining}\NormalTok{().}\FunctionTok{length}
      \NormalTok{\}));}
    \NormalTok{\},}
    \NormalTok{...}
\end{Highlighting}
\end{Shaded}

Above, we map the second template for this project,
\texttt{templates/stats.html} to \texttt{statsTemplate} which is used
for rendering the overall \texttt{done} and \texttt{remaining} states.
This works by simply passing our template the length of our overall
Todos collection (\texttt{Todos.length} - the number of Todo items
created so far) and similarly the length (counts) for items that have
been completed (\texttt{Todos.done().length}) or are remaining
(\texttt{Todos.remaining().length}).

The contents of our \texttt{statsTemplate} can be seen below. It's
nothing too complicated, but does use ternary conditions to evaluate
whether we should state there's ``1 item'' or ``2 items'' in a
particular state.

\begin{verbatim}
<% if (total) { %>
        <span class="todo-count">
          <span class="number"><%= remaining %></span>
          <span class="word"><%= remaining == 1 ? 'item' : 'items' %></span> left.
        </span>
      <% } %>
      <% if (done) { %>
        <span class="todo-clear">
          <a href="#">
            Clear <span class="number-done"><%= done %></span>
            completed <span class="word-done"><%= done == 1 ? 'item' : 'items' %></span>
          </a>
        </span>
      <% } %>
\end{verbatim}

The rest of the source for the Todo app mainly consists of code for
handling user and application events, but that rounds up most of the
core concepts for this practical.

To see how everything ties together, feel free to grab the source by
cloning this repo or browse it
\href{https://github.com/addyosmani/backbone-fundamentals/tree/master/practicals/modular-todo-app}{online}
to learn more. I hope you find it helpful!

\textbf{Note:} While this first practical doesn't use a build profile as
outlined in the chapter on using the RequireJS optimizer, we will be
using one in the section on building mobile Backbone applications.

\subsection{Route-based module
loading}\label{route-based-module-loading}

This section will discuss a route based approach to module loading as
implemented in \href{http://walmartlabs.github.com/lumbar}{Lumbar} by
Kevin Decker. Like RequireJS, Lumbar is also a modular build system, but
the pattern it implements for loading routes may be used with any build
system.

The specifics of the Lumbar build tool are not discussed in this book.
To see a complete Lumbar based project with the loader and build system
see \href{http://thoraxjs.org}{Thorax} which provides boilerplate
projects for various environments including Lumbar.

\subsubsection{JSON-based module
configuration}\label{json-based-module-configuration}

RequireJS defines dependencies per file, while Lumbar defines a list of
files for each module in a central JSON configuration file, outputting a
single JavaScript file for each defined module. Lumbar requires that
each module (except the base module) define a single router and a list
of routes. An example file might look like:

\begin{verbatim}
 {
    "modules": {
        "base": {
            "scripts": [
                "js/lib/underscore.js",
                "js/lib/backbone.js",
                "etc"
            ]
        },
        "pages": {
            "scripts": [
                "js/routers/pages.js",
                "js/views/pages/index.js",
                "etc"
            ],
            "routes": {
                "": "index",
                "contact": "contact"
            }
        }
    }
}
\end{verbatim}

Every JavaScript file defined in a module will have a \texttt{module}
object in scope which contains the \texttt{name} and \texttt{routes} for
the module. In \texttt{js/routers/pages.js} we could define a Backbone
router for our \texttt{pages} module like so:

\begin{verbatim}
new (Backbone.Router.extend({
    routes: module.routes,
    index: function() {},
    contact: function() {}
}));
\end{verbatim}

\subsubsection{Module loader Router}\label{module-loader-router}

A little used feature of \texttt{Backbone.Router} is its ability to
create multiple routers that listen to the same set of routes. Lumbar
uses this feature to create a router that listens to all routes in the
application. When a route is matched, this master router checks to see
if the needed module is loaded. If the module is already loaded, then
the master router takes no action and the router defined by the module
will handle the route. If the needed module has not yet been loaded, it
will be loaded, then \texttt{Backbone.history.loadUrl} will be called.
This reloads the route, causes the master router to take no further
action and the router defined in the freshly loaded module to respond.

A sample implementation is provided below. The \texttt{config} object
would need to contain the data from our sample configuration JSON file
above, and the \texttt{loader} object would need to implement
\texttt{isLoaded} and \texttt{loadModule} methods. Note that Lumbar
provides all of these implementations, the examples are provided to
create your own implementation.

\begin{verbatim}
// Create an object that will be used as the prototype
// for our master router
var handlers = {
    routes: {}
};

_.each(config.modules, function(module, moduleName) {
    if (module.routes) {
        // Generate a loading callback for the module
        var callbackName = "loader_" moduleName;
        handlers[callbackName] = function() {
            if (loader.isLoaded(moduleName)) {
                // Do nothing if the module is loaded
                return;
            } else {
                //the module needs to be loaded
                loader.loadModule(moduleName, function() {
                    // Module is loaded, reloading the route
                    // will trigger callback in the module's
                    // router
                    Backbone.history.loadUrl();
                });
            }
        };
        // Each route in the module should trigger the
        // loading callback
        _.each(module.routes, function(methodName, route) {
            handlers.routes[route] = callbackName;
        });
    }
});

// Create the master router
new (Backbone.Router.extend(handlers));
\end{verbatim}

\subsubsection{Using NodeJS to handle
pushState}\label{using-nodejs-to-handle-pushstate}

\texttt{window.history.pushState} support (serving Backbone routes
without a hash mark) requires that the server be aware of what URLs your
Backbone application will handle, since the user should be able to enter
the app at any of those routes (or hit reload after navigating to a
pushState URL).

Another advantage to defining all routes in a single location is that
the same JSON configuration file provided above could be loaded by the
server, listening to each route. A sample implementation in NodeJS and
Express:

\begin{verbatim}
var fs = require('fs'),
    _ = require('underscore'),
    express = require('express'),
    server = express(),
    config = JSON.parse(fs.readFileSync('path/to/config.json'));

_.each(config.modules, function(module, moduleName) {
    if (module.routes) {
        _.each(module.routes, function(methodName, route) {
            server.get(route, function(req, res) {
                  res.sendFile('public/index.html');
            });
        });
    }
});
\end{verbatim}

This assumes that index.html will be serving out your Backbone
application. The \texttt{Backbone.History} object can handle the rest of
the routing logic as long as a \texttt{root} option is specified. A
sample configuration for a simple application that lives at the root
might look like:

\begin{verbatim}
Backbone.history || (Backbone.history = new Backbone.History());
Backbone.history.start({
  pushState: true,
  root: '/'
});
\end{verbatim}

\subsection{An asset package alternative for dependency
management}\label{an-asset-package-alternative-for-dependency-management}

For more than trivial views, DocumentCloud have a home-built asset
packager called \href{https://github.com/documentcloud/jammit}{Jammit},
which has easy integration with Underscore.js templates and can also be
used for dependency management.

Jammit expects your JavaScript templates (JST) to live alongside any ERB
templates you're using in the form of .jst files. It packages the
templates into a global JST object which can be used to render templates
into strings. Making Jammit aware of your templates is straight-forward
- just add an entry for something like \texttt{views/**/*.jst} to your
app package in assets.yml.

To provide Jammit dependencies you simply write out an assets.yml file
that either listed the dependencies in order or used a combination of
free capture directories (for example: \texttt{/**/*.js},
\texttt{templates/*.js}, and specific files).

A template using Jammit can derive it's data from the collection object
that is passed to it:

\begin{verbatim}
this.$el.html(JST.myTemplate({ collection: this.collection }));
\end{verbatim}

\section{Paginating Backbone.js Requests \&
Collections}\label{paginating-backbone.js-requests-collections}

\subsection{Introduction}\label{introduction-2}

Pagination is a ubiquitous problem we often find ourselves needing to
solve on the web - perhaps most predominantly when working with service
APIs and JavaScript-heavy clients which consume them. It's also a
problem that is often under-refined as most of us consider pagination
relatively easy to get right. This isn't however always the case as
pagination tends to get more tricky than it initially seems.

Before we dive into solutions for paginating data for your Backbone
applications, let's define exactly what we consider pagination to be:

Pagination is a control system allowing users to browse through pages of
search results (or any type of content) which is continued. Search
results are the canonical example, but pagination today is found on news
sites, blogs, and discussion boards, often in the form of Previous and
Next links. More complete pagination systems offer granular control of
the specific pages you can navigate to, giving the user more power to
find what they are looking for.

It isn't a problem limited to pages requiring some visual controls for
pagination either - sites like Facebook, Pinterest, and Twitter have
demonstrated that there are many contexts where infinite paging is also
useful. Infinite paging is, of course, when we pre-fetch (or appear to
pre-fetch) content from a subsequent page and add it directly to the
user's current page, making the experience feel ``infinite''.

Pagination is very context-specific and depends on the content being
displayed. In the Google search results, pagination is important as they
want to offer you the most relevant set of results in the first 1-2
pages. After that, you might be a little more selective (or random) with
the page you choose to navigate to. This differs from cases where you'll
want to cycle through consecutive pages for (e.g., for a news article or
blog post).

Pagination is almost certainly content and context-specific, but as
Faruk Ates has \href{https://gist.github.com/mislav/622561}{previously}
pointed out the principles of good pagination apply no matter what the
content or context is. As with everything extensible when it comes to
Backbone, you can write your own pagination to address many of these
content-specific types of pagination problems. That said, you'll
probably spend quite a bit of time on this and sometimes you just want
to use a tried and tested solution that just works.

On this topic, we're going to go through a set of pagination components
I (and a group of
\href{https://github.com/addyosmani/backbone.paginator/contributors}{contributors})
wrote for Backbone.js, which should hopefully come in useful if you're
working on applications which need to page Backbone Collections. They're
part of an extension called
\href{http://github.com/addyosmani/backbone.paginator}{Backbone.Paginator}.

\subsubsection{Backbone.Paginator}\label{backbone.paginator}

\textbf{Note:} As of Backbone.Paginator
\href{https://github.com/backbone-paginator/backbone.paginator/releases}{2.0},
the API to the project has changed and includes updated which break
backwards compatibility. The below section refers to Backbone.Paginator
1.0 which can still be downloaded
\href{https://github.com/backbone-paginator/backbone.paginator/releases/tag/v1.0.0}{here}.

When working with data on the client-side, the three types of pagination
we are most likely to run into are:

\textbf{Requests to a service layer (API)} - For example, query for
results containing the term `Paul' - if 5,000 results are available only
display 20 results per page (leaving us with 250 possible result pages
that can be navigated to).

This problem actually has quite a great deal more to it, such as
maintaining persistence of other URL parameters (e.g sort, query, order)
which can change based on a user's search configuration in a UI. One
also has to think of a clean way of hooking views up to this pagination
so you can easily navigate between pages (e.g., First, Last, Next,
Previous, 1,2,3), manage the number of results displayed per page and so
on.

\textbf{Further client-side pagination of data returned -} e.g we've
been returned a JSON response containing 100 results. Rather than
displaying all 100 to the user, we only display 20 of these results
within a navigable UI in the browser.

Similar to the request problem, client-pagination has its own challenges
like navigation once again (Next, Previous, 1,2,3), sorting, order,
switching the number of results to display per page and so on.

\textbf{Infinite results} - with services such as Facebook, the concept
of numeric pagination is instead replaced with a `Load More' or `View
More' button. Triggering this normally fetches the next `page' of N
results but rather than replacing the previous set of results loaded
entirely, we simply append to them instead.

A request pager which simply appends results in a view rather than
replacing on each new fetch is effectively an `infinite' pager.

\textbf{Let's now take a look at exactly what we're getting out of the
box:}

Backbone.Paginator is a set of opinionated components for paginating
collections of data using Backbone.js. It aims to provide both solutions
for assisting with pagination of requests to a server (e.g an API) as
well as pagination of single-loads of data, where we may wish to further
paginate a collection of N results into M pages within a view.

\begin{figure}[htbp]
\centering
\includegraphics{img/paginator-ui.png}
\end{figure}

Backbone.Paginator supports two main pagination components:

\begin{itemize}
\itemsep1pt\parskip0pt\parsep0pt
\item
  \textbf{Backbone.Paginator.requestPager}: For pagination of requests
  between a client and a server-side API
\item
  \textbf{Backbone.Paginator.clientPager}: For pagination of data
  returned from a server which you would like to further paginate within
  the UI (e.g 60 results are returned, paginate into 3 pages of 20)
\end{itemize}

\subsubsection{Live Examples}\label{live-examples}

If you would like to look at examples built using the components
included in the project, links to official demos are included below and
use the Netflix API so that you can see them working with an actual data
source.

\begin{itemize}
\itemsep1pt\parskip0pt\parsep0pt
\item
  \href{http://addyosmani.github.com/backbone.paginator/examples/netflix-request-paging/index.html}{Backbone.Paginator.requestPager()}
\item
  \href{http://addyosmani.github.com/backbone.paginator/examples/netflix-client-paging/index.html}{Backbone.Paginator.clientPager()}
\item
  \href{http://addyosmani.github.com/backbone.paginator/examples/netflix-infinite-paging/index.html}{Infinite
  Pagination (Backbone.Paginator.requestPager())}
\item
  \href{http://addyosmani.github.com/backbone.paginator/examples/google-diacritic/index.html}{Diacritic
  Plugin}
\end{itemize}

\subsection{Paginator.requestPager}\label{paginator.requestpager}

In this section we're going to walk through using the requestPager. You
would use this component when working with a service API which itself
supports pagination. This component allows users to control the
pagination settings for requests to this API (i.e navigate to the next,
previous, N pages) via the client-side.

The idea is that pagination, searching, and filtering of data can all be
done from your Backbone application without the need for a page reload.

\begin{figure}[htbp]
\centering
\includegraphics{img/paginator-request.png}
\end{figure}

\paragraph{1. Create a new Paginated
collection}\label{create-a-new-paginated-collection}

First, we define a new Paginated collection using
\texttt{Backbone.Paginator.requestPager()} as follows:

\begin{Shaded}
\begin{Highlighting}[]

\KeywordTok{var} \NormalTok{PaginatedCollection = }\OtherTok{Backbone}\NormalTok{.}\OtherTok{Paginator}\NormalTok{.}\OtherTok{requestPager}\NormalTok{.}\FunctionTok{extend}\NormalTok{(\{}
\end{Highlighting}
\end{Shaded}

\paragraph{2. Set the model for the collection as
normal}\label{set-the-model-for-the-collection-as-normal}

Within our collection, we then (as normal) specify the model to be used
with this collection followed by the URL (or base URL) for the service
providing our data (e.g the Netflix API).

\begin{Shaded}
\begin{Highlighting}[]

        \NormalTok{model: model,}
\end{Highlighting}
\end{Shaded}

\paragraph{3. Configure the base URL and the type of the
request}\label{configure-the-base-url-and-the-type-of-the-request}

We need to set a base URL. The \texttt{type} of the request is
\texttt{GET} by default, and the \texttt{dataType} is \texttt{jsonp} in
order to enable cross-domain requests.

\begin{Shaded}
\begin{Highlighting}[]
    \NormalTok{paginator_core: \{}
      \CommentTok{// the type of the request (GET by default)}
      \DataTypeTok{type}\NormalTok{: }\StringTok{'GET'}\NormalTok{,}

      \CommentTok{// the type of reply (jsonp by default)}
      \DataTypeTok{dataType}\NormalTok{: }\StringTok{'jsonp'}\NormalTok{,}

      \CommentTok{// the URL (or base URL) for the service}
      \CommentTok{// if you want to have a more dynamic URL, you can make this a function}
      \CommentTok{// that returns a string}
      \DataTypeTok{url}\NormalTok{: }\StringTok{'http://odata.netflix.com/Catalog/People(49446)/TitlesActedIn?'}
    \NormalTok{\},}
\end{Highlighting}
\end{Shaded}

\subsection{Gotchas!}\label{gotchas}

If you use \texttt{dataType} \textbf{NOT} jsonp, please remove the
callback custom parameter inside the \texttt{server\_api} configuration.

\paragraph{4. Configure how the library will show the
results}\label{configure-how-the-library-will-show-the-results}

We need to tell the library how many items per page we would like to
see, etc\ldots{}

\begin{Shaded}
\begin{Highlighting}[]
    \NormalTok{paginator_ui: \{}
      \CommentTok{// the lowest page index your API allows to be accessed}
      \DataTypeTok{firstPage}\NormalTok{: }\DecValTok{0}\NormalTok{,}

      \CommentTok{// which page should the paginator start from}
      \CommentTok{// (also, the actual page the paginator is on)}
      \DataTypeTok{currentPage}\NormalTok{: }\DecValTok{0}\NormalTok{,}

      \CommentTok{// how many items per page should be shown}
      \DataTypeTok{perPage}\NormalTok{: }\DecValTok{3}\NormalTok{,}

      \CommentTok{// a default number of total pages to query in case the API or}
      \CommentTok{// service you are using does not support providing the total}
      \CommentTok{// number of pages for us.}
      \CommentTok{// 10 as a default in case your service doesn't return the total}
      \DataTypeTok{totalPages}\NormalTok{: }\DecValTok{10}
    \NormalTok{\},}
\end{Highlighting}
\end{Shaded}

\paragraph{5. Configure the parameters we want to send to the
server}\label{configure-the-parameters-we-want-to-send-to-the-server}

Only the base URL won't be enough for most cases, so you can pass more
parameters to the server. Note how you can use functions instead of
hardcoded values, and you can also refer to the values you specified in
\texttt{paginator\_ui}.

\begin{Shaded}
\begin{Highlighting}[]
    \NormalTok{server_api: \{}
      \CommentTok{// the query field in the request}
      \StringTok{'$filter'}\NormalTok{: }\StringTok{''}\NormalTok{,}

      \CommentTok{// number of items to return per request/page}
      \StringTok{'$top'}\NormalTok{: }\KeywordTok{function}\NormalTok{() \{ }\KeywordTok{return} \KeywordTok{this}\NormalTok{.}\FunctionTok{perPage} \NormalTok{\},}

      \CommentTok{// how many results the request should skip ahead to}
      \CommentTok{// customize as needed. For the Netflix API, skipping ahead based on}
      \CommentTok{// page * number of results per page was necessary.}
      \StringTok{'$skip'}\NormalTok{: }\KeywordTok{function}\NormalTok{() \{ }\KeywordTok{return} \KeywordTok{this}\NormalTok{.}\FunctionTok{currentPage} \NormalTok{* }\KeywordTok{this}\NormalTok{.}\FunctionTok{perPage} \NormalTok{\},}

      \CommentTok{// field to sort by}
      \StringTok{'$orderby'}\NormalTok{: }\StringTok{'ReleaseYear'}\NormalTok{,}

      \CommentTok{// what format would you like to request results in?}
      \StringTok{'$format'}\NormalTok{: }\StringTok{'json'}\NormalTok{,}

      \CommentTok{// custom parameters}
      \StringTok{'$inlinecount'}\NormalTok{: }\StringTok{'allpages'}\NormalTok{,}
      \StringTok{'$callback'}\NormalTok{: }\StringTok{'callback'}
    \NormalTok{\},}
\end{Highlighting}
\end{Shaded}

\subsection{Gotchas!}\label{gotchas-1}

If you use \texttt{\$callback}, please ensure that you did use the jsonp
as a \texttt{dataType} inside your \texttt{paginator\_core}
configuration.

\paragraph{6. Finally, configure Collection.parse() and we're
done}\label{finally-configure-collection.parse-and-were-done}

The last thing we need to do is configure our collection's
\texttt{parse()} method. We want to ensure we're returning the correct
part of our JSON response containing the data our collection will be
populated with, which below is \texttt{response.d.results} (for the
Netflix API).

You might also notice that we're setting \texttt{this.totalPages} to the
total page count returned by the API. This allows us to define the
maximum number of (result) pages available for the current/last request
so that we can clearly display this in the UI. It also allows us to
influence whether clicking say, a `next' button should proceed with a
request or not.

\begin{Shaded}
\begin{Highlighting}[]
        \NormalTok{parse: }\KeywordTok{function} \NormalTok{(response) \{}
            \CommentTok{// Be sure to change this based on how your results}
            \CommentTok{// are structured (e.g d.results is Netflix specific)}
            \KeywordTok{var} \NormalTok{tags = }\OtherTok{response}\NormalTok{.}\OtherTok{d}\NormalTok{.}\FunctionTok{results}\NormalTok{;}
            \CommentTok{//Normally this.totalPages would equal response.d.__count}
            \CommentTok{//but as this particular NetFlix request only returns a}
            \CommentTok{//total count of items for the search, we divide.}
            \KeywordTok{this}\NormalTok{.}\FunctionTok{totalPages} \NormalTok{= }\OtherTok{Math}\NormalTok{.}\FunctionTok{ceil}\NormalTok{(}\OtherTok{response}\NormalTok{.}\OtherTok{d}\NormalTok{.}\FunctionTok{__count} \NormalTok{/ }\OtherTok{this}\NormalTok{.}\FunctionTok{perPage}\NormalTok{);}
            \KeywordTok{return} \NormalTok{tags;}
        \NormalTok{\}}
    \NormalTok{\});}

\NormalTok{\});}
\end{Highlighting}
\end{Shaded}

\paragraph{Convenience methods:}\label{convenience-methods}

For your convenience, the following methods are made available for use
in your views to interact with the \texttt{requestPager}:

\begin{itemize}
\itemsep1pt\parskip0pt\parsep0pt
\item
  \textbf{Collection.goTo( n, options )} - go to a specific page
\item
  \textbf{Collection.nextPage( options )} - go to the next page
\item
  \textbf{Collection.prevPage( options )} - go to the previous page
\item
  \textbf{Collection.howManyPer( n )} - set the number of items to
  display per page
\end{itemize}

\textbf{requestPager} collection's methods \texttt{.goTo()},
\texttt{.nextPage()} and \texttt{.prevPage()} are all extensions of the
original
\href{http://documentcloud.github.com/backbone/\#Collection-fetch}{Backbone
Collection.fetch() methods}. As so, they all can take the same option
object as a parameter.

This option object can use \texttt{success} and \texttt{error}
parameters to pass a function to be executed after server answer.

\begin{Shaded}
\begin{Highlighting}[]
\OtherTok{Collection}\NormalTok{.}\FunctionTok{goTo}\NormalTok{(n, \{}
  \DataTypeTok{success}\NormalTok{: }\KeywordTok{function}\NormalTok{( collection, response ) \{}
    \CommentTok{// called is server request success}
  \NormalTok{\},}
  \DataTypeTok{error}\NormalTok{: }\KeywordTok{function}\NormalTok{( collection, response ) \{}
    \CommentTok{// called if server request fail}
  \NormalTok{\}}
\NormalTok{\});}
\end{Highlighting}
\end{Shaded}

To manage callback, you could also use the
\href{http://api.jquery.com/jQuery.ajax/\#jqXHR}{jqXHR} returned by
these methods to manage callback.

\begin{Shaded}
\begin{Highlighting}[]
\NormalTok{Collection}
  \NormalTok{.}\FunctionTok{requestNextPage}\NormalTok{()}
  \NormalTok{.}\FunctionTok{done}\NormalTok{(}\KeywordTok{function}\NormalTok{( data, textStatus, jqXHR ) \{}
    \CommentTok{// called is server request success}
  \NormalTok{\})}
  \NormalTok{.}\FunctionTok{fail}\NormalTok{(}\KeywordTok{function}\NormalTok{( data, textStatus, jqXHR ) \{}
    \CommentTok{// called if server request fail}
  \NormalTok{\})}
  \NormalTok{.}\FunctionTok{always}\NormalTok{(}\KeywordTok{function}\NormalTok{( data, textStatus, jqXHR ) \{}
    \CommentTok{// do something after server request is complete}
  \NormalTok{\});}
\NormalTok{\});}
\end{Highlighting}
\end{Shaded}

If you'd like to add the incoming models to the current collection,
instead of replacing the collection's contents, pass
\texttt{\{update: true, remove: false\}} as options to these methods.

\begin{Shaded}
\begin{Highlighting}[]
\OtherTok{Collection}\NormalTok{.}\FunctionTok{prevPage}\NormalTok{(\{ }\DataTypeTok{update}\NormalTok{: }\KeywordTok{true}\NormalTok{, }\DataTypeTok{remove}\NormalTok{: }\KeywordTok{false} \NormalTok{\});}
\end{Highlighting}
\end{Shaded}

\subsection{Paginator.clientPager}\label{paginator.clientpager}

The clientPager is used to further paginate data that has already been
returned by the service API. Say you've requested 100 results from the
service and wish to split this into 5 pages of paginated results, each
containing 20 results at a client level - the clientPager makes it
trivial to do this.

\begin{figure}[htbp]
\centering
\includegraphics{img/paginator-client.png}
\end{figure}

Use the clientPager when you prefer to get results in a single ``load''
and thus avoid making additional network requests each time your users
want to fetch the next ``page'' of items. As the results have all
already been requested, it's just a case of switching between the ranges
of data actually presented to the user.

\paragraph{1. Create a new paginated collection with a model and
URL}\label{create-a-new-paginated-collection-with-a-model-and-url}

As with \texttt{requestPager}, let's first create a new Paginated
\texttt{Backbone.Paginator.clientPager} collection, with a model:

\begin{Shaded}
\begin{Highlighting}[]
    \KeywordTok{var} \NormalTok{PaginatedCollection = }\OtherTok{Backbone}\NormalTok{.}\OtherTok{Paginator}\NormalTok{.}\OtherTok{clientPager}\NormalTok{.}\FunctionTok{extend}\NormalTok{(\{}

        \DataTypeTok{model}\NormalTok{: model,}
\end{Highlighting}
\end{Shaded}

\paragraph{2. Configure the base URL and the type of the
request}\label{configure-the-base-url-and-the-type-of-the-request-1}

We need to set a base URL. The \texttt{type} of the request is
\texttt{GET} by default, and the \texttt{dataType} is \texttt{jsonp} in
order to enable cross-domain requests.

\begin{Shaded}
\begin{Highlighting}[]
    \NormalTok{paginator_core: \{}
      \CommentTok{// the type of the request (GET by default)}
      \DataTypeTok{type}\NormalTok{: }\StringTok{'GET'}\NormalTok{,}

      \CommentTok{// the type of reply (jsonp by default)}
      \DataTypeTok{dataType}\NormalTok{: }\StringTok{'jsonp'}\NormalTok{,}

      \CommentTok{// the URL (or base URL) for the service}
      \DataTypeTok{url}\NormalTok{: }\StringTok{'http://odata.netflix.com/v2/Catalog/Titles?&'}
    \NormalTok{\},}
\end{Highlighting}
\end{Shaded}

\paragraph{3. Configure how the library will show the
results}\label{configure-how-the-library-will-show-the-results-1}

We need to tell the library how many items per page we would like to
see, etc\ldots{}

\begin{Shaded}
\begin{Highlighting}[]
    \NormalTok{paginator_ui: \{}
      \CommentTok{// the lowest page index your API allows to be accessed}
      \DataTypeTok{firstPage}\NormalTok{: }\DecValTok{1}\NormalTok{,}

      \CommentTok{// which page should the paginator start from}
      \CommentTok{// (also, the actual page the paginator is on)}
      \DataTypeTok{currentPage}\NormalTok{: }\DecValTok{1}\NormalTok{,}

      \CommentTok{// how many items per page should be shown}
      \DataTypeTok{perPage}\NormalTok{: }\DecValTok{3}\NormalTok{,}

      \CommentTok{// a default number of total pages to query in case the API or}
      \CommentTok{// service you are using does not support providing the total}
      \CommentTok{// number of pages for us.}
      \CommentTok{// 10 as a default in case your service doesn't return the total}
      \DataTypeTok{totalPages}\NormalTok{: }\DecValTok{10}\NormalTok{,}

      \CommentTok{// The total number of pages to be shown as a pagination}
      \CommentTok{// list is calculated by (pagesInRange * 2) + 1.}
      \DataTypeTok{pagesInRange}\NormalTok{: }\DecValTok{4}
    \NormalTok{\},}
\end{Highlighting}
\end{Shaded}

\paragraph{4. Configure the parameters we want to send to the
server}\label{configure-the-parameters-we-want-to-send-to-the-server-1}

Only the base URL won't be enough for most cases, so you can pass more
parameters to the server. Note how you can use functions instead of
hardcoded values, and you can also refer to the values you specified in
\texttt{paginator\_ui}.

\begin{Shaded}
\begin{Highlighting}[]
    \NormalTok{server_api: \{}
      \CommentTok{// the query field in the request}
      \StringTok{'$filter'}\NormalTok{: }\StringTok{'substringof(}\CharTok{\textbackslash{}'}\StringTok{america}\CharTok{\textbackslash{}'}\StringTok{,Name)'}\NormalTok{,}

      \CommentTok{// number of items to return per request/page}
      \StringTok{'$top'}\NormalTok{: }\KeywordTok{function}\NormalTok{() \{ }\KeywordTok{return} \KeywordTok{this}\NormalTok{.}\FunctionTok{perPage} \NormalTok{\},}

      \CommentTok{// how many results the request should skip ahead to}
      \CommentTok{// customize as needed. For the Netflix API, skipping ahead based on}
      \CommentTok{// page * number of results per page was necessary.}
      \StringTok{'$skip'}\NormalTok{: }\KeywordTok{function}\NormalTok{() \{ }\KeywordTok{return} \KeywordTok{this}\NormalTok{.}\FunctionTok{currentPage} \NormalTok{* }\KeywordTok{this}\NormalTok{.}\FunctionTok{perPage} \NormalTok{\},}

      \CommentTok{// field to sort by}
      \StringTok{'$orderby'}\NormalTok{: }\StringTok{'ReleaseYear'}\NormalTok{,}

      \CommentTok{// what format would you like to request results in?}
      \StringTok{'$format'}\NormalTok{: }\StringTok{'json'}\NormalTok{,}

      \CommentTok{// custom parameters}
      \StringTok{'$inlinecount'}\NormalTok{: }\StringTok{'allpages'}\NormalTok{,}
      \StringTok{'$callback'}\NormalTok{: }\StringTok{'callback'}
    \NormalTok{\},}
\end{Highlighting}
\end{Shaded}

\paragraph{5. Finally, configure Collection.parse() and we're
done}\label{finally-configure-collection.parse-and-were-done-1}

And finally we have our \texttt{parse()} method, which in this case
isn't concerned with the total number of result pages available on the
server as we have our own total count of pages for the paginated data in
the UI.

\begin{Shaded}
\begin{Highlighting}[]
    \NormalTok{parse: }\KeywordTok{function} \NormalTok{(response) \{}
            \KeywordTok{var} \NormalTok{tags = }\OtherTok{response}\NormalTok{.}\OtherTok{d}\NormalTok{.}\FunctionTok{results}\NormalTok{;}
            \KeywordTok{return} \NormalTok{tags;}
        \NormalTok{\}}

    \NormalTok{\});}
\end{Highlighting}
\end{Shaded}

\paragraph{Convenience methods:}\label{convenience-methods-1}

As mentioned, your views can hook into a number of convenience methods
to navigate around UI-paginated data. For \texttt{clientPager} these
include:

\begin{itemize}
\itemsep1pt\parskip0pt\parsep0pt
\item
  \textbf{Collection.goTo(n, options)} - go to a specific page
\item
  \textbf{Collection.prevPage(options)} - go to the previous page
\item
  \textbf{Collection.nextPage(options)} - go to the next page
\item
  \textbf{Collection.howManyPer(n)} - set how many items to display per
  page
\item
  \textbf{Collection.setSort(sortBy, sortDirection)} - update sort on
  the current view. Sorting will automatically detect if you're trying
  to sort numbers (even if they're strored as strings) and will do the
  right thing.
\item
  \textbf{Collection.setFilter(filterFields, filterWords)} - filter the
  current view. Filtering supports multiple words without any specific
  order, so you'll basically get a full-text search ability. Also, you
  can pass it only one field from the model, or you can pass an array
  with fields and all of them will get filtered. Last option is to pass
  it an object containing a comparison method and rules. Currently, only
  \texttt{levenshtein} method is available.
\end{itemize}

The \texttt{goTo()}, \texttt{prevPage()}, and \texttt{nextPage()}
functions do not require the \texttt{options} param since they will be
executed synchronously. However, when specified, the success callback
will be invoked before the function returns. For example:

\begin{Shaded}
\begin{Highlighting}[]
\FunctionTok{nextPage}\NormalTok{(); }\CommentTok{// this works just fine!}
\FunctionTok{nextPage}\NormalTok{(\{}\DataTypeTok{success}\NormalTok{: }\KeywordTok{function}\NormalTok{() \{ \}\}); }\CommentTok{// this will call the success function}
\end{Highlighting}
\end{Shaded}

The options param exists to preserve (some) interface unification
between the requestPaginator and clientPaginator so that they may be
used interchangeably in your Backbone.Views.

\begin{Shaded}
\begin{Highlighting}[]
  \KeywordTok{this}\NormalTok{.}\OtherTok{collection}\NormalTok{.}\FunctionTok{setFilter}\NormalTok{(}
    \NormalTok{\{}\StringTok{'Name'}\NormalTok{: \{}\DataTypeTok{cmp_method}\NormalTok{: }\StringTok{'levenshtein'}\NormalTok{, }\DataTypeTok{max_distance}\NormalTok{: }\DecValTok{7}\NormalTok{\}\}}
    \NormalTok{, }\StringTok{"Amreican P"} \CommentTok{// Note the switched 'r' and 'e', and the 'P' from 'Pie'}
  \NormalTok{);}
\end{Highlighting}
\end{Shaded}

Also note that the Levenshtein plugin should be loaded and enabled using
the \texttt{useLevenshteinPlugin} variable. Last but not less important:
performing Levenshtein comparison returns the \texttt{distance} between
two strings. It won't let you \emph{search} lengthy text. The distance
between two strings means the number of characters that should be added,
removed or moved to the left or to the right so the strings get equal.
That means that comparing ``Something'' in ``This is a test that could
show something'' will return 32, which is bigger than comparing
``Something'' and ``ABCDEFG'' (9). Use Levenshtein only for short texts
(titles, names, etc).

\begin{itemize}
\item
  \textbf{Collection.doFakeFilter(filterFields, filterWords)} - returns
  the models count after fake-applying a call to
  \texttt{Collection.setFilter}.
\item
  \textbf{Collection.setFieldFilter(rules)} - filter each value of each
  model according to \texttt{rules} that you pass as argument. Example:
  You have a collection of books with `release year' and `author'. You
  can filter only the books that were released between 1999 and 2003.
  And then you can add another \texttt{rule} that will filter those
  books only to authors who's name start with `A'. Possible rules:
  function, required, min, max, range, minLength, maxLength,
  rangeLength, oneOf, equalTo, containsAllOf, pattern. Passing this an
  empty rules set will remove any FieldFilter rules applied.
\end{itemize}

\begin{Shaded}
\begin{Highlighting}[]

  \OtherTok{my_collection}\NormalTok{.}\FunctionTok{setFieldFilter}\NormalTok{([}
    \NormalTok{\{}\DataTypeTok{field}\NormalTok{: }\StringTok{'release_year'}\NormalTok{, }\DataTypeTok{type}\NormalTok{: }\StringTok{'range'}\NormalTok{, }\DataTypeTok{value}\NormalTok{: \{}\DataTypeTok{min}\NormalTok{: }\StringTok{'1999'}\NormalTok{, }\DataTypeTok{max}\NormalTok{: }\StringTok{'2003'}\NormalTok{\}\},}
    \NormalTok{\{}\DataTypeTok{field}\NormalTok{: }\StringTok{'author'}\NormalTok{, }\DataTypeTok{type}\NormalTok{: }\StringTok{'pattern'}\NormalTok{, }\DataTypeTok{value}\NormalTok{: }\KeywordTok{new} \FunctionTok{RegExp}\NormalTok{(}\StringTok{'A*'}\NormalTok{, }\StringTok{'igm'}\NormalTok{)\}}
  \NormalTok{]);}

  \CommentTok{//Rules:}
  \CommentTok{//}
  \CommentTok{//var my_var = 'green';}
  \CommentTok{//}
  \CommentTok{//\{field: 'color', type: 'equalTo', value: my_var\}}
  \CommentTok{//\{field: 'color', type: 'function', value: function(field_value)\{ return field_value == my_var; \} \}}
  \CommentTok{//\{field: 'color', type: 'required'\}}
  \CommentTok{//\{field: 'number_of_colors', type: 'min', value: '2'\}}
  \CommentTok{//\{field: 'number_of_colors', type: 'max', value: '4'\}}
  \CommentTok{//\{field: 'number_of_colors', type: 'range', value: \{min: '2', max: '4'\} \}}
  \CommentTok{//\{field: 'color_name', type: 'minLength', value: '4'\}}
  \CommentTok{//\{field: 'color_name', type: 'maxLength', value: '6'\}}
  \CommentTok{//\{field: 'color_name', type: 'rangeLength', value: \{min: '4', max: '6'\}\}}
  \CommentTok{//\{field: 'color_name', type: 'oneOf', value: ['green', 'yellow']\}}
  \CommentTok{//\{field: 'color_name', type: 'pattern', value: new RegExp('gre*', 'ig')\}}
  \CommentTok{//\{field: 'color_name', type: 'containsAllOf', value: ['green', 'yellow', 'blue']\}}
\end{Highlighting}
\end{Shaded}

\begin{itemize}
\itemsep1pt\parskip0pt\parsep0pt
\item
  \textbf{Collection.doFakeFieldFilter(rules)} - returns the models
  count after fake-applying a call to
  \texttt{Collection.setFieldFilter}.
\end{itemize}

\paragraph{Implementation notes:}\label{implementation-notes}

You can use some variables in your \texttt{View} to represent the actual
state of the paginator.

\begin{itemize}
\itemsep1pt\parskip0pt\parsep0pt
\item
  \texttt{totalUnfilteredRecords} - Contains the number of records,
  including all records filtered in any way. (Only available in
  \texttt{clientPager})
\item
  \texttt{totalRecords} - Contains the number of records
\item
  \texttt{currentPage} - The actual page were the paginator is at.
\item
  \texttt{perPage} - The number of records the paginator will show per
  page.
\item
  \texttt{totalPages} - The number of total pages.
\item
  \texttt{startRecord} - The position of the first record shown in the
  current page (eg 41 to 50 from 2000 records) (Only available in
  \texttt{clientPager})
\item
  \texttt{endRecord} - The position of the last record shown in the
  current page (eg 41 to 50 from 2000 records) (Only available in
  \texttt{clientPager})
\item
  \texttt{pagesInRange} - The number of pages to be drawn on each side
  of the current page. So if \texttt{pagesInRange} is 3 and
  \texttt{currentPage} is 13 you will get the numbers 10, 11, 12,
  13(selected), 14, 15, 16.
\end{itemize}

\begin{Shaded}
\begin{Highlighting}[]
\CommentTok{<!-- sample template for pagination UI -->}
\KeywordTok{<script}\OtherTok{ type=}\StringTok{"text/html"}\OtherTok{ id=}\StringTok{"tmpServerPagination"}\KeywordTok{>}

  \NormalTok{<div }\KeywordTok{class}\NormalTok{=}\StringTok{"row-fluid"}\NormalTok{>}

    \NormalTok{<div }\KeywordTok{class}\NormalTok{=}\StringTok{"pagination span8"}\NormalTok{>}
      \NormalTok{<ul>}
\ErrorTok{        <% _.each (pageSet, function (p) \{ %>}
        \NormalTok{<% }\KeywordTok{if} \NormalTok{(currentPage == p) \{ %>}
          \NormalTok{<li }\KeywordTok{class}\NormalTok{=}\StringTok{"active"}\NormalTok{><span><%= p %><}\OtherTok{/span></li}\NormalTok{>}
        \NormalTok{<% \} }\KeywordTok{else} \NormalTok{\{ %>}
          \NormalTok{<li><a href=}\StringTok{"#"} \KeywordTok{class}\NormalTok{=}\StringTok{"page"}\NormalTok{><%= p %><}\OtherTok{/a></li}\NormalTok{>}
        \NormalTok{<% \} %>}
        \NormalTok{<% \}); %>}
      \NormalTok{<}\OtherTok{/ul>}
\OtherTok{    </div}\NormalTok{>}

    \NormalTok{<div }\KeywordTok{class}\NormalTok{=}\StringTok{"pagination span4"}\NormalTok{>}
      \NormalTok{<ul>}
        \NormalTok{<% }\KeywordTok{if} \NormalTok{(currentPage > firstPage) \{ %>}
          \NormalTok{<li><a href=}\StringTok{"#"} \KeywordTok{class}\NormalTok{=}\StringTok{"serverprevious"}\NormalTok{>Previous<}\OtherTok{/a></li}\NormalTok{>}
        \NormalTok{<% \}}\KeywordTok{else}\NormalTok{\{ %>}
          \NormalTok{<li><span>Previous<}\OtherTok{/span></li}\NormalTok{>}
        \NormalTok{<% \}%>}
        \NormalTok{<% }\KeywordTok{if} \NormalTok{(currentPage < totalPages) \{ %>}
          \NormalTok{<li><a href=}\StringTok{"#"} \KeywordTok{class}\NormalTok{=}\StringTok{"servernext"}\NormalTok{>Next<}\OtherTok{/a></li}\NormalTok{>}
        \NormalTok{<% \} }\KeywordTok{else} \NormalTok{\{ %>}
          \NormalTok{<li><span>Next<}\OtherTok{/span></li}\NormalTok{>}
        \NormalTok{<% \} %>}
        \NormalTok{<% }\KeywordTok{if} \NormalTok{(firstPage != currentPage) \{ %>}
          \NormalTok{<li><a href=}\StringTok{"#"} \KeywordTok{class}\NormalTok{=}\StringTok{"serverfirst"}\NormalTok{>First<}\OtherTok{/a></li}\NormalTok{>}
        \NormalTok{<% \} }\KeywordTok{else} \NormalTok{\{ %>}
          \NormalTok{<li><span>First<}\OtherTok{/span></li}\NormalTok{>}
        \NormalTok{<% \} %>}
        \NormalTok{<% }\KeywordTok{if} \NormalTok{(totalPages != currentPage) \{ %>}
          \NormalTok{<li><a href=}\StringTok{"#"} \KeywordTok{class}\NormalTok{=}\StringTok{"serverlast"}\NormalTok{>Last<}\OtherTok{/a></li}\NormalTok{>}
        \NormalTok{<% \} }\KeywordTok{else} \NormalTok{\{ %>}
          \NormalTok{<li><span>Last<}\OtherTok{/span></li}\NormalTok{>}
        \NormalTok{<% \} %>}
      \NormalTok{<}\OtherTok{/ul>}
\OtherTok{    </div}\NormalTok{>}

  \NormalTok{<}\OtherTok{/div>}

\OtherTok{  <span class="cell serverhowmany"> Show <a href="#"}
\OtherTok{    class="selected">18</a}\NormalTok{> | <a href=}\StringTok{"#"} \KeywordTok{class}\NormalTok{=}\StringTok{""}\NormalTok{>}\DecValTok{9}\NormalTok{<}\OtherTok{/a> }\FloatTok{|}\OtherTok{ <a href="#" class="">12</a}\NormalTok{> per page}
  \NormalTok{<}\OtherTok{/span>}

\OtherTok{  <span class="divider">/}\NormalTok{<}\OtherTok{/span>}

\OtherTok{  <span class="cell first records">}
\OtherTok{    Page: <span class="label"><%= currentPage %></span}\NormalTok{> of <span }\KeywordTok{class}\NormalTok{=}\StringTok{"label"}\NormalTok{><%= totalPages %><}\OtherTok{/span> shown}
\OtherTok{  </span}\NormalTok{>}

\KeywordTok{</script>}
\end{Highlighting}
\end{Shaded}

\subsubsection{Plugins}\label{plugins}

\textbf{Diacritic.js}

A plugin for Backbone.Paginator that replaces diacritic characters
?? 
%(\texttt{´}, \texttt{˝}, \texttt{̏},
%\texttt{˚},\texttt{\textasciitilde{}} etc.) 
with characters that match
them most closely. This is particularly useful for filtering.

\begin{figure}[htbp]
\centering
\includegraphics{img/paginator-dia.png}
\end{figure}

To enable the plugin, set \texttt{this.useDiacriticsPlugin} to true, as
can be seen in the example below:

\begin{Shaded}
\begin{Highlighting}[]
\OtherTok{Paginator}\NormalTok{.}\FunctionTok{clientPager} \NormalTok{= }\OtherTok{Backbone}\NormalTok{.}\OtherTok{Collection}\NormalTok{.}\FunctionTok{extend}\NormalTok{(\{}

    \CommentTok{// Default values used when sorting and/or filtering.}
    \DataTypeTok{initialize}\NormalTok{: }\KeywordTok{function}\NormalTok{()\{}
      \KeywordTok{this}\NormalTok{.}\FunctionTok{useDiacriticsPlugin} \NormalTok{= }\KeywordTok{true}\NormalTok{; }\CommentTok{// use diacritics plugin if available}
    \NormalTok{...}
\end{Highlighting}
\end{Shaded}

\subsubsection{Bootstrapping}\label{bootstrapping}

By default, both the clientPager and requestPager will make an initial
request to the server in order to populate their internal paging data.
In order to avoid this additional request, it may be beneficial to
bootstrap your Backbone.Paginator instance from data that already exists
in the dom.

\textbf{Backbone.Paginator.clientPager:}

\begin{Shaded}
\begin{Highlighting}[]

\CommentTok{// Extend the Backbone.Paginator.clientPager with your own configuration options}
\KeywordTok{var} \NormalTok{MyClientPager =  }\OtherTok{Backbone}\NormalTok{.}\OtherTok{Paginator}\NormalTok{.}\OtherTok{clientPager}\NormalTok{.}\FunctionTok{extend}\NormalTok{(\{}\DataTypeTok{paginator_ui}\NormalTok{: \{\}\});}
\CommentTok{// Create an instance of your class and populate with the models of your entire collection}
\KeywordTok{var} \NormalTok{aClientPager = }\KeywordTok{new} \FunctionTok{MyClientPager}\NormalTok{([\{}\DataTypeTok{id}\NormalTok{: }\DecValTok{1}\NormalTok{, }\DataTypeTok{title}\NormalTok{: }\StringTok{'foo'}\NormalTok{\}, \{}\DataTypeTok{id}\NormalTok{: }\DecValTok{2}\NormalTok{, }\DataTypeTok{title}\NormalTok{: }\StringTok{'bar'}\NormalTok{\}]);}
\CommentTok{// Invoke the bootstrap function}
\OtherTok{aClientPager}\NormalTok{.}\FunctionTok{bootstrap}\NormalTok{();}
\end{Highlighting}
\end{Shaded}

Note: If you intend to bootstrap a clientPager, there is no need to
specify a `paginator\_core' object in your configuration (since you
should have already populated the clientPager with the entirety of its
necessary data)

\textbf{Backbone.Paginator.requestPager:}

\begin{Shaded}
\begin{Highlighting}[]

\CommentTok{// Extend the Backbone.Paginator.requestPager with your own configuration options}
\KeywordTok{var} \NormalTok{MyRequestPager =  }\OtherTok{Backbone}\NormalTok{.}\OtherTok{Paginator}\NormalTok{.}\OtherTok{requestPager}\NormalTok{.}\FunctionTok{extend}\NormalTok{(\{}\DataTypeTok{paginator_ui}\NormalTok{: \{\}\});}
\CommentTok{// Create an instance of your class with the first page of data}
\KeywordTok{var} \NormalTok{aRequestPager = }\KeywordTok{new} \FunctionTok{MyRequestPager}\NormalTok{([\{}\DataTypeTok{id}\NormalTok{: }\DecValTok{1}\NormalTok{, }\DataTypeTok{title}\NormalTok{: }\StringTok{'foo'}\NormalTok{\}, \{}\DataTypeTok{id}\NormalTok{: }\DecValTok{2}\NormalTok{, }\DataTypeTok{title}\NormalTok{: }\StringTok{'bar'}\NormalTok{\}]);}
\CommentTok{// Invoke the bootstrap function and configure requestPager with 'totalRecords'}
\OtherTok{aRequestPager}\NormalTok{.}\FunctionTok{bootstrap}\NormalTok{(\{}\DataTypeTok{totalRecords}\NormalTok{: }\DecValTok{50}\NormalTok{\});}
\end{Highlighting}
\end{Shaded}

Note: Both the clientPager and requestPager \texttt{bootstrap} function
will accept an options param that will be extended by your
Backbone.Paginator instance. However the `totalRecords' property will be
set implicitly by the clientPager.

\href{http://ricostacruz.com/backbone-patterns/\#bootstrapping_data}{More
on Backbone bootstrapping}

\subsubsection{Styling}\label{styling}

You're of course free to customize the overall look and feel of the
paginators as much as you wish. By default, all sample applications make
use of the \href{http://twitter.github.com/bootstrap}{Twitter Bootstrap}
for styling links, buttons and drop-downs.

CSS classes are available to style record counts, filters, sorting and
more:

\begin{figure}[htbp]
\centering
\includegraphics{img/paginator-styling2.png}
\end{figure}

Classes are also available for styling more granular elements like page
counts within \texttt{breadcrumb \textgreater{} pages} e.g
\texttt{.page}, \texttt{.page selected}:

\begin{figure}[htbp]
\centering
\includegraphics{img/paginator-classes.png}
\end{figure}

There's a tremendous amount of flexibility available for styling and as
you're in control of templating too, your paginators can be made to look
as visually simple or complex as needed.

\subsubsection{Conclusions}\label{conclusions}

Although it's certainly possible to write your own custom pagination
classes to work with Backbone Collections, Backbone.Paginator tries to
take care of much of this for you.

It's highly configurable, avoiding the need to write your own paging
when working with Collections of data sourced from your database or API.
Use the plugin to help tame large lists of data into more manageable,
easily navigatable, paginated lists.

Additionally, if you have any questions about Backbone.Paginator (or
would like to help improve it), feel free to post to the project
\href{https://github.com/addyosmani/backbone.paginator}{issues} list.

\section{Backbone Boilerplate And
Grunt-BBB}\label{backbone-boilerplate-and-grunt-bbb}

Boilerplates provide us a starting point for working on projects.
They're a base for building upon using the minimum required code to get
something functional put together. When you're working on a new Backbone
application, a new Model typically only takes a few lines of code to get
working.

That alone probably isn't enough however, as you'll need a Collection to
group those models, a View to render them and perhaps a router if you're
looking to making specific views of your Collection data bookmarkable.
If you're starting on a completely fresh project, you may also need a
build process in place to produce an optimized version of your app that
can be pushed to production.

This is where boilerplate solutions are useful. Rather than having to
manually write out the initial code for each piece of your Backbone app,
a boilerplate could do this for you, also ideally taking care of the
build process.

\href{https://github.com/tbranyen/backbone-boilerplate/}{Backbone
Boilerplate} (or just BB) provides just this. It is an excellent set of
best practices and utilities for building Backbone.js applications,
created by Backbone contributor \href{https://github.com/tbranyen}{Tim
Branyen}. He took the the gotchas, pitfalls and common tasks he ran into
while heavily using Backbone to build apps and crafted BB as a result of
this experience.

\href{https://github.com/backbone-boilerplate/grunt-bbb}{Grunt-BBB or
Boilerplate Build Buddy} is the companion tool to BB, which offers
scaffolding, file watcher and build capabilities. Used together with BB
it provides an excellent base for quickly starting new Backbone
applications.

\begin{figure}[htbp]
\centering
\includegraphics{img/bbb.png}
\end{figure}

Out of the box, BB and Grunt-BBB provide provide us with:

\begin{itemize}
\itemsep1pt\parskip0pt\parsep0pt
\item
  Backbone, \href{https://github.com/bestiejs/lodash}{Lodash} (an
  \href{http://underscorejs.org/}{Underscore.js} alternative) and
  \href{http://jquery.com}{jQuery} with an
  \href{http://html5boilerplate.com}{HTML5 Boilerplate} foundation
\item
  Boilerplate and scaffolding support, allowing us to spend minimal time
  writing boilerplate for modules, collections and so on.
\item
  A build tool for template pre-compilation and, concatenation \&
  minification of all our libraries, application code and stylesheets
\item
  A Lightweight node.js webserver
\end{itemize}

Notes on build tool steps:

\begin{itemize}
\itemsep1pt\parskip0pt\parsep0pt
\item
  Template pre-compilation: using a template library such as Underscore
  micro-templating or Handlebars.js generally involves three steps: (1)
  reading a raw template, (2) compiling it into a JavaScript function
  and (3) running the compiled template with your desired data.
  Precompiling eliminates the second step from runtime, by moving this
  process into a build step.
\item
  Concatenation is the process of combining a number of assets (in our
  case, script files) into a single (or fewer number) of files to reduce
  the number of HTTP requests required to obtain them.
\item
  Minification is the process of removing unnecessary characters (e.g
  white space, new lines, comments) from code and compressing it to
  reduce the overall size of the scripts being served.
\end{itemize}

\subsection{Getting Started}\label{getting-started}

\subsubsection{Backbone Boilerplate and
Grunt-BBB}\label{backbone-boilerplate-and-grunt-bbb-1}

To get started we're going to install Grunt-BBB, which will include
Backbone Boilerplate and any third-party dependencies it might need such
as the Grunt build tool.

We can install Grunt-bBB via NPM by running:

\begin{verbatim}
npm install -g bbb
\end{verbatim}

That's it. We should now be good to go.

A typical workflow for using grunt-bbb, which we will use later on is:

\begin{itemize}
\itemsep1pt\parskip0pt\parsep0pt
\item
  Initialize a new project (\texttt{bbb init})
\item
  Add new modules and templates (\texttt{bbb init:module})
\item
  Preview changes using the built in server (\texttt{bbb server})
\item
  Run the build tool (\texttt{bbb build})
\item
  Lint JavaScript, compile templates, build your application using r.js,
  minify CSS and JavaScript (using \texttt{bbb release})
\end{itemize}

\subsection{Creating a new project}\label{creating-a-new-project}

Let's create a new directory for our project and run \texttt{bbb init}
to kick things off. A number of project sub-directories and files will
be stubbed out for us, as shown below:

\begin{verbatim}
$ bbb init
Running "init" task
This task will create one or more files in the current directory, based on the
environment and the answers to a few questions. Note that answering "?" to any
question will show question-specific help and answering "none" to most questions
will leave its value blank.

"bbb" template notes:
This tool will help you install, configure, build, and maintain your Backbone
Boilerplate project.
Writing app/app.js...OK
Writing app/config.js...OK
Writing app/main.js...OK
Writing app/router.js...OK
Writing app/styles/index.css...OK
Writing favicon.ico...OK
Writing grunt.js...OK
Writing index.html...OK
Writing package.json...OK
Writing readme.md...OK
Writing test/jasmine/index.html...OK
Writing test/jasmine/spec/example.js...OK
Writing test/jasmine/vendor/jasmine-html.js...OK
Writing test/jasmine/vendor/jasmine.css...OK
Writing test/jasmine/vendor/jasmine.js...OK
Writing test/jasmine/vendor/jasmine_favicon.png...OK
Writing test/jasmine/vendor/MIT.LICENSE...OK
Writing test/qunit/index.html...OK
Writing test/qunit/tests/example.js...OK
Writing test/qunit/vendor/qunit.css...OK
Writing test/qunit/vendor/qunit.js...OK
Writing vendor/h5bp/css/main.css...OK
Writing vendor/h5bp/css/normalize.css...OK
Writing vendor/jam/backbone/backbone.js...OK
Writing vendor/jam/backbone/package.json...OK
Writing vendor/jam/backbone.layoutmanager/backbone.layoutmanager.js...OK
Writing vendor/jam/backbone.layoutmanager/package.json...OK
Writing vendor/jam/jquery/jquery.js...OK
Writing vendor/jam/jquery/package.json...OK
Writing vendor/jam/lodash/lodash.js...OK
Writing vendor/jam/lodash/lodash.min.js...OK
Writing vendor/jam/lodash/lodash.underscore.min.js...OK
Writing vendor/jam/lodash/package.json...OK
Writing vendor/jam/require.config.js...OK
Writing vendor/jam/require.js...OK
Writing vendor/js/libs/almond.js...OK
Writing vendor/js/libs/require.js...OK

Initialized from template "bbb".

Done, without errors.
\end{verbatim}

Let's review what has been generated.

\subsubsection{index.html}\label{index.html}

This is a fairly standard stripped-down HTML5 Boilerplate foundation
with the notable exception of including
\href{http://requirejs.org}{RequireJS} at the bottom of the page.

\begin{Shaded}
\begin{Highlighting}[]
\ErrorTok{<}\NormalTok{!doctype html>}
\KeywordTok{<html}\OtherTok{ lang=}\StringTok{"en"}\KeywordTok{>}
\KeywordTok{<head>}
  \KeywordTok{<meta}\OtherTok{ charset=}\StringTok{"utf-8"}\KeywordTok{>}
  \KeywordTok{<meta}\OtherTok{ http-equiv=}\StringTok{"X-UA-Compatible"}\OtherTok{ content=}\StringTok{"IE=edge,chrome=1"}\KeywordTok{>}
  \KeywordTok{<meta}\OtherTok{ name=}\StringTok{"viewport"}\OtherTok{ content=}\StringTok{"width=device-width,initial-scale=1"}\KeywordTok{>}

  \KeywordTok{<title>}\NormalTok{Backbone Boilerplate}\KeywordTok{</title>}

  \CommentTok{<!-- Application styles. -->}
  \CommentTok{<!--(if target dummy)><!-->}
  \KeywordTok{<link}\OtherTok{ rel=}\StringTok{"stylesheet"}\OtherTok{ href=}\StringTok{"/app/styles/index.css"}\KeywordTok{>}
  \CommentTok{<!--<!(endif)-->}
\KeywordTok{</head>}
\KeywordTok{<body>}
  \CommentTok{<!-- Application container. -->}
  \KeywordTok{<main}\OtherTok{ role=}\StringTok{"main"}\OtherTok{ id=}\StringTok{"main"}\KeywordTok{></main>}

  \CommentTok{<!-- Application source. -->}
  \CommentTok{<!--(if target dummy)><!-->}
  \KeywordTok{<script}\OtherTok{ data-main=}\StringTok{"/app/config"}\OtherTok{ src=}\StringTok{"/vendor/js/libs/require.js"}\KeywordTok{></script>}
  \CommentTok{<!--<!(endif)-->}

\KeywordTok{</body>}
\KeywordTok{</html>}
\end{Highlighting}
\end{Shaded}

RequireJS - the \href{https://github.com/amdjs/amdjs-api/wiki/AMD}{AMD}
(Asynchronous Module Definition) module and script loader - will assist
us with managing the modules in our application. We've already covered
it in the last chapter, but let's recap what this particular block does
in terms of the Boilerplate:

\begin{verbatim}
<script data-main="/app/config" src="/vendor/js/libs/require.js"></script>
\end{verbatim}

The \texttt{data-main} attribute is used to inform RequireJS to load
\texttt{app/config.js} (a configuration object) after it has finished
loading itself. You'll notice that we've omitted the \texttt{.js}
extension here as RequireJS can automatically add this for us, however
it will respect your paths if we do choose to include it regardless.
Let's now look at the config file being referenced.

\subsubsection{config.js}\label{config.js}

A RequireJS configuration object allows us to specify aliases and paths
for dependencies we're likely to reference often (e.g., jQuery),
bootstrap properties like our base application URL, and \texttt{shim}
libraries that don't support AMD natively.

This is what the config file in Backbone Boilerplate looks like:

\begin{Shaded}
\begin{Highlighting}[]
\CommentTok{// Set the require.js configuration for your application.}
\OtherTok{require}\NormalTok{.}\FunctionTok{config}\NormalTok{(\{}

  \CommentTok{// Initialize the application with the main application file and the JamJS}
  \CommentTok{// generated configuration file.}
  \DataTypeTok{deps}\NormalTok{: [}\StringTok{"../vendor/jam/require.config"}\NormalTok{, }\StringTok{"main"}\NormalTok{],}

  \DataTypeTok{paths}\NormalTok{: \{}
    \CommentTok{// Put paths here.}
  \NormalTok{\},}

  \DataTypeTok{shim}\NormalTok{: \{}
    \CommentTok{// Put shims here.}
  \NormalTok{\}}

\NormalTok{\});}
\end{Highlighting}
\end{Shaded}

The first option defined in the above config is
\texttt{deps: {[}"../vendor/jam/require.config", "main"{]}}. This
informs RequireJS to load up additional RequireJS configuration as well
a a main.js file, which is considered the entry point for our
application.

You may notice that we haven't specified any other path information for
\texttt{main}. Require will infer the default \texttt{baseUrl} using the
path from our \texttt{data-main} attribute in index.html. In other
words, our \texttt{baseUrl} is \texttt{app/} and any scripts we require
will be loaded relative to this location. We could use the
\texttt{baseUrl} option to override this default if we wanted to use a
different location.

The next block is \texttt{paths}, which we can use to specify paths
relative to the \texttt{baseUrl} as well as the paths/aliases to
dependencies we're likely to regularly reference.

After this comes \texttt{shim}, an important part of our RequireJS
configuration which allows us to load libraries which are not AMD
compliant. The basic idea here is that rather than requiring all
libraries to implement support for AMD, the \texttt{shim} takes care of
the hard work for us.

Going back to \texttt{deps}, the contents of our \texttt{require.config}
file can be seen below.

\begin{Shaded}
\begin{Highlighting}[]
\KeywordTok{var} \NormalTok{jam = \{}
    \StringTok{"packages"}\NormalTok{: [}
        \NormalTok{\{}
            \StringTok{"name"}\NormalTok{: }\StringTok{"backbone"}\NormalTok{,}
            \StringTok{"location"}\NormalTok{: }\StringTok{"../vendor/jam/backbone"}\NormalTok{,}
            \StringTok{"main"}\NormalTok{: }\StringTok{"backbone.js"}
        \NormalTok{\},}
        \NormalTok{\{}
            \StringTok{"name"}\NormalTok{: }\StringTok{"backbone.layoutmanager"}\NormalTok{,}
            \StringTok{"location"}\NormalTok{: }\StringTok{"../vendor/jam/backbone.layoutmanager"}\NormalTok{,}
            \StringTok{"main"}\NormalTok{: }\StringTok{"backbone.layoutmanager.js"}
        \NormalTok{\},}
        \NormalTok{\{}
            \StringTok{"name"}\NormalTok{: }\StringTok{"jquery"}\NormalTok{,}
            \StringTok{"location"}\NormalTok{: }\StringTok{"../vendor/jam/jquery"}\NormalTok{,}
            \StringTok{"main"}\NormalTok{: }\StringTok{"jquery.js"}
        \NormalTok{\},}
        \NormalTok{\{}
            \StringTok{"name"}\NormalTok{: }\StringTok{"lodash"}\NormalTok{,}
            \StringTok{"location"}\NormalTok{: }\StringTok{"../vendor/jam/lodash"}\NormalTok{,}
            \StringTok{"main"}\NormalTok{: }\StringTok{"./lodash.js"}
        \NormalTok{\}}
    \NormalTok{],}
    \StringTok{"version"}\NormalTok{: }\StringTok{"0.2.11"}\NormalTok{,}
    \StringTok{"shim"}\NormalTok{: \{}
        \StringTok{"backbone"}\NormalTok{: \{}
            \StringTok{"deps"}\NormalTok{: [}
                \StringTok{"jquery"}\NormalTok{,}
                \StringTok{"lodash"}
            \NormalTok{],}
            \StringTok{"exports"}\NormalTok{: }\StringTok{"Backbone"}
        \NormalTok{\},}
        \StringTok{"backbone.layoutmanager"}\NormalTok{: \{}
            \StringTok{"deps"}\NormalTok{: [}
                \StringTok{"jquery"}\NormalTok{,}
                \StringTok{"backbone"}\NormalTok{,}
                \StringTok{"lodash"}
            \NormalTok{],}
            \StringTok{"exports"}\NormalTok{: }\StringTok{"Backbone.LayoutManager"}
        \NormalTok{\}}
    \NormalTok{\}}
\NormalTok{\};}
\end{Highlighting}
\end{Shaded}

The \texttt{jam} object is to support configuration of
\href{http://jamjs.org/}{Jam} - a package manager for the front-end
which helps install, upgrade and configure the dependencies used by your
project. It is currently the package manager of choice for Backbone
Boilerplate.

Under the \texttt{packages} array, a number of dependencies are
specified for inclusion, such as Backbone, the Backbone.LayoutManager
plugin, jQuery and Lo-dash.

For those curious about
\href{https://github.com/tbranyen/backbone.layoutmanager}{Backbone.LayoutManager},
it's a Backbone plugin that provides a foundation for assembling layouts
and views within Backbone.

Additional packages you install using Jam will have a corresponding
entry added to \texttt{packages}.

\subsubsection{main.js}\label{main.js}

Next, we have \texttt{main.js}, which defines the entry point for our
application. We use a global \texttt{require()} method to load an array
containing any other scripts needed, such as our application
\texttt{app.js} and our main router \texttt{router.js}. Note that most
of the time, we will only use \texttt{require()} for bootstrapping an
application and a similar method called \texttt{define()} for all other
purposes.

The function defined after our array of dependencies is a callback which
doesn't fire until these scripts have loaded. Notice how we're able to
locally alias references to ``app'' and ``router'' as \texttt{app} and
\texttt{Router} for convenience.

\begin{Shaded}
\begin{Highlighting}[]
\FunctionTok{require}\NormalTok{([}
  \CommentTok{// Application.}
  \StringTok{"app"}\NormalTok{,}

  \CommentTok{// Main Router.}
  \StringTok{"router"}
\NormalTok{],}

\KeywordTok{function}\NormalTok{(app, Router) \{}

  \CommentTok{// Define your master router on the application namespace and trigger all}
  \CommentTok{// navigation from this instance.}
  \OtherTok{app}\NormalTok{.}\FunctionTok{router} \NormalTok{= }\KeywordTok{new} \FunctionTok{Router}\NormalTok{();}

  \CommentTok{// Trigger the initial route and enable HTML5 History API support, set the}
  \CommentTok{// root folder to '/' by default.  Change in app.js.}
  \OtherTok{Backbone}\NormalTok{.}\OtherTok{history}\NormalTok{.}\FunctionTok{start}\NormalTok{(\{ }\DataTypeTok{pushState}\NormalTok{: }\KeywordTok{true}\NormalTok{, }\DataTypeTok{root}\NormalTok{: }\OtherTok{app}\NormalTok{.}\FunctionTok{root} \NormalTok{\});}

  \CommentTok{// All navigation that is relative should be passed through the navigate}
  \CommentTok{// method, to be processed by the router. If the link has a `data-bypass`}
  \CommentTok{// attribute, bypass the delegation completely.}
  \FunctionTok{$}\NormalTok{(document).}\FunctionTok{on}\NormalTok{(}\StringTok{"click"}\NormalTok{, }\StringTok{"a[href]:not([data-bypass])"}\NormalTok{, }\KeywordTok{function}\NormalTok{(evt) \{}
    \CommentTok{// Get the absolute anchor href.}
    \KeywordTok{var} \NormalTok{href = \{ }\DataTypeTok{prop}\NormalTok{: }\FunctionTok{$}\NormalTok{(}\KeywordTok{this}\NormalTok{).}\FunctionTok{prop}\NormalTok{(}\StringTok{"href"}\NormalTok{), }\DataTypeTok{attr}\NormalTok{: }\FunctionTok{$}\NormalTok{(}\KeywordTok{this}\NormalTok{).}\FunctionTok{attr}\NormalTok{(}\StringTok{"href"}\NormalTok{) \};}
    \CommentTok{// Get the absolute root.}
    \KeywordTok{var} \NormalTok{root = }\OtherTok{location}\NormalTok{.}\FunctionTok{protocol} \NormalTok{+ }\StringTok{"//"} \NormalTok{+ }\OtherTok{location}\NormalTok{.}\FunctionTok{host} \NormalTok{+ }\OtherTok{app}\NormalTok{.}\FunctionTok{root}\NormalTok{;}

    \CommentTok{// Ensure the root is part of the anchor href, meaning it's relative.}
    \KeywordTok{if} \NormalTok{(}\OtherTok{href}\NormalTok{.}\OtherTok{prop}\NormalTok{.}\FunctionTok{slice}\NormalTok{(}\DecValTok{0}\NormalTok{, }\OtherTok{root}\NormalTok{.}\FunctionTok{length}\NormalTok{) === root) \{}
      \CommentTok{// Stop the default event to ensure the link will not cause a page}
      \CommentTok{// refresh.}
      \OtherTok{evt}\NormalTok{.}\FunctionTok{preventDefault}\NormalTok{();}

      \CommentTok{// `Backbone.history.navigate` is sufficient for all Routers and will}
      \CommentTok{// trigger the correct events. The Router's internal `navigate` method}
      \CommentTok{// calls this anyways.  The fragment is sliced from the root.}
      \OtherTok{Backbone}\NormalTok{.}\OtherTok{history}\NormalTok{.}\FunctionTok{navigate}\NormalTok{(}\OtherTok{href}\NormalTok{.}\FunctionTok{attr}\NormalTok{, }\KeywordTok{true}\NormalTok{);}
    \NormalTok{\}}
  \NormalTok{\});}

\NormalTok{\});}
\end{Highlighting}
\end{Shaded}

Inline, Backbone Boilerplate includes boilerplate code for initializing
our router with HTML5 History API support and handling other navigation
scenarios, so we don't have to.

\subsubsection{app.js}\label{app.js}

Let us now look at our \texttt{app.js} module. Typically, in
non-Backbone Boilerplate applications, an \texttt{app.js} file may
contain the core logic or module references needed to kick start an app.

In this case however, this file is used to define templating and layout
configuration options as well as utilities for consuming layouts. To a
beginner, this might look like a lot of code to comprehend, but the good
news is that for basic apps, you're unlikely to need to heavily modify
this. Instead, you'll be more concerned with modules for your app, which
we'll look at next.

\begin{Shaded}
\begin{Highlighting}[]
\FunctionTok{define}\NormalTok{([}
  \StringTok{"backbone.layoutmanager"}
\NormalTok{], }\KeywordTok{function}\NormalTok{() \{}

  \CommentTok{// Provide a global location to place configuration settings and module}
  \CommentTok{// creation.}
  \KeywordTok{var} \NormalTok{app = \{}
    \CommentTok{// The root path to run the application.}
    \DataTypeTok{root}\NormalTok{: }\StringTok{"/"}
  \NormalTok{\};}

  \CommentTok{// Localize or create a new JavaScript Template object.}
  \KeywordTok{var} \NormalTok{JST = }\OtherTok{window}\NormalTok{.}\FunctionTok{JST} \NormalTok{= }\OtherTok{window}\NormalTok{.}\FunctionTok{JST} \NormalTok{|| \{\};}

  \CommentTok{// Configure LayoutManager with Backbone Boilerplate defaults.}
  \OtherTok{Backbone}\NormalTok{.}\OtherTok{LayoutManager}\NormalTok{.}\FunctionTok{configure}\NormalTok{(\{}
    \CommentTok{// Allow LayoutManager to augment Backbone.View.prototype.}
    \DataTypeTok{manage}\NormalTok{: }\KeywordTok{true}\NormalTok{,}

    \DataTypeTok{prefix}\NormalTok{: }\StringTok{"app/templates/"}\NormalTok{,}

    \DataTypeTok{fetch}\NormalTok{: }\KeywordTok{function}\NormalTok{(path) \{}
      \CommentTok{// Concatenate the file extension.}
      \NormalTok{path = path + }\StringTok{".html"}\NormalTok{;}

      \CommentTok{// If cached, use the compiled template.}
      \KeywordTok{if} \NormalTok{(JST[path]) \{}
        \KeywordTok{return} \NormalTok{JST[path];}
      \NormalTok{\}}

      \CommentTok{// Put fetch into `async-mode`.}
      \KeywordTok{var} \NormalTok{done = }\KeywordTok{this}\NormalTok{.}\FunctionTok{async}\NormalTok{();}

      \CommentTok{// Seek out the template asynchronously.}
      \OtherTok{$}\NormalTok{.}\FunctionTok{get}\NormalTok{(}\OtherTok{app}\NormalTok{.}\FunctionTok{root} \NormalTok{+ path, }\KeywordTok{function}\NormalTok{(contents) \{}
        \FunctionTok{done}\NormalTok{(JST[path] = }\OtherTok{_}\NormalTok{.}\FunctionTok{template}\NormalTok{(contents));}
      \NormalTok{\});}
    \NormalTok{\}}
  \NormalTok{\});}

  \CommentTok{// Mix Backbone.Events, modules, and layout management into the app object.}
  \KeywordTok{return} \OtherTok{_}\NormalTok{.}\FunctionTok{extend}\NormalTok{(app, \{}
    \CommentTok{// Create a custom object with a nested Views object.}
    \DataTypeTok{module}\NormalTok{: }\KeywordTok{function}\NormalTok{(additionalProps) \{}
      \KeywordTok{return} \OtherTok{_}\NormalTok{.}\FunctionTok{extend}\NormalTok{(\{ }\DataTypeTok{Views}\NormalTok{: \{\} \}, additionalProps);}
    \NormalTok{\},}

    \CommentTok{// Helper for using layouts.}
    \DataTypeTok{useLayout}\NormalTok{: }\KeywordTok{function}\NormalTok{(name, options) \{}
      \CommentTok{// Enable variable arity by allowing the first argument to be the options}
      \CommentTok{// object and omitting the name argument.}
      \KeywordTok{if} \NormalTok{(}\OtherTok{_}\NormalTok{.}\FunctionTok{isObject}\NormalTok{(name)) \{}
        \NormalTok{options = name;}
      \NormalTok{\}}

      \CommentTok{// Ensure options is an object.}
      \NormalTok{options = options || \{\};}

      \CommentTok{// If a name property was specified use that as the template.}
      \KeywordTok{if} \NormalTok{(}\OtherTok{_}\NormalTok{.}\FunctionTok{isString}\NormalTok{(name)) \{}
        \OtherTok{options}\NormalTok{.}\FunctionTok{template} \NormalTok{= name;}
      \NormalTok{\}}

      \CommentTok{// Create a new Layout with options.}
      \KeywordTok{var} \NormalTok{layout = }\KeywordTok{new} \OtherTok{Backbone}\NormalTok{.}\FunctionTok{Layout}\NormalTok{(}\OtherTok{_}\NormalTok{.}\FunctionTok{extend}\NormalTok{(\{}
        \DataTypeTok{el}\NormalTok{: }\StringTok{"#main"}
      \NormalTok{\}, options));}

      \CommentTok{// Cache the reference.}
      \KeywordTok{return} \KeywordTok{this}\NormalTok{.}\FunctionTok{layout} \NormalTok{= layout;}
    \NormalTok{\}}
  \NormalTok{\}, }\OtherTok{Backbone}\NormalTok{.}\FunctionTok{Events}\NormalTok{);}

\NormalTok{\});}
\end{Highlighting}
\end{Shaded}

Note: JST stands for JavaScript templates and generally refers to
templates which have been (or will be) precompiled as part of a build
step. When running \texttt{bbb release} or \texttt{bbb debug},
Underscore/Lo-dash templates will be precompiled to avoid the need to
compile them at runtime within the browser.

\subsubsection{Creating Backbone Boilerplate
Modules}\label{creating-backbone-boilerplate-modules}

Not to be confused with simply being just an AMD module, a Backbone
Boilerplate \texttt{module} is a script composed of a:

\begin{itemize}
\itemsep1pt\parskip0pt\parsep0pt
\item
  Model
\item
  Collection
\item
  Views (optional)
\end{itemize}

We can easily create a new Boilerplate module using \texttt{grunt-bbb}
once again using \texttt{init}:

\begin{verbatim}
# Create a new module
$ bbb init:module

# Grunt prompt
Please answer the following:
[?] Module Name foo
[?] Do you need to make any changes to the above before continuing? (y/N)

Writing app/modules/foo.js...OK
Writing app/styles/foo.styl...OK
Writing app/templates/foo.html...OK

Initialized from template "module".

Done, without errors.
\end{verbatim}

This will generate a module \texttt{foo.js} as follows:

\begin{Shaded}
\begin{Highlighting}[]
\CommentTok{// Foo module}
\FunctionTok{define}\NormalTok{([}
  \CommentTok{// Application.}
  \StringTok{"app"}
\NormalTok{],}

\CommentTok{// Map dependencies from above array.}
\KeywordTok{function}\NormalTok{(app) \{}

  \CommentTok{// Create a new module.}
  \KeywordTok{var} \NormalTok{Foo = }\OtherTok{app}\NormalTok{.}\FunctionTok{module}\NormalTok{();}

  \CommentTok{// Default Model.}
  \OtherTok{Foo}\NormalTok{.}\FunctionTok{Model} \NormalTok{= }\OtherTok{Backbone}\NormalTok{.}\OtherTok{Model}\NormalTok{.}\FunctionTok{extend}\NormalTok{(\{}

  \NormalTok{\});}

  \CommentTok{// Default Collection.}
  \OtherTok{Foo}\NormalTok{.}\FunctionTok{Collection} \NormalTok{= }\OtherTok{Backbone}\NormalTok{.}\OtherTok{Collection}\NormalTok{.}\FunctionTok{extend}\NormalTok{(\{}
    \DataTypeTok{model}\NormalTok{: }\OtherTok{Foo}\NormalTok{.}\FunctionTok{Model}
  \NormalTok{\});}

  \CommentTok{// Default View.}
  \OtherTok{Foo}\NormalTok{.}\OtherTok{Views}\NormalTok{.}\FunctionTok{Layout} \NormalTok{= }\OtherTok{Backbone}\NormalTok{.}\OtherTok{Layout}\NormalTok{.}\FunctionTok{extend}\NormalTok{(\{}
    \DataTypeTok{template}\NormalTok{: }\StringTok{"foo"}
  \NormalTok{\});}

  \CommentTok{// Return the module for AMD compliance.}
  \KeywordTok{return} \NormalTok{Foo;}

\NormalTok{\});}
\end{Highlighting}
\end{Shaded}

Notice how boilerplate code for a model, collection and view have been
scaffolded out for us.

Optionally, we may also wish to include references to plugins such as
the Backbone LocalStorage or Offline adapters. One clean way of
including a plugin in the above boilerplate could be:

\begin{Shaded}
\begin{Highlighting}[]
\CommentTok{// Foo module}
\FunctionTok{define}\NormalTok{([}
  \CommentTok{// Application.}
  \StringTok{"app"}\NormalTok{,}
  \CommentTok{// Plugins}
  \StringTok{'plugins/backbone-localstorage'}
\NormalTok{],}

\CommentTok{// Map dependencies from above array.}
\KeywordTok{function}\NormalTok{(app) \{}

  \CommentTok{// Create a new module.}
  \KeywordTok{var} \NormalTok{Foo = }\OtherTok{app}\NormalTok{.}\FunctionTok{module}\NormalTok{();}

  \CommentTok{// Default Model.}
  \OtherTok{Foo}\NormalTok{.}\FunctionTok{Model} \NormalTok{= }\OtherTok{Backbone}\NormalTok{.}\OtherTok{Model}\NormalTok{.}\FunctionTok{extend}\NormalTok{(\{}
    \CommentTok{// Save all of the items under the `"foo"` namespace.}
    \DataTypeTok{localStorage}\NormalTok{: }\KeywordTok{new} \FunctionTok{Store}\NormalTok{(}\StringTok{'foo-backbone'}\NormalTok{),}
  \NormalTok{\});}

  \CommentTok{// Default Collection.}
  \OtherTok{Foo}\NormalTok{.}\FunctionTok{Collection} \NormalTok{= }\OtherTok{Backbone}\NormalTok{.}\OtherTok{Collection}\NormalTok{.}\FunctionTok{extend}\NormalTok{(\{}
    \DataTypeTok{model}\NormalTok{: }\OtherTok{Foo}\NormalTok{.}\FunctionTok{Model}
  \NormalTok{\});}

  \CommentTok{// Default View.}
  \OtherTok{Foo}\NormalTok{.}\OtherTok{Views}\NormalTok{.}\FunctionTok{Layout} \NormalTok{= }\OtherTok{Backbone}\NormalTok{.}\OtherTok{Layout}\NormalTok{.}\FunctionTok{extend}\NormalTok{(\{}
    \DataTypeTok{template}\NormalTok{: }\StringTok{"foo"}
  \NormalTok{\});}

  \CommentTok{// Return the module for AMD compliance.}
  \KeywordTok{return} \NormalTok{Foo;}

\NormalTok{\});}
\end{Highlighting}
\end{Shaded}

\subsubsection{router.js}\label{router.js}

Finally, let's look at our application router which is used for handling
navigation. The default router Backbone Boilerplate generates for us
includes sane defaults out of the box and can be easily extended.

\begin{Shaded}
\begin{Highlighting}[]
\FunctionTok{define}\NormalTok{([}
  \CommentTok{// Application.}
  \StringTok{"app"}
\NormalTok{],}

\KeywordTok{function}\NormalTok{(app) \{}

  \CommentTok{// Defining the application router, you can attach sub routers here.}
  \KeywordTok{var} \NormalTok{Router = }\OtherTok{Backbone}\NormalTok{.}\OtherTok{Router}\NormalTok{.}\FunctionTok{extend}\NormalTok{(\{}
    \DataTypeTok{routes}\NormalTok{: \{}
      \StringTok{""}\NormalTok{: }\StringTok{"index"}
    \NormalTok{\},}

    \DataTypeTok{index}\NormalTok{: }\KeywordTok{function}\NormalTok{() \{}

    \NormalTok{\}}
  \NormalTok{\});}

  \KeywordTok{return} \NormalTok{Router;}

\NormalTok{\});}
\end{Highlighting}
\end{Shaded}

If however we would like to execute some module-specific logic, when the
page loads (i.e when a user hits the default route), we can pull in a
module as a dependency and optionally use the Backbone LayoutManager to
attach Views to our layout as follows:

\begin{Shaded}
\begin{Highlighting}[]
\FunctionTok{define}\NormalTok{([}
  \CommentTok{// Application.}
  \StringTok{'app'}\NormalTok{,}

  \CommentTok{// Modules}
  \StringTok{'modules/foo'}
\NormalTok{],}

\KeywordTok{function}\NormalTok{(app, Foo) \{}

  \CommentTok{// Defining the application router, you can attach sub routers here.}
  \KeywordTok{var} \NormalTok{Router = }\OtherTok{Backbone}\NormalTok{.}\OtherTok{Router}\NormalTok{.}\FunctionTok{extend}\NormalTok{(\{}
    \DataTypeTok{routes}\NormalTok{: \{}
      \StringTok{''}\NormalTok{: }\StringTok{'index'}
    \NormalTok{\},}

    \DataTypeTok{index}\NormalTok{: }\KeywordTok{function}\NormalTok{() \{}
            \CommentTok{// Create a new Collection}
            \KeywordTok{var} \NormalTok{collection = }\KeywordTok{new} \OtherTok{Foo}\NormalTok{.}\FunctionTok{Collection}\NormalTok{();}

            \CommentTok{// Use and configure a 'main' layout}
            \OtherTok{app}\NormalTok{.}\FunctionTok{useLayout}\NormalTok{(}\StringTok{'main'}\NormalTok{).}\FunctionTok{setViews}\NormalTok{(\{}
                    \CommentTok{// Attach the bar View into the content View}
                    \StringTok{'.bar'}\NormalTok{: }\KeywordTok{new} \OtherTok{Foo}\NormalTok{.}\OtherTok{Views}\NormalTok{.}\FunctionTok{Bar}\NormalTok{(\{}
                            \DataTypeTok{collection}\NormalTok{: collection}
                    \NormalTok{\})}
             \NormalTok{\}).}\FunctionTok{render}\NormalTok{();}
    \NormalTok{\}}
  \NormalTok{\});}

  \CommentTok{// Fetch data (e.g., from localStorage)}
  \OtherTok{collection}\NormalTok{.}\FunctionTok{fetch}\NormalTok{();}

  \KeywordTok{return} \NormalTok{Router;}

\NormalTok{\});}
\end{Highlighting}
\end{Shaded}

\subsection{Other Useful Tools \&
Projects}\label{other-useful-tools-projects}

When working with Backbone, you usually need to write a number of
different classes and files for your application. Scaffolding tools such
as Grunt-BBB can help automate this process by generating basic
boilerplates for the files you need for you.

\subsubsection{Yeoman}\label{yeoman}

If you appreciated Grunt-BBB but would like to explore a tool for
assisting with your broader development workflow, I'm happy to recommend
a tool I've been helping with called \href{http://yeoman.io}{Yeoman}.

\begin{figure}[htbp]
\centering
\includegraphics{img/yeoman.png}
\end{figure}

Yeoman is a workflow comprised of a collection of tools and best
practices for helping you develop more efficiently. It's comprised of yo
(a scaffolding tool), \href{http://gruntjs.com}{Grunt}(a build tool) and
\href{http://bower.io}{Bower} (a client-side package manager).

Where Grunt-BBB focuses on offering an opionated start for Backbone
projects, Yeoman allows you to scaffold apps using Backbone (or other
libraries and frameworks), get Backbone plugins directly from the
command-line and compile your CoffeeScript, Sass or other abstractions
without additional effort.

\begin{figure}[htbp]
\centering
\includegraphics{img/bower.png}
\end{figure}

You may also be interested in \href{http://brunch.io/}{Brunch}, a
similar project which uses skeleton boilerplates to generate new
applications.

\subsubsection{Backbone DevTools}\label{backbone-devtools}

When building an application with Backbone, there's some additional
tooling available for your day-to-day debugging workflow.

Backbone DevTools was created to help with this and is a Chrome DevTools
extension allowing you to inspect events, syncs, View-DOM bindings and
what objects have been instantiated.

A useful View hierarchy is displayed in the Elements panel. Also, when
you inspect a DOM element the closest View will be exposed via \$view in
the console.

\begin{figure}[htbp]
\centering
\includegraphics{img/bbdevtools.jpg}
\end{figure}

At the time of writing, the project is currently available on
\href{https://github.com/spect88/backbone-devtools}{GitHub}.

\subsection{Conclusions}\label{conclusions-1}

In this section we reviewed Backbone Boilerplate and learned how to use
the \texttt{bbb} tool to help us scaffold out our application.

If you would like to learn more about how this project helps structure
your app, BBB includes some built-in boilerplate sample apps that can be
easily generated for review.

These include a boilerplate tutorial project
(\texttt{bbb init:tutorial}) and an implementation of my
\href{http://todomvc}{TodoMVC} project (\texttt{bbb init:todomvc}). I
recommend checking these out as they'll provide you with a more complete
picture of how Backbone Boilerplate, its templates, and so on fit into
the overall setup for a web app.

For more about Grunt-BBB, remember to take a look at the official
project
\href{https://github.com/backbone-boilerplate/grunt-bbb}{repository}.
There is also a related
\href{https://dl.dropbox.com/u/79007/talks/Modern_Web_Applications/slides/index.html}{slide-deck}
available for those interested in reading more.

\section{Backbone \& jQuery Mobile}\label{backbone-jquery-mobile}

\subsubsection{Mobile app development with jQuery
Mobile}\label{mobile-app-development-with-jquery-mobile}

The mobile web is huge and it is continuing to grow at an impressive
rate. Along with the massive growth of the mobile internet comes a
striking diversity of devices and browsers. As a result, making your
applications cross-platform and mobile-ready is both important and
challenging. Creating native apps is expensive. It is very costly in
terms of time and it usually requires varied experiences in programming
languages like Objective C , C\#, Java and JavaScript to support
multiple runtime environments.

HTML, CSS, and JavaScript enable you to build a single application
targeting a common runtime environment: the browser. This approach
supports a broad range of mobile devices such as tablets, smartphones,
and notebooks along with traditional PCs.

The challenging task is not only to adapt contents like text and
pictures properly to various screen resolutions but also to have same
user experience across native apps under different operating systems.
Like jQueryUI, jQuery Mobile is a user interface framework based on
jQuery that works across all popular phone, tablet, e-Reader, and
desktop platforms. It is built with accessibility and universal access
in mind.

The main idea of the framework is to enable anyone to create a mobile
app using only HTML. Knowledge of a programming language is not required
and there is no need to write complex, device specific CSS. For this
reason jQMobile follows two main principles we first need to understand
in order to integrate the framework to Backbone: \emph{progressive
enhancement} and \emph{responsive web design}.

\paragraph{The Principle of progressive widget enhancement by
jQMobile}\label{the-principle-of-progressive-widget-enhancement-by-jqmobile}

JQuery Mobile follows progressive enhancement and responsive web design
principles using HTML-5 markup-driven definitions and configurations.

A page in jQuery Mobile consists of an element with a
\texttt{data-role="page"} attribute. Within the \texttt{page} container,
any valid HTML markup can be used, but for typical pages in jQM, the
immediate children are divs with \texttt{data-role="header"},
\texttt{data-role="content"}, and \texttt{data-role="footer"}. The
baseline requirement for a page is only a page wrapper to support the
navigation system, the rest is optional.

An initial HTML page looks like this:

\begin{Shaded}
\begin{Highlighting}[]
\DataTypeTok{<!DOCTYPE }\NormalTok{html}\DataTypeTok{>}
\KeywordTok{<html>}
\KeywordTok{<head>}
    \KeywordTok{<title>}\NormalTok{Page Title}\KeywordTok{</title>}

    \KeywordTok{<meta}\OtherTok{ name=}\StringTok{"viewport"}\OtherTok{ content=}\StringTok{"width=device-width, initial-scale=1"}\KeywordTok{>}

    \KeywordTok{<link}\OtherTok{ rel=}\StringTok{"stylesheet"}\OtherTok{ href=}\StringTok{"http://code.jquery.com/mobile/1.3.0/jquery.mobile-1.3.0.min.css"} \KeywordTok{/>}
    \KeywordTok{<script}\OtherTok{ src=}\StringTok{"http://code.jquery.com/jquery-1.9.1.min.js"}\KeywordTok{></script>}
    \KeywordTok{<script}\OtherTok{ src=}\StringTok{"http://code.jquery.com/mobile/1.3.0/jquery.mobile-1.3.0.min.js"}\KeywordTok{></script>}
\KeywordTok{</head>}
\KeywordTok{<body>}

\KeywordTok{<div}\OtherTok{ data-role=}\StringTok{"page"}\KeywordTok{>}
  \KeywordTok{<div}\OtherTok{ data-role=}\StringTok{"header"}\KeywordTok{>}
    \KeywordTok{<h1>}\NormalTok{Page Title}\KeywordTok{</h1>}
  \KeywordTok{</div>}
  \KeywordTok{<div}\OtherTok{ data-role=}\StringTok{"content"}\KeywordTok{>}
     \KeywordTok{<p>}\NormalTok{Page content goes here.}\KeywordTok{</p>}
     \KeywordTok{<form>}
       \KeywordTok{<label}\OtherTok{ for=}\StringTok{"slider-1"}\KeywordTok{>}\NormalTok{Slider with tooltip:}\KeywordTok{</label>}
       \KeywordTok{<input}\OtherTok{ type=}\StringTok{"range"}\OtherTok{ name=}\StringTok{"slider-1"}\OtherTok{ id=}\StringTok{"slider-1"}\OtherTok{ min=}\StringTok{"0"}\OtherTok{ max=}\StringTok{"100"}\OtherTok{ value=}\StringTok{"50"} 
\OtherTok{        data-popup-enabled=}\StringTok{"true"}\KeywordTok{>}
     \KeywordTok{</form>}
  \KeywordTok{</div>}
  \KeywordTok{<div}\OtherTok{ data-role=}\StringTok{"footer"}\KeywordTok{>}
     \KeywordTok{<h4>}\NormalTok{Page Footer}\KeywordTok{</h4>}
  \KeywordTok{</div>}
\KeywordTok{</div>}
\KeywordTok{</body>}
\KeywordTok{</html>}
\end{Highlighting}
\end{Shaded}

\emph{Example HTML setup of a basic jQuery Mobile page}

JQuery Mobile will transform the written HTML definition to the rendered
HTML and CSS using its Progressive Widget Enhancement API. It also
executes JavaScript which is conditioned by configurations, attribute
properties, and runtime specific settings.

This implies: Whenever HTML content is added or changed, it needs to be
handled by the progressive widget enhancement of jQuery Mobile.

\begin{figure}[htbp]
\centering
\includegraphics{img/chapter10-1-1-1.png}
\end{figure}

\emph{Comparison of the user interface of the default HTML to the jQuery
Mobile enhanced version}

\paragraph{Understanding jQuery Mobile
Navigation}\label{understanding-jquery-mobile-navigation}

The jQuery Mobile navigation system controls its application's lifecycle
by automatically ``hijacking'' standard links and form submissions and
turning them into AJAX requests. Whenever a link is clicked or a form is
submitted, that event is automatically intercepted and used to issue an
AJAX request based on the href or form action instead of reloading the
page.

When the page document is requested, jQuery Mobile searches the document
for all elements with the \texttt{data-role="page"} attribute, parses
its contents, and inserts that code into the DOM of the original page.
Once the new page is prepared, jQuery Mobile's JavaScript triggers a
transition that shows the new page and hides the HTML of the previous
page in the DOM.

Next, any widgets in the incoming page are enhanced to apply all the
styles and behavior. The rest of the incoming page is discarded so any
scripts, stylesheets, or other information will not be included.

Via the \emph{multi-page templating feature}, you can add as many pages
as you want to the same HTML file within the body tag by defining divs
with \texttt{data-role="page"} or \texttt{data-role="dialog"} attributes
along with an \texttt{id} which can be used in links (preceded by a
hashbang):

\begin{Shaded}
\begin{Highlighting}[]
\KeywordTok{<html>}
  \KeywordTok{<head>}\NormalTok{...}\KeywordTok{</head>}
  \KeywordTok{<body>}
  \NormalTok{...}
  \KeywordTok{<div}\OtherTok{ data-role=}\StringTok{"page"}\OtherTok{ id=}\StringTok{"firstpage"}\KeywordTok{>}
    \NormalTok{...}
   \KeywordTok{<div}\OtherTok{ data-role=}\StringTok{"content"}\KeywordTok{>} 
     \KeywordTok{<a}\OtherTok{ href=}\StringTok{"#secondpage"}\KeywordTok{>}\NormalTok{go to secondpage}\KeywordTok{</a>}
   \KeywordTok{</div>}
  \KeywordTok{</div>}
  \KeywordTok{<div}\OtherTok{ data-role=}\StringTok{"page"}\OtherTok{ id=}\StringTok{"secondpage"}\KeywordTok{>}
    \NormalTok{...}
    \KeywordTok{<div}\OtherTok{ data-role=}\StringTok{"content"} \KeywordTok{>}
       \KeywordTok{<a}\OtherTok{ href=}\StringTok{"#firstdialog"}\OtherTok{ data-rel=}\StringTok{"dialog"} \KeywordTok{>}\NormalTok{open a page as a dialog}\KeywordTok{</a>}
    \KeywordTok{</div>}
  \KeywordTok{</div>}
  \KeywordTok{<div}\OtherTok{ data-role=}\StringTok{"dialog"}\OtherTok{ id=}\StringTok{"firstdialog"}\KeywordTok{>}
    \NormalTok{...}
     \KeywordTok{<div}\OtherTok{ data-role=}\StringTok{"content"}\KeywordTok{>}
       \KeywordTok{<a}\OtherTok{ href=}\StringTok{"#firstpage"}\KeywordTok{>}\NormalTok{leave dialog and go to first page}\KeywordTok{</a>}
     \KeywordTok{</div>}
  \KeywordTok{</div>}
\KeywordTok{</body>}
\KeywordTok{</html>}
\end{Highlighting}
\end{Shaded}

\emph{jQuery Mobile multi-page templating example}

To, for example, navigate to \emph{secondpage} and have it appear in a
modal dialog using a fade-transition, you would just add the
\texttt{data-rel="dialog"}, \texttt{data-transition="fade"}, and
\texttt{href="index.html\#secondpage"} attributes to an anchor tag.

Roughly speaking, having its own event cycle, jQuery Mobile is a tiny
MVC framework which includes features like progressive widget
enhancement, pre-fetching, caching, and multi-page templating by HTML
configurations innately. In general, a Backbone.js developer does not
need to know about its internal event workflow, but will need to know
how to apply HTML-based configurations which will take action within the
event phase. The \emph{Intercepting jQuery Mobile Events} section goes
into detail regarding how to handle special scenarios when fine-grained
JavaScript adaptions need to be applied.

For further introduction and explanations about jQuery Mobile visit:

\begin{itemize}
\itemsep1pt\parskip0pt\parsep0pt
\item
  \url{http://view.jquerymobile.com/1.3.0/docs/intro/}
\item
  \url{http://view.jquerymobile.com/1.3.0/docs/widgets/pages/}
\item
  \url{http://view.jquerymobile.com/1.3.0/docs/intro/rwd.php}
\end{itemize}

\subsubsection{Basic Backbone app setup for jQuery
Mobile}\label{basic-backbone-app-setup-for-jquery-mobile}

The first major hurdle developers typically run into when building
applications with jQuery Mobile and an MV* framework is that both
frameworks want to handle application navigation.

To combine Backbone and jQuery Mobile, we first need to disable jQuery
Mobile's navigation system and progressive enhancement. The second step
will then be to make use of jQM's custom API to apply configurations and
enhance components during Backbone's application lifecycle instead.

The mobile app example presented here is based on the existing codebase
of the TodoMVC Backbone-Require.js example, which was discussed in an
earlier chapter, and is enhanced to support jQuery Mobile.

\begin{figure}[htbp]
\centering
\includegraphics{img/chapter10-1-1.png}
\end{figure}

\emph{Screenshot of the TodoMVC app with jQuery Mobile}

This implementation makes use of Grunt-BBB as well as Handlebars.js.
Additional utilities useful for mobile applications will be provided,
which can be easily combined and extended. (see the \emph{Backbone
Boilerplate \& Grunt-BBB} and \emph{Backbone Extensions} chapters)

\begin{figure}[htbp]
\centering
\includegraphics{img/chapter10-1-2.png}
\end{figure}

\emph{Workspace of the TodoMVC app with jQueryMobile and Backbone}

The order of the files loaded by Require.js is as follows:

\begin{enumerate}
\def\labelenumi{\arabic{enumi}.}
\itemsep1pt\parskip0pt\parsep0pt
\item
  jQuery
\item
  Underscore/Lodash
\item
  handlebars.compiled
\item
  TodoRouter (instantiates specific views)
\item
  jQueryMobile
\item
  JqueryMobileCustomInitConfig
\item
  Instantiation of the Backbone Router
\end{enumerate}

By opening the console in the project directory and then running the
Grunt-Backbone command \texttt{grunt handlebars} or \texttt{grunt watch}
in the console, it will combine and compile all template files to
\texttt{dist/debug/handlebars\_packaged}. To start the application, run
\texttt{grunt server}.

Files instantiated, when redirected from the Backbone-Router are:

\begin{enumerate}
\def\labelenumi{\alph{enumi})}
\itemsep1pt\parskip0pt\parsep0pt
\item
  \emph{BasicView.js} and \emph{basic\_page\_simple.template}
\end{enumerate}

The BasicView is responsible for the Handlebars multipage-template
processing. Its implementation of \texttt{render} calls the jQuery
Mobile API \texttt{\$.mobile.changePage} to handle page navigation and
progressive widget enhancement.

\begin{enumerate}
\def\labelenumi{\alph{enumi})}
\setcounter{enumi}{1}
\itemsep1pt\parskip0pt\parsep0pt
\item
  Concrete view with its template partial
\end{enumerate}

E.g., \texttt{EditTodoPage.js} and
\texttt{editTodoView.template\_partial}

The head section of \texttt{index.html} needs to load the
\texttt{jquerymobile.css} as well as the \texttt{base.css}, which is
used by all Todo-MVC apps, and the \texttt{index.css} for some
project-specific custom CSS.

\begin{Shaded}
\begin{Highlighting}[]
\KeywordTok{<html>}
\KeywordTok{<head>}
    \KeywordTok{<meta}\OtherTok{ charset=}\StringTok{"utf-8"}\KeywordTok{>}
    \KeywordTok{<meta}\OtherTok{ http-equiv=}\StringTok{"X-UA-Compatible"}\OtherTok{ content=}\StringTok{"IE=edge,chrome=1"}\KeywordTok{>}
    \KeywordTok{<meta}\OtherTok{ name=}\StringTok{"viewport"}\OtherTok{ content=}\StringTok{"width=device-width,initial-scale=1"}\KeywordTok{>}

    \KeywordTok{<title>}\NormalTok{TodoMVC Jquery Mobile}\KeywordTok{</title>}

\CommentTok{<!-- widget and responsive design styles -->}
    \KeywordTok{<link}\OtherTok{ rel=}\StringTok{"stylesheet"}\OtherTok{ href=}\StringTok{"/assets/css/jquerymobile.css"}\KeywordTok{>}
\CommentTok{<!-- used by all TodoMVC apps -->}
    \KeywordTok{<link}\OtherTok{ rel=}\StringTok{"stylesheet"}\OtherTok{ href=}\StringTok{"/assets/css/base.css"}\KeywordTok{>}
\CommentTok{<!-- custom css -->}
    \KeywordTok{<link}\OtherTok{ rel=}\StringTok{"stylesheet"}\OtherTok{ href=}\StringTok{"/assets/css/index.css"}\KeywordTok{>}
\KeywordTok{</head>}

\KeywordTok{<body>}
    \KeywordTok{<script}\OtherTok{ data-main=}\StringTok{"/app/config"}\OtherTok{ src=}\StringTok{"/assets/js/libs/require.js"}\KeywordTok{></script>}
\KeywordTok{</body>}
\KeywordTok{</html>}
\end{Highlighting}
\end{Shaded}

\emph{index.html}

\subsubsection{Workflow with Backbone and
jQueryMobile}\label{workflow-with-backbone-and-jquerymobile}

By delegating the routing and navigation functions of the jQuery Mobile
Framework to Backbone, we can profit from its clear separation of
application structure to later on easily share application logic between
a desktop webpage, tablets, and mobile apps.

We now need to contend with the different ways in which Backbone and
jQuery Mobile handle requests. \texttt{Backbone.Router} offers an
explicit way to define custom navigation routes, while jQuery Mobile
uses URL hash fragments to reference separate pages or views in the same
document.

Some of the ideas that have been previously proposed to work-around this
problem included manually patching Backbone and jQuery Mobile. The
solution demonstrated below will not only simplify the handling of the
jQuery Mobile component initialization event-cycle, but also enables use
of existing Backbone Router handlers.

To adapt the navigation control from jQuery Mobile to Backbone, we first
need to apply some specific settings to the \texttt{mobileinit} event
which occurs after the framework has loaded in order to let the Backbone
Router decide which page to load.

A configuration which will get jQM to delegate navigation to Backbone
and which will also enable manual widget creation triggering is given
below:

\begin{Shaded}
\begin{Highlighting}[]
\FunctionTok{$}\NormalTok{(document).}\FunctionTok{bind}\NormalTok{(}\StringTok{"mobileinit"}\NormalTok{, }\KeywordTok{function}\NormalTok{()\{}

\CommentTok{// Disable jQM routing and component creation events   }
   \CommentTok{// disable hash-routing}
   \OtherTok{$}\NormalTok{.}\OtherTok{mobile}\NormalTok{.}\FunctionTok{hashListeningEnabled} \NormalTok{= }\KeywordTok{false}\NormalTok{;}
   \CommentTok{// disable anchor-control}
   \OtherTok{$}\NormalTok{.}\OtherTok{mobile}\NormalTok{.}\FunctionTok{linkBindingEnabled} \NormalTok{= }\KeywordTok{false}\NormalTok{;}
   \CommentTok{// can cause calling object creation twice and back button issues are solved}
   \OtherTok{$}\NormalTok{.}\OtherTok{mobile}\NormalTok{.}\FunctionTok{ajaxEnabled} \NormalTok{= }\KeywordTok{false}\NormalTok{;}
   \CommentTok{// Otherwise after mobileinit, it tries to load a landing page}
   \OtherTok{$}\NormalTok{.}\OtherTok{mobile}\NormalTok{.}\FunctionTok{autoInitializePage} \NormalTok{= }\KeywordTok{false}\NormalTok{;}
   \CommentTok{// we want to handle caching and cleaning the DOM ourselves}
   \OtherTok{$}\NormalTok{.}\OtherTok{mobile}\NormalTok{.}\OtherTok{page}\NormalTok{.}\OtherTok{prototype}\NormalTok{.}\OtherTok{options}\NormalTok{.}\FunctionTok{domCache} \NormalTok{= }\KeywordTok{false}\NormalTok{;}

\CommentTok{// consider due to compatibility issues}
   \CommentTok{// not supported by all browsers}
   \OtherTok{$}\NormalTok{.}\OtherTok{mobile}\NormalTok{.}\FunctionTok{pushStateEnabled} \NormalTok{= }\KeywordTok{false}\NormalTok{;}
   \CommentTok{// Solves phonegap issues with the back-button}
   \OtherTok{$}\NormalTok{.}\OtherTok{mobile}\NormalTok{.}\FunctionTok{phonegapNavigationEnabled} \NormalTok{= }\KeywordTok{true}\NormalTok{;}
   \CommentTok{//no native datepicker will conflict with the jQM component}
   \OtherTok{$}\NormalTok{.}\OtherTok{mobile}\NormalTok{.}\OtherTok{page}\NormalTok{.}\OtherTok{prototype}\NormalTok{.}\OtherTok{options}\NormalTok{.}\OtherTok{degradeInputs}\NormalTok{.}\FunctionTok{date} \NormalTok{= }\KeywordTok{true}\NormalTok{;}
\NormalTok{\});}
\end{Highlighting}
\end{Shaded}

\emph{jquerymobile.config.js}

The behaviour and usage of the new workflow will be explained below,
grouped by its functionalities:

\begin{enumerate}
\def\labelenumi{\alph{enumi})}
\item
  Routing to a concrete View-page
\item
  Management of mobile page templates
\item
  DOM management
\item
  \$.mobile.changePage
\end{enumerate}

In the following discussion, the steps 1-11 in the text refer to the new
workflow diagram of the mobile application below.

\begin{figure}[htbp]
\centering
\includegraphics{img/chapter10-2-1.png}
\end{figure}

\emph{Workflow of TodoMVC, with Backbone and jQueryMobile}

\paragraph{Routing to a concrete View page, Inheriting from
BasicView}\label{routing-to-a-concrete-view-page-inheriting-from-basicview}

When the hash URL changes, e.g., a link is clicked, the configuration
above prevents jQM from triggering its events. Instead, the Backbone
Router listens to the hash changes and decides which view to request.

Experience has shown that, for mobile pages, it is a good practice to
create basic prototypes for jQM components such as basic pages, popups,
and dialogs, as well as for using the jQuery Validation Plugin. This
makes it much easier to exchange device-specific view logic at runtime
and adopt general strategies. This will also help to add syntax and to
support multi-chaining of prototype inheritance with JavaScript and
Backbone.

By creating a \texttt{BasicView} superclass, we enable all inheriting
view-pages to share a common way of handling jQM along with common usage
of a template engine and specific view handling.

When building with Grunt/Yeoman, the semantic templates are compiled by
Handlebar.js and the AMDs template files are combined into a single
file. By merging all page definitions into a single-file-app, it becomes
offline capable, which is important for mobile app.

\paragraph{Management of Mobile Page
Templates}\label{management-of-mobile-page-templates}

Within a concrete View page, you can override properties for static
values and functions to return dynamic values of the super class
\texttt{BasicView}. These values will be processed later by the
BasicView to construct the HTML of a jQuery Mobile page with the help of
Handlebars.

Additional dynamic template parameters, e.g., Backbone model
information, will be taken from the specific View and merged with the
ones from the BasicView.

A concrete View might look like:

\begin{Shaded}
\begin{Highlighting}[]
\FunctionTok{define}\NormalTok{([}
    \StringTok{"backbone"}\NormalTok{, }\StringTok{"modules/view/abstract/BasicView"}\NormalTok{],}
    \KeywordTok{function} \NormalTok{(Backbone, BasicView) \{}
        \KeywordTok{return} \OtherTok{BasicView}\NormalTok{.}\FunctionTok{extend}\NormalTok{(\{}
            \DataTypeTok{id }\NormalTok{: }\StringTok{"editTodoView"}\NormalTok{, }
            \DataTypeTok{getHeaderTitle }\NormalTok{: }\KeywordTok{function} \NormalTok{() \{}
                \KeywordTok{return} \StringTok{"Edit Todo"}\NormalTok{;}
            \NormalTok{\},}
            \DataTypeTok{getSpecificTemplateValues }\NormalTok{: }\KeywordTok{function} \NormalTok{() \{}
                \KeywordTok{return} \OtherTok{_}\NormalTok{.}\FunctionTok{clone}\NormalTok{(}\KeywordTok{this}\NormalTok{.}\OtherTok{model}\NormalTok{.}\FunctionTok{attributes}\NormalTok{);}
            \NormalTok{\},}
            \DataTypeTok{events }\NormalTok{: }\KeywordTok{function} \NormalTok{() \{}
                \CommentTok{// merged events of BasicView, to add an older fix for back button functionality}
                \KeywordTok{return} \OtherTok{_}\NormalTok{.}\FunctionTok{extend}\NormalTok{(\{}
                    \StringTok{'click #saveDescription'} \NormalTok{: }\StringTok{'saveDescription'}
                \NormalTok{\}, }\KeywordTok{this}\NormalTok{.}\OtherTok{constructor}\NormalTok{.}\OtherTok{__super__}\NormalTok{.}\FunctionTok{events}\NormalTok{);}
            \NormalTok{\},}
            \DataTypeTok{saveDescription }\NormalTok{: }\KeywordTok{function} \NormalTok{(clickEvent) \{}
                \KeywordTok{this}\NormalTok{.}\OtherTok{model}\NormalTok{.}\FunctionTok{save}\NormalTok{(\{}
                    \DataTypeTok{title }\NormalTok{: }\FunctionTok{$}\NormalTok{(}\StringTok{"#todoDescription"}\NormalTok{, }\KeywordTok{this}\NormalTok{.}\FunctionTok{el}\NormalTok{).}\FunctionTok{val}\NormalTok{()}
                \NormalTok{\});}
                \KeywordTok{return} \KeywordTok{true}\NormalTok{;}
            \NormalTok{\}}
        \NormalTok{\});}
    \NormalTok{\});}
\end{Highlighting}
\end{Shaded}

\emph{A concrete View (EditTodoPage.js)}

By default, the BasicView uses \texttt{basic\_page\_simple.template} as
the Handlebars template. If you need to use a custom template or want to
introduce a new Super abstract View with an alternate template, override
the \texttt{getTemplateID} function:

\begin{Shaded}
\begin{Highlighting}[]
\NormalTok{getTemplateID : }\KeywordTok{function}\NormalTok{()\{}
  \KeywordTok{return} \StringTok{"custom_page_template"}\NormalTok{;}
\NormalTok{\}}
\end{Highlighting}
\end{Shaded}

By convention, the \texttt{id} attribute will be taken as the id of the
jQM page as well as the filename of the corresponding template file to
be inserted as a partial in the \texttt{basic\_page\_simple} template.
In the case of the \texttt{EditTodoPage} view, the name of the file will
be \texttt{editTodoPage.template\_partial}.

Every concrete page is meant to be a partial, which will be inserted in
the \texttt{data-role="content"} element, where the parameter
\texttt{templatePartialPageID} is located.

Later on, the result of the \texttt{getHeaderTitle} function from
\texttt{EditTodoPage} will replace the \emph{headerTitle} in the
abstract template.

\begin{Shaded}
\begin{Highlighting}[]
\NormalTok{<div data-role=}\StringTok{"header"}\NormalTok{>}
        \NormalTok{\{\{whatis }\StringTok{"Specific loaded Handlebars parameters:"}\NormalTok{\}\}}
        \NormalTok{\{\{whatis }\KeywordTok{this}\NormalTok{\}\}}
        \NormalTok{<h2>\{\{headerTitle\}\}<}\OtherTok{/h2>}
\OtherTok{        <a id="backButton" href="href="javascript:history.go}\FloatTok{(}\OtherTok{-1}\FloatTok{)}\OtherTok{;" data-icon="star" data-rel="back" >back</a}\NormalTok{>}
    \NormalTok{<}\OtherTok{/div>}
\OtherTok{    <div data-role="content">}
\OtherTok{        \{\{whatis "Template page trying to load:"\}\}}
\OtherTok{        \{\{whatis templatePartialPageID\}\}}
\OtherTok{        \{\{> templatePartialPageID\}\}}
\OtherTok{    </div}\NormalTok{>}
    \NormalTok{<div data-role=}\StringTok{"footer"}\NormalTok{>}
        \NormalTok{\{\{footerContent\}\}}
\NormalTok{<}\OtherTok{/div>}
\end{Highlighting}
\end{Shaded}

\emph{basic\_page\_simple.template}

\emph{Note: The \texttt{whatis} Handlebars View helper does simple
logging of parameters.}

All the additional parameters being returned by
\texttt{getSpecificTemplateValues} will be inserted into the concrete
template \texttt{editTodoPage.template\_partial}.

Because \texttt{footerContent} is expected to be used rarely, its
content is returned by \texttt{getSpecificTemplateValues}.

In the case of the EditTodoPage view, all the model information is being
returned and \texttt{title} is used in the concrete partial page:

\begin{Shaded}
\begin{Highlighting}[]
\KeywordTok{<div}\OtherTok{ data-role=}\StringTok{"fieldcontain"}\KeywordTok{>}
    \KeywordTok{<label}\OtherTok{ for=}\StringTok{"todoDescription"}\KeywordTok{>}\NormalTok{Todo Description}\KeywordTok{</label>}
    \KeywordTok{<input}\OtherTok{ type=}\StringTok{"text"}\OtherTok{ name=}\StringTok{"todoDescription"}\OtherTok{ id=}\StringTok{"todoDescription"}\OtherTok{ value=}\StringTok{"\{\{title\}\}"} \KeywordTok{/>}
\KeywordTok{</div>}
    \KeywordTok{<a}\OtherTok{ id=}\StringTok{"saveDescription"}\OtherTok{ href=}\StringTok{"#"}\OtherTok{ data-role=}\StringTok{"button"}\OtherTok{ data-mini=}\StringTok{"true"}\KeywordTok{>}\NormalTok{Save}\KeywordTok{</a>}
\end{Highlighting}
\end{Shaded}

\emph{editTodoView.template\_partial}

When \texttt{render} is triggered, the
\texttt{basic\_page\_simple.template} and
\texttt{editTodoView.template\_partial} templates will be loaded and the
parameters from \texttt{EditTodoPage} and \texttt{BasicView} will be
combined and generated by Handlebars to generate:

\begin{Shaded}
\begin{Highlighting}[]
    \KeywordTok{<div}\OtherTok{ data-role=}\StringTok{"header"}\KeywordTok{>}
        \KeywordTok{<h2>}\NormalTok{Edit Todo}\KeywordTok{</h2>}
        \KeywordTok{<a}\OtherTok{ id=}\StringTok{"backButton"}\OtherTok{ href=}\StringTok{"href="}\ErrorTok{javascript:history.go(-1);"}\OtherTok{ data-icon=}\StringTok{"star"}\OtherTok{ data-rel=}\StringTok{"back"} \KeywordTok{>}\NormalTok{back}\KeywordTok{</a>}
    \KeywordTok{</div>}
    \KeywordTok{<div}\OtherTok{ data-role=}\StringTok{"content"}\KeywordTok{>}
      \KeywordTok{<div}\OtherTok{ data-role=}\StringTok{"fieldcontain"}\KeywordTok{>}
       \KeywordTok{<label}\OtherTok{ for=}\StringTok{"todoDescription"}\KeywordTok{>}\NormalTok{Todo Description}\KeywordTok{</label>}
       \KeywordTok{<input}\OtherTok{ type=}\StringTok{"text"}\OtherTok{ name=}\StringTok{"todoDescription"}\OtherTok{ id=}\StringTok{"todoDescription"}\OtherTok{ value=}\StringTok{"Cooking"} \KeywordTok{/>}
      \KeywordTok{</div>}
      \KeywordTok{<a}\OtherTok{ id=}\StringTok{"saveDescription"}\OtherTok{ href=}\StringTok{"#"}\OtherTok{ data-role=}\StringTok{"button"}\OtherTok{ data-mini=}\StringTok{"true"}\KeywordTok{>}\NormalTok{Save}\KeywordTok{</a>}
    \KeywordTok{</div>}
    \KeywordTok{<div}\OtherTok{ data-role=}\StringTok{"footer"}\KeywordTok{>}
        \NormalTok{Footer}
    \KeywordTok{</div>}
\end{Highlighting}
\end{Shaded}

\emph{Final HTML definition resulting from basic\_page\_simple\_template
and editTodoView.template\_partial}

The next section explains how the template parameters are collected by
the \texttt{BasicView} and the HTML definition is loaded.

\paragraph{DOM management and
\$.mobile.changePage}\label{dom-management-and-.mobile.changepage}

When \texttt{render} is executed (line 29 is the source code listing
below), \texttt{BasicView} first cleans up the DOM by removing the
previous page (line 70). To delete the elements from the DOM,
\texttt{\$.remove} cannot be used, but \texttt{\$previousEl.detach()}
can be since \texttt{detach} does not remove the element's attached
events and data.

This is important, because jQuery Mobile still needs information (e.g.,
to trigger transition effects when switching to another page). Keep in
mind that the DOM data and events should be cleared later on as well to
avoid possible performance issues.

Other strategies than the one used in the function
\texttt{cleanupPossiblePageDuplicationInDOM} to cleanup the DOM are
viable. To only remove the old page having the same id as the current
from the DOM, when it was already requested before, would also be a
working strategy of preventing DOM duplication. Depending on what fits
best to your application needs, it is also possibly a one-banana problem
to exchange it using a caching mechanism.

Next, \texttt{BasicView} collects all template parameters from the
concrete View implementation and inserts the HTML of the requested page
into the body. This is done in steps 4, 5, 6, and 7 in the diagram above
(between lines 23 and 51 in the source listing).

Additionally, the \texttt{data-role} will be set on the jQuery Mobile
page. Commonly used attribute values are page, dialog, or popup.

As you can see, (starting at line 74), the \texttt{goBackInHistory}
function contains a manual implementation to handle the back button's
action. In certain scenarios, the back button navigation functionality
of jQuery Mobile was not working with older versions and disabled
jQMobile's navigation system.

\begin{Shaded}
\begin{Highlighting}[]
 \DecValTok{1} \FunctionTok{define}\NormalTok{([}
 \DecValTok{2}     \StringTok{"lodash"}\NormalTok{,}
 \DecValTok{3}     \StringTok{"backbone"}\NormalTok{,}
 \DecValTok{4}     \StringTok{"handlebars"}\NormalTok{,}
 \DecValTok{5}     \StringTok{"handlebars_helpers"}
 \DecValTok{6} \NormalTok{],}
 \DecValTok{7} 
 \DecValTok{8} \KeywordTok{function} \NormalTok{(_, Backbone, Handlebars) \{}
 \DecValTok{9}     \KeywordTok{var} \NormalTok{BasicView = }\OtherTok{Backbone}\NormalTok{.}\OtherTok{View}\NormalTok{.}\FunctionTok{extend}\NormalTok{(\{}
\DecValTok{10}         \DataTypeTok{initialize}\NormalTok{: }\KeywordTok{function} \NormalTok{() \{}
\DecValTok{11}             \OtherTok{_}\NormalTok{.}\FunctionTok{bindAll}\NormalTok{();}
\DecValTok{12}             \KeywordTok{this}\NormalTok{.}\FunctionTok{render}\NormalTok{();}
\DecValTok{13}         \NormalTok{\},}
\DecValTok{14}         \DataTypeTok{events}\NormalTok{: \{}
\DecValTok{15}             \StringTok{"click #backButton"}\NormalTok{: }\StringTok{"goBackInHistory"}
\DecValTok{16}         \NormalTok{\},}
\DecValTok{17}         \DataTypeTok{role}\NormalTok{: }\StringTok{"page"}\NormalTok{,}
\DecValTok{18}         \DataTypeTok{attributes}\NormalTok{: }\KeywordTok{function} \NormalTok{() \{}
\DecValTok{19}             \KeywordTok{return} \NormalTok{\{}
\DecValTok{20}                 \StringTok{"data-role"}\NormalTok{: }\KeywordTok{this}\NormalTok{.}\FunctionTok{role}
\DecValTok{21}             \NormalTok{\};}
\DecValTok{22}         \NormalTok{\},}
\DecValTok{23}         \DataTypeTok{getHeaderTitle}\NormalTok{: }\KeywordTok{function} \NormalTok{() \{}
\DecValTok{24}             \KeywordTok{return} \KeywordTok{this}\NormalTok{.}\FunctionTok{getSpecificTemplateValues}\NormalTok{().}\FunctionTok{headerTitle}\NormalTok{;}
\DecValTok{25}         \NormalTok{\},}
\DecValTok{26}         \DataTypeTok{getTemplateID}\NormalTok{: }\KeywordTok{function} \NormalTok{() \{}
\DecValTok{27}             \KeywordTok{return} \StringTok{"basic_page_simple"}\NormalTok{;}
\DecValTok{28}         \NormalTok{\},}
\DecValTok{29}         \DataTypeTok{render}\NormalTok{: }\KeywordTok{function} \NormalTok{() \{}
\DecValTok{30}             \KeywordTok{this}\NormalTok{.}\FunctionTok{cleanupPossiblePageDuplicationInDOM}\NormalTok{();}
\DecValTok{31}             \FunctionTok{$}\NormalTok{(}\KeywordTok{this}\NormalTok{.}\FunctionTok{el}\NormalTok{).}\FunctionTok{html}\NormalTok{(}\KeywordTok{this}\NormalTok{.}\FunctionTok{getBasicPageTemplateResult}\NormalTok{());}
\DecValTok{32}             \KeywordTok{this}\NormalTok{.}\FunctionTok{addPageToDOMAndRenderJQM}\NormalTok{();}
\DecValTok{33}             \KeywordTok{this}\NormalTok{.}\FunctionTok{enhanceJQMComponentsAPI}\NormalTok{();}
\DecValTok{34}         \NormalTok{\},}
\DecValTok{35} \CommentTok{// Generate HTML using the Handlebars templates}
\DecValTok{36}         \DataTypeTok{getTemplateResult}\NormalTok{: }\KeywordTok{function} \NormalTok{(templateDefinitionID, templateValues) \{}
\DecValTok{37}             \KeywordTok{return} \OtherTok{window}\NormalTok{.}\FunctionTok{JST}\NormalTok{[templateDefinitionID](templateValues);}
\DecValTok{38}         \NormalTok{\},}
\DecValTok{39} \CommentTok{// Collect all template paramters and merge them}
\DecValTok{40}         \DataTypeTok{getBasicPageTemplateResult}\NormalTok{: }\KeywordTok{function} \NormalTok{() \{}
\DecValTok{41}             \KeywordTok{var} \NormalTok{templateValues = \{}
\DecValTok{42}                 \DataTypeTok{templatePartialPageID}\NormalTok{: }\KeywordTok{this}\NormalTok{.}\FunctionTok{id}\NormalTok{,}
\DecValTok{43}                 \DataTypeTok{headerTitle}\NormalTok{: }\KeywordTok{this}\NormalTok{.}\FunctionTok{getHeaderTitle}\NormalTok{()}
\DecValTok{44}             \NormalTok{\};}
\DecValTok{45}             \KeywordTok{var} \NormalTok{specific = }\KeywordTok{this}\NormalTok{.}\FunctionTok{getSpecificTemplateValues}\NormalTok{();}
\DecValTok{46}             \OtherTok{$}\NormalTok{.}\FunctionTok{extend}\NormalTok{(templateValues, }\KeywordTok{this}\NormalTok{.}\FunctionTok{getSpecificTemplateValues}\NormalTok{());}
\DecValTok{47}             \KeywordTok{return} \KeywordTok{this}\NormalTok{.}\FunctionTok{getTemplateResult}\NormalTok{(}\KeywordTok{this}\NormalTok{.}\FunctionTok{getTemplateID}\NormalTok{(), templateValues);}
\DecValTok{48}         \NormalTok{\},}
\DecValTok{49}         \DataTypeTok{getRequestedPageTemplateResult}\NormalTok{: }\KeywordTok{function} \NormalTok{() \{}
\DecValTok{50}             \KeywordTok{this}\NormalTok{.}\FunctionTok{getBasicPageTemplateResult}\NormalTok{();}
\DecValTok{51}         \NormalTok{\},}
\DecValTok{52}         \DataTypeTok{enhanceJQMComponentsAPI}\NormalTok{: }\KeywordTok{function} \NormalTok{() \{}
\DecValTok{53} \CommentTok{// changePage}
\DecValTok{54}             \OtherTok{$}\NormalTok{.}\OtherTok{mobile}\NormalTok{.}\FunctionTok{changePage}\NormalTok{(}\StringTok{"#"} \NormalTok{+ }\KeywordTok{this}\NormalTok{.}\FunctionTok{id}\NormalTok{, \{}
\DecValTok{55}                 \DataTypeTok{changeHash}\NormalTok{: }\KeywordTok{false}\NormalTok{,}
\DecValTok{56}                 \DataTypeTok{role}\NormalTok{: }\KeywordTok{this}\NormalTok{.}\FunctionTok{role}
\DecValTok{57}             \NormalTok{\});}
\DecValTok{58}         \NormalTok{\},}
\DecValTok{59} \CommentTok{// Add page to DOM}
\DecValTok{60}         \DataTypeTok{addPageToDOMAndRenderJQM}\NormalTok{: }\KeywordTok{function} \NormalTok{() \{}
\DecValTok{61}             \FunctionTok{$}\NormalTok{(}\StringTok{"body"}\NormalTok{).}\FunctionTok{append}\NormalTok{(}\FunctionTok{$}\NormalTok{(}\KeywordTok{this}\NormalTok{.}\FunctionTok{el}\NormalTok{));}
\DecValTok{62}             \FunctionTok{$}\NormalTok{(}\StringTok{"#"} \NormalTok{+ }\KeywordTok{this}\NormalTok{.}\FunctionTok{id}\NormalTok{).}\FunctionTok{page}\NormalTok{();}
\DecValTok{63}         \NormalTok{\},}
\DecValTok{64} \CommentTok{// Cleanup DOM strategy}
\DecValTok{65}         \DataTypeTok{cleanupPossiblePageDuplicationInDOM}\NormalTok{: }\KeywordTok{function} \NormalTok{() \{}
\DecValTok{66}         \CommentTok{// Can also be moved to the event "pagehide": or "onPageHide"}
\DecValTok{67}             \KeywordTok{var} \NormalTok{$previousEl = }\FunctionTok{$}\NormalTok{(}\StringTok{"#"} \NormalTok{+ }\KeywordTok{this}\NormalTok{.}\FunctionTok{id}\NormalTok{);}
\DecValTok{68}             \KeywordTok{var} \NormalTok{alreadyInDom = }\OtherTok{$previousEl}\NormalTok{.}\FunctionTok{length} \NormalTok{>= }\DecValTok{0}\NormalTok{;}
\DecValTok{69}             \KeywordTok{if} \NormalTok{(alreadyInDom) \{}
\DecValTok{70}                 \OtherTok{$previousEl}\NormalTok{.}\FunctionTok{detach}\NormalTok{();}
\DecValTok{71}             \NormalTok{\}}
\DecValTok{72}         \NormalTok{\},}
\DecValTok{73} \CommentTok{// Strategy to always support back button with disabled navigation}
\DecValTok{74}         \DataTypeTok{goBackInHistory}\NormalTok{: }\KeywordTok{function} \NormalTok{(clickEvent) \{}
\DecValTok{75}             \OtherTok{history}\NormalTok{.}\FunctionTok{go}\NormalTok{(-}\DecValTok{1}\NormalTok{);}
\DecValTok{76}             \KeywordTok{return} \KeywordTok{false}\NormalTok{;}
\DecValTok{77}         \NormalTok{\}}
\DecValTok{78}     \NormalTok{\});}
\DecValTok{79} 
\DecValTok{80}     \KeywordTok{return} \NormalTok{BasicView;}
\DecValTok{81} \NormalTok{\});}
\end{Highlighting}
\end{Shaded}

\emph{BasicView.js}

After the dynamic HTML is added to the DOM,
\texttt{\$.mobile.changePage} has to be applied at step 8 (code line
54).

This is the most important API call, because it triggers the jQuery
Mobile component creation for the current page.

Next, the page will be displayed to the user at step 9.

\begin{Shaded}
\begin{Highlighting}[]
\NormalTok{<a data-mini=}\StringTok{"true"} \NormalTok{data-role=}\StringTok{"button"} \NormalTok{href=}\StringTok{"#"} \NormalTok{id=}\StringTok{"saveDescription"} \NormalTok{data-corners=}\StringTok{"true"} 
\NormalTok{data-shadow=}\StringTok{"true"} \NormalTok{data-iconshadow=}\StringTok{"true"} \NormalTok{data-wrapperels=}\StringTok{"span"} \NormalTok{data-theme=}\StringTok{"c"} 
\KeywordTok{class}\NormalTok{=}\StringTok{"ui-btn ui-shadow ui-btn-corner-all ui-mini ui-btn-up-c"}\NormalTok{>}
    \NormalTok{<span }\KeywordTok{class}\NormalTok{=}\StringTok{"ui-btn-inner"}\NormalTok{>}
         \NormalTok{<span }\KeywordTok{class}\NormalTok{=}\StringTok{"ui-btn-text"}\NormalTok{>Save<}\OtherTok{/span>}
\OtherTok{     </span}\NormalTok{>}
\NormalTok{<}\OtherTok{/a>}
\end{Highlighting}
\end{Shaded}

\begin{figure}[htbp]
\centering
\includegraphics{img/chapter10-2-2.png}
\end{figure}

\emph{Look and feel of the written HTML code and the jQuery Mobile
enhanced Todo description page}

UI enhancement is done in the \texttt{enhanceJQMComponentsAPI} function
in line 52:

\begin{Shaded}
\begin{Highlighting}[]
\OtherTok{$}\NormalTok{.}\OtherTok{mobile}\NormalTok{.}\FunctionTok{changePage}\NormalTok{(}\StringTok{"#"} \NormalTok{+ }\KeywordTok{this}\NormalTok{.}\FunctionTok{id}\NormalTok{, \{}
                      \DataTypeTok{changeHash}\NormalTok{: }\KeywordTok{false}\NormalTok{,}
                      \DataTypeTok{role}\NormalTok{: }\KeywordTok{this}\NormalTok{.}\FunctionTok{role}
                    \NormalTok{\});}
\end{Highlighting}
\end{Shaded}

To retain control of hash routing, \texttt{changeHash} has to be set to
false and the proper \texttt{role} parameter provided to guarantee
proper page appearance. Finally, \texttt{changePage} will show the new
page with its defined transition to the user.

For the basic use cases, it is advised to have one View per page, and
always render the complete page again by calling
\texttt{\$.mobile.changePage} when widget enhancement needs to be done.

To progress component enrichment of a newly added HTML-fragment into the
DOM, advanced techniques need to be applied to guarantee correct
appearance of the mobile components. You need to be very careful when
creating partial HTML code and updating values on UI elements. The next
section will explain how to handle these situations.

\subsubsection{Applying advanced jQM techniques to
Backbone}\label{applying-advanced-jqm-techniques-to-backbone}

\paragraph{Dynamic DOM Scripting}\label{dynamic-dom-scripting}

The solution described above solves the issues of handling routing with
Backbone by calling \texttt{\$.mobile.changePage('pageID')}.
Additionally, it guarantees that the HTML page will be completely
enhanced by the markup for jQuery Mobile.

The second tricky part with jQuery Mobile is to dynamically manipulate
specific DOM contents (e.g.~after loading in content with Ajax). We
suggest you use this technique only if there is evidence for an
appreciable performance gain.

With the current version (1.3), jQM provides three ways, documented and
explained below in the official API, on forums, and blogs.

\begin{itemize}
\item
  \textbf{\$(``pageId'').trigger(``pagecreate'')}

  \emph{Creates markup of header, content as well as footer}
\item
  \textbf{\$(``anyElement'').trigger(``create'')}

  \emph{Creates markup of the element as well as all children}
\item
  \textbf{\$(``myListElement'').listview(``refresh'')}
\item
  \textbf{\$(`{[}type=``radio''{]}').checkboxradio()}
\item
  \textbf{\$(`{[}type=``text''{]}').textinput()}
\item
  \textbf{\$(`{[}type=``button''{]}').button()}
\item
  \textbf{\$(`{[}data-role=``navbar''{]}').navbar()}
\item
  \textbf{\$(`{[}type=``range''{]}').slider()}
\item
  \textbf{\$(`select').selectmenu()}

  \emph{Every component of jQM offers plugins methods they can invoke to
  update the state of specific UI elements.}
\end{itemize}

Sometimes, when creating a component from scratch, the following error
can occur: `cannot call methods on listview prior to initialization'.
This can be avoided, with component initialization prior to markup
enhancement, by calling it in the following way:

\begin{Shaded}
\begin{Highlighting}[]
 \FunctionTok{$}\NormalTok{(}\StringTok{'#mylist'}\NormalTok{).}\FunctionTok{listview}\NormalTok{().}\FunctionTok{listview}\NormalTok{(}\StringTok{'refresh'}\NormalTok{)}
\end{Highlighting}
\end{Shaded}

To see more details and enhancements for further scripting pages of JQM
read their API and follow the release notes frequently.

\begin{itemize}
\itemsep1pt\parskip0pt\parsep0pt
\item
  \href{http://jquerymobile.com/test/docs/pages/page-scripting.html}{jQuery
  Mobile: Page Scripting}
\item
  \href{http://stackoverflow.com/questions/14468659/jquery-mobile-document-ready-vs-page-events/}{jQuery
  Mobile: Document Ready vs.~Page Events}
\item
  \href{http://stackoverflow.com/questions/14550396/jquery-mobile-markup-enhancement-of-dynamically-added-content}{StackOverflow:
  Markup Enhancement of Dynamically Added Content}
\end{itemize}

If you consider using a \texttt{Model-Binding Plugin}, you will need to
come up with an automated mechanism to enrich single components.

After having a look at the previous section about Dynamic DOM Scripting,
it might not be acceptable to completely re-create a component (e.g a
Listview) which takes a longer time to load and to reduce the complexity
of event-delegation. Instead, the component-specific plugins, which will
only update the needed parts of the HTML and CSS, should be used.

In the case of a Listview, you would need to call the following function
to update the list of added, edited, or removed entries:

\begin{Shaded}
\begin{Highlighting}[]
\FunctionTok{$}\NormalTok{(}\StringTok{'#mylist'}\NormalTok{).}\FunctionTok{listview}\NormalTok{()}
\end{Highlighting}
\end{Shaded}

You need to come up with a means of detecting the component type to in
order to decide which plugin method needs to be called. The jQuery
Mobile Angular.js Adapter provides such a strategy and solution as well.

\href{https://github.com/tigbro/jquery-mobile-angular-adapter/blob/master/src/main/webapp/integration/jqmWidgetPatches.js}{Example
of Model Binding with jQuery Mobile}

\paragraph{Intercepting jQuery Mobile
Events}\label{intercepting-jquery-mobile-events}

In special situations you will need to take action on a triggered jQuery
Mobile event, which can be done as follows:

\begin{Shaded}
\begin{Highlighting}[]
\FunctionTok{$}\NormalTok{(}\StringTok{'#myPage'}\NormalTok{).}\FunctionTok{live}\NormalTok{(}\StringTok{'pagebeforecreate'}\NormalTok{, }\KeywordTok{function}\NormalTok{(event)\{}
         \OtherTok{console}\NormalTok{.}\FunctionTok{log}\NormalTok{(}\StringTok{'page was inserted into the DOM'}\NormalTok{);}
    \CommentTok{//run your own enhancement scripting here...}
          \CommentTok{// prevent the page plugin from making its manipulations}
    \KeywordTok{return} \KeywordTok{false}\NormalTok{;}
\NormalTok{\});}

\FunctionTok{$}\NormalTok{(}\StringTok{'#myPage'}\NormalTok{).}\FunctionTok{live}\NormalTok{(}\StringTok{'pagecreate'}\NormalTok{, }\KeywordTok{function}\NormalTok{(event)\{}
          \OtherTok{console}\NormalTok{.}\FunctionTok{log}\NormalTok{(‘page was enhanced by jQM}\StringTok{');}
\NormalTok{\});}
\end{Highlighting}
\end{Shaded}

In such scenarios, it is important to know when the jQuery Mobile events
occur. The following diagram depicts the event cycle (page A is the
outgoing page and page B is the incoming page).

\begin{figure}[htbp]
\centering
\includegraphics{img/chapter10-3-1.png}
\end{figure}

\emph{jQuery Mobile Event Cycle}

An alternative is the jQuery Mobile Router project, which you might use
to replace the Backbone Router. With the help of the jQM Router project,
you could achieve a powerful way to intercept and route one of the
various jQM events. It is an extension to jQuery Mobile, which can be
used independently.

Be aware that jQM-Router misses some features of Backbone.Router and is
tightly coupled with the jQuery Mobile framework. For these reasons, we
did not use it for the TodoMVC app. If you intend to use it, consider
using a Backbone.js custom build to exclude the router code. This might
save around 25\% relative to the max compressed size of 17,1 KB.

\href{http://gregfranko.com/backbone/customBuild/}{Backbone's Custom
Builder}

\paragraph{Performance}\label{performance}

Performance is an important topic on mobile devices. jQuery Mobile
provides various tools that create performance logs which can give you a
good overview of the actual time spent in routing logic, component
enhancement, and visual effects.

Depending on the device, the time spent on transitions can take up to
90\% of the load time. To disable all transitions, you can either pass
the transition \texttt{none} to \texttt{\$.mobile.changePage()}, in the
configuration code block:

\begin{Shaded}
\begin{Highlighting}[]
\FunctionTok{$}\NormalTok{(document).}\FunctionTok{bind}\NormalTok{(}\StringTok{"mobileinit"}\NormalTok{, }\KeywordTok{function}\NormalTok{()\{}
\NormalTok{…}
\CommentTok{// Otherwise, depending on takes up to 90% of loadtime}
  \OtherTok{$}\NormalTok{.}\OtherTok{mobile}\NormalTok{.}\FunctionTok{defaultPageTransition} \NormalTok{= }\StringTok{"none"}\NormalTok{;}
  \OtherTok{$}\NormalTok{.}\OtherTok{mobile}\NormalTok{.}\FunctionTok{defaultDialogTransition} \NormalTok{= }\StringTok{"none"}\NormalTok{;}
    \NormalTok{\});}
  \NormalTok{\})}
\end{Highlighting}
\end{Shaded}

or consider adding device-specific settings, for example:

\begin{Shaded}
\begin{Highlighting}[]
\FunctionTok{$}\NormalTok{(document).}\FunctionTok{bind}\NormalTok{(}\StringTok{"mobileinit"}\NormalTok{, }\KeywordTok{function}\NormalTok{()\{}

  \KeywordTok{var} \NormalTok{iosDevice =((}\OtherTok{navigator}\NormalTok{.}\OtherTok{userAgent}\NormalTok{.}\FunctionTok{match}\NormalTok{(}\OtherTok{/iPhone/i}\NormalTok{))}
  \NormalTok{|| (}\OtherTok{navigator}\NormalTok{.}\OtherTok{userAgent}\NormalTok{.}\FunctionTok{match}\NormalTok{(}\OtherTok{/iPod/i}\NormalTok{))) ? }\KeywordTok{true} \NormalTok{: }\KeywordTok{false}\NormalTok{;}

  \OtherTok{$}\NormalTok{.}\FunctionTok{extend}\NormalTok{(  }\OtherTok{$}\NormalTok{.}\FunctionTok{mobile} \NormalTok{, \{}
    \DataTypeTok{slideText }\NormalTok{:  (iosDevice) ? }\StringTok{"slide"} \NormalTok{: }\StringTok{"none"}\NormalTok{,}
    \DataTypeTok{slideUpText }\NormalTok{:  (iosDevice) ? }\StringTok{"slideup"} \NormalTok{: }\StringTok{"none"}\NormalTok{,}
    \DataTypeTok{defaultPageTransition}\NormalTok{:(iosDevice) ? }\StringTok{"slide"} \NormalTok{: }\StringTok{"none"}\NormalTok{,}
    \DataTypeTok{defaultDialogTransition}\NormalTok{:(iosDevice) ? }\StringTok{"slide"} \NormalTok{: }\StringTok{"none"}
  \NormalTok{\});}
\end{Highlighting}
\end{Shaded}

Also, consider doing your own pre-caching of enhanced jQuery Mobile
pages.

The jQuery Mobile API is frequently enhanced with regards to this topic
in each new release. We suggest you take a look at the latest updated
API to determine an optimal caching strategy with dynamic scripting that
best fits your needs.

For further information on performance, see the following:

\begin{itemize}
\itemsep1pt\parskip0pt\parsep0pt
\item
  \href{https://github.com/jquery/jquery-mobile/tree/master/tools}{jQuery
  Mobile Profiling Tools}
\item
  \href{http://backbonefu.com/2012/01/jquery-mobile-and-backbone-js-the-ugly/}{Device
  specific jQuery Mobile configuations}
\item
  \href{http://www.objectpartners.com/2012/11/02/use-jquery-mobile\%E2\%80\%99s-tools-suite-to-help-you-debug-and-improve-your-jquery-mobile-application/}{jQuery
  Mobile Debugging tools}
\item
  \href{http://jquerymobile.com/demos/1.2.0/docs/pages/page-cache.html}{jQuery
  Mobile precaching functionalities}
\end{itemize}

\paragraph{Clever Multi-Platform Support
Management}\label{clever-multi-platform-support-management}

Nowadays, a company typically has an existing webpage and management
decides to provide an additional mobile app to customers. The code of
the web page and the mobile app become independent of each other and the
time required for content or feature changes becomes much higher than
for the webpage alone.

As the trend is towards an increasing number of mobile platforms and
dimensions, the effort required to support them is only increasing as
well. Ultimately, creating per-device experiences is not always viable.
However, it is essential that content is available to all users,
regardless of their browser and platform. This principle must be kept in
mind during the design phase.

\emph{\href{http://www.lukew.com/ff/entry.asp?933}{Responsive Design}}
and \emph{\href{http://www.abookapart.com/products/mobile-first}{Mobile
First}} approaches address these challenges.

The mobile app architecture presented in this chapter takes care of a
lot of the actual heavy lifting required, as it supports responsive
layouts out of the box and even supports browsers which cannot handle
media queries. It might not be obvious that jQM is a UI framework not
dissimilar to jQuery UI. jQuery Mobile is using the widget factory and
is capable of being used for more than just mobile environments.

To support multi-platform browsers using jQuery Mobile and Backbone, you
can, in order of increasing time and effort:

\begin{enumerate}
\def\labelenumi{\arabic{enumi}.}
\itemsep1pt\parskip0pt\parsep0pt
\item
  Ideally, have one code project, where only CSS differs for different
  devices.
\item
  Same code project, and at runtime different HTML templates and
  super-classes are exchanged per device type.
\item
  Same code project, and the Responsive Design API and most widgets of
  jQuery Mobile will be reused. For the desktop browser, some components
  will be added by another widget framework (e.g. \emph{jQueryUI} or
  \emph{Twitter Boostrap}), e.g.~controlled by the HTML templating.
\item
  Same code project, and at runtime, jQuery Mobile will be completely
  replaced by another widget framework (e.g. \emph{jQueryUI} or
  \emph{Twitter Boostrap}). Super-classes and configurations, as well as
  concrete Backbone.View code snippets need to be replaced.
\item
  Different code projects, but common modules are reused.
\item
  For the desktop app, there is a completely separate code project.
  Reasons might be the usage of complete different programming
  languagages and/or frameworks, lack of Responsive Design knowledge or
  legacy of pollution.
\end{enumerate}

The ideal solution, to build a nice-looking desktop application with
only one mobile framework, sounds crazy, but is feasible.

If you have a look at the jQuery Mobile API page in a desktop browser,
it does not look anything like a mobile application.

\begin{figure}[htbp]
\centering
\includegraphics{img/chapter10-3-3.png}
\end{figure}

\emph{Desktop view of the jQuery Mobile API and Docs application
(http://view.jquerymobile.com/1.3.0/)}

The same goes for the jQuery Mobile design examples, where jQuery Mobile
intends to add further user interface experiences.

\begin{figure}[htbp]
\centering
\includegraphics{img/chapter10-3-4.png}
\end{figure}

\emph{Design examples of jQuery Mobile for desktop environments,
http://jquerymobile.com/designs/\#desktop}

The accordions, date-pickers, sliders - everything in the desktop UI is
re-using what jQM would be providing users on mobile devices. By way of
example, adding the attribute \texttt{data-mini="true"} on components
will lose the clumsiness of the mobile widgets on a desktop browser.

See http://jquerymobile.com/demos/1.2.0/docs/forms/forms-all-mini.html,
Mini-widgets for desktop applications by jQuery Mobile.

Thanks to some media queries, the desktop UI can make optimal use of
whitespace, expanding component blocks out and providing alternative
layouts while still making use of jQM as the component framework.

The benefit of this is that you don't need to pull in another widget
framework (e.g., jQuery UI) separately to be able to take advantage of
these features. Thanks to the ThemeRoller, the components can look
pretty much exactly how you would like them to and users of the app can
get a jQM UI for lower-resolutions and a jQM-ish UI for everything else.

The take away here is just to remember that if you are not already going
through the hassle of conditional script/style loading based on
screen-resolution (using matchMedia.js, etc.), there are simpler
approaches that can be taken to cross-device component theming. At least
the Responsive Design API of jQuery Mobile, which was added since
version 1.3.0, is always reasonable because it will work for mobile as
well as for desktop. In summary, you can manage jQuery Mobile components
to give users a typical desktop appearance and they will not realize a
difference.

\href{http://view.jquerymobile.com/1.3.0/docs/intro/rwd.php}{Responsive
Design with jQuery Mobile}

Also, if you hit your limits of CSS-styling and configurations of your
jQuery Mobile application for desktop browsers, the additional effort to
use jQuery Mobile and Twitter Bootstrap together can be quite simple. In
the case that a desktop browser requests the page and Twitter Bootstrap
has been loaded, the mobile TodoMVC app would need conditional code to
not trigger the jQM Widget processive enhancement plugins API
(demonstrated in the \emph{Dynamic DOM Scripting} section) in the
Backbone.View implementation. Therefore, as explained in the previous
sections, we recommend triggering widget enhancements by
\texttt{\$.mobile.changePage} only once to load the complete page.

An example of such a widget hybrid usage can be seen here:

\begin{figure}[htbp]
\centering
\includegraphics{img/chapter10-3-2.png}
\end{figure}

\emph{\href{http://appengine.beecoss.com}{Appengine boilerplate, desktop
and mobile appearance}}

Although it is using server-side technologies for templating using the
programming language Python, the principle of triggering progressive
enhancement at page load is the same as \texttt{\$mobile.changePage}.

As you can see, the JavaScript and even the CSS stays the same. The only
device-specific conditions and differences in implementations are for
selecting the appropriate framework imports, which are located in the
HTML template:

\begin{Shaded}
\begin{Highlighting}[]
\NormalTok{...}
 \NormalTok{\{% if is_mobile %\}}
    \KeywordTok{<link}\OtherTok{ rel=}\StringTok{"stylesheet"}\OtherTok{ href=}\StringTok{"/mobile/jquery.mobile-1.1.0.min.css"} \KeywordTok{/>}
    \NormalTok{\{% else %\}}
      \KeywordTok{<link}\OtherTok{ rel=}\StringTok{"apple-touch-icon"}\OtherTok{ href=}\StringTok{"/apple-touch-icon.png"} \KeywordTok{/>}
      \KeywordTok{<link}\OtherTok{ rel=}\StringTok{"stylesheet"}\OtherTok{ href=}\StringTok{"/css/style.css"} \KeywordTok{/>}
      \KeywordTok{<link}\OtherTok{ rel=}\StringTok{"stylesheet"}\OtherTok{ href=}\StringTok{"/css/bootstrap.min.css"}\KeywordTok{>}
      \KeywordTok{<link}\OtherTok{ rel=}\StringTok{"stylesheet"}\OtherTok{ href=}\StringTok{"/css/bootstrap-responsive.min.css"}\KeywordTok{>}
    \NormalTok{\{% endif %\}}
      \KeywordTok{<link}\OtherTok{ rel=}\StringTok{"stylesheet"}\OtherTok{ href=}\StringTok{"/css/main.css"} \KeywordTok{/>}

    \NormalTok{\{% block mediaCSS %\}\{% endblock %\}}
\NormalTok{...}
 \NormalTok{\{% if is_mobile %\}}
      \KeywordTok{<script}\OtherTok{ src=}\StringTok{"/mobile/jquery.mobile-1.1.0.min.js"}\KeywordTok{></script>}
    \NormalTok{\{% else %\}}
      \KeywordTok{<script}\OtherTok{ src=}\StringTok{"/js/libs/bootstrap.min.js"}\KeywordTok{></script>}
    \NormalTok{\{% endif %\}}
\NormalTok{...}
\end{Highlighting}
\end{Shaded}

\section{Unit Testing}\label{unit-testing}

One definition of unit testing is the process of taking the smallest
piece of testable code in an application, isolating it from the
remainder of your codebase, and determining if it behaves exactly as
expected.

For an application to be considered `well-tested', each function should
ideally have its own separate unit tests where it's tested against the
different conditions you expect it to handle. All tests must pass before
functionality is considered `complete'. This allows developers to both
modify a unit of code and its dependencies with a level of confidence
about whether these changes have caused any breakage.

A basic example of unit testing is where a developer asserts that
passing specific values to a sum function results in the correct value
being returned. For an example more relevant to this book, we may wish
to assert that adding a new Todo item to a list correctly adds a Model
of a specific type to a Todos Collection.

When building modern web-applications, it's typically considered
best-practice to include automated unit testing as a part of your
development process. In the following chapters we are going to look at
three different solutions for unit testing your Backbone.js apps -
Jasmine, QUnit and SinonJS.

\section{Jasmine}\label{jasmine}

\subsection{Behavior-Driven
Development}\label{behavior-driven-development}

In this section, we'll be taking a look at how to unit test Backbone
applications using a popular JavaScript testing framework called
\href{http://pivotal.github.com/jasmine/}{Jasmine} from Pivotal Labs.

Jasmine describes itself as a behavior-driven development (BDD)
framework for testing JavaScript code. Before we jump into how the
framework works, it's useful to understand exactly what
\href{http://en.wikipedia.org/wiki/Behavior_Driven_Development}{BDD} is.

BDD is a second-generation testing approach first described by
\href{http://dannorth.net/introducing-bdd/}{Dan North} (the authority on
BDD) which attempts to test the behavior of software. It's considered
second-generation as it came out of merging ideas from Domain driven
design (DDD) and lean software development. BDD helps teams deliver
high-quality software by answering many of the more confusing questions
early on in the agile process. Such questions commonly include those
concerning documentation and testing.

If you were to read a book on BDD, it's likely that it would be
described as being `outside-in and pull-based'. The reason for this is
that it borrows the idea of `pulling features' from Lean manufacturing
which effectively ensures that the right software solutions are being
written by a) focusing on the expected outputs of the system and b)
ensuring these outputs are achieved.

BDD recognizes that there are usually multiple stakeholders in a project
and not a single amorphous user of the system. These different groups
will be affected by the software being written in differing ways and
will have varying opinions of what quality in the system means to them.
It's for this reason that it's important to understand who the software
will be bringing value to and exactly what in it will be valuable to
them.

Finally, BDD relies on automation. Once you've defined the quality
expected, your team will want to check on the functionality of the
solution being built regularly and compare it to the results they
expect. In order to facilitate this efficiently, the process has to be
automated. BDD relies heavily on the automation of specification-testing
and Jasmine is a tool which can assist with this.

BDD helps both developers and non-technical stakeholders:

\begin{itemize}
\itemsep1pt\parskip0pt\parsep0pt
\item
  Better understand and represent the models of the problems being
  solved
\item
  Explain supported test cases in a language that non-developers can
  read
\item
  Focus on minimizing translation of the technical code being written
  and the domain language spoken by the business
\end{itemize}

What this means is that developers should be able to show Jasmine unit
tests to a project stakeholder and (at a high level, thanks to a common
vocabulary being used) they'll ideally be able to understand what the
code supports.

Developers often implement BDD in unison with another testing paradigm
known as
\href{http://en.wikipedia.org/wiki/Test-driven_development}{TDD}
(test-driven development). The main idea behind TDD is using the
following development process:

\begin{enumerate}
\def\labelenumi{\arabic{enumi}.}
\itemsep1pt\parskip0pt\parsep0pt
\item
  Write unit tests which describe the functionality you would like your
  code to support
\item
  Watch these tests fail (as the code to support them hasn't yet been
  written)
\item
  Write code to make the tests pass
\item
  Rinse, repeat, and refactor
\end{enumerate}

In this chapter we're going to use BDD (with TDD) to write unit tests
for a Backbone application.

\textbf{\emph{Note:}} I've seen a lot of developers also opt for writing
tests to validate behavior of their code after having written it. While
this is fine, note that it can come with pitfalls such as only testing
for behavior your code currently supports, rather than the behavior
needed to fully solve the problem.

\subsection{Suites, Specs, \& Spies}\label{suites-specs-spies}

When using Jasmine, you'll be writing suites and specifications (specs).
Suites basically describe scenarios while specs describe what can be
done in these scenarios.

Each spec is a JavaScript function, described with a call to
\texttt{it()} using a description string and a function. The description
should describe the behaviour the particular unit of code should exhibit
and, keeping in mind BDD, it should ideally be meaningful. Here's an
example of a basic spec:

\begin{Shaded}
\begin{Highlighting}[]
\FunctionTok{it}\NormalTok{(}\StringTok{'should be incrementing in value'}\NormalTok{, }\KeywordTok{function}\NormalTok{()\{}
    \KeywordTok{var} \NormalTok{counter = }\DecValTok{0}\NormalTok{;}
    \NormalTok{counter++;}
\NormalTok{\});}
\end{Highlighting}
\end{Shaded}

On its own, a spec isn't particularly useful until expectations are set
about the behavior of the code. Expectations in specs are defined using
the \texttt{expect()} function and an
\href{https://github.com/pivotal/jasmine/wiki/Matchers}{expectation
matcher} (e.g., \texttt{toEqual()}, \texttt{toBeTruthy()},
\texttt{toContain()}). A revised example using an expectation matcher
would look like:

\begin{Shaded}
\begin{Highlighting}[]
\FunctionTok{it}\NormalTok{(}\StringTok{'should be incrementing in value'}\NormalTok{, }\KeywordTok{function}\NormalTok{()\{}
    \KeywordTok{var} \NormalTok{counter = }\DecValTok{0}\NormalTok{;}
    \NormalTok{counter++;}
    \FunctionTok{expect}\NormalTok{(counter).}\FunctionTok{toEqual}\NormalTok{(}\DecValTok{1}\NormalTok{);}
\NormalTok{\});}
\end{Highlighting}
\end{Shaded}

The above code passes our behavioral expectation as \texttt{counter}
equals 1. Notice how easy it was to read the expectation on the last
line (you probably grokked it without any explanation).

Specs are grouped into suites which we describe using Jasmine's
\texttt{describe()} function, again passing a string as a description
and a function as we did for \texttt{it()}. The name/description for
your suite is typically that of the component or module you're testing.

Jasmine will use the description as the group name when it reports the
results of the specs you've asked it to run. A simple suite containing
our sample spec could look like:

\begin{Shaded}
\begin{Highlighting}[]
\FunctionTok{describe}\NormalTok{(}\StringTok{'Stats'}\NormalTok{, }\KeywordTok{function}\NormalTok{()\{}
    \FunctionTok{it}\NormalTok{(}\StringTok{'can increment a number'}\NormalTok{, }\KeywordTok{function}\NormalTok{()\{}
        \NormalTok{...}
    \NormalTok{\});}

    \FunctionTok{it}\NormalTok{(}\StringTok{'can subtract a number'}\NormalTok{, }\KeywordTok{function}\NormalTok{()\{}
        \NormalTok{...}
    \NormalTok{\});}
\NormalTok{\});}
\end{Highlighting}
\end{Shaded}

Suites also share a functional scope, so it's possible to declare
variables and functions inside a describe block which are accessible
within specs:

\begin{Shaded}
\begin{Highlighting}[]
\FunctionTok{describe}\NormalTok{(}\StringTok{'Stats'}\NormalTok{, }\KeywordTok{function}\NormalTok{()\{}
    \KeywordTok{var} \NormalTok{counter = }\DecValTok{1}\NormalTok{;}

    \FunctionTok{it}\NormalTok{(}\StringTok{'can increment a number'}\NormalTok{, }\KeywordTok{function}\NormalTok{()\{}
        \CommentTok{// the counter was = 1}
        \NormalTok{counter = counter + }\DecValTok{1}\NormalTok{;}
        \FunctionTok{expect}\NormalTok{(counter).}\FunctionTok{toEqual}\NormalTok{(}\DecValTok{2}\NormalTok{);}
    \NormalTok{\});}

    \FunctionTok{it}\NormalTok{(}\StringTok{'can subtract a number'}\NormalTok{, }\KeywordTok{function}\NormalTok{()\{}
        \CommentTok{// the counter was = 2}
        \NormalTok{counter = counter - }\DecValTok{1}\NormalTok{;}
        \FunctionTok{expect}\NormalTok{(counter).}\FunctionTok{toEqual}\NormalTok{(}\DecValTok{1}\NormalTok{);}
    \NormalTok{\});}
\NormalTok{\});}
\end{Highlighting}
\end{Shaded}

\textbf{\emph{Note:}} Suites are executed in the order in which they are
described, which can be useful to know if you would prefer to see test
results for specific parts of your application reported first.

Jasmine also supports \textbf{spies} - a way to mock, spy, and fake
behavior in our unit tests. Spies replace the function they're spying
on, allowing us to simulate behavior we would like to mock (i.e., test
without using the actual implementation).

In the example below, we're spying on the \texttt{setComplete} method of
a dummy Todo function to test that arguments can be passed to it as
expected.

\begin{Shaded}
\begin{Highlighting}[]
\KeywordTok{var} \NormalTok{Todo = }\KeywordTok{function}\NormalTok{()\{}
\NormalTok{\};}

\OtherTok{Todo}\NormalTok{.}\OtherTok{prototype}\NormalTok{.}\FunctionTok{setComplete} \NormalTok{= }\KeywordTok{function} \NormalTok{(arg)\{}
    \KeywordTok{return} \NormalTok{arg;}
\NormalTok{\}}

\FunctionTok{describe}\NormalTok{(}\StringTok{'a simple spy'}\NormalTok{, }\KeywordTok{function}\NormalTok{()\{}
    \FunctionTok{it}\NormalTok{(}\StringTok{'should spy on an instance method of a Todo'}\NormalTok{, }\KeywordTok{function}\NormalTok{()\{}
        \KeywordTok{var} \NormalTok{myTodo = }\KeywordTok{new} \FunctionTok{Todo}\NormalTok{();}
        \FunctionTok{spyOn}\NormalTok{(myTodo, }\StringTok{'setComplete'}\NormalTok{);}
        \OtherTok{myTodo}\NormalTok{.}\FunctionTok{setComplete}\NormalTok{(}\StringTok{'foo bar'}\NormalTok{);}

        \FunctionTok{expect}\NormalTok{(}\OtherTok{myTodo}\NormalTok{.}\FunctionTok{setComplete}\NormalTok{).}\FunctionTok{toHaveBeenCalledWith}\NormalTok{(}\StringTok{'foo bar'}\NormalTok{);}

        \KeywordTok{var} \NormalTok{myTodo2 = }\KeywordTok{new} \FunctionTok{Todo}\NormalTok{();}
        \FunctionTok{spyOn}\NormalTok{(myTodo2, }\StringTok{'setComplete'}\NormalTok{);}

        \FunctionTok{expect}\NormalTok{(}\OtherTok{myTodo2}\NormalTok{.}\FunctionTok{setComplete}\NormalTok{).}\OtherTok{not}\NormalTok{.}\FunctionTok{toHaveBeenCalled}\NormalTok{();}

    \NormalTok{\});}
\NormalTok{\});}
\end{Highlighting}
\end{Shaded}

You are more likely to use spies for testing
\href{http://en.wikipedia.org/wiki/Asynchronous_communication}{asynchronous}
behavior in your application such as AJAX requests. Jasmine supports:

\begin{itemize}
\itemsep1pt\parskip0pt\parsep0pt
\item
  Writing tests which can mock AJAX requests using spies. This allows us
  to test both the code that initiates the AJAX request and the code
  executed upon its completion. It's also possible to mock/fake the
  server responses. The benefit of this type of testing is that it's
  faster as no real calls are being made to a server. The ability to
  simulate any response from the server is also of great benefit.
\item
  Asynchronous tests which don't rely on spies
\end{itemize}

This example of the first kind of test shows how to fake an AJAX request
and verify that the request was both calling the correct URL and
executed a callback where one was provided.

\begin{Shaded}
\begin{Highlighting}[]
\FunctionTok{it}\NormalTok{(}\StringTok{'the callback should be executed on success'}\NormalTok{, }\KeywordTok{function} \NormalTok{() \{}

    \CommentTok{// `andCallFake()` calls a passed function when a spy}
    \CommentTok{// has been called}
    \FunctionTok{spyOn}\NormalTok{($, }\StringTok{'ajax'}\NormalTok{).}\FunctionTok{andCallFake}\NormalTok{(}\KeywordTok{function}\NormalTok{(options) \{}
        \OtherTok{options}\NormalTok{.}\FunctionTok{success}\NormalTok{();}
    \NormalTok{\});}

    \CommentTok{// Create a new spy}
    \KeywordTok{var} \NormalTok{callback = }\OtherTok{jasmine}\NormalTok{.}\FunctionTok{createSpy}\NormalTok{();}

    \CommentTok{// Exexute the spy callback if the}
    \CommentTok{// request for Todo 15 is successful}
    \FunctionTok{getTodo}\NormalTok{(}\DecValTok{15}\NormalTok{, callback);}

    \CommentTok{// Verify that the URL of the most recent call}
    \CommentTok{// matches our expected Todo item.}
    \FunctionTok{expect}\NormalTok{(}\OtherTok{$}\NormalTok{.}\OtherTok{ajax}\NormalTok{.}\OtherTok{mostRecentCall}\NormalTok{.}\FunctionTok{args}\NormalTok{[}\DecValTok{0}\NormalTok{][}\StringTok{'url'}\NormalTok{]).}\FunctionTok{toEqual}\NormalTok{(}\StringTok{'/todos/15'}\NormalTok{);}

    \CommentTok{// `expect(x).toHaveBeenCalled()` will pass if `x` is a}
    \CommentTok{// spy and was called.}
    \FunctionTok{expect}\NormalTok{(callback).}\FunctionTok{toHaveBeenCalled}\NormalTok{();}
\NormalTok{\});}

\KeywordTok{function} \FunctionTok{getTodo}\NormalTok{(id, callback) \{}
    \OtherTok{$}\NormalTok{.}\FunctionTok{ajax}\NormalTok{(\{}
        \DataTypeTok{type}\NormalTok{: }\StringTok{'GET'}\NormalTok{,}
        \DataTypeTok{url}\NormalTok{: }\StringTok{'/todos/'} \NormalTok{+ id,}
        \DataTypeTok{dataType}\NormalTok{: }\StringTok{'json'}\NormalTok{,}
        \DataTypeTok{success}\NormalTok{: callback}
    \NormalTok{\});}
\NormalTok{\}}
\end{Highlighting}
\end{Shaded}

All of these are Spy-specific matchers and are documented on the Jasmine
\href{https://github.com/pivotal/jasmine/wiki/Spies}{wiki}.

For the second type of test (asynchronous tests), we can take the above
further by taking advantage of three other methods Jasmine supports:

\begin{itemize}
\itemsep1pt\parskip0pt\parsep0pt
\item
  \href{https://github.com/pivotal/jasmine/wiki/Asynchronous-specs}{waits(timeout)}
  - a native timeout before the next block is run
\item
  \href{https://github.com/pivotal/jasmine/wiki/Asynchronous-specs}{waitsFor(function,
  optional message, optional timeout)} - a way to pause specs until some
  other work has completed. Jasmine waits until the supplied function
  returns true here before it moves on to the next block.
\item
  \href{https://github.com/pivotal/jasmine/wiki/Asynchronous-specs}{runs(function)}
  - a block which runs as if it was directly called. They exist so that
  we can test asynchronous processes.
\end{itemize}

\begin{Shaded}
\begin{Highlighting}[]
\FunctionTok{it}\NormalTok{(}\StringTok{'should make an actual AJAX request to a server'}\NormalTok{, }\KeywordTok{function} \NormalTok{() \{}

    \CommentTok{// Create a new spy}
    \KeywordTok{var} \NormalTok{callback = }\OtherTok{jasmine}\NormalTok{.}\FunctionTok{createSpy}\NormalTok{();}

    \CommentTok{// Exexute the spy callback if the}
    \CommentTok{// request for Todo 16 is successful}
    \FunctionTok{getTodo}\NormalTok{(}\DecValTok{16}\NormalTok{, callback);}

    \CommentTok{// Pause the spec until the callback count is}
    \CommentTok{// greater than 0}
    \FunctionTok{waitsFor}\NormalTok{(}\KeywordTok{function}\NormalTok{() \{}
        \KeywordTok{return} \OtherTok{callback}\NormalTok{.}\FunctionTok{callCount} \NormalTok{> }\DecValTok{0}\NormalTok{;}
    \NormalTok{\});}

    \CommentTok{// Once the wait is complete, our runs() block}
    \CommentTok{// will check to ensure our spy callback has been}
    \CommentTok{// called}
    \FunctionTok{runs}\NormalTok{(}\KeywordTok{function}\NormalTok{() \{}
        \FunctionTok{expect}\NormalTok{(callback).}\FunctionTok{toHaveBeenCalled}\NormalTok{();}
    \NormalTok{\});}
\NormalTok{\});}

\KeywordTok{function} \FunctionTok{getTodo}\NormalTok{(id, callback) \{}
    \OtherTok{$}\NormalTok{.}\FunctionTok{ajax}\NormalTok{(\{}
        \DataTypeTok{type}\NormalTok{: }\StringTok{'GET'}\NormalTok{,}
        \DataTypeTok{url}\NormalTok{: }\StringTok{'todos.json'}\NormalTok{,}
        \DataTypeTok{dataType}\NormalTok{: }\StringTok{'json'}\NormalTok{,}
        \DataTypeTok{success}\NormalTok{: callback}
    \NormalTok{\});}
\NormalTok{\}}
\end{Highlighting}
\end{Shaded}

\textbf{\emph{Note:}} It's useful to remember that when making real
requests to a web server in your unit tests, this has the potential to
massively slow down the speed at which tests run (due to many factors
including server latency). As this also introduces an external
dependency that can (and should) be minimized in your unit testing, it
is strongly recommended that you opt for spies to remove the dependency
on a web server.

\subsection{beforeEach() and
afterEach()}\label{beforeeach-and-aftereach}

Jasmine also supports specifying code that can be run before each
(\texttt{beforeEach()}) and after each (\texttt{afterEach()}) test. This
is useful for enforcing consistent conditions (such as resetting
variables that may be required by specs). In the following example,
\texttt{beforeEach()} is used to create a new sample Todo model which
specs can use for testing attributes.

\begin{Shaded}
\begin{Highlighting}[]
\FunctionTok{beforeEach}\NormalTok{(}\KeywordTok{function}\NormalTok{()\{}
   \KeywordTok{this}\NormalTok{.}\FunctionTok{todo} \NormalTok{= }\KeywordTok{new} \OtherTok{Backbone}\NormalTok{.}\FunctionTok{Model}\NormalTok{(\{}
      \DataTypeTok{text}\NormalTok{: }\StringTok{'Buy some more groceries'}\NormalTok{,}
      \DataTypeTok{done}\NormalTok{: }\KeywordTok{false}
   \NormalTok{\});}
\NormalTok{\});}

\FunctionTok{it}\NormalTok{(}\StringTok{'should contain a text value if not the default value'}\NormalTok{, }\KeywordTok{function}\NormalTok{()\{}
   \FunctionTok{expect}\NormalTok{(}\KeywordTok{this}\NormalTok{.}\OtherTok{todo}\NormalTok{.}\FunctionTok{get}\NormalTok{(}\StringTok{'text'}\NormalTok{)).}\FunctionTok{toEqual}\NormalTok{(}\StringTok{'Buy some more groceries'}\NormalTok{);}
\NormalTok{\});}
\end{Highlighting}
\end{Shaded}

Each nested \texttt{describe()} in your tests can have their own
\texttt{beforeEach()} and \texttt{afterEach()} methods which support
including setup and teardown methods relevant to a particular suite.

\texttt{beforeEach()} and \texttt{afterEach()} can be used together to
write tests verifying that our Backbone routes are being correctly
triggered when we navigate to the URL. We can start with the
\texttt{index} action:

\begin{Shaded}
\begin{Highlighting}[]
\FunctionTok{describe}\NormalTok{(}\StringTok{'Todo routes'}\NormalTok{, }\KeywordTok{function}\NormalTok{()\{}

   \FunctionTok{beforeEach}\NormalTok{(}\KeywordTok{function}\NormalTok{()\{}

        \CommentTok{// Create a new router}
        \KeywordTok{this}\NormalTok{.}\FunctionTok{router} \NormalTok{= }\KeywordTok{new} \OtherTok{App}\NormalTok{.}\FunctionTok{TodoRouter}\NormalTok{();}

        \CommentTok{// Create a new spy}
        \KeywordTok{this}\NormalTok{.}\FunctionTok{routerSpy} \NormalTok{= }\OtherTok{jasmine}\NormalTok{.}\FunctionTok{spy}\NormalTok{();}

        \CommentTok{// Begin monitoring hashchange events}
        \KeywordTok{try}\NormalTok{\{}
            \OtherTok{Backbone}\NormalTok{.}\OtherTok{history}\NormalTok{.}\FunctionTok{start}\NormalTok{(\{}
                \DataTypeTok{silent}\NormalTok{:}\KeywordTok{true}\NormalTok{,}
                \DataTypeTok{pushState}\NormalTok{: }\KeywordTok{true}
            \NormalTok{\});}
        \NormalTok{\}}\KeywordTok{catch}\NormalTok{(e)\{}
           \CommentTok{// ...}
        \NormalTok{\}}

        \CommentTok{// Navigate to a URL}
        \KeywordTok{this}\NormalTok{.}\OtherTok{router}\NormalTok{.}\FunctionTok{navigate}\NormalTok{(}\StringTok{'/js/spec/SpecRunner.html'}\NormalTok{);}
   \NormalTok{\}); }

   \FunctionTok{afterEach}\NormalTok{(}\KeywordTok{function}\NormalTok{()\{}

        \CommentTok{// Navigate back to the URL}
        \KeywordTok{this}\NormalTok{.}\OtherTok{router}\NormalTok{.}\FunctionTok{navigate}\NormalTok{(}\StringTok{'/js/spec/SpecRunner.html'}\NormalTok{);}

        \CommentTok{// Disable Backbone.history temporarily.}
        \CommentTok{// Note that this is not really useful in real apps but is}
        \CommentTok{// good for testing routers}
        \OtherTok{Backbone}\NormalTok{.}\OtherTok{history}\NormalTok{.}\FunctionTok{stop}\NormalTok{();}
   \NormalTok{\});}

   \FunctionTok{it}\NormalTok{(}\StringTok{'should call the index route correctly'}\NormalTok{, }\KeywordTok{function}\NormalTok{()\{}
        \KeywordTok{this}\NormalTok{.}\OtherTok{router}\NormalTok{.}\FunctionTok{bind}\NormalTok{(}\StringTok{'route:index'}\NormalTok{, }\KeywordTok{this}\NormalTok{.}\FunctionTok{routerSpy}\NormalTok{, }\KeywordTok{this}\NormalTok{);}
        \KeywordTok{this}\NormalTok{.}\OtherTok{router}\NormalTok{.}\FunctionTok{navigate}\NormalTok{(}\StringTok{''}\NormalTok{, \{}\DataTypeTok{trigger}\NormalTok{: }\KeywordTok{true}\NormalTok{\});}

        \CommentTok{// If everything in our beforeEach() and afterEach()}
        \CommentTok{// calls have been correctly executed, the following}
        \CommentTok{// should now pass.}
        \FunctionTok{expect}\NormalTok{(}\KeywordTok{this}\NormalTok{.}\FunctionTok{routerSpy}\NormalTok{).}\FunctionTok{toHaveBeenCalledOnce}\NormalTok{();}
        \FunctionTok{expect}\NormalTok{(}\KeywordTok{this}\NormalTok{.}\FunctionTok{routerSpy}\NormalTok{).}\FunctionTok{toHaveBeenCalledWith}\NormalTok{();}
   \NormalTok{\});}

\NormalTok{\});}
\end{Highlighting}
\end{Shaded}

The actual TodoRouter for that would make the above test pass looks
like:

\begin{Shaded}
\begin{Highlighting}[]
\KeywordTok{var} \NormalTok{App = App || \{\};}
\OtherTok{App}\NormalTok{.}\FunctionTok{TodoRouter} \NormalTok{= }\OtherTok{Backbone}\NormalTok{.}\OtherTok{Router}\NormalTok{.}\FunctionTok{extend}\NormalTok{(\{}
    \DataTypeTok{routes}\NormalTok{:\{}
        \StringTok{''}\NormalTok{: }\StringTok{'index'}
    \NormalTok{\},}
    \DataTypeTok{index}\NormalTok{: }\KeywordTok{function}\NormalTok{()\{}
        \CommentTok{//...}
    \NormalTok{\}}
\NormalTok{\});}
\end{Highlighting}
\end{Shaded}

\subsection{Shared scope}\label{shared-scope}

Let's imagine we have a Suite where we wish to check for the existence
of a new Todo item instance. This could be done by duplicating the spec
as follows:

\begin{Shaded}
\begin{Highlighting}[]
\FunctionTok{describe}\NormalTok{(}\StringTok{"Todo tests"}\NormalTok{, }\KeywordTok{function}\NormalTok{()\{}
   
   \CommentTok{// Spec}
   \FunctionTok{it}\NormalTok{(}\StringTok{"Should be defined when we create it"}\NormalTok{, }\KeywordTok{function}\NormalTok{()\{}
        \CommentTok{// A Todo item we are testing}
        \KeywordTok{var} \NormalTok{todo = }\KeywordTok{new} \FunctionTok{Todo}\NormalTok{(}\StringTok{"Get the milk"}\NormalTok{, }\StringTok{"Tuesday"}\NormalTok{);}
        \FunctionTok{expect}\NormalTok{(todo).}\FunctionTok{toBeDefined}\NormalTok{();}
   \NormalTok{\}); }

   \FunctionTok{it}\NormalTok{(}\StringTok{"Should have the correct title"}\NormalTok{, }\KeywordTok{function}\NormalTok{()\{}
        \CommentTok{// Where we introduce code duplication}
        \KeywordTok{var} \NormalTok{todo = }\KeywordTok{new} \FunctionTok{Todo}\NormalTok{(}\StringTok{"Get the milk"}\NormalTok{, }\StringTok{"Tuesday"}\NormalTok{);}
        \FunctionTok{expect}\NormalTok{(}\OtherTok{todo}\NormalTok{.}\FunctionTok{title}\NormalTok{).}\FunctionTok{toBe}\NormalTok{(}\StringTok{"Get the milk"}\NormalTok{);}
   \NormalTok{\});}

\NormalTok{\});}
\end{Highlighting}
\end{Shaded}

As you can see, we've introduced duplication that should ideally be
refactored into something cleaner. We can do this using Jasmine's Suite
(Shared) Functional Scope.

All of the specs within the same Suite share the same functional scope,
meaning that variables declared within the Suite itself are available to
all of the Specs in that suite. This gives us a way to work around our
duplication problem by moving the creation of our Todo objects into the
common functional scope:

\begin{Shaded}
\begin{Highlighting}[]
\FunctionTok{describe}\NormalTok{(}\StringTok{"Todo tests"}\NormalTok{, }\KeywordTok{function}\NormalTok{()\{}
    
    \CommentTok{// The instance of Todo, the object we wish to test}
    \CommentTok{// is now in the shared functional scope}
    \KeywordTok{var} \NormalTok{todo = }\KeywordTok{new} \FunctionTok{Todo}\NormalTok{(}\StringTok{"Get the milk"}\NormalTok{, }\StringTok{"Tuesday"}\NormalTok{);}

    \CommentTok{// Spec}
    \FunctionTok{it}\NormalTok{(}\StringTok{"should be correctly defined"}\NormalTok{, }\KeywordTok{function}\NormalTok{()\{}
        \FunctionTok{expect}\NormalTok{(todo).}\FunctionTok{toBeDefined}\NormalTok{();}
    \NormalTok{\});}

    \FunctionTok{it}\NormalTok{(}\StringTok{"should have the correct title"}\NormalTok{, }\KeywordTok{function}\NormalTok{()\{}
        \FunctionTok{expect}\NormalTok{(}\OtherTok{todo}\NormalTok{.}\FunctionTok{title}\NormalTok{).}\FunctionTok{toBe}\NormalTok{(}\StringTok{"Get the milk"}\NormalTok{);}
    \NormalTok{\});}

\NormalTok{\});}
\end{Highlighting}
\end{Shaded}

In the previous section you may have noticed that we initially declared
\texttt{this.todo} within the scope of our \texttt{beforeEach()} call
and were then able to continue using this reference in
\texttt{afterEach()}.

This is again down to shared function scope, which allows such
declaractions to be common to all blocks (including \texttt{runs()}).

Variables declared outside of the shared scope (i.e within the local
scope \texttt{var todo=...}) will however not be shared.

\subsection{Getting set up}\label{getting-set-up-1}

Now that we've reviewed some fundamentals, let's go through downloading
Jasmine and getting everything set up to write tests.

A standalone release of Jasmine can be
\href{https://github.com/pivotal/jasmine/releases/}{downloaded} from the
official release page.

You'll need a file called SpecRunner.html in addition to the release. It
can be downloaded from
https://github.com/pivotal/jasmine/tree/master/lib/jasmine-core/example
or as part of a download of the complete Jasmine
\href{https://github.com/pivotal/jasmine/zipball/master}{repo}.
Alternatively, you can \texttt{git clone} the main Jasmine repository
from https://github.com/pivotal/jasmine.git.

Let's review
\href{https://github.com/pivotal/jasmine/blob/master/lib/templates/SpecRunner.html.jst}{SpecRunner.html.jst}:

It first includes both Jasmine and the necessary CSS required for
reporting:

\begin{verbatim}
<link rel="stylesheet" type="text/css" href="lib/jasmine-<%= jasmineVersion %>/jasmine.css">
<script src="lib/jasmine-<%= jasmineVersion %>/jasmine.js"></script>
<script src="lib/jasmine-<%= jasmineVersion %>/jasmine-html.js"></script>
<script src="lib/jasmine-<%= jasmineVersion %>/boot.js"></script>
\end{verbatim}

Next come the sources being tested:

\begin{verbatim}
<!-- include source files here... -->
<script src="src/Player.js"></script>
<script src="src/Song.js"></script>
\end{verbatim}

Finally, some sample tests are included:

\begin{verbatim}
<!-- include spec files here... -->
<script src="spec/SpecHelper.js"></script>
<script src="spec/PlayerSpec.js"></script>
\end{verbatim}

\textbf{\emph{Note:}} Below this section of SpecRunner is code
responsible for running the actual tests. Given that we won't be
covering modifying this code, I'm going to skip reviewing it. I do
however encourage you to take a look through
\href{https://github.com/pivotal/jasmine/blob/master/lib/jasmine-core/example/spec/PlayerSpec.js}{PlayerSpec.js}
and
\href{https://github.com/pivotal/jasmine/blob/master/lib/jasmine-core/example/spec/SpecHelper.js}{SpecHelper.js}.
They're a useful basic example to go through how a minimal set of tests
might work.

Also note that for the purposes of introduction, some of the examples in
this section will be testing aspects of Backbone.js itself, just to give
you a feel for how Jasmine works. You generally will not need to write
testing ensuring a framework is working as expected.

\subsection{TDD With Backbone}\label{tdd-with-backbone}

When developing applications with Backbone, it can be necessary to test
both individual modules of code as well as models, views, collections,
and routers. Taking a TDD approach to testing, let's review some specs
for testing these Backbone components using the popular Backbone
\href{https://github.com/addyosmani/todomvc/tree/master/todo-example/backbone}{Todo}
application.

\subsection{Models}\label{models-2}

The complexity of Backbone models can vary greatly depending on what
your application is trying to achieve. In the following example, we're
going to test default values, attributes, state changes, and validation
rules.

First, we begin our suite for model testing using \texttt{describe()}:

\begin{Shaded}
\begin{Highlighting}[]
\FunctionTok{describe}\NormalTok{(}\StringTok{'Tests for Todo'}\NormalTok{, }\KeywordTok{function}\NormalTok{() \{}
\end{Highlighting}
\end{Shaded}

Models should ideally have default values for attributes. This helps
ensure that when creating instances without a value set for any specific
attribute, a default one (e.g., an empty string) is used instead. The
idea here is to allow your application to interact with models without
any unexpected behavior.

In the following spec, we create a new Todo without any attributes
passed then check to find out what the value of the \texttt{text}
attribute is. As no value has been set, we expect a default value of
\texttt{''} to be returned.

\begin{Shaded}
\begin{Highlighting}[]
\FunctionTok{it}\NormalTok{(}\StringTok{'Can be created with default values for its attributes.'}\NormalTok{, }\KeywordTok{function}\NormalTok{() \{}
    \KeywordTok{var} \NormalTok{todo = }\KeywordTok{new} \FunctionTok{Todo}\NormalTok{();}
    \FunctionTok{expect}\NormalTok{(}\OtherTok{todo}\NormalTok{.}\FunctionTok{get}\NormalTok{(}\StringTok{'text'}\NormalTok{)).}\FunctionTok{toBe}\NormalTok{(}\StringTok{''}\NormalTok{);}
\NormalTok{\});}
\end{Highlighting}
\end{Shaded}

If testing this spec before your models have been written, you'll incur
a failing test, as expected. What's required for the spec to pass is a
default value for the attribute \texttt{text}. We can set this and some
other useful defaults (which we'll be using shortly) in our Todo model
as follows:

\begin{Shaded}
\begin{Highlighting}[]
\OtherTok{window}\NormalTok{.}\FunctionTok{Todo} \NormalTok{= }\OtherTok{Backbone}\NormalTok{.}\OtherTok{Model}\NormalTok{.}\FunctionTok{extend}\NormalTok{(\{}

    \DataTypeTok{defaults}\NormalTok{: \{}
      \DataTypeTok{text}\NormalTok{: }\StringTok{''}\NormalTok{,}
      \DataTypeTok{done}\NormalTok{:  }\KeywordTok{false}\NormalTok{,}
      \DataTypeTok{order}\NormalTok{: }\DecValTok{0}
    \NormalTok{\}}
\end{Highlighting}
\end{Shaded}

Next, it is common to include validation logic in your models to ensure
that input passed from users or other modules in the application are
valid.

A Todo app may wish to validate the text input supplied in case it
contains rude words. Similarly if we're storing the \texttt{done} state
of a Todo item using booleans, we need to validate that truthy/falsy
values are passed and not just any arbitrary string.

In the following spec, we take advantage of the fact that validations
which fail model.validate() trigger an ``invalid'' event. This allows us
to test if validations are correctly failing when invalid input is
supplied.

We create an errorCallback spy using Jasmine's built in
\texttt{createSpy()} method which allows us to spy on the invalid event
as follows:

\begin{Shaded}
\begin{Highlighting}[]
\FunctionTok{it}\NormalTok{(}\StringTok{'Can contain custom validation rules, and will trigger an invalid event on failed validation.'}\NormalTok{, }\KeywordTok{function}\NormalTok{() \{}

    \KeywordTok{var} \NormalTok{errorCallback = }\OtherTok{jasmine}\NormalTok{.}\FunctionTok{createSpy}\NormalTok{(}\StringTok{'-invalid event callback-'}\NormalTok{);}

    \KeywordTok{var} \NormalTok{todo = }\KeywordTok{new} \FunctionTok{Todo}\NormalTok{();}

    \OtherTok{todo}\NormalTok{.}\FunctionTok{on}\NormalTok{(}\StringTok{'invalid'}\NormalTok{, errorCallback);}

    \CommentTok{// What would you need to set on the todo properties to}
    \CommentTok{// cause validation to fail?}

    \OtherTok{todo}\NormalTok{.}\FunctionTok{set}\NormalTok{(\{}\DataTypeTok{done}\NormalTok{:}\StringTok{'a non-boolean value'}\NormalTok{\});}

    \KeywordTok{var} \NormalTok{errorArgs = }\OtherTok{errorCallback}\NormalTok{.}\OtherTok{mostRecentCall}\NormalTok{.}\FunctionTok{args}\NormalTok{;}

    \FunctionTok{expect}\NormalTok{(errorArgs).}\FunctionTok{toBeDefined}\NormalTok{();}
    \FunctionTok{expect}\NormalTok{(errorArgs[}\DecValTok{0}\NormalTok{]).}\FunctionTok{toBe}\NormalTok{(todo);}
    \FunctionTok{expect}\NormalTok{(errorArgs[}\DecValTok{1}\NormalTok{]).}\FunctionTok{toBe}\NormalTok{(}\StringTok{'Todo.done must be a boolean value.'}\NormalTok{);}
\NormalTok{\});}
\end{Highlighting}
\end{Shaded}

The code to make the above failing test support validation is relatively
simple. In our model, we override the validate() method (as recommended
in the Backbone docs), checking to make sure a model both has a `done'
property and that its value is a valid boolean before allowing it to
pass.

\begin{Shaded}
\begin{Highlighting}[]
\NormalTok{validate: }\KeywordTok{function}\NormalTok{(attrs) \{}
    \KeywordTok{if} \NormalTok{(}\OtherTok{attrs}\NormalTok{.}\FunctionTok{hasOwnProperty}\NormalTok{(}\StringTok{'done'}\NormalTok{) && !}\OtherTok{_}\NormalTok{.}\FunctionTok{isBoolean}\NormalTok{(}\OtherTok{attrs}\NormalTok{.}\FunctionTok{done}\NormalTok{)) \{}
        \KeywordTok{return} \StringTok{'Todo.done must be a boolean value.'}\NormalTok{;}
    \NormalTok{\}}
\NormalTok{\}}
\end{Highlighting}
\end{Shaded}

If you would like to review the final code for our Todo model, you can
find it below:

\begin{Shaded}
\begin{Highlighting}[]

\OtherTok{window}\NormalTok{.}\FunctionTok{Todo} \NormalTok{= }\OtherTok{Backbone}\NormalTok{.}\OtherTok{Model}\NormalTok{.}\FunctionTok{extend}\NormalTok{(\{}

    \DataTypeTok{defaults}\NormalTok{: \{}
      \DataTypeTok{text}\NormalTok{: }\StringTok{''}\NormalTok{,}
      \DataTypeTok{done}\NormalTok{:  }\KeywordTok{false}\NormalTok{,}
      \DataTypeTok{order}\NormalTok{: }\DecValTok{0}
    \NormalTok{\},}

    \DataTypeTok{initialize}\NormalTok{: }\KeywordTok{function}\NormalTok{() \{}
        \KeywordTok{this}\NormalTok{.}\FunctionTok{set}\NormalTok{(\{}\DataTypeTok{text}\NormalTok{: }\KeywordTok{this}\NormalTok{.}\FunctionTok{get}\NormalTok{(}\StringTok{'text'}\NormalTok{)\}, \{}\DataTypeTok{silent}\NormalTok{: }\KeywordTok{true}\NormalTok{\});}
    \NormalTok{\},}

    \DataTypeTok{validate}\NormalTok{: }\KeywordTok{function}\NormalTok{(attrs) \{}
        \KeywordTok{if} \NormalTok{(}\OtherTok{attrs}\NormalTok{.}\FunctionTok{hasOwnProperty}\NormalTok{(}\StringTok{'done'}\NormalTok{) && !}\OtherTok{_}\NormalTok{.}\FunctionTok{isBoolean}\NormalTok{(}\OtherTok{attrs}\NormalTok{.}\FunctionTok{done}\NormalTok{)) \{}
            \KeywordTok{return} \StringTok{'Todo.done must be a boolean value.'}\NormalTok{;}
        \NormalTok{\}}
    \NormalTok{\},}

    \DataTypeTok{toggle}\NormalTok{: }\KeywordTok{function}\NormalTok{() \{}
        \KeywordTok{this}\NormalTok{.}\FunctionTok{save}\NormalTok{(\{}\DataTypeTok{done}\NormalTok{: !}\KeywordTok{this}\NormalTok{.}\FunctionTok{get}\NormalTok{(}\StringTok{'done'}\NormalTok{)\});}
    \NormalTok{\}}

\NormalTok{\});}
\end{Highlighting}
\end{Shaded}

\subsection{Collections}\label{collections-1}

We now need to define specs to test a Backbone collection of Todo models
(a TodoList). Collections are responsible for a number of list tasks
including managing order and filtering.

A few specific specs that come to mind when working with collections
are:

\begin{itemize}
\itemsep1pt\parskip0pt\parsep0pt
\item
  Making sure we can add new Todo models as both objects and arrays
\item
  Attribute testing to make sure attributes such as the base URL of the
  collection are values we expect
\item
  Purposefully adding items with a status of \texttt{done:true} and
  checking against how many items the collection thinks have been
  completed vs.~those that are remaining
\end{itemize}

In this section we're going to cover the first two of these with the
third left as an extended exercise you can try on your own.

Testing that Todo models can be added to a collection as objects or
arrays is relatively trivial. First, we initialize a new TodoList
collection and check to make sure its length (i.e., the number of Todo
models it contains) is 0. Next, we add new Todos, both as objects and
arrays, checking the length property of the collection at each stage to
ensure the overall count is what we expect:

\begin{Shaded}
\begin{Highlighting}[]
\FunctionTok{describe}\NormalTok{(}\StringTok{'Tests for TodoList'}\NormalTok{, }\KeywordTok{function}\NormalTok{() \{}

    \FunctionTok{it}\NormalTok{(}\StringTok{'Can add Model instances as objects and arrays.'}\NormalTok{, }\KeywordTok{function}\NormalTok{() \{}
        \KeywordTok{var} \NormalTok{todos = }\KeywordTok{new} \FunctionTok{TodoList}\NormalTok{();}

        \FunctionTok{expect}\NormalTok{(}\OtherTok{todos}\NormalTok{.}\FunctionTok{length}\NormalTok{).}\FunctionTok{toBe}\NormalTok{(}\DecValTok{0}\NormalTok{);}

        \OtherTok{todos}\NormalTok{.}\FunctionTok{add}\NormalTok{(\{ }\DataTypeTok{text}\NormalTok{: }\StringTok{'Clean the kitchen'} \NormalTok{\});}

        \CommentTok{// how many todos have been added so far?}
        \FunctionTok{expect}\NormalTok{(}\OtherTok{todos}\NormalTok{.}\FunctionTok{length}\NormalTok{).}\FunctionTok{toBe}\NormalTok{(}\DecValTok{1}\NormalTok{);}

        \OtherTok{todos}\NormalTok{.}\FunctionTok{add}\NormalTok{([}
            \NormalTok{\{ }\DataTypeTok{text}\NormalTok{: }\StringTok{'Do the laundry'}\NormalTok{, }\DataTypeTok{done}\NormalTok{: }\KeywordTok{true} \NormalTok{\},}
            \NormalTok{\{ }\DataTypeTok{text}\NormalTok{: }\StringTok{'Go to the gym'}\NormalTok{\}}
        \NormalTok{]);}

        \CommentTok{// how many are there in total now?}
        \FunctionTok{expect}\NormalTok{(}\OtherTok{todos}\NormalTok{.}\FunctionTok{length}\NormalTok{).}\FunctionTok{toBe}\NormalTok{(}\DecValTok{3}\NormalTok{);}
    \NormalTok{\});}
\NormalTok{...}
\end{Highlighting}
\end{Shaded}

Similar to model attributes, it's also quite straight-forward to test
attributes in collections. Here we have a spec that ensures the
collection url (i.e., the url reference to the collection's location on
the server) is what we expect it to be:

\begin{Shaded}
\begin{Highlighting}[]
\FunctionTok{it}\NormalTok{(}\StringTok{'Can have a url property to define the basic url structure for all contained models.'}\NormalTok{, }\KeywordTok{function}\NormalTok{() \{}
        \KeywordTok{var} \NormalTok{todos = }\KeywordTok{new} \FunctionTok{TodoList}\NormalTok{();}

        \CommentTok{// what has been specified as the url base in our model?}
        \FunctionTok{expect}\NormalTok{(}\OtherTok{todos}\NormalTok{.}\FunctionTok{url}\NormalTok{).}\FunctionTok{toBe}\NormalTok{(}\StringTok{'/todos/'}\NormalTok{);}
\NormalTok{\});}
\end{Highlighting}
\end{Shaded}

For the third spec (which you will write as an exercise), note that the
implementation for our collection will have methods for filtering how
many Todo items are done and how many are remaining - we'll call these
\texttt{done()} and \texttt{remaining()}. Consider writing a spec which
creates a new collection and adds one new model that has a preset
\texttt{done} state of \texttt{true} and two others that have the
default \texttt{done} state of \texttt{false}. Testing the length of
what's returned using \texttt{done()} and \texttt{remaining()} will tell
us whether the state management in our application is working or needs a
little tweaking.

The final implementation for our TodoList collection can be found below:

\begin{Shaded}
\begin{Highlighting}[]
 \OtherTok{window}\NormalTok{.}\FunctionTok{TodoList} \NormalTok{= }\OtherTok{Backbone}\NormalTok{.}\OtherTok{Collection}\NormalTok{.}\FunctionTok{extend}\NormalTok{(\{}

        \DataTypeTok{model}\NormalTok{: Todo,}

        \DataTypeTok{url}\NormalTok{: }\StringTok{'/todos/'}\NormalTok{,}

        \DataTypeTok{done}\NormalTok{: }\KeywordTok{function}\NormalTok{() \{}
            \KeywordTok{return} \KeywordTok{this}\NormalTok{.}\FunctionTok{filter}\NormalTok{(}\KeywordTok{function}\NormalTok{(todo) \{ }\KeywordTok{return} \OtherTok{todo}\NormalTok{.}\FunctionTok{get}\NormalTok{(}\StringTok{'done'}\NormalTok{); \});}
        \NormalTok{\},}

        \DataTypeTok{remaining}\NormalTok{: }\KeywordTok{function}\NormalTok{() \{}
            \KeywordTok{return} \KeywordTok{this}\NormalTok{.}\OtherTok{without}\NormalTok{.}\FunctionTok{apply}\NormalTok{(}\KeywordTok{this}\NormalTok{, }\KeywordTok{this}\NormalTok{.}\FunctionTok{done}\NormalTok{());}
        \NormalTok{\},}

        \DataTypeTok{nextOrder}\NormalTok{: }\KeywordTok{function}\NormalTok{() \{}
            \KeywordTok{if} \NormalTok{(!}\KeywordTok{this}\NormalTok{.}\FunctionTok{length}\NormalTok{) \{}
                \KeywordTok{return} \DecValTok{1}\NormalTok{;}
            \NormalTok{\}}

            \KeywordTok{return} \KeywordTok{this}\NormalTok{.}\FunctionTok{last}\NormalTok{().}\FunctionTok{get}\NormalTok{(}\StringTok{'order'}\NormalTok{) + }\DecValTok{1}\NormalTok{;}
        \NormalTok{\},}

        \DataTypeTok{comparator}\NormalTok{: }\KeywordTok{function}\NormalTok{(todo) \{}
            \KeywordTok{return} \OtherTok{todo}\NormalTok{.}\FunctionTok{get}\NormalTok{(}\StringTok{'order'}\NormalTok{);}
        \NormalTok{\}}

    \NormalTok{\});}
\end{Highlighting}
\end{Shaded}

\subsection{Views}\label{views-2}

Before we take a look at testing Backbone views, let's briefly review a
jQuery plugin that can assist with writing Jasmine specs for them.

\textbf{The Jasmine jQuery Plugin}

As we know our Todo application will be using jQuery for DOM
manipulation, there's a useful jQuery plugin called
\href{https://github.com/velesin/jasmine-jquery}{jasmine-jquery} we can
use to help simplify BDD testing of the rendering performed by our
views.

The plugin provides a number of additional Jasmine
\href{https://github.com/pivotal/jasmine/wiki/Matchers}{matchers} to
help test jQuery-wrapped sets such as:

\begin{itemize}
\itemsep1pt\parskip0pt\parsep0pt
\item
  \texttt{toBe(jQuerySelector)} e.g.,
  \texttt{expect(\$('\textless{}div id="some-id"\textgreater{}\textless{}/div\textgreater{}')).toBe('div\#some-id')}
\item
  \texttt{toBeChecked()} e.g.,
  \texttt{expect(\$('\textless{}input type="checkbox" checked="checked"/\textgreater{}')).toBeChecked()}
\item
  \texttt{toBeSelected()} e.g.,
  \texttt{expect(\$('\textless{}option selected="selected"\textgreater{}\textless{}/option\textgreater{}')).toBeSelected()}
\end{itemize}

and \href{https://github.com/velesin/jasmine-jquery}{many others}. The
complete list of matchers supported can be found on the project
homepage. It's useful to know that similar to the standard Jasmine
matchers, the custom matchers above can be inverted using the .not
prefix (i.e \texttt{expect(x).not.toBe(y)}):

\begin{Shaded}
\begin{Highlighting}[]
\FunctionTok{expect}\NormalTok{(}\FunctionTok{$}\NormalTok{(}\StringTok{'<div>I am an example</div>'}\NormalTok{)).}\OtherTok{not}\NormalTok{.}\FunctionTok{toHaveText}\NormalTok{(}\OtherTok{/other/}\NormalTok{)}
\end{Highlighting}
\end{Shaded}

jasmine-jquery also includes a fixtures module that can be used to load
arbitrary HTML content we wish to use in our tests. Fixtures can be used
as follows:

Include some HTML in an external fixtures file:

some.fixture.html:
\texttt{\textless{}div id="sample-fixture"\textgreater{}some HTML content\textless{}/div\textgreater{}}

Then inside our actual test we would load it as follows:

\begin{Shaded}
\begin{Highlighting}[]
\FunctionTok{loadFixtures}\NormalTok{(}\StringTok{'some.fixture.html'}\NormalTok{)}
\FunctionTok{$}\NormalTok{(}\StringTok{'some-fixture'}\NormalTok{).}\FunctionTok{myTestedPlugin}\NormalTok{();}
\FunctionTok{expect}\NormalTok{(}\FunctionTok{$}\NormalTok{(}\StringTok{'#some-fixture'}\NormalTok{)).}\FunctionTok{to}\NormalTok{<the rest of your matcher would go here>}
\end{Highlighting}
\end{Shaded}

The jasmine-jquery plugin loads fixtures from a directory named
spec/javascripts/fixtures by default. If you wish to configure this path
you can do so by initially setting
\texttt{jasmine.getFixtures().fixturesPath = 'your custom path'}.

Finally, jasmine-jquery includes support for spying on jQuery events
without the need for any extra plumbing work. This can be done using the
\texttt{spyOnEvent()} and
\texttt{assert(eventName).toHaveBeenTriggered(selector)} functions. For
example:

\begin{Shaded}
\begin{Highlighting}[]
\FunctionTok{spyOnEvent}\NormalTok{(}\FunctionTok{$}\NormalTok{(}\StringTok{'#el'}\NormalTok{), }\StringTok{'click'}\NormalTok{);}
\FunctionTok{$}\NormalTok{(}\StringTok{'#el'}\NormalTok{).}\FunctionTok{click}\NormalTok{();}
\FunctionTok{expect}\NormalTok{(}\StringTok{'click'}\NormalTok{).}\FunctionTok{toHaveBeenTriggeredOn}\NormalTok{(}\FunctionTok{$}\NormalTok{(}\StringTok{'#el'}\NormalTok{));}
\end{Highlighting}
\end{Shaded}

\subsubsection{View testing}\label{view-testing}

In this section we will review the three dimensions of specs writing for
Backbone Views: initial setup, view rendering, and templating. The
latter two of these are the most commonly tested, however we'll see
shortly why writing specs for the initialization of your views can also
be of benefit.

\paragraph{Initial setup}\label{initial-setup}

At their most basic, specs for Backbone views should validate that they
are being correctly tied to specific DOM elements and are backed by
valid data models. The reason to consider doing this is that these specs
can identify issues which will trip up more complex tests later on.
Also, they're fairly simple to write given the overall value offered.

To help ensure a consistent testing setup for our specs, we use
\texttt{beforeEach()} to append both an empty
\texttt{\textless{}ul\textgreater{}} (\#todoList) to the DOM and
initialize a new instance of a TodoView using an empty Todo model.
\texttt{afterEach()} is used to remove the previous \#todoList
\texttt{\textless{}ul\textgreater{}} as well as the previous instance of
the view.

\begin{Shaded}
\begin{Highlighting}[]
\FunctionTok{describe}\NormalTok{(}\StringTok{'Tests for TodoView'}\NormalTok{, }\KeywordTok{function}\NormalTok{() \{}

    \FunctionTok{beforeEach}\NormalTok{(}\KeywordTok{function}\NormalTok{() \{}
        \FunctionTok{$}\NormalTok{(}\StringTok{'body'}\NormalTok{).}\FunctionTok{append}\NormalTok{(}\StringTok{'<ul id="todoList"></ul>'}\NormalTok{);}
        \KeywordTok{this}\NormalTok{.}\FunctionTok{todoView} \NormalTok{= }\KeywordTok{new} \FunctionTok{TodoView}\NormalTok{(\{ }\DataTypeTok{model}\NormalTok{: }\KeywordTok{new} \FunctionTok{Todo}\NormalTok{() \});}
    \NormalTok{\});}


    \FunctionTok{afterEach}\NormalTok{(}\KeywordTok{function}\NormalTok{() \{}
        \KeywordTok{this}\NormalTok{.}\OtherTok{todoView}\NormalTok{.}\FunctionTok{remove}\NormalTok{();}
        \FunctionTok{$}\NormalTok{(}\StringTok{'#todoList'}\NormalTok{).}\FunctionTok{remove}\NormalTok{();}
    \NormalTok{\});}

\NormalTok{...}
\end{Highlighting}
\end{Shaded}

The first spec useful to write is a check that the TodoView we've
created is using the correct \texttt{tagName} (element or className).
The purpose of this test is to make sure it's been correctly tied to a
DOM element when it was created.

Backbone views typically create empty DOM elements once initialized,
however these elements are not attached to the visible DOM in order to
allow them to be constructed without an impact on the performance of
rendering.

\begin{Shaded}
\begin{Highlighting}[]
\FunctionTok{it}\NormalTok{(}\StringTok{'Should be tied to a DOM element when created, based off the property provided.'}\NormalTok{, }\KeywordTok{function}\NormalTok{() \{}
    \CommentTok{//what html element tag name represents this view?}
    \FunctionTok{expect}\NormalTok{(}\KeywordTok{this}\NormalTok{.}\OtherTok{todoView}\NormalTok{.}\OtherTok{el}\NormalTok{.}\OtherTok{tagName}\NormalTok{.}\FunctionTok{toLowerCase}\NormalTok{()).}\FunctionTok{toBe}\NormalTok{(}\StringTok{'li'}\NormalTok{);}
\NormalTok{\});}
\end{Highlighting}
\end{Shaded}

Once again, if the TodoView has not already been written, we will
experience failing specs. Thankfully, solving this is as simple as
creating a new Backbone.View with a specific \texttt{tagName}.

\begin{Shaded}
\begin{Highlighting}[]
\KeywordTok{var} \NormalTok{todoView = }\OtherTok{Backbone}\NormalTok{.}\OtherTok{View}\NormalTok{.}\FunctionTok{extend}\NormalTok{(\{}
    \DataTypeTok{tagName}\NormalTok{:  }\StringTok{'li'}
\NormalTok{\});}
\end{Highlighting}
\end{Shaded}

If instead of testing against the \texttt{tagName} you would prefer to
use a className instead, we can take advantage of jasmine-jquery's
\texttt{toHaveClass()} matcher:

\begin{verbatim}
it('Should have a class of "todos"', function(){
   expect(this.todoView.$el).toHaveClass('todos');
});
\end{verbatim}

The \texttt{toHaveClass()} matcher operates on jQuery objects and if the
plugin hadn't been used, an exception would have been thrown. It is of
course also possible to test for the className by accessing el.className
if you don't use jasmine-jquery.

You may have noticed that in \texttt{beforeEach()}, we passed our view
an initial (albeit unfilled) Todo model. Views should be backed by a
model instance which provides data. As this is quite important to our
view's ability to function, we can write a spec to ensure a model is
defined (using the \texttt{toBeDefined()} matcher) and then test
attributes of the model to ensure defaults both exist and are the values
we expect them to be.

\begin{Shaded}
\begin{Highlighting}[]
\FunctionTok{it}\NormalTok{(}\StringTok{'Is backed by a model instance, which provides the data.'}\NormalTok{, }\KeywordTok{function}\NormalTok{() \{}

    \FunctionTok{expect}\NormalTok{(}\KeywordTok{this}\NormalTok{.}\OtherTok{todoView}\NormalTok{.}\FunctionTok{model}\NormalTok{).}\FunctionTok{toBeDefined}\NormalTok{();}

    \CommentTok{// what's the value for Todo.get('done') here?}
    \FunctionTok{expect}\NormalTok{(}\KeywordTok{this}\NormalTok{.}\OtherTok{todoView}\NormalTok{.}\OtherTok{model}\NormalTok{.}\FunctionTok{get}\NormalTok{(}\StringTok{'done'}\NormalTok{)).}\FunctionTok{toBe}\NormalTok{(}\KeywordTok{false}\NormalTok{); }\CommentTok{//or toBeFalsy()}
\NormalTok{\});}
\end{Highlighting}
\end{Shaded}

\paragraph{View rendering}\label{view-rendering}

Next we're going to take a look at writing specs for view rendering.
Specifically, we want to test that our TodoView elements are actually
rendering as expected.

In smaller applications, those new to BDD might argue that visual
confirmation of view rendering could replace unit testing of views. The
reality is that when dealing with applications that might grow to a
large number of views, it makes sense to automate this process as much
as possible from the get-go. There are also aspects of rendering that
require verification beyond what is visually presented on-screen (which
we'll see very shortly).

We're going to begin testing views by writing two specs. The first spec
will check that the view's \texttt{render()} method is correctly
returning the view instance, which is necessary for chaining. Our second
spec will check that the HTML produced is exactly what we expect based
on the properties of the model instance that's been associated with our
TodoView.

Unlike some of the previous specs we've covered, this section will make
greater use of \texttt{beforeEach()} to both demonstrate how to use
nested suites and also ensure a consistent set of conditions for our
specs. In our first example we're simply going to create a sample model
(based on Todo) and instantiate a TodoView with it.

\begin{Shaded}
\begin{Highlighting}[]
\FunctionTok{describe}\NormalTok{(}\StringTok{'TodoView'}\NormalTok{, }\KeywordTok{function}\NormalTok{() \{}

  \FunctionTok{beforeEach}\NormalTok{(}\KeywordTok{function}\NormalTok{() \{}
    \KeywordTok{this}\NormalTok{.}\FunctionTok{model} \NormalTok{= }\KeywordTok{new} \OtherTok{Backbone}\NormalTok{.}\FunctionTok{Model}\NormalTok{(\{}
      \DataTypeTok{text}\NormalTok{: }\StringTok{'My Todo'}\NormalTok{,}
      \DataTypeTok{order}\NormalTok{: }\DecValTok{1}\NormalTok{,}
      \DataTypeTok{done}\NormalTok{: }\KeywordTok{false}
    \NormalTok{\});}
    \KeywordTok{this}\NormalTok{.}\FunctionTok{view} \NormalTok{= }\KeywordTok{new} \FunctionTok{TodoView}\NormalTok{(\{}\DataTypeTok{model}\NormalTok{:}\KeywordTok{this}\NormalTok{.}\FunctionTok{model}\NormalTok{\});}
  \NormalTok{\});}

  \FunctionTok{describe}\NormalTok{(}\StringTok{'Rendering'}\NormalTok{, }\KeywordTok{function}\NormalTok{() \{}

    \FunctionTok{it}\NormalTok{(}\StringTok{'returns the view object'}\NormalTok{, }\KeywordTok{function}\NormalTok{() \{}
      \FunctionTok{expect}\NormalTok{(}\KeywordTok{this}\NormalTok{.}\OtherTok{view}\NormalTok{.}\FunctionTok{render}\NormalTok{()).}\FunctionTok{toEqual}\NormalTok{(}\KeywordTok{this}\NormalTok{.}\FunctionTok{view}\NormalTok{);}
    \NormalTok{\});}

    \FunctionTok{it}\NormalTok{(}\StringTok{'produces the correct HTML'}\NormalTok{, }\KeywordTok{function}\NormalTok{() \{}
      \KeywordTok{this}\NormalTok{.}\OtherTok{view}\NormalTok{.}\FunctionTok{render}\NormalTok{();}

      \CommentTok{// let's use jasmine-jquery's toContain() to avoid}
      \CommentTok{// testing for the complete content of a todo's markup}
      \FunctionTok{expect}\NormalTok{(}\KeywordTok{this}\NormalTok{.}\OtherTok{view}\NormalTok{.}\OtherTok{el}\NormalTok{.}\FunctionTok{innerHTML}\NormalTok{)}
        \NormalTok{.}\FunctionTok{toContain}\NormalTok{(}\StringTok{'<label class="todo-content">My Todo</label>'}\NormalTok{);}
    \NormalTok{\});}

  \NormalTok{\});}

\NormalTok{\});}
\end{Highlighting}
\end{Shaded}

When these specs are run, only the second one (`produces the correct
HTML') fails. Our first spec (`returns the view object'), which is
testing that the TodoView instance is returned from \texttt{render()},
passes since this is Backbone's default behavior and we haven't
overwritten the \texttt{render()} method with our own version yet.

\textbf{Note:} For the purposes of maintaining readability, all template
examples in this section will use a minimal version of the following
Todo view template. As it's relatively trivial to expand this, please
feel free to refer to this sample if needed:

\begin{verbatim}
<div class="todo <%= done ? 'done' : '' %>">
        <div class="display">
          <input class="check" type="checkbox" <%= done ? 'checked="checked"' : '' %> />
          <label class="todo-content"><%= text %></label>
          <span class="todo-destroy"></span>
        </div>
        <div class="edit">
          <input class="todo-input" type="text" value="<%= content %>" />
        </div>
</div>
\end{verbatim}

The second spec fails with the following message:

\texttt{Expected '' to contain '\textless{}label class="todo-content"\textgreater{}My Todo\textless{}/label\textgreater{}'.}

The reason for this is the default behavior for render() doesn't create
any markup. Let's write a replacement for render() which fixes this:

\begin{Shaded}
\begin{Highlighting}[]
\NormalTok{render: }\KeywordTok{function}\NormalTok{() \{}
  \KeywordTok{var} \NormalTok{template = }\StringTok{'<label class="todo-content">+++PLACEHOLDER+++</label>'}\NormalTok{;}
  \KeywordTok{var} \NormalTok{output = template}
    \NormalTok{.}\FunctionTok{replace}\NormalTok{(}\StringTok{'+++PLACEHOLDER+++'}\NormalTok{, }\KeywordTok{this}\NormalTok{.}\OtherTok{model}\NormalTok{.}\FunctionTok{get}\NormalTok{(}\StringTok{'text'}\NormalTok{));}
  \KeywordTok{this}\NormalTok{.}\OtherTok{$el}\NormalTok{.}\FunctionTok{html}\NormalTok{(output);}
  \KeywordTok{return} \KeywordTok{this}\NormalTok{;}
\NormalTok{\}}
\end{Highlighting}
\end{Shaded}

The above specifies an inline string template and replaces fields found
in the template within the ``+++PLACEHOLDER+++'' blocks with their
corresponding values from the associated model. As we're also returning
the TodoView instance from the method, the first spec will still pass.

It would be impossible to discuss unit testing without mentioning
fixtures. Fixtures typically contain test data (e.g., HTML) that is
loaded in when needed (either locally or from an external file) for unit
testing. So far we've been establishing jQuery expectations based on the
view's el property. This works for a number of cases, however, there are
instances where it may be necessary to render markup into the document.
The most optimal way to handle this within specs is through using
fixtures (another feature brought to us by the jasmine-jquery plugin).

Re-writing the last spec to use fixtures would look as follows:

\begin{Shaded}
\begin{Highlighting}[]
\FunctionTok{describe}\NormalTok{(}\StringTok{'TodoView'}\NormalTok{, }\KeywordTok{function}\NormalTok{() \{}

  \FunctionTok{beforeEach}\NormalTok{(}\KeywordTok{function}\NormalTok{() \{}
    \NormalTok{...}
    \FunctionTok{setFixtures}\NormalTok{(}\StringTok{'<ul class="todos"></ul>'}\NormalTok{);}
  \NormalTok{\});}

  \NormalTok{...}

  \FunctionTok{describe}\NormalTok{(}\StringTok{'Template'}\NormalTok{, }\KeywordTok{function}\NormalTok{() \{}

    \FunctionTok{beforeEach}\NormalTok{(}\KeywordTok{function}\NormalTok{() \{}
      \FunctionTok{$}\NormalTok{(}\StringTok{'.todos'}\NormalTok{).}\FunctionTok{append}\NormalTok{(}\KeywordTok{this}\NormalTok{.}\OtherTok{view}\NormalTok{.}\FunctionTok{render}\NormalTok{().}\FunctionTok{el}\NormalTok{);}
    \NormalTok{\});}

    \FunctionTok{it}\NormalTok{(}\StringTok{'has the correct text content'}\NormalTok{, }\KeywordTok{function}\NormalTok{() \{}
      \FunctionTok{expect}\NormalTok{(}\FunctionTok{$}\NormalTok{(}\StringTok{'.todos'}\NormalTok{).}\FunctionTok{find}\NormalTok{(}\StringTok{'.todo-content'}\NormalTok{))}
        \NormalTok{.}\FunctionTok{toHaveText}\NormalTok{(}\StringTok{'My Todo'}\NormalTok{);}
    \NormalTok{\});}

  \NormalTok{\});}

\NormalTok{\});}
\end{Highlighting}
\end{Shaded}

What we're now doing in the above spec is appending the rendered todo
item into the fixture. We then set expectations against the fixture,
which may be something desirable when a view is setup against an element
which already exists in the DOM. It would be necessary to provide both
the fixture and test the \texttt{el} property correctly picking up the
element expected when the view is instantiated.

\paragraph{Rendering with a templating
system}\label{rendering-with-a-templating-system}

When a user marks a Todo item as complete (done), we may wish to provide
them with visual feedback (such as a striked line through the text) to
differentiate the item from those that are remaining. This can be done
by attaching a new class to the item. Let's begin by writing a test:

\begin{Shaded}
\begin{Highlighting}[]
\FunctionTok{describe}\NormalTok{(}\StringTok{'When a todo is done'}\NormalTok{, }\KeywordTok{function}\NormalTok{() \{}

  \FunctionTok{beforeEach}\NormalTok{(}\KeywordTok{function}\NormalTok{() \{}
    \KeywordTok{this}\NormalTok{.}\OtherTok{model}\NormalTok{.}\FunctionTok{set}\NormalTok{(\{}\DataTypeTok{done}\NormalTok{: }\KeywordTok{true}\NormalTok{\}, \{}\DataTypeTok{silent}\NormalTok{: }\KeywordTok{true}\NormalTok{\});}
    \FunctionTok{$}\NormalTok{(}\StringTok{'.todos'}\NormalTok{).}\FunctionTok{append}\NormalTok{(}\KeywordTok{this}\NormalTok{.}\OtherTok{view}\NormalTok{.}\FunctionTok{render}\NormalTok{().}\FunctionTok{el}\NormalTok{);}
  \NormalTok{\});}

  \FunctionTok{it}\NormalTok{(}\StringTok{'has a done class'}\NormalTok{, }\KeywordTok{function}\NormalTok{() \{}
    \FunctionTok{expect}\NormalTok{(}\FunctionTok{$}\NormalTok{(}\StringTok{'.todos .todo-content:first-child'}\NormalTok{))}
      \NormalTok{.}\FunctionTok{toHaveClass}\NormalTok{(}\StringTok{'done'}\NormalTok{);}
  \NormalTok{\});}

\NormalTok{\});}
\end{Highlighting}
\end{Shaded}

This will fail with the following message:

\texttt{Expected '\textless{}label class="todo-content"\textgreater{}My Todo\textless{}/label\textgreater{}' to have class 'done'.}

which can be fixed in the existing render() method as follows:

\begin{Shaded}
\begin{Highlighting}[]
\NormalTok{render: }\KeywordTok{function}\NormalTok{() \{}
  \KeywordTok{var} \NormalTok{template = }\StringTok{'<label class="todo-content">'} \NormalTok{+}
    \StringTok{'<%= text %></label>'}\NormalTok{;}
  \KeywordTok{var} \NormalTok{output = template}
    \NormalTok{.}\FunctionTok{replace}\NormalTok{(}\StringTok{'<%= text %>'}\NormalTok{, }\KeywordTok{this}\NormalTok{.}\OtherTok{model}\NormalTok{.}\FunctionTok{get}\NormalTok{(}\StringTok{'text'}\NormalTok{));}
  \KeywordTok{this}\NormalTok{.}\OtherTok{$el}\NormalTok{.}\FunctionTok{html}\NormalTok{(output);}
  \KeywordTok{if} \NormalTok{(}\KeywordTok{this}\NormalTok{.}\OtherTok{model}\NormalTok{.}\FunctionTok{get}\NormalTok{(}\StringTok{'done'}\NormalTok{)) \{}
    \KeywordTok{this}\NormalTok{.}\FunctionTok{$}\NormalTok{(}\StringTok{'.todo-content'}\NormalTok{).}\FunctionTok{addClass}\NormalTok{(}\StringTok{'done'}\NormalTok{);}
  \NormalTok{\}}
  \KeywordTok{return} \KeywordTok{this}\NormalTok{;}
\NormalTok{\}}
\end{Highlighting}
\end{Shaded}

However, this can get unwieldy fairly quickly. As the level of
complexity and logic in our templates increase, so do the challenges
associated with testing them. We can ease this process by taking
advantage of modern templating libraries, many of which have already
been demonstrated to work well with testing solutions such as Jasmine.

JavaScript templating systems (such as
\href{http://handlebarsjs.com/}{Handlebars},
\href{http://mustache.github.com/}{Mustache}, and Underscore's own
\href{http://underscorejs.org/\#template}{micro-templating}) support
conditional logic in template strings. What this effectively means is
that we can add if/else/ternery expressions inline which can then be
evaluated as needed, allowing us to build even more powerful templates.

In our case, we are going to use the micro-templating found in
Underscore.js as no additional files are required to use it and we can
easily modify our existing specs to use it without a great deal of
effort.

Assuming our template is defined using a script tag of ID
\texttt{myTemplate}:

\begin{verbatim}
<script type="text/template" id="myTemplate">
    <div class="todo <%= done ? 'done' : '' %>">
            <div class="display">
              <input class="check" type="checkbox" <%= done ? 'checked="checked"' : '' %> />
              <label class="todo-content"><%= text %></label>
              <span class="todo-destroy"></span>
            </div>
            <div class="edit">
              <input class="todo-input" type="text" value="<%= content %>" />
            </div>
    </div>
</script>
\end{verbatim}

Our TodoView can be modified to use Underscore templating as follows:

\begin{Shaded}
\begin{Highlighting}[]
\KeywordTok{var} \NormalTok{TodoView = }\OtherTok{Backbone}\NormalTok{.}\OtherTok{View}\NormalTok{.}\FunctionTok{extend}\NormalTok{(\{}

  \DataTypeTok{tagName}\NormalTok{: }\StringTok{'li'}\NormalTok{,}
  \DataTypeTok{template}\NormalTok{: }\OtherTok{_}\NormalTok{.}\FunctionTok{template}\NormalTok{(}\FunctionTok{$}\NormalTok{(}\StringTok{'#myTemplate'}\NormalTok{).}\FunctionTok{html}\NormalTok{()),}

  \DataTypeTok{initialize}\NormalTok{: }\KeywordTok{function}\NormalTok{(options) \{}
    \CommentTok{// ...}
  \NormalTok{\},}

  \DataTypeTok{render}\NormalTok{: }\KeywordTok{function}\NormalTok{() \{}
    \KeywordTok{this}\NormalTok{.}\OtherTok{$el}\NormalTok{.}\FunctionTok{html}\NormalTok{(}\KeywordTok{this}\NormalTok{.}\FunctionTok{template}\NormalTok{(}\KeywordTok{this}\NormalTok{.}\OtherTok{model}\NormalTok{.}\FunctionTok{attributes}\NormalTok{));}
    \KeywordTok{return} \KeywordTok{this}\NormalTok{;}
  \NormalTok{\},}

  \NormalTok{...}

\NormalTok{\});}
\end{Highlighting}
\end{Shaded}

So, what's going on here? We're first defining our template in a script
tag with a custom script type (e.g., type=``text/template''). As this
isn't a script type any browser understands, it's simply ignored,
however referencing the script by an id attribute allows the template to
be kept separate to other parts of the page.

In our view, we're the using the Underscore \texttt{\_.template()}
method to compile our template into a function that we can easily pass
model data to later on. In the line \texttt{this.model.toJSON()} we are
simply returning a copy of the model's attributes for JSON
stringification to the \texttt{template} method, creating a block of
HTML that can now be appended to the DOM.

Note: Ideally all of your template logic should exist outside of your
specs, either in individual template files or embedded using script tags
within your SpecRunner. This is generally more maintainable.

If you are working with much smaller templates and are not doing this,
there is however a useful trick that can be applied to automatically
create or extend templates in the Jasmine shared functional scope for
each test.

By creating a new directory (say, `templates') in the `spec' folder and
including a new script file with the following contents into
SpecRunner.html, we can manually add custom attributes representing
smaller templates we wish to use:

\begin{Shaded}
\begin{Highlighting}[]
\FunctionTok{beforeEach}\NormalTok{(}\KeywordTok{function}\NormalTok{() \{}
  \KeywordTok{this}\NormalTok{.}\FunctionTok{templates} \NormalTok{= }\OtherTok{_}\NormalTok{.}\FunctionTok{extend}\NormalTok{(}\KeywordTok{this}\NormalTok{.}\FunctionTok{templates} \NormalTok{|| \{\}, \{}
    \DataTypeTok{todo}\NormalTok{: }\StringTok{'<label class="todo-content">'} \NormalTok{+}
            \StringTok{'<%= text %>'} \NormalTok{+}
          \StringTok{'</label>'}
  \NormalTok{\});}
\NormalTok{\});}
\end{Highlighting}
\end{Shaded}

To finish this off, we simply update our existing spec to reference the
template when instantiating the TodoView:

\begin{Shaded}
\begin{Highlighting}[]
\FunctionTok{describe}\NormalTok{(}\StringTok{'TodoView'}\NormalTok{, }\KeywordTok{function}\NormalTok{() \{}

  \FunctionTok{beforeEach}\NormalTok{(}\KeywordTok{function}\NormalTok{() \{}
    \NormalTok{...}
    \KeywordTok{this}\NormalTok{.}\FunctionTok{view} \NormalTok{= }\KeywordTok{new} \FunctionTok{TodoView}\NormalTok{(\{}
      \DataTypeTok{model}\NormalTok{: }\KeywordTok{this}\NormalTok{.}\FunctionTok{model}\NormalTok{,}
      \DataTypeTok{template}\NormalTok{: }\KeywordTok{this}\NormalTok{.}\OtherTok{templates}\NormalTok{.}\FunctionTok{todo}
    \NormalTok{\});}
  \NormalTok{\});}

  \NormalTok{...}

\NormalTok{\});}
\end{Highlighting}
\end{Shaded}

The existing specs we've looked at would continue to pass using this
approach, leaving us free to adjust the template with some additional
conditional logic for Todos with a status of `done':

\begin{Shaded}
\begin{Highlighting}[]
\FunctionTok{beforeEach}\NormalTok{(}\KeywordTok{function}\NormalTok{() \{}
  \KeywordTok{this}\NormalTok{.}\FunctionTok{templates} \NormalTok{= }\OtherTok{_}\NormalTok{.}\FunctionTok{extend}\NormalTok{(}\KeywordTok{this}\NormalTok{.}\FunctionTok{templates} \NormalTok{|| \{\}, \{}
    \DataTypeTok{todo}\NormalTok{: }\StringTok{'<label class="todo-content <%= done ? '}\NormalTok{done}\StringTok{' : '' %>"'} \NormalTok{+}
            \StringTok{'<%= text %>'} \NormalTok{+}
          \StringTok{'</label>'}
  \NormalTok{\});}
\NormalTok{\});}
\end{Highlighting}
\end{Shaded}

This will now also pass without any issues, however as mentioned, this
last approach probably only makes sense if you're working with smaller,
highly dynamic templates.

\subsection{Conclusions}\label{conclusions-2}

We have now covered how to write Jasmine tests for Backbone.js models,
collections, and views. While testing routing can at times be desirable,
some developers feel it can be more optimal to leave this to third-party
tools such as Selenium, so do keep this in mind.

\subsection{Exercise}\label{exercise}

As an exercise, I recommend now trying the Jasmine Koans in
\texttt{practicals\textbackslash{}jasmine-koans} and trying to fix some
of the purposefully failing tests it has to offer. This is an excellent
way of not just learning how Jasmine specs and suites work, but working
through the examples (without peeking back) will also put your Backbone
skills to the test too.

\subsection{Further reading}\label{further-reading-1}

\begin{itemize}
\itemsep1pt\parskip0pt\parsep0pt
\item
  \href{http://tinnedfruit.com/2011/04/26/testing-backbone-apps-with-jasmine-sinon-3.html}{Testing
  Backbone Apps With SinonJS} by James Newbry
\item
  \href{http://japhr.blogspot.com/2011/11/jasmine-backbonejs-revisited.html}{Jasmine
  + Backbone Revisited}
\item
  \href{http://japhr.blogspot.com/2011/12/phantomjs-and-backbonejs-and-requirejs.html}{Backbone,
  PhantomJS and Jasmine}
\end{itemize}

\section{QUnit}\label{qunit}

\subsection{Introduction}\label{introduction-3}

QUnit is a powerful JavaScript test suite written by jQuery team member
\href{http://bassistance.de/}{Jörn Zaefferer} and used by many large
open-source projects (such as jQuery and Backbone.js) to test their
code. It's both capable of testing standard JavaScript code in the
browser as well as code on the server-side (where environments supported
include Rhino, V8 and SpiderMonkey). This makes it a robust solution for
a large number of use-cases.

Quite a few Backbone.js contributors feel that QUnit is a better
introductory framework for testing if you don't wish to start off with
Jasmine and BDD right away. As we'll see later on in this chapter, QUnit
can also be combined with third-party solutions such as SinonJS to
produce an even more powerful testing solution supporting spies and
mocks, which some say is preferable over Jasmine.

My personal recommendation is that it's worth comparing both frameworks
and opting for the solution that you feel the most comfortable with.

\subsection{Getting Setup}\label{getting-setup}

Luckily, getting QUnit setup is a fairly straight-forward process that
will take less than 5 minutes.

We first setup a testing environment composed of three files:

\begin{itemize}
\itemsep1pt\parskip0pt\parsep0pt
\item
  An HTML \textbf{structure} for displaying test results
\item
  The \textbf{qunit.js} file composing the testing framework
\item
  The \textbf{qunit.css} file for styling test results
\end{itemize}

The latter two of these can be downloaded from the
\href{http://qunitjs.com}{QUnit website}.

If you would prefer, you can use a hosted version of the QUnit source
files for testing purposes. The hosted URLs can be found at
\url{http://github.com/jquery/qunit/raw/master/qunit/}.

\paragraph{Sample HTML with QUnit-compatible
markup:}\label{sample-html-with-qunit-compatible-markup}

\begin{Shaded}
\begin{Highlighting}[]
\DataTypeTok{<!DOCTYPE }\NormalTok{html}\DataTypeTok{>}
\KeywordTok{<html>}
\KeywordTok{<head>}
    \KeywordTok{<title>}\NormalTok{QUnit Test Suite}\KeywordTok{</title>}

     \KeywordTok{<link}\OtherTok{ rel=}\StringTok{"stylesheet"}\OtherTok{ href=}\StringTok{"qunit.css"}\KeywordTok{>}
     \KeywordTok{<script}\OtherTok{ src=}\StringTok{"qunit.js"}\KeywordTok{></script>}

     \CommentTok{<!-- Your application -->}
     \KeywordTok{<script}\OtherTok{ src=}\StringTok{"app.js"}\KeywordTok{></script>}

     \CommentTok{<!-- Your tests -->}
     \KeywordTok{<script}\OtherTok{ src=}\StringTok{"tests.js"}\KeywordTok{></script>}
\KeywordTok{</head>}
\KeywordTok{<body>}
    \KeywordTok{<h1}\OtherTok{ id=}\StringTok{"qunit-header"}\KeywordTok{>}\NormalTok{QUnit Test Suite}\KeywordTok{</h1>}
    \KeywordTok{<h2}\OtherTok{ id=}\StringTok{"qunit-banner"}\KeywordTok{></h2>}
    \KeywordTok{<div}\OtherTok{ id=}\StringTok{"qunit-testrunner-toolbar"}\KeywordTok{></div>}
    \KeywordTok{<h2}\OtherTok{ id=}\StringTok{"qunit-userAgent"}\KeywordTok{></h2>}
    \KeywordTok{<ol}\OtherTok{ id=}\StringTok{"qunit-tests"}\KeywordTok{>}\NormalTok{test markup, hidden.}\KeywordTok{</ol>}
\KeywordTok{</body>}
\KeywordTok{</html>}
\end{Highlighting}
\end{Shaded}

Let's go through the elements above with qunit mentioned in their ID.
When QUnit is running:

\begin{itemize}
\itemsep1pt\parskip0pt\parsep0pt
\item
  \textbf{qunit-header} shows the name of the test suite
\item
  \textbf{qunit-banner} shows up as red if a test fails and green if all
  tests pass
\item
  \textbf{qunit-testrunner-toolbar} contains additional options for
  configuring the display of tests
\item
  \textbf{qunit-userAgent} displays the navigator.userAgent property
\item
  \textbf{qunit-tests} is a container for our test results
\end{itemize}

When running correctly, the above test runner looks as follows:

\begin{figure}[htbp]
\centering
\includegraphics{img/7d4de12.png}
\caption{screenshot 1}
\end{figure}

The numbers of the form (a, b, c) after each test name correspond to a)
failed asserts, b) passed asserts and c) total asserts. Clicking on a
test name expands it to display all of the assertions for that test
case. Assertions in green have successfully passed.

\begin{figure}[htbp]
\centering
\includegraphics{img/9df4.png}
\caption{screenshot 2}
\end{figure}

If however any tests fail, the test gets highlighted (and the
qunit-banner at the top switches to red):

\begin{figure}[htbp]
\centering
\includegraphics{img/3e5545.png}
\caption{screenshot 3}
\end{figure}

\subsection{Assertions}\label{assertions}

QUnit supports a number of basic \textbf{assertions}, which are used in
tests to verify that the result being returned by our code is what we
expect. If an assertion fails, we know that a bug exists. Similar to
Jasmine, QUnit can be used to easily test for regressions. Specifically,
when a bug is found one can write an assertion to test the existence of
the bug, write a patch, and then commit both. If subsequent changes to
the code break the test you'll know what was responsible and be able to
address it more easily.

Some of the supported QUnit assertions we're going to look at first are:

\begin{itemize}
\itemsep1pt\parskip0pt\parsep0pt
\item
  \texttt{ok ( state, message )} - passes if the first argument is
  truthy
\item
  \texttt{equal ( actual, expected, message )} - a simple comparison
  assertion with type coercion
\item
  \texttt{notEqual ( actual, expected, message )} - the opposite of the
  above
\item
  \texttt{expect( amount )} - the number of assertions expected to run
  within each test
\item
  \texttt{strictEqual( actual, expected, message)} - offers a much
  stricter comparison than \texttt{equal()} and is considered the
  preferred method of checking equality as it avoids stumbling on subtle
  coercion bugs
\item
  \texttt{deepEqual( actual, expected, message )} - similar to
  \texttt{strictEqual}, comparing the contents (with \texttt{===}) of
  the given objects, arrays and primitives.
\end{itemize}

\paragraph{Basic test case using test( name, callback
)}\label{basic-test-case-using-test-name-callback}

Creating new test cases with QUnit is relatively straight-forward and
can be done using \texttt{test()}, which constructs a test where the
first argument is the \texttt{name} of the test to be displayed in our
results and the second is a \texttt{callback} function containing all of
our assertions. This is called as soon as QUnit is running.

\begin{Shaded}
\begin{Highlighting}[]
\KeywordTok{var} \NormalTok{myString = }\StringTok{'Hello Backbone.js'}\NormalTok{;}

\FunctionTok{test}\NormalTok{( }\StringTok{'Our first QUnit test - asserting results'}\NormalTok{, }\KeywordTok{function}\NormalTok{()\{}

    \CommentTok{// ok( boolean, message )}
    \FunctionTok{ok}\NormalTok{( }\KeywordTok{true}\NormalTok{, }\StringTok{'the test succeeds'}\NormalTok{);}
    \FunctionTok{ok}\NormalTok{( }\KeywordTok{false}\NormalTok{, }\StringTok{'the test fails'}\NormalTok{);}

    \CommentTok{// equal( actualValue, expectedValue, message )}
    \FunctionTok{equal}\NormalTok{( myString, }\StringTok{'Hello Backbone.js'}\NormalTok{, }\StringTok{'The value expected is Hello Backbone.js!'}\NormalTok{);}
\NormalTok{\});}
\end{Highlighting}
\end{Shaded}

What we're doing in the above is defining a variable with a specific
value and then testing to ensure the value was what we expected it to
be. This was done using the comparison assertion, \texttt{equal()},
which expects its first argument to be a value being tested and the
second argument to be the expected value. We also used \texttt{ok()},
which allows us to easily test against functions or variables that
evaluate to booleans.

Note: Optionally in our test case, we could have passed an `expected'
value to \texttt{test()} defining the number of assertions we expect to
run. This takes the form: \texttt{test( name, {[}expected{]}, test );}
or by manually settings the expectation at the top of the test function,
like so: \texttt{expect( 1 )}. I recommend you make a habit of always
defining how many assertions you expect. More on this later.

\paragraph{Comparing the actual output of a function against the
expected
output}\label{comparing-the-actual-output-of-a-function-against-the-expected-output}

As testing a simple static variable is fairly trivial, we can take this
further to test actual functions. In the following example we test the
output of a function that reverses a string to ensure that the output is
correct using \texttt{equal()} and \texttt{notEqual()}:

\begin{Shaded}
\begin{Highlighting}[]
\KeywordTok{function} \FunctionTok{reverseString}\NormalTok{( str )\{}
    \KeywordTok{return} \OtherTok{str}\NormalTok{.}\FunctionTok{split}\NormalTok{(}\StringTok{''}\NormalTok{).}\FunctionTok{reverse}\NormalTok{().}\FunctionTok{join}\NormalTok{(}\StringTok{''}\NormalTok{);}
\NormalTok{\}}

\FunctionTok{test}\NormalTok{( }\StringTok{'reverseString()'}\NormalTok{, }\KeywordTok{function}\NormalTok{() \{}
    \FunctionTok{expect}\NormalTok{( }\DecValTok{5} \NormalTok{);}
    \FunctionTok{equal}\NormalTok{( }\FunctionTok{reverseString}\NormalTok{(}\StringTok{'hello'}\NormalTok{), }\StringTok{'olleh'}\NormalTok{, }\StringTok{'The value expected was olleh'} \NormalTok{);}
    \FunctionTok{equal}\NormalTok{( }\FunctionTok{reverseString}\NormalTok{(}\StringTok{'foobar'}\NormalTok{), }\StringTok{'raboof'}\NormalTok{, }\StringTok{'The value expected was raboof'} \NormalTok{);}
    \FunctionTok{equal}\NormalTok{( }\FunctionTok{reverseString}\NormalTok{(}\StringTok{'world'}\NormalTok{), }\StringTok{'dlrow'}\NormalTok{, }\StringTok{'The value expected was dlrow'} \NormalTok{);}
    \FunctionTok{notEqual}\NormalTok{( }\FunctionTok{reverseString}\NormalTok{(}\StringTok{'world'}\NormalTok{), }\StringTok{'dlroo'}\NormalTok{, }\StringTok{'The value was expected to not be dlroo'} \NormalTok{);}
    \FunctionTok{equal}\NormalTok{( }\FunctionTok{reverseString}\NormalTok{(}\StringTok{'bubble'}\NormalTok{), }\StringTok{'double'}\NormalTok{, }\StringTok{'The value expected was elbbub'} \NormalTok{);}
\NormalTok{\})}
\end{Highlighting}
\end{Shaded}

Running these tests in the QUnit test runner (which you would see when
your HTML test page was loaded) we would find that four of the
assertions pass while the last one does not. The reason the test against
\texttt{'double'} fails is because it was purposefully written
incorrectly. In your own projects if a test fails to pass and your
assertions are correct, you've probably just found a bug!

\subsection{Adding structure to
assertions}\label{adding-structure-to-assertions}

Housing all of our assertions in one test case can quickly become
difficult to maintain, but luckily QUnit supports structuring blocks of
assertions more cleanly. This can be done using \texttt{module()} - a
method that allows us to easily group tests together. A typical approach
to grouping might be keeping multiple tests for a specific method as
part of the same group (module).

\paragraph{Basic QUnit Modules}\label{basic-qunit-modules}

\begin{Shaded}
\begin{Highlighting}[]
\FunctionTok{module}\NormalTok{( }\StringTok{'Module One'} \NormalTok{);}
\FunctionTok{test}\NormalTok{( }\StringTok{'first test'}\NormalTok{, }\KeywordTok{function}\NormalTok{() \{\} );}
\FunctionTok{test}\NormalTok{( }\StringTok{'another test'}\NormalTok{, }\KeywordTok{function}\NormalTok{() \{\} );}

\FunctionTok{module}\NormalTok{( }\StringTok{'Module Two'} \NormalTok{);}
\FunctionTok{test}\NormalTok{( }\StringTok{'second test'}\NormalTok{, }\KeywordTok{function}\NormalTok{() \{\} );}
\FunctionTok{test}\NormalTok{( }\StringTok{'another test'}\NormalTok{, }\KeywordTok{function}\NormalTok{() \{\} );}

\FunctionTok{module}\NormalTok{( }\StringTok{'Module Three'} \NormalTok{);}
\FunctionTok{test}\NormalTok{( }\StringTok{'third test'}\NormalTok{, }\KeywordTok{function}\NormalTok{() \{\} );}
\FunctionTok{test}\NormalTok{( }\StringTok{'another test'}\NormalTok{, }\KeywordTok{function}\NormalTok{() \{\} );}
\end{Highlighting}
\end{Shaded}

We can take this further by introducing \texttt{setup()} and
\texttt{teardown()} callbacks to our modules, where \texttt{setup()} is
run before each test and \texttt{teardown()} is run after each test.

\paragraph{Using setup() and teardown()}\label{using-setup-and-teardown}

\begin{Shaded}
\begin{Highlighting}[]
\FunctionTok{module}\NormalTok{( }\StringTok{'Module One'}\NormalTok{, \{}
    \DataTypeTok{setup}\NormalTok{: }\KeywordTok{function}\NormalTok{() \{}
        \CommentTok{// run before}
    \NormalTok{\},}
    \DataTypeTok{teardown}\NormalTok{: }\KeywordTok{function}\NormalTok{() \{}
        \CommentTok{// run after}
    \NormalTok{\}}
\NormalTok{\});}

\FunctionTok{test}\NormalTok{(}\StringTok{'first test'}\NormalTok{, }\KeywordTok{function}\NormalTok{() \{}
    \CommentTok{// run the first test}
\NormalTok{\});}
\end{Highlighting}
\end{Shaded}

These callbacks can be used to define (or clear) any components we wish
to instantiate for use in one or more of our tests. As we'll see
shortly, this is ideal for defining new instances of views, collections,
models, or routers from a project that we can then reference across
multiple tests.

\paragraph{Using setup() and teardown() for instantiation and
clean-up}\label{using-setup-and-teardown-for-instantiation-and-clean-up}

\begin{Shaded}
\begin{Highlighting}[]
\CommentTok{// Define a simple model and collection modeling a store and}
\CommentTok{// list of stores}

\KeywordTok{var} \NormalTok{Store = }\OtherTok{Backbone}\NormalTok{.}\OtherTok{Model}\NormalTok{.}\FunctionTok{extend}\NormalTok{(\{\});}

\KeywordTok{var} \NormalTok{StoreList = }\OtherTok{Backbone}\NormalTok{.}\OtherTok{Collection}\NormalTok{.}\FunctionTok{extend}\NormalTok{(\{}
    \DataTypeTok{model}\NormalTok{: Store,}
    \DataTypeTok{comparator}\NormalTok{: }\KeywordTok{function}\NormalTok{( Store ) \{ }\KeywordTok{return} \OtherTok{Store}\NormalTok{.}\FunctionTok{get}\NormalTok{(}\StringTok{'name'}\NormalTok{) \}}
\NormalTok{\});}

\CommentTok{// Define a group for our tests}
\FunctionTok{module}\NormalTok{( }\StringTok{'StoreList sanity check'}\NormalTok{, \{}
    \DataTypeTok{setup}\NormalTok{: }\KeywordTok{function}\NormalTok{() \{}
        \KeywordTok{this}\NormalTok{.}\FunctionTok{list} \NormalTok{= }\KeywordTok{new} \NormalTok{StoreList;}
        \KeywordTok{this}\NormalTok{.}\OtherTok{list}\NormalTok{.}\FunctionTok{add}\NormalTok{(}\KeywordTok{new} \FunctionTok{Store}\NormalTok{(\{ }\DataTypeTok{name}\NormalTok{: }\StringTok{'Costcutter'} \NormalTok{\}));}
        \KeywordTok{this}\NormalTok{.}\OtherTok{list}\NormalTok{.}\FunctionTok{add}\NormalTok{(}\KeywordTok{new} \FunctionTok{Store}\NormalTok{(\{ }\DataTypeTok{name}\NormalTok{: }\StringTok{'Target'} \NormalTok{\}));}
        \KeywordTok{this}\NormalTok{.}\OtherTok{list}\NormalTok{.}\FunctionTok{add}\NormalTok{(}\KeywordTok{new} \FunctionTok{Store}\NormalTok{(\{ }\DataTypeTok{name}\NormalTok{: }\StringTok{'Walmart'} \NormalTok{\}));}
        \KeywordTok{this}\NormalTok{.}\OtherTok{list}\NormalTok{.}\FunctionTok{add}\NormalTok{(}\KeywordTok{new} \FunctionTok{Store}\NormalTok{(\{ }\DataTypeTok{name}\NormalTok{: }\StringTok{'Barnes & Noble'} \NormalTok{\}));}
    \NormalTok{\},}
    \DataTypeTok{teardown}\NormalTok{: }\KeywordTok{function}\NormalTok{() \{}
        \OtherTok{window}\NormalTok{.}\FunctionTok{errors} \NormalTok{= }\KeywordTok{null}\NormalTok{;}
    \NormalTok{\}}
\NormalTok{\});}

\CommentTok{// Test the order of items added}
\FunctionTok{test}\NormalTok{( }\StringTok{'test ordering'}\NormalTok{, }\KeywordTok{function}\NormalTok{() \{}
    \FunctionTok{expect}\NormalTok{( }\DecValTok{1} \NormalTok{);}
    \KeywordTok{var} \NormalTok{expected = [}\StringTok{'Barnes & Noble'}\NormalTok{, }\StringTok{'Costcutter'}\NormalTok{, }\StringTok{'Target'}\NormalTok{, }\StringTok{'Walmart'}\NormalTok{];}
    \KeywordTok{var} \NormalTok{actual = }\KeywordTok{this}\NormalTok{.}\OtherTok{list}\NormalTok{.}\FunctionTok{pluck}\NormalTok{(}\StringTok{'name'}\NormalTok{);}
    \FunctionTok{deepEqual}\NormalTok{( actual, expected, }\StringTok{'is maintained by comparator'} \NormalTok{);}
\NormalTok{\});}
\end{Highlighting}
\end{Shaded}

Here, a list of stores is created and stored on \texttt{setup()}. A
\texttt{teardown()} callback is used to simply clear a list of errors we
might be storing within the window scope, but is otherwise not needed.

\subsection{Assertion examples}\label{assertion-examples}

Before we continue any further, let's review some more examples of how
QUnit's various assertions can be correctly used when writing tests:

\subsubsection{equal - a comparison assertion. It passes if actual ==
expected}\label{equal---a-comparison-assertion.-it-passes-if-actual-expected}

\begin{Shaded}
\begin{Highlighting}[]
\FunctionTok{test}\NormalTok{( }\StringTok{'equal'}\NormalTok{, }\DecValTok{2}\NormalTok{, }\KeywordTok{function}\NormalTok{() \{}
  \KeywordTok{var} \NormalTok{actual = }\DecValTok{6} \NormalTok{- }\DecValTok{5}\NormalTok{;}
  \FunctionTok{equal}\NormalTok{( actual, }\KeywordTok{true}\NormalTok{,  }\StringTok{'passes as 1 == true'} \NormalTok{);}
  \FunctionTok{equal}\NormalTok{( actual, }\DecValTok{1}\NormalTok{,     }\StringTok{'passes as 1 == 1'} \NormalTok{);}
\NormalTok{\});}
\end{Highlighting}
\end{Shaded}

\subsubsection{notEqual - a comparison assertion. It passes if actual !=
expected}\label{notequal---a-comparison-assertion.-it-passes-if-actual-expected}

\begin{Shaded}
\begin{Highlighting}[]
\FunctionTok{test}\NormalTok{( }\StringTok{'notEqual'}\NormalTok{, }\DecValTok{2}\NormalTok{, }\KeywordTok{function}\NormalTok{() \{}
  \KeywordTok{var} \NormalTok{actual = }\DecValTok{6} \NormalTok{- }\DecValTok{5}\NormalTok{;}
  \FunctionTok{notEqual}\NormalTok{( actual, }\KeywordTok{false}\NormalTok{, }\StringTok{'passes as 1 != false'} \NormalTok{);}
  \FunctionTok{notEqual}\NormalTok{( actual, }\DecValTok{0}\NormalTok{,     }\StringTok{'passes as 1 != 0'} \NormalTok{);}
\NormalTok{\});}
\end{Highlighting}
\end{Shaded}

\subsubsection{strictEqual - a comparison assertion. It passes if actual
===
expected}\label{strictequal---a-comparison-assertion.-it-passes-if-actual-expected}

\begin{Shaded}
\begin{Highlighting}[]
\FunctionTok{test}\NormalTok{( }\StringTok{'strictEqual'}\NormalTok{, }\DecValTok{2}\NormalTok{, }\KeywordTok{function}\NormalTok{() \{}
  \KeywordTok{var} \NormalTok{actual = }\DecValTok{6} \NormalTok{- }\DecValTok{5}\NormalTok{;}
  \FunctionTok{strictEqual}\NormalTok{( actual, }\KeywordTok{true}\NormalTok{,  }\StringTok{'fails as 1 !== true'} \NormalTok{);}
  \FunctionTok{strictEqual}\NormalTok{( actual, }\DecValTok{1}\NormalTok{,     }\StringTok{'passes as 1 === 1'} \NormalTok{);}
\NormalTok{\});}
\end{Highlighting}
\end{Shaded}

\subsubsection{notStrictEqual - a comparison assertion. It passes if
actual !==
expected}\label{notstrictequal---a-comparison-assertion.-it-passes-if-actual-expected}

\begin{Shaded}
\begin{Highlighting}[]
\FunctionTok{test}\NormalTok{(}\StringTok{'notStrictEqual'}\NormalTok{, }\DecValTok{2}\NormalTok{, }\KeywordTok{function}\NormalTok{() \{}
  \KeywordTok{var} \NormalTok{actual = }\DecValTok{6} \NormalTok{- }\DecValTok{5}\NormalTok{;}
  \FunctionTok{notStrictEqual}\NormalTok{( actual, }\KeywordTok{true}\NormalTok{,  }\StringTok{'passes as 1 !== true'} \NormalTok{);}
  \FunctionTok{notStrictEqual}\NormalTok{( actual, }\DecValTok{1}\NormalTok{,     }\StringTok{'fails as 1 === 1'} \NormalTok{);}
\NormalTok{\});}
\end{Highlighting}
\end{Shaded}

\subsubsection{deepEqual - a recursive comparison assertion. Unlike
strictEqual(), it works on objects, arrays and
primitives.}\label{deepequal---a-recursive-comparison-assertion.-unlike-strictequal-it-works-on-objects-arrays-and-primitives.}

\begin{Shaded}
\begin{Highlighting}[]
\FunctionTok{test}\NormalTok{(}\StringTok{'deepEqual'}\NormalTok{, }\DecValTok{4}\NormalTok{, }\KeywordTok{function}\NormalTok{() \{}
  \KeywordTok{var} \NormalTok{actual = \{}\DataTypeTok{q}\NormalTok{: }\StringTok{'foo'}\NormalTok{, }\DataTypeTok{t}\NormalTok{: }\StringTok{'bar'}\NormalTok{\};}
  \KeywordTok{var} \NormalTok{el =  }\FunctionTok{$}\NormalTok{(}\StringTok{'div'}\NormalTok{);}
  \KeywordTok{var} \NormalTok{children = }\FunctionTok{$}\NormalTok{(}\StringTok{'div'}\NormalTok{).}\FunctionTok{children}\NormalTok{();}

  \FunctionTok{equal}\NormalTok{( actual, \{}\DataTypeTok{q}\NormalTok{: }\StringTok{'foo'}\NormalTok{, }\DataTypeTok{t}\NormalTok{: }\StringTok{'bar'}\NormalTok{\},   }\StringTok{'fails - objects are not equal using equal()'} \NormalTok{);}
  \FunctionTok{deepEqual}\NormalTok{( actual, \{}\DataTypeTok{q}\NormalTok{: }\StringTok{'foo'}\NormalTok{, }\DataTypeTok{t}\NormalTok{: }\StringTok{'bar'}\NormalTok{\},   }\StringTok{'passes - objects are equal'} \NormalTok{);}
  \FunctionTok{equal}\NormalTok{( el, children, }\StringTok{'fails - jQuery objects are not the same'} \NormalTok{);}
  \FunctionTok{deepEqual}\NormalTok{(el, children, }\StringTok{'fails - objects not equivalent'} \NormalTok{);}

\NormalTok{\});}
\end{Highlighting}
\end{Shaded}

\subsubsection{notDeepEqual - a comparison assertion. This returns the
opposite of
deepEqual}\label{notdeepequal---a-comparison-assertion.-this-returns-the-opposite-of-deepequal}

\begin{Shaded}
\begin{Highlighting}[]
\FunctionTok{test}\NormalTok{(}\StringTok{'notDeepEqual'}\NormalTok{, }\DecValTok{2}\NormalTok{, }\KeywordTok{function}\NormalTok{() \{}
  \KeywordTok{var} \NormalTok{actual = \{}\DataTypeTok{q}\NormalTok{: }\StringTok{'foo'}\NormalTok{, }\DataTypeTok{t}\NormalTok{: }\StringTok{'bar'}\NormalTok{\};}
  \FunctionTok{notEqual}\NormalTok{( actual, \{}\DataTypeTok{q}\NormalTok{: }\StringTok{'foo'}\NormalTok{, }\DataTypeTok{t}\NormalTok{: }\StringTok{'bar'}\NormalTok{\},   }\StringTok{'passes - objects are not equal'} \NormalTok{);}
  \FunctionTok{notDeepEqual}\NormalTok{( actual, \{}\DataTypeTok{q}\NormalTok{: }\StringTok{'foo'}\NormalTok{, }\DataTypeTok{t}\NormalTok{: }\StringTok{'bar'}\NormalTok{\},   }\StringTok{'fails - objects are equivalent'} \NormalTok{);}
\NormalTok{\});}
\end{Highlighting}
\end{Shaded}

\subsubsection{raises - an assertion which tests if a callback throws
any
exceptions}\label{raises---an-assertion-which-tests-if-a-callback-throws-any-exceptions}

\begin{Shaded}
\begin{Highlighting}[]
\FunctionTok{test}\NormalTok{(}\StringTok{'raises'}\NormalTok{, }\DecValTok{1}\NormalTok{, }\KeywordTok{function}\NormalTok{() \{}
  \FunctionTok{raises}\NormalTok{(}\KeywordTok{function}\NormalTok{() \{}
    \KeywordTok{throw} \KeywordTok{new} \FunctionTok{Error}\NormalTok{( }\StringTok{'Oh no! It`s an error!'} \NormalTok{);}
  \NormalTok{\}, }\StringTok{'passes - an error was thrown inside our callback'}\NormalTok{);}
\NormalTok{\});}
\end{Highlighting}
\end{Shaded}

\subsection{Fixtures}\label{fixtures}

From time to time we may need to write tests that modify the DOM.
Managing the clean-up of such operations between tests can be a genuine
pain, but thankfully QUnit has a solution to this problem in the form of
the \texttt{\#qunit-fixture} element, seen below.

\paragraph{Fixture markup:}\label{fixture-markup}

\begin{Shaded}
\begin{Highlighting}[]
\DataTypeTok{<!DOCTYPE }\NormalTok{html}\DataTypeTok{>}
\KeywordTok{<html>}
\KeywordTok{<head>}
    \KeywordTok{<title>}\NormalTok{QUnit Test}\KeywordTok{</title>}
    \KeywordTok{<link}\OtherTok{ rel=}\StringTok{"stylesheet"}\OtherTok{ href=}\StringTok{"qunit.css"}\KeywordTok{>}
    \KeywordTok{<script}\OtherTok{ src=}\StringTok{"qunit.js"}\KeywordTok{></script>}
    \KeywordTok{<script}\OtherTok{ src=}\StringTok{"app.js"}\KeywordTok{></script>}
    \KeywordTok{<script}\OtherTok{ src=}\StringTok{"tests.js"}\KeywordTok{></script>}
\KeywordTok{</head>}
\KeywordTok{<body>}
    \KeywordTok{<h1}\OtherTok{ id=}\StringTok{"qunit-header"}\KeywordTok{>}\NormalTok{QUnit Test}\KeywordTok{</h1>}
    \KeywordTok{<h2}\OtherTok{ id=}\StringTok{"qunit-banner"}\KeywordTok{></h2>}
    \KeywordTok{<div}\OtherTok{ id=}\StringTok{"qunit-testrunner-toolbar"}\KeywordTok{></div>}
    \KeywordTok{<h2}\OtherTok{ id=}\StringTok{"qunit-userAgent"}\KeywordTok{></h2>}
    \KeywordTok{<ol}\OtherTok{ id=}\StringTok{"qunit-tests"}\KeywordTok{></ol>}
    \KeywordTok{<div}\OtherTok{ id=}\StringTok{"qunit-fixture"}\KeywordTok{></div>}
\KeywordTok{</body>}
\KeywordTok{</html>}
\end{Highlighting}
\end{Shaded}

We can either opt to place static markup in the fixture or just
insert/append any DOM elements we may need to it. QUnit will
automatically reset the \texttt{innerHTML} of the fixture after each
test to its original value. In case you're using jQuery, it's useful to
know that QUnit checks for its availability and will opt to use
\texttt{\$(el).html()} instead, which will cleanup any jQuery event
handlers too.

\subsubsection{Fixtures example:}\label{fixtures-example}

Let us now go through a more complete example of using fixtures. One
thing that most of us are used to doing in jQuery is working with lists
- they're often used to define the markup for menus, grids, and a number
of other components. You may have used jQuery plugins before that
manipulated a given list in a particular way and it can be useful to
test that the final (manipulated) output of the plugin is what was
expected.

For the purposes of our next example, we're going to use Ben Alman's
\texttt{\$.enumerate()} plugin, which can prepend each item in a list by
its index, optionally allowing us to set what the first number in the
list is. The code snippet for the plugin can be found below, followed by
an example of the output it generates:

\begin{Shaded}
\begin{Highlighting}[]
\OtherTok{$}\NormalTok{.}\OtherTok{fn}\NormalTok{.}\FunctionTok{enumerate} \NormalTok{= }\KeywordTok{function}\NormalTok{( start ) \{}
      \KeywordTok{if} \NormalTok{( }\KeywordTok{typeof} \NormalTok{start !== }\StringTok{'undefined'} \NormalTok{) \{}
        \CommentTok{// Since `start` value was provided, enumerate and return}
        \CommentTok{// the initial jQuery object to allow chaining.}

        \KeywordTok{return} \KeywordTok{this}\NormalTok{.}\FunctionTok{each}\NormalTok{(}\KeywordTok{function}\NormalTok{(i)\{}
          \FunctionTok{$}\NormalTok{(}\KeywordTok{this}\NormalTok{).}\FunctionTok{prepend}\NormalTok{( }\StringTok{'<b>'} \NormalTok{+ ( i + start ) + }\StringTok{'</b> '} \NormalTok{);}
        \NormalTok{\});}

      \NormalTok{\} }\KeywordTok{else} \NormalTok{\{}
        \CommentTok{// Since no `start` value was provided, function as a}
        \CommentTok{// getter, returning the appropriate value from the first}
        \CommentTok{// selected element.}

        \KeywordTok{var} \NormalTok{val = }\KeywordTok{this}\NormalTok{.}\FunctionTok{eq}\NormalTok{( }\DecValTok{0} \NormalTok{).}\FunctionTok{children}\NormalTok{( }\StringTok{'b'} \NormalTok{).}\FunctionTok{eq}\NormalTok{( }\DecValTok{0} \NormalTok{).}\FunctionTok{text}\NormalTok{();}
        \KeywordTok{return} \FunctionTok{Number}\NormalTok{( val );}
      \NormalTok{\}}
    \NormalTok{\};}

\CommentTok{/*}
\CommentTok{    <ul>}
\CommentTok{      <li>1. hello</li>}
\CommentTok{      <li>2. world</li>}
\CommentTok{      <li>3. i</li>}
\CommentTok{      <li>4. am</li>}
\CommentTok{      <li>5. foo</li>}
\CommentTok{    </ul>}
\CommentTok{*/}
\end{Highlighting}
\end{Shaded}

Let's now write some tests for the plugin. First, we define the markup
for a list containing some sample items inside our
\texttt{qunit-fixture} element:

\begin{Shaded}
\begin{Highlighting}[]
\KeywordTok{<div}\OtherTok{ id=}\StringTok{"qunit-fixture"}\KeywordTok{>}
    \KeywordTok{<ul>}
      \KeywordTok{<li>}\NormalTok{hello}\KeywordTok{</li>}
      \KeywordTok{<li>}\NormalTok{world}\KeywordTok{</li>}
      \KeywordTok{<li>}\NormalTok{i}\KeywordTok{</li>}
      \KeywordTok{<li>}\NormalTok{am}\KeywordTok{</li>}
      \KeywordTok{<li>}\NormalTok{foo}\KeywordTok{</li>}
    \KeywordTok{</ul>}
 \KeywordTok{</div>}
\end{Highlighting}
\end{Shaded}

Next, we need to think about what should be tested.
\texttt{\$.enumerate()} supports a few different use cases, including:

\begin{itemize}
\itemsep1pt\parskip0pt\parsep0pt
\item
  \textbf{No arguments passed} - i.e., \texttt{\$(el).enumerate()}
\item
  \textbf{0 passed as an argument} - i.e., \texttt{\$(el).enumerate(0)}
\item
  \textbf{1 passed as an argument} - i.e., \texttt{\$(el).enumerate(1)}
\end{itemize}

As the text value for each list item is of the form ``n. item-text'' and
we only require this to test against the expected output, we can simply
access the content using \texttt{\$(el).eq(index).text()} (for more
information on .eq() see \href{http://api.jquery.com/eq/}{here}).

and finally, here are our test cases:

\begin{Shaded}
\begin{Highlighting}[]
\FunctionTok{module}\NormalTok{(}\StringTok{'jQuery#enumerate'}\NormalTok{);}

\FunctionTok{test}\NormalTok{( }\StringTok{'No arguments passed'}\NormalTok{, }\DecValTok{5}\NormalTok{, }\KeywordTok{function}\NormalTok{() \{}
  \KeywordTok{var} \NormalTok{items = }\FunctionTok{$}\NormalTok{(}\StringTok{'#qunit-fixture li'}\NormalTok{).}\FunctionTok{enumerate}\NormalTok{(); }\CommentTok{// 0}
  \FunctionTok{equal}\NormalTok{( }\OtherTok{items}\NormalTok{.}\FunctionTok{eq}\NormalTok{(}\DecValTok{0}\NormalTok{).}\FunctionTok{text}\NormalTok{(), }\StringTok{'0. hello'}\NormalTok{, }\StringTok{'first item should have index 0'} \NormalTok{);}
  \FunctionTok{equal}\NormalTok{( }\OtherTok{items}\NormalTok{.}\FunctionTok{eq}\NormalTok{(}\DecValTok{1}\NormalTok{).}\FunctionTok{text}\NormalTok{(), }\StringTok{'1. world'}\NormalTok{, }\StringTok{'second item should have index 1'} \NormalTok{);}
  \FunctionTok{equal}\NormalTok{( }\OtherTok{items}\NormalTok{.}\FunctionTok{eq}\NormalTok{(}\DecValTok{2}\NormalTok{).}\FunctionTok{text}\NormalTok{(), }\StringTok{'2. i'}\NormalTok{, }\StringTok{'third item should have index 2'} \NormalTok{);}
  \FunctionTok{equal}\NormalTok{( }\OtherTok{items}\NormalTok{.}\FunctionTok{eq}\NormalTok{(}\DecValTok{3}\NormalTok{).}\FunctionTok{text}\NormalTok{(), }\StringTok{'3. am'}\NormalTok{, }\StringTok{'fourth item should have index 3'} \NormalTok{);}
  \FunctionTok{equal}\NormalTok{( }\OtherTok{items}\NormalTok{.}\FunctionTok{eq}\NormalTok{(}\DecValTok{4}\NormalTok{).}\FunctionTok{text}\NormalTok{(), }\StringTok{'4. foo'}\NormalTok{, }\StringTok{'fifth item should have index 4'} \NormalTok{);}
\NormalTok{\});}

\FunctionTok{test}\NormalTok{( }\StringTok{'0 passed as an argument'}\NormalTok{, }\DecValTok{5}\NormalTok{, }\KeywordTok{function}\NormalTok{() \{}
  \KeywordTok{var} \NormalTok{items = }\FunctionTok{$}\NormalTok{(}\StringTok{'#qunit-fixture li'}\NormalTok{).}\FunctionTok{enumerate}\NormalTok{( }\DecValTok{0} \NormalTok{);}
  \FunctionTok{equal}\NormalTok{( }\OtherTok{items}\NormalTok{.}\FunctionTok{eq}\NormalTok{(}\DecValTok{0}\NormalTok{).}\FunctionTok{text}\NormalTok{(), }\StringTok{'0. hello'}\NormalTok{, }\StringTok{'first item should have index 0'} \NormalTok{);}
  \FunctionTok{equal}\NormalTok{( }\OtherTok{items}\NormalTok{.}\FunctionTok{eq}\NormalTok{(}\DecValTok{1}\NormalTok{).}\FunctionTok{text}\NormalTok{(), }\StringTok{'1. world'}\NormalTok{, }\StringTok{'second item should have index 1'} \NormalTok{);}
  \FunctionTok{equal}\NormalTok{( }\OtherTok{items}\NormalTok{.}\FunctionTok{eq}\NormalTok{(}\DecValTok{2}\NormalTok{).}\FunctionTok{text}\NormalTok{(), }\StringTok{'2. i'}\NormalTok{, }\StringTok{'third item should have index 2'} \NormalTok{);}
  \FunctionTok{equal}\NormalTok{( }\OtherTok{items}\NormalTok{.}\FunctionTok{eq}\NormalTok{(}\DecValTok{3}\NormalTok{).}\FunctionTok{text}\NormalTok{(), }\StringTok{'3. am'}\NormalTok{, }\StringTok{'fourth item should have index 3'} \NormalTok{);}
  \FunctionTok{equal}\NormalTok{( }\OtherTok{items}\NormalTok{.}\FunctionTok{eq}\NormalTok{(}\DecValTok{4}\NormalTok{).}\FunctionTok{text}\NormalTok{(), }\StringTok{'4. foo'}\NormalTok{, }\StringTok{'fifth item should have index 4'} \NormalTok{);}
\NormalTok{\});}

\FunctionTok{test}\NormalTok{( }\StringTok{'1 passed as an argument'}\NormalTok{, }\DecValTok{3}\NormalTok{, }\KeywordTok{function}\NormalTok{() \{}
  \KeywordTok{var} \NormalTok{items = }\FunctionTok{$}\NormalTok{(}\StringTok{'#qunit-fixture li'}\NormalTok{).}\FunctionTok{enumerate}\NormalTok{( }\DecValTok{1} \NormalTok{);}
  \FunctionTok{equal}\NormalTok{( }\OtherTok{items}\NormalTok{.}\FunctionTok{eq}\NormalTok{(}\DecValTok{0}\NormalTok{).}\FunctionTok{text}\NormalTok{(), }\StringTok{'1. hello'}\NormalTok{, }\StringTok{'first item should have index 1'} \NormalTok{);}
  \FunctionTok{equal}\NormalTok{( }\OtherTok{items}\NormalTok{.}\FunctionTok{eq}\NormalTok{(}\DecValTok{1}\NormalTok{).}\FunctionTok{text}\NormalTok{(), }\StringTok{'2. world'}\NormalTok{, }\StringTok{'second item should have index 2'} \NormalTok{);}
  \FunctionTok{equal}\NormalTok{( }\OtherTok{items}\NormalTok{.}\FunctionTok{eq}\NormalTok{(}\DecValTok{2}\NormalTok{).}\FunctionTok{text}\NormalTok{(), }\StringTok{'3. i'}\NormalTok{, }\StringTok{'third item should have index 3'} \NormalTok{);}
  \FunctionTok{equal}\NormalTok{( }\OtherTok{items}\NormalTok{.}\FunctionTok{eq}\NormalTok{(}\DecValTok{3}\NormalTok{).}\FunctionTok{text}\NormalTok{(), }\StringTok{'4. am'}\NormalTok{, }\StringTok{'fourth item should have index 4'} \NormalTok{);}
  \FunctionTok{equal}\NormalTok{( }\OtherTok{items}\NormalTok{.}\FunctionTok{eq}\NormalTok{(}\DecValTok{4}\NormalTok{).}\FunctionTok{text}\NormalTok{(), }\StringTok{'5. foo'}\NormalTok{, }\StringTok{'fifth item should have index 5'} \NormalTok{);}
\NormalTok{\});}
\end{Highlighting}
\end{Shaded}

\subsection{Asynchronous code}\label{asynchronous-code}

As with Jasmine, the effort required to run synchronous tests with QUnit
is fairly minimal. That said, what about tests that require asynchronous
callbacks (such as expensive processes, Ajax requests, and so on)? When
we're dealing with asynchronous code, rather than letting QUnit control
when the next test runs, we can tell it that we need it to stop running
and wait until it's okay to continue once again.

Remember: running asynchronous code without any special considerations
can cause incorrect assertions to appear in other tests, so we want to
make sure we get it right.

Writing QUnit tests for asynchronous code is made possible using the
\texttt{start()} and \texttt{stop()} methods, which programmatically set
the start and stop points during such tests. Here's a simple example:

\begin{Shaded}
\begin{Highlighting}[]
\FunctionTok{test}\NormalTok{(}\StringTok{'An async test'}\NormalTok{, }\KeywordTok{function}\NormalTok{()\{}
   \FunctionTok{stop}\NormalTok{();}
   \FunctionTok{expect}\NormalTok{( }\DecValTok{1} \NormalTok{);}
   \OtherTok{$}\NormalTok{.}\FunctionTok{ajax}\NormalTok{(\{}
        \DataTypeTok{url}\NormalTok{: }\StringTok{'/test'}\NormalTok{,}
        \DataTypeTok{dataType}\NormalTok{: }\StringTok{'json'}\NormalTok{,}
        \DataTypeTok{success}\NormalTok{: }\KeywordTok{function}\NormalTok{( data )\{}
            \FunctionTok{deepEqual}\NormalTok{(data, \{}
               \DataTypeTok{topic}\NormalTok{: }\StringTok{'hello'}\NormalTok{,}
               \DataTypeTok{message}\NormalTok{: }\StringTok{'hi there!''}
            \NormalTok{\});}
            \FunctionTok{ok}\NormalTok{(}\KeywordTok{true}\NormalTok{, }\StringTok{'Asynchronous test passed!'}\NormalTok{);}
            \FunctionTok{start}\NormalTok{();}
        \NormalTok{\}}
    \NormalTok{\});}
\NormalTok{\});}
\end{Highlighting}
\end{Shaded}

A jQuery \texttt{\$.ajax()} request is used to connect to a test
resource and assert that the data returned is correct.
\texttt{deepEqual()} is used here as it allows us to compare different
data types (e.g., objects, arrays) and ensures that what is returned is
exactly what we're expecting. We know that our Ajax request is
asynchronous and so we first call \texttt{stop()}, then run the code
making the request, and finally, at the very end of our callback, inform
QUnit that it is okay to continue running other tests.

Note: rather than including \texttt{stop()}, we can simply exclude it
and substitute \texttt{test()} with \texttt{asyncTest()} if we prefer.
This improves readability when dealing with a mixture of asynchronous
and synchronous tests in your suite. While this setup should work fine
for many use-cases, there is no guarantee that the callback in our
\texttt{\$.ajax()} request will actually get called. To factor this into
our tests, we can use \texttt{expect()} once again to define how many
assertions we expect to see within our test. This is a healthy safety
blanket as it ensures that if a test completes with an insufficient
number of assertions, we know something went wrong and can fix it.

\section{SinonJS}\label{sinonjs}

Similar to the section on testing Backbone.js apps using the Jasmine BDD
framework, we're nearly ready to take what we've learned and write a
number of QUnit tests for our Todo application.

Before we start though, you may have noticed that QUnit doesn't support
test spies. Test spies are functions which record arguments, exceptions,
and return values for any of their calls. They're typically used to test
callbacks and how functions may be used in the application being tested.
In testing frameworks, spies usually are anonymous functions or wrappers
around functions which already exist.

\subsection{What is SinonJS?}\label{what-is-sinonjs}

In order for us to substitute support for spies in QUnit, we will be
taking advantage of a mocking framework called
\href{http://sinonjs.org/}{SinonJS} by Christian Johansen. We will also
be using the \href{http://sinonjs.org/qunit/}{SinonJS-QUnit adapter}
which provides seamless integration with QUnit (meaning setup is
minimal). Sinon.JS is completely test-framework agnostic and should be
easy to use with any testing framework, so it's ideal for our needs.

The framework supports three features we'll be taking advantage of for
unit testing our application:

\begin{itemize}
\itemsep1pt\parskip0pt\parsep0pt
\item
  \textbf{Anonymous spies}
\item
  \textbf{Spying on existing methods}
\item
  \textbf{A rich inspection interface}
\end{itemize}

\paragraph{Basic Spies}\label{basic-spies}

Using \texttt{this.spy()} without any arguments creates an anonymous
spy. This is comparable to \texttt{jasmine.createSpy()}. We can observe
basic usage of a SinonJS spy in the following example:

\begin{Shaded}
\begin{Highlighting}[]
\FunctionTok{test}\NormalTok{(}\StringTok{'should call all subscribers for a message exactly once'}\NormalTok{, }\KeywordTok{function} \NormalTok{() \{}
    \KeywordTok{var} \NormalTok{message = }\FunctionTok{getUniqueString}\NormalTok{();}
    \KeywordTok{var} \NormalTok{spy = }\KeywordTok{this}\NormalTok{.}\FunctionTok{spy}\NormalTok{();}

    \OtherTok{PubSub}\NormalTok{.}\FunctionTok{subscribe}\NormalTok{( message, spy );}
    \OtherTok{PubSub}\NormalTok{.}\FunctionTok{publishSync}\NormalTok{( message, }\StringTok{'Hello World'} \NormalTok{);}

    \FunctionTok{ok}\NormalTok{( }\OtherTok{spy}\NormalTok{.}\FunctionTok{calledOnce}\NormalTok{, }\StringTok{'the subscriber was called once'} \NormalTok{);}
\NormalTok{\});}
\end{Highlighting}
\end{Shaded}

\paragraph{Spying On Existing
Functions}\label{spying-on-existing-functions}

We can also use \texttt{this.spy()} to spy on existing functions (like
jQuery's \texttt{\$.ajax}) in the example below. When spying on a
function which already exists, the function behaves normally but we get
access to data about its calls which can be very useful for testing
purposes.

\begin{Shaded}
\begin{Highlighting}[]
\FunctionTok{test}\NormalTok{( }\StringTok{'should inspect the jQuery.getJSON usage of jQuery.ajax'}\NormalTok{, }\KeywordTok{function} \NormalTok{() \{}
    \KeywordTok{this}\NormalTok{.}\FunctionTok{spy}\NormalTok{( jQuery, }\StringTok{'ajax'} \NormalTok{);}

    \OtherTok{jQuery}\NormalTok{.}\FunctionTok{getJSON}\NormalTok{( }\StringTok{'/todos/completed'} \NormalTok{);}

    \FunctionTok{ok}\NormalTok{( }\OtherTok{jQuery}\NormalTok{.}\OtherTok{ajax}\NormalTok{.}\FunctionTok{calledOnce} \NormalTok{);}
    \FunctionTok{equals}\NormalTok{( }\OtherTok{jQuery}\NormalTok{.}\OtherTok{ajax}\NormalTok{.}\FunctionTok{getCall}\NormalTok{(}\DecValTok{0}\NormalTok{).}\FunctionTok{args}\NormalTok{[}\DecValTok{0}\NormalTok{].}\FunctionTok{url}\NormalTok{, }\StringTok{'/todos/completed'} \NormalTok{);}
    \FunctionTok{equals}\NormalTok{( }\OtherTok{jQuery}\NormalTok{.}\OtherTok{ajax}\NormalTok{.}\FunctionTok{getCall}\NormalTok{(}\DecValTok{0}\NormalTok{).}\FunctionTok{args}\NormalTok{[}\DecValTok{0}\NormalTok{].}\FunctionTok{dataType}\NormalTok{, }\StringTok{'json'} \NormalTok{);}
\NormalTok{\});}
\end{Highlighting}
\end{Shaded}

\paragraph{Inspection Interface}\label{inspection-interface}

SinonJS comes with a rich spy interface which allows us to test whether
a spy was called with a specific argument, if it was called a specific
number of times, and test against the values of arguments. A complete
list of features supported in the interface can be found on
\href{http://sinonjs.org/docs/}{SinonJS.org}, but let's take a look at
some examples demonstrating some of the most commonly used ones:

\textbf{Matching arguments: test a spy was called with a specific set of
arguments:}

\begin{Shaded}
\begin{Highlighting}[]
\FunctionTok{test}\NormalTok{( }\StringTok{'Should call a subscriber with standard matching'}\NormalTok{: }\KeywordTok{function} \NormalTok{() \{}
    \KeywordTok{var} \NormalTok{spy = }\OtherTok{sinon}\NormalTok{.}\FunctionTok{spy}\NormalTok{();}

    \OtherTok{PubSub}\NormalTok{.}\FunctionTok{subscribe}\NormalTok{( }\StringTok{'message'}\NormalTok{, spy );}
    \OtherTok{PubSub}\NormalTok{.}\FunctionTok{publishSync}\NormalTok{( }\StringTok{'message'}\NormalTok{, \{ }\DataTypeTok{id}\NormalTok{: }\DecValTok{45} \NormalTok{\} );}

    \FunctionTok{assertTrue}\NormalTok{( }\OtherTok{spy}\NormalTok{.}\FunctionTok{calledWith}\NormalTok{( \{ }\DataTypeTok{id}\NormalTok{: }\DecValTok{45} \NormalTok{\} ) );}
\NormalTok{\});}
\end{Highlighting}
\end{Shaded}

\textbf{Stricter argument matching: test a spy was called at least once
with specific arguments and no others:}

\begin{Shaded}
\begin{Highlighting}[]
\FunctionTok{test}\NormalTok{( }\StringTok{'Should call a subscriber with strict matching'}\NormalTok{: }\KeywordTok{function} \NormalTok{() \{}
    \KeywordTok{var} \NormalTok{spy = }\OtherTok{sinon}\NormalTok{.}\FunctionTok{spy}\NormalTok{();}

    \OtherTok{PubSub}\NormalTok{.}\FunctionTok{subscribe}\NormalTok{( }\StringTok{'message'}\NormalTok{, spy );}
    \OtherTok{PubSub}\NormalTok{.}\FunctionTok{publishSync}\NormalTok{( }\StringTok{'message'}\NormalTok{, }\StringTok{'many'}\NormalTok{, }\StringTok{'arguments'} \NormalTok{);}
    \OtherTok{PubSub}\NormalTok{.}\FunctionTok{publishSync}\NormalTok{( }\StringTok{'message'}\NormalTok{, }\DecValTok{12}\NormalTok{, }\DecValTok{34} \NormalTok{);}

    \CommentTok{// This passes}
    \FunctionTok{assertTrue}\NormalTok{( }\OtherTok{spy}\NormalTok{.}\FunctionTok{calledWith}\NormalTok{(}\StringTok{'many'}\NormalTok{) );}

    \CommentTok{// This however, fails}
    \FunctionTok{assertTrue}\NormalTok{( }\OtherTok{spy}\NormalTok{.}\FunctionTok{calledWithExactly}\NormalTok{( }\StringTok{'many'} \NormalTok{) );}
\NormalTok{\});}
\end{Highlighting}
\end{Shaded}

\textbf{Testing call order: testing if a spy was called before or after
another spy:}

\begin{Shaded}
\begin{Highlighting}[]
\FunctionTok{test}\NormalTok{( }\StringTok{'Should call a subscriber and maintain call order'}\NormalTok{: }\KeywordTok{function} \NormalTok{() \{}
    \KeywordTok{var} \NormalTok{a = }\OtherTok{sinon}\NormalTok{.}\FunctionTok{spy}\NormalTok{();}
    \KeywordTok{var} \NormalTok{b = }\OtherTok{sinon}\NormalTok{.}\FunctionTok{spy}\NormalTok{();}

    \OtherTok{PubSub}\NormalTok{.}\FunctionTok{subscribe}\NormalTok{( }\StringTok{'message'}\NormalTok{, a );}
    \OtherTok{PubSub}\NormalTok{.}\FunctionTok{subscribe}\NormalTok{( }\StringTok{'event'}\NormalTok{, b );}

    \OtherTok{PubSub}\NormalTok{.}\FunctionTok{publishSync}\NormalTok{( }\StringTok{'message'}\NormalTok{, \{ }\DataTypeTok{id}\NormalTok{: }\DecValTok{45} \NormalTok{\} );}
    \OtherTok{PubSub}\NormalTok{.}\FunctionTok{publishSync}\NormalTok{( }\StringTok{'event'}\NormalTok{, [}\DecValTok{1}\NormalTok{, }\DecValTok{2}\NormalTok{, }\DecValTok{3}\NormalTok{] );}

    \FunctionTok{assertTrue}\NormalTok{( }\OtherTok{a}\NormalTok{.}\FunctionTok{calledBefore}\NormalTok{(b) );}
    \FunctionTok{assertTrue}\NormalTok{( }\OtherTok{b}\NormalTok{.}\FunctionTok{calledAfter}\NormalTok{(a) );}
\NormalTok{\});}
\end{Highlighting}
\end{Shaded}

\textbf{Match execution counts: test a spy was called a specific number
of times:}

\begin{Shaded}
\begin{Highlighting}[]
\FunctionTok{test}\NormalTok{( }\StringTok{'Should call a subscriber and check call counts'}\NormalTok{, }\KeywordTok{function} \NormalTok{() \{}
    \KeywordTok{var} \NormalTok{message = }\FunctionTok{getUniqueString}\NormalTok{();}
    \KeywordTok{var} \NormalTok{spy = }\KeywordTok{this}\NormalTok{.}\FunctionTok{spy}\NormalTok{();}

    \OtherTok{PubSub}\NormalTok{.}\FunctionTok{subscribe}\NormalTok{( message, spy );}
    \OtherTok{PubSub}\NormalTok{.}\FunctionTok{publishSync}\NormalTok{( message, }\StringTok{'some payload'} \NormalTok{);}


    \CommentTok{// Passes if spy was called once and only once.}
    \FunctionTok{ok}\NormalTok{( }\OtherTok{spy}\NormalTok{.}\FunctionTok{calledOnce} \NormalTok{); }\CommentTok{// calledTwice and calledThrice are also supported}

    \CommentTok{// The number of recorded calls.}
    \FunctionTok{equal}\NormalTok{( }\OtherTok{spy}\NormalTok{.}\FunctionTok{callCount}\NormalTok{, }\DecValTok{1} \NormalTok{);}

    \CommentTok{// Directly checking the arguments of the call}
    \FunctionTok{equals}\NormalTok{( }\OtherTok{spy}\NormalTok{.}\FunctionTok{getCall}\NormalTok{(}\DecValTok{0}\NormalTok{).}\FunctionTok{args}\NormalTok{[}\DecValTok{0}\NormalTok{], message );}
\NormalTok{\});}
\end{Highlighting}
\end{Shaded}

\subsection{Stubs and mocks}\label{stubs-and-mocks}

SinonJS also supports two other powerful features: stubs and mocks. Both
stubs and mocks implement all of the features of the spy API, but have
some added functionality.

\subsubsection{Stubs}\label{stubs}

A stub allows us to replace any existing behaviour for a specific method
with something else. They can be very useful for simulating exceptions
and are most often used to write test cases when certain dependencies of
your code-base may not yet be written.

Let us briefly re-explore our Backbone Todo application, which contained
a Todo model and a TodoList collection. For the purpose of this
walkthrough, we want to isolate our TodoList collection and fake the
Todo model to test how adding new models might behave.

We can pretend that the models have yet to be written just to
demonstrate how stubbing might be carried out. A shell collection just
containing a reference to the model to be used might look like this:

\begin{Shaded}
\begin{Highlighting}[]
\KeywordTok{var} \NormalTok{TodoList = }\OtherTok{Backbone}\NormalTok{.}\OtherTok{Collection}\NormalTok{.}\FunctionTok{extend}\NormalTok{(\{}
    \DataTypeTok{model}\NormalTok{: Todo}
\NormalTok{\});}

\CommentTok{// Let's assume our instance of this collection is}
\KeywordTok{this}\NormalTok{.}\FunctionTok{todoList}\NormalTok{;}
\end{Highlighting}
\end{Shaded}

Assuming our collection is instantiating new models itself, it's
necessary for us to stub the model's constructor function for the the
test. This can be done by creating a simple stub as follows:

\begin{Shaded}
\begin{Highlighting}[]
\KeywordTok{this}\NormalTok{.}\FunctionTok{todoStub} \NormalTok{= }\OtherTok{sinon}\NormalTok{.}\FunctionTok{stub}\NormalTok{( window, }\StringTok{'Todo'} \NormalTok{);}
\end{Highlighting}
\end{Shaded}

The above creates a stub of the Todo method on the window object. When
stubbing a persistent object, it's necessary to restore it to its
original state. This can be done in a \texttt{teardown()} as follows:

\begin{Shaded}
\begin{Highlighting}[]
\KeywordTok{this}\NormalTok{.}\OtherTok{todoStub}\NormalTok{.}\FunctionTok{restore}\NormalTok{();}
\end{Highlighting}
\end{Shaded}

After this, we need to alter what the constructor returns, which can be
efficiently done using a plain \texttt{Backbone.Model} constructor.
While this isn't a Todo model, it does still provide us an actual
Backbone model.

\begin{Shaded}
\begin{Highlighting}[]
\NormalTok{setup: }\KeywordTok{function}\NormalTok{() \{}
    \KeywordTok{this}\NormalTok{.}\FunctionTok{model} \NormalTok{= }\KeywordTok{new} \OtherTok{Backbone}\NormalTok{.}\FunctionTok{Model}\NormalTok{(\{}
      \DataTypeTok{id}\NormalTok{: }\DecValTok{2}\NormalTok{,}
      \DataTypeTok{title}\NormalTok{: }\StringTok{'Hello world'}
    \NormalTok{\});}
    \KeywordTok{this}\NormalTok{.}\OtherTok{todoStub}\NormalTok{.}\FunctionTok{returns}\NormalTok{( }\KeywordTok{this}\NormalTok{.}\FunctionTok{model} \NormalTok{);}
\NormalTok{\});}
\end{Highlighting}
\end{Shaded}

The expectation here might be that this snippet would ensure our
TodoList collection always instantiates a stubbed Todo model, but
because a reference to the model in the collection is already present,
we need to reset the model property of our collection as follows:

\begin{Shaded}
\begin{Highlighting}[]
\KeywordTok{this}\NormalTok{.}\OtherTok{todoList}\NormalTok{.}\FunctionTok{model} \NormalTok{= Todo;}
\end{Highlighting}
\end{Shaded}

The result of this is that when our TodoList collection instantiates new
Todo models, it will return our plain Backbone model instance as
desired. This allows us to write a test for the addition of new model
literals as follows:

\begin{Shaded}
\begin{Highlighting}[]
\FunctionTok{module}\NormalTok{( }\StringTok{'Should function when instantiated with model literals'}\NormalTok{, \{}

  \DataTypeTok{setup}\NormalTok{:}\KeywordTok{function}\NormalTok{() \{}

    \KeywordTok{this}\NormalTok{.}\FunctionTok{todoStub} \NormalTok{= }\OtherTok{sinon}\NormalTok{.}\FunctionTok{stub}\NormalTok{(window, }\StringTok{'Todo'}\NormalTok{);}
    \KeywordTok{this}\NormalTok{.}\FunctionTok{model} \NormalTok{= }\KeywordTok{new} \OtherTok{Backbone}\NormalTok{.}\FunctionTok{Model}\NormalTok{(\{}
      \DataTypeTok{id}\NormalTok{: }\DecValTok{2}\NormalTok{,}
      \DataTypeTok{title}\NormalTok{: }\StringTok{'Hello world'}
    \NormalTok{\});}

    \KeywordTok{this}\NormalTok{.}\OtherTok{todoStub}\NormalTok{.}\FunctionTok{returns}\NormalTok{(}\KeywordTok{this}\NormalTok{.}\FunctionTok{model}\NormalTok{);}
    \KeywordTok{this}\NormalTok{.}\FunctionTok{todos} \NormalTok{= }\KeywordTok{new} \FunctionTok{TodoList}\NormalTok{();}

    \CommentTok{// Let's reset the relationship to use a stub}
    \KeywordTok{this}\NormalTok{.}\OtherTok{todos}\NormalTok{.}\FunctionTok{model} \NormalTok{= Todo;}
    
    \CommentTok{// add a model}
    \KeywordTok{this}\NormalTok{.}\OtherTok{todos}\NormalTok{.}\FunctionTok{add}\NormalTok{(\{}
      \DataTypeTok{id}\NormalTok{: }\DecValTok{2}\NormalTok{,}
      \DataTypeTok{title}\NormalTok{: }\StringTok{'Hello world'}
    \NormalTok{\});}
  \NormalTok{\},}

  \DataTypeTok{teardown}\NormalTok{: }\KeywordTok{function}\NormalTok{() \{}
    \KeywordTok{this}\NormalTok{.}\OtherTok{todoStub}\NormalTok{.}\FunctionTok{restore}\NormalTok{();}
  \NormalTok{\}}

\NormalTok{\});}

\FunctionTok{test}\NormalTok{(}\StringTok{'should add a model'}\NormalTok{, }\KeywordTok{function}\NormalTok{() \{}
    \FunctionTok{equal}\NormalTok{( }\KeywordTok{this}\NormalTok{.}\OtherTok{todos}\NormalTok{.}\FunctionTok{length}\NormalTok{, }\DecValTok{1} \NormalTok{);}
\NormalTok{\});}

\FunctionTok{test}\NormalTok{(}\StringTok{'should find a model by id'}\NormalTok{, }\KeywordTok{function}\NormalTok{() \{}
    \FunctionTok{equal}\NormalTok{( }\KeywordTok{this}\NormalTok{.}\OtherTok{todos}\NormalTok{.}\FunctionTok{get}\NormalTok{(}\DecValTok{5}\NormalTok{).}\FunctionTok{get}\NormalTok{(}\StringTok{'id'}\NormalTok{), }\DecValTok{5} \NormalTok{);}
  \NormalTok{\});}
\NormalTok{\});}
\end{Highlighting}
\end{Shaded}

\subsubsection{Mocks}\label{mocks}

Mocks are effectively the same as stubs, however they mock a complete
API and have some built-in expectations for how they should be used. The
difference between a mock and a spy is that as the expectations for
their use are pre-defined and the test will fail if any of these are not
met.

Here's a snippet with sample usage of a mock based on PubSubJS. Here, we
have a \texttt{clearTodo()} method as a callback and use mocks to verify
its behavior.

\begin{Shaded}
\begin{Highlighting}[]
\FunctionTok{test}\NormalTok{(}\StringTok{'should call all subscribers when exceptions'}\NormalTok{, }\KeywordTok{function} \NormalTok{() \{}
    \KeywordTok{var} \NormalTok{myAPI = \{ }\DataTypeTok{clearTodo}\NormalTok{: }\KeywordTok{function} \NormalTok{() \{\} \};}

    \KeywordTok{var} \NormalTok{spy = }\KeywordTok{this}\NormalTok{.}\FunctionTok{spy}\NormalTok{();}
    \KeywordTok{var} \NormalTok{mock = }\KeywordTok{this}\NormalTok{.}\FunctionTok{mock}\NormalTok{( myAPI );}
    \OtherTok{mock}\NormalTok{.}\FunctionTok{expects}\NormalTok{( }\StringTok{'clearTodo'} \NormalTok{).}\FunctionTok{once}\NormalTok{().}\FunctionTok{throws}\NormalTok{();}

    \OtherTok{PubSub}\NormalTok{.}\FunctionTok{subscribe}\NormalTok{( }\StringTok{'message'}\NormalTok{, }\OtherTok{myAPI}\NormalTok{.}\FunctionTok{clearTodo} \NormalTok{);}
    \OtherTok{PubSub}\NormalTok{.}\FunctionTok{subscribe}\NormalTok{( }\StringTok{'message'}\NormalTok{, spy );}
    \OtherTok{PubSub}\NormalTok{.}\FunctionTok{publishSync}\NormalTok{( }\StringTok{'message'}\NormalTok{, }\KeywordTok{undefined} \NormalTok{);}

    \OtherTok{mock}\NormalTok{.}\FunctionTok{verify}\NormalTok{();}
    \FunctionTok{ok}\NormalTok{( }\OtherTok{spy}\NormalTok{.}\FunctionTok{calledOnce} \NormalTok{);}
\NormalTok{\});}
\end{Highlighting}
\end{Shaded}

\subsection{Exercise}\label{exercise-1}

We can now begin writing tests for our Todo application, which are
listed and separated by component (e.g., Models, Collections, etc.).
It's useful to pay attention to the name of the test, the logic being
tested, and most importantly the assertions being made as this will give
you some insight into how what we've learned can be applied to a
complete application.

To get the most out of this section, I recommend looking at the QUnit
Koans included in the \texttt{practicals/qunit-koans} folder - this is a
port of the Backbone.js Jasmine Koans over to QUnit.

\emph{In case you haven't had a chance to try out one of the Koans kits
as yet, they are a set of unit tests using a specific testing framework
that both demonstrate how a set of tests for an application may be
written, but also leave some tests unfilled so that you can complete
them as an exercise.}

\subsubsection{Models}\label{models-3}

For our models we want to at minimum test that:

\begin{itemize}
\itemsep1pt\parskip0pt\parsep0pt
\item
  New instances can be created with the expected default values
\item
  Attributes can be set and retrieved correctly
\item
  Changes to state correctly fire off custom events where needed
\item
  Validation rules are correctly enforced
\end{itemize}

\begin{Shaded}
\begin{Highlighting}[]
\FunctionTok{module}\NormalTok{( }\StringTok{'About Backbone.Model'}\NormalTok{);}

\FunctionTok{test}\NormalTok{(}\StringTok{'Can be created with default values for its attributes.'}\NormalTok{, }\KeywordTok{function}\NormalTok{() \{}
    \FunctionTok{expect}\NormalTok{( }\DecValTok{3} \NormalTok{);}

    \KeywordTok{var} \NormalTok{todo = }\KeywordTok{new} \FunctionTok{Todo}\NormalTok{();}
    \FunctionTok{equal}\NormalTok{( }\OtherTok{todo}\NormalTok{.}\FunctionTok{get}\NormalTok{(}\StringTok{'text'}\NormalTok{), }\StringTok{''} \NormalTok{);}
    \FunctionTok{equal}\NormalTok{( }\OtherTok{todo}\NormalTok{.}\FunctionTok{get}\NormalTok{(}\StringTok{'done'}\NormalTok{), }\KeywordTok{false} \NormalTok{);}
    \FunctionTok{equal}\NormalTok{( }\OtherTok{todo}\NormalTok{.}\FunctionTok{get}\NormalTok{(}\StringTok{'order'}\NormalTok{), }\DecValTok{0} \NormalTok{);}
\NormalTok{\});}

\FunctionTok{test}\NormalTok{(}\StringTok{'Will set attributes on the model instance when created.'}\NormalTok{, }\KeywordTok{function}\NormalTok{() \{}
    \FunctionTok{expect}\NormalTok{( }\DecValTok{1} \NormalTok{);}

    \KeywordTok{var} \NormalTok{todo = }\KeywordTok{new} \FunctionTok{Todo}\NormalTok{( \{ }\DataTypeTok{text}\NormalTok{: }\StringTok{'Get oil change for car.'} \NormalTok{\} );}
    \FunctionTok{equal}\NormalTok{( }\OtherTok{todo}\NormalTok{.}\FunctionTok{get}\NormalTok{(}\StringTok{'text'}\NormalTok{), }\StringTok{'Get oil change for car.'} \NormalTok{);}

\NormalTok{\});}

\FunctionTok{test}\NormalTok{(}\StringTok{'Will call a custom initialize function on the model instance when created.'}\NormalTok{, }\KeywordTok{function}\NormalTok{() \{}
    \FunctionTok{expect}\NormalTok{( }\DecValTok{1} \NormalTok{);}

    \KeywordTok{var} \NormalTok{toot = }\KeywordTok{new} \FunctionTok{Todo}\NormalTok{(\{ }\DataTypeTok{text}\NormalTok{: }\StringTok{'Stop monkeys from throwing their own crap!'} \NormalTok{\});}
    \FunctionTok{equal}\NormalTok{( }\OtherTok{toot}\NormalTok{.}\FunctionTok{get}\NormalTok{(}\StringTok{'text'}\NormalTok{), }\StringTok{'Stop monkeys from throwing their own rainbows!'} \NormalTok{);}
\NormalTok{\});}

\FunctionTok{test}\NormalTok{(}\StringTok{'Fires a custom event when the state changes.'}\NormalTok{, }\KeywordTok{function}\NormalTok{() \{}
    \FunctionTok{expect}\NormalTok{( }\DecValTok{1} \NormalTok{);}

    \KeywordTok{var} \NormalTok{spy = }\KeywordTok{this}\NormalTok{.}\FunctionTok{spy}\NormalTok{();}
    \KeywordTok{var} \NormalTok{todo = }\KeywordTok{new} \FunctionTok{Todo}\NormalTok{();}

    \OtherTok{todo}\NormalTok{.}\FunctionTok{on}\NormalTok{( }\StringTok{'change'}\NormalTok{, spy );}
    \CommentTok{// Change the model state}
    \OtherTok{todo}\NormalTok{.}\FunctionTok{set}\NormalTok{( \{ }\DataTypeTok{text}\NormalTok{: }\StringTok{'new text'} \NormalTok{\} );}

    \FunctionTok{ok}\NormalTok{( }\OtherTok{spy}\NormalTok{.}\FunctionTok{calledOnce}\NormalTok{, }\StringTok{'A change event callback was correctly triggered'} \NormalTok{);}
\NormalTok{\});}


\FunctionTok{test}\NormalTok{(}\StringTok{'Can contain custom validation rules, and will trigger an invalid event on failed validation.'}\NormalTok{, }\KeywordTok{function}\NormalTok{() \{}
    \FunctionTok{expect}\NormalTok{( }\DecValTok{3} \NormalTok{);}

    \KeywordTok{var} \NormalTok{errorCallback = }\KeywordTok{this}\NormalTok{.}\FunctionTok{spy}\NormalTok{();}
    \KeywordTok{var} \NormalTok{todo = }\KeywordTok{new} \FunctionTok{Todo}\NormalTok{();}

    \OtherTok{todo}\NormalTok{.}\FunctionTok{on}\NormalTok{(}\StringTok{'invalid'}\NormalTok{, errorCallback);}
    \CommentTok{// Change the model state in such a way that validation will fail}
    \OtherTok{todo}\NormalTok{.}\FunctionTok{set}\NormalTok{( \{ }\DataTypeTok{done}\NormalTok{: }\StringTok{'not a boolean'} \NormalTok{\} );}

    \FunctionTok{ok}\NormalTok{( }\OtherTok{errorCallback}\NormalTok{.}\FunctionTok{called}\NormalTok{, }\StringTok{'A failed validation correctly triggered an error'} \NormalTok{);}
    \FunctionTok{notEqual}\NormalTok{( }\OtherTok{errorCallback}\NormalTok{.}\FunctionTok{getCall}\NormalTok{(}\DecValTok{0}\NormalTok{), }\KeywordTok{undefined} \NormalTok{);}
    \FunctionTok{equal}\NormalTok{( }\OtherTok{errorCallback}\NormalTok{.}\FunctionTok{getCall}\NormalTok{(}\DecValTok{0}\NormalTok{).}\FunctionTok{args}\NormalTok{[}\DecValTok{1}\NormalTok{], }\StringTok{'Todo.done must be a boolean value.'} \NormalTok{);}

\NormalTok{\});}
\end{Highlighting}
\end{Shaded}

\subsubsection{Collections}\label{collections-2}

For our collection we'll want to test that:

\begin{itemize}
\itemsep1pt\parskip0pt\parsep0pt
\item
  The Collection has a Todo Model
\item
  Uses localStorage for syncing
\item
  That done(), remaining() and clear() work as expected
\item
  The order for Todos is numerically correct
\end{itemize}

\begin{Shaded}
\begin{Highlighting}[]
  \FunctionTok{describe}\NormalTok{(}\StringTok{'Test Collection'}\NormalTok{, }\KeywordTok{function}\NormalTok{() \{}

    \FunctionTok{beforeEach}\NormalTok{(}\KeywordTok{function}\NormalTok{() \{}

      \CommentTok{// Define new todos}
      \KeywordTok{this}\NormalTok{.}\FunctionTok{todoOne} \NormalTok{= }\KeywordTok{new} \NormalTok{Todo;}
      \KeywordTok{this}\NormalTok{.}\FunctionTok{todoTwo} \NormalTok{= }\KeywordTok{new} \FunctionTok{Todo}\NormalTok{(\{}
        \DataTypeTok{title}\NormalTok{: }\StringTok{"Buy some milk"}
      \NormalTok{\});}

      \CommentTok{// Create a new collection of todos for testing}
      \KeywordTok{return} \KeywordTok{this}\NormalTok{.}\FunctionTok{todos} \NormalTok{= }\KeywordTok{new} \FunctionTok{TodoList}\NormalTok{([}\KeywordTok{this}\NormalTok{.}\FunctionTok{todoOne}\NormalTok{, }\KeywordTok{this}\NormalTok{.}\FunctionTok{todoTwo}\NormalTok{]);}
    \NormalTok{\});}

    \FunctionTok{it}\NormalTok{(}\StringTok{'Has the Todo model'}\NormalTok{, }\KeywordTok{function}\NormalTok{() \{}
      \KeywordTok{return} \FunctionTok{expect}\NormalTok{(}\KeywordTok{this}\NormalTok{.}\OtherTok{todos}\NormalTok{.}\FunctionTok{model}\NormalTok{).}\FunctionTok{toBe}\NormalTok{(Todo);}
    \NormalTok{\});}

    \FunctionTok{it}\NormalTok{(}\StringTok{'Uses local storage'}\NormalTok{, }\KeywordTok{function}\NormalTok{() \{}
      \KeywordTok{return} \FunctionTok{expect}\NormalTok{(}\KeywordTok{this}\NormalTok{.}\OtherTok{todos}\NormalTok{.}\FunctionTok{localStorage}\NormalTok{).}\FunctionTok{toEqual}\NormalTok{(}\KeywordTok{new} \FunctionTok{Store}\NormalTok{(}\StringTok{'todos-backbone'}\NormalTok{));}
    \NormalTok{\});}

    \FunctionTok{describe}\NormalTok{(}\StringTok{'done'}\NormalTok{, }\KeywordTok{function}\NormalTok{() \{}
      \KeywordTok{return} \FunctionTok{it}\NormalTok{(}\StringTok{'returns an array of the todos that are done'}\NormalTok{, }\KeywordTok{function}\NormalTok{() \{}
        \KeywordTok{this}\NormalTok{.}\OtherTok{todoTwo}\NormalTok{.}\FunctionTok{done} \NormalTok{= }\KeywordTok{true}\NormalTok{;}
        \KeywordTok{return} \FunctionTok{expect}\NormalTok{(}\KeywordTok{this}\NormalTok{.}\OtherTok{todos}\NormalTok{.}\FunctionTok{done}\NormalTok{()).}\FunctionTok{toEqual}\NormalTok{([}\KeywordTok{this}\NormalTok{.}\FunctionTok{todoTwo}\NormalTok{]);}
      \NormalTok{\});}
    \NormalTok{\});}

    \FunctionTok{describe}\NormalTok{(}\StringTok{'remaining'}\NormalTok{, }\KeywordTok{function}\NormalTok{() \{}
      \KeywordTok{return} \FunctionTok{it}\NormalTok{(}\StringTok{'returns an array of the todos that are not done'}\NormalTok{, }\KeywordTok{function}\NormalTok{() \{}
        \KeywordTok{this}\NormalTok{.}\OtherTok{todoTwo}\NormalTok{.}\FunctionTok{done} \NormalTok{= }\KeywordTok{true}\NormalTok{;}
        \KeywordTok{return} \FunctionTok{expect}\NormalTok{(}\KeywordTok{this}\NormalTok{.}\OtherTok{todos}\NormalTok{.}\FunctionTok{remaining}\NormalTok{()).}\FunctionTok{toEqual}\NormalTok{([}\KeywordTok{this}\NormalTok{.}\FunctionTok{todoOne}\NormalTok{]);}
      \NormalTok{\});}
    \NormalTok{\});}

    \FunctionTok{describe}\NormalTok{(}\StringTok{'clear'}\NormalTok{, }\KeywordTok{function}\NormalTok{() \{}
      \KeywordTok{return} \FunctionTok{it}\NormalTok{(}\StringTok{'destroys the current todo from local storage'}\NormalTok{, }\KeywordTok{function}\NormalTok{() \{}
        \FunctionTok{expect}\NormalTok{(}\KeywordTok{this}\NormalTok{.}\OtherTok{todos}\NormalTok{.}\FunctionTok{models}\NormalTok{).}\FunctionTok{toEqual}\NormalTok{([}\KeywordTok{this}\NormalTok{.}\FunctionTok{todoOne}\NormalTok{, }\KeywordTok{this}\NormalTok{.}\FunctionTok{todoTwo}\NormalTok{]);}
        \KeywordTok{this}\NormalTok{.}\OtherTok{todos}\NormalTok{.}\FunctionTok{clear}\NormalTok{(}\KeywordTok{this}\NormalTok{.}\FunctionTok{todoOne}\NormalTok{);}
        \KeywordTok{return} \FunctionTok{expect}\NormalTok{(}\KeywordTok{this}\NormalTok{.}\OtherTok{todos}\NormalTok{.}\FunctionTok{models}\NormalTok{).}\FunctionTok{toEqual}\NormalTok{([}\KeywordTok{this}\NormalTok{.}\FunctionTok{todoTwo}\NormalTok{]);}
      \NormalTok{\});}
    \NormalTok{\});}

    \KeywordTok{return} \FunctionTok{describe}\NormalTok{(}\StringTok{'Order sets the order on todos ascending numerically'}\NormalTok{, }\KeywordTok{function}\NormalTok{() \{}
      \FunctionTok{it}\NormalTok{(}\StringTok{'defaults to one when there arent any items in the collection'}\NormalTok{, }\KeywordTok{function}\NormalTok{() \{}
        \KeywordTok{this}\NormalTok{.}\FunctionTok{emptyTodos} \NormalTok{= }\KeywordTok{new} \OtherTok{TodoApp}\NormalTok{.}\OtherTok{Collections}\NormalTok{.}\FunctionTok{TodoList}\NormalTok{;}
        \KeywordTok{return} \FunctionTok{expect}\NormalTok{(}\KeywordTok{this}\NormalTok{.}\OtherTok{emptyTodos}\NormalTok{.}\FunctionTok{order}\NormalTok{()).}\FunctionTok{toEqual}\NormalTok{(}\DecValTok{0}\NormalTok{);}
      \NormalTok{\});}

      \KeywordTok{return} \FunctionTok{it}\NormalTok{(}\StringTok{'Increments the order by one each time'}\NormalTok{, }\KeywordTok{function}\NormalTok{() \{}
        \FunctionTok{expect}\NormalTok{(}\KeywordTok{this}\NormalTok{.}\OtherTok{todos}\NormalTok{.}\FunctionTok{order}\NormalTok{(}\KeywordTok{this}\NormalTok{.}\FunctionTok{todoOne}\NormalTok{)).}\FunctionTok{toEqual}\NormalTok{(}\DecValTok{1}\NormalTok{);}
        \KeywordTok{return} \FunctionTok{expect}\NormalTok{(}\KeywordTok{this}\NormalTok{.}\OtherTok{todos}\NormalTok{.}\FunctionTok{order}\NormalTok{(}\KeywordTok{this}\NormalTok{.}\FunctionTok{todoTwo}\NormalTok{)).}\FunctionTok{toEqual}\NormalTok{(}\DecValTok{2}\NormalTok{);}
      \NormalTok{\});}
    \NormalTok{\});}

  \NormalTok{\});}
\end{Highlighting}
\end{Shaded}

\subsubsection{Views}\label{views-3}

For our views we want to ensure:

\begin{itemize}
\itemsep1pt\parskip0pt\parsep0pt
\item
  They are being correctly tied to a DOM element when created
\item
  They can render, after which the DOM representation of the view should
  be visible
\item
  They support wiring up view methods to DOM elements
\end{itemize}

One could also take this further and test that user interactions with
the view correctly result in any models that need to be changed being
updated correctly.

\begin{Shaded}
\begin{Highlighting}[]
\FunctionTok{module}\NormalTok{( }\StringTok{'About Backbone.View'}\NormalTok{, \{}
    \DataTypeTok{setup}\NormalTok{: }\KeywordTok{function}\NormalTok{() \{}
        \FunctionTok{$}\NormalTok{(}\StringTok{'body'}\NormalTok{).}\FunctionTok{append}\NormalTok{(}\StringTok{'<ul id="todoList"></ul>'}\NormalTok{);}
        \KeywordTok{this}\NormalTok{.}\FunctionTok{todoView} \NormalTok{= }\KeywordTok{new} \FunctionTok{TodoView}\NormalTok{(\{ }\DataTypeTok{model}\NormalTok{: }\KeywordTok{new} \FunctionTok{Todo}\NormalTok{() \});}
    \NormalTok{\},}
    \DataTypeTok{teardown}\NormalTok{: }\KeywordTok{function}\NormalTok{() \{}
        \KeywordTok{this}\NormalTok{.}\OtherTok{todoView}\NormalTok{.}\FunctionTok{remove}\NormalTok{();}
        \FunctionTok{$}\NormalTok{(}\StringTok{'#todoList'}\NormalTok{).}\FunctionTok{remove}\NormalTok{();}
    \NormalTok{\}}
\NormalTok{\});}

\FunctionTok{test}\NormalTok{(}\StringTok{'Should be tied to a DOM element when created, based off the property provided.'}\NormalTok{, }\KeywordTok{function}\NormalTok{() \{}
    \FunctionTok{expect}\NormalTok{( }\DecValTok{1} \NormalTok{);}
    \FunctionTok{equal}\NormalTok{( }\KeywordTok{this}\NormalTok{.}\OtherTok{todoView}\NormalTok{.}\OtherTok{el}\NormalTok{.}\OtherTok{tagName}\NormalTok{.}\FunctionTok{toLowerCase}\NormalTok{(), }\StringTok{'li'} \NormalTok{);}
\NormalTok{\});}

\FunctionTok{test}\NormalTok{(}\StringTok{'Is backed by a model instance, which provides the data.'}\NormalTok{, }\KeywordTok{function}\NormalTok{() \{}
    \FunctionTok{expect}\NormalTok{( }\DecValTok{2} \NormalTok{);}
    \FunctionTok{notEqual}\NormalTok{( }\KeywordTok{this}\NormalTok{.}\OtherTok{todoView}\NormalTok{.}\FunctionTok{model}\NormalTok{, }\KeywordTok{undefined} \NormalTok{);}
    \FunctionTok{equal}\NormalTok{( }\KeywordTok{this}\NormalTok{.}\OtherTok{todoView}\NormalTok{.}\OtherTok{model}\NormalTok{.}\FunctionTok{get}\NormalTok{(}\StringTok{'done'}\NormalTok{), }\KeywordTok{false} \NormalTok{);}
\NormalTok{\});}

\FunctionTok{test}\NormalTok{(}\StringTok{'Can render, after which the DOM representation of the view will be visible.'}\NormalTok{, }\KeywordTok{function}\NormalTok{() \{}
   \KeywordTok{this}\NormalTok{.}\OtherTok{todoView}\NormalTok{.}\FunctionTok{render}\NormalTok{();}

    \CommentTok{// Append the DOM representation of the view to ul#todoList}
    \FunctionTok{$}\NormalTok{(}\StringTok{'ul#todoList'}\NormalTok{).}\FunctionTok{append}\NormalTok{(}\KeywordTok{this}\NormalTok{.}\OtherTok{todoView}\NormalTok{.}\FunctionTok{el}\NormalTok{);}

    \CommentTok{// Check the number of li items rendered to the list}
    \FunctionTok{equal}\NormalTok{(}\FunctionTok{$}\NormalTok{(}\StringTok{'#todoList'}\NormalTok{).}\FunctionTok{find}\NormalTok{(}\StringTok{'li'}\NormalTok{).}\FunctionTok{length}\NormalTok{, }\DecValTok{1}\NormalTok{);}
\NormalTok{\});}

\FunctionTok{asyncTest}\NormalTok{(}\StringTok{'Can wire up view methods to DOM elements.'}\NormalTok{, }\KeywordTok{function}\NormalTok{() \{}
    \FunctionTok{expect}\NormalTok{( }\DecValTok{2} \NormalTok{);}
    \KeywordTok{var} \NormalTok{viewElt;}

    \FunctionTok{$}\NormalTok{(}\StringTok{'#todoList'}\NormalTok{).}\FunctionTok{append}\NormalTok{( }\KeywordTok{this}\NormalTok{.}\OtherTok{todoView}\NormalTok{.}\FunctionTok{render}\NormalTok{().}\FunctionTok{el} \NormalTok{);}

    \FunctionTok{setTimeout}\NormalTok{(}\KeywordTok{function}\NormalTok{() \{}
        \NormalTok{viewElt = }\FunctionTok{$}\NormalTok{(}\StringTok{'#todoList li input.check'}\NormalTok{).}\FunctionTok{filter}\NormalTok{(}\StringTok{':first'}\NormalTok{);}

        \FunctionTok{equal}\NormalTok{(}\OtherTok{viewElt}\NormalTok{.}\FunctionTok{length} \NormalTok{> }\DecValTok{0}\NormalTok{, }\KeywordTok{true}\NormalTok{);}

        \CommentTok{// Ensure QUnit knows we can continue}
        \FunctionTok{start}\NormalTok{();}
    \NormalTok{\}, }\DecValTok{1000}\NormalTok{, }\StringTok{'Expected DOM Elt to exist'}\NormalTok{);}

    \CommentTok{// Trigget the view to toggle the 'done' status on an item or items}
    \FunctionTok{$}\NormalTok{(}\StringTok{'#todoList li input.check'}\NormalTok{).}\FunctionTok{click}\NormalTok{();}

    \CommentTok{// Check the done status for the model is true}
    \FunctionTok{equal}\NormalTok{( }\KeywordTok{this}\NormalTok{.}\OtherTok{todoView}\NormalTok{.}\OtherTok{model}\NormalTok{.}\FunctionTok{get}\NormalTok{(}\StringTok{'done'}\NormalTok{), }\KeywordTok{true} \NormalTok{);}
\NormalTok{\});}
\end{Highlighting}
\end{Shaded}

\subsubsection{App}\label{app}

It can also be useful to write tests for any application bootstrap you
may have in place. For the following module, our setup instantiates and
appends to a TodoApp view and we can test anything from local instances
of views being correctly defined to application interactions correctly
resulting in changes to instances of local collections.

\begin{Shaded}
\begin{Highlighting}[]
\FunctionTok{module}\NormalTok{( }\StringTok{'About Backbone Applications'} \NormalTok{, \{}
    \DataTypeTok{setup}\NormalTok{: }\KeywordTok{function}\NormalTok{() \{}
        \OtherTok{Backbone}\NormalTok{.}\FunctionTok{localStorageDB} \NormalTok{= }\KeywordTok{new} \FunctionTok{Store}\NormalTok{(}\StringTok{'testTodos'}\NormalTok{);}
        \FunctionTok{$}\NormalTok{(}\StringTok{'#qunit-fixture'}\NormalTok{).}\FunctionTok{append}\NormalTok{(}\StringTok{'<div id="app"></div>'}\NormalTok{);}
        \KeywordTok{this}\NormalTok{.}\FunctionTok{App} \NormalTok{= }\KeywordTok{new} \FunctionTok{TodoApp}\NormalTok{(\{ }\DataTypeTok{appendTo}\NormalTok{: }\FunctionTok{$}\NormalTok{(}\StringTok{'#app'}\NormalTok{) \});}
    \NormalTok{\},}

    \DataTypeTok{teardown}\NormalTok{: }\KeywordTok{function}\NormalTok{() \{}
        \KeywordTok{this}\NormalTok{.}\OtherTok{App}\NormalTok{.}\OtherTok{todos}\NormalTok{.}\FunctionTok{reset}\NormalTok{();}
        \FunctionTok{$}\NormalTok{(}\StringTok{'#app'}\NormalTok{).}\FunctionTok{remove}\NormalTok{();}
    \NormalTok{\}}
\NormalTok{\});}

\FunctionTok{test}\NormalTok{(}\StringTok{'Should bootstrap the application by initializing the Collection.'}\NormalTok{, }\KeywordTok{function}\NormalTok{() \{}
    \FunctionTok{expect}\NormalTok{( }\DecValTok{2} \NormalTok{);}

    \CommentTok{// The todos collection should not be undefined}
    \FunctionTok{notEqual}\NormalTok{( }\KeywordTok{this}\NormalTok{.}\OtherTok{App}\NormalTok{.}\FunctionTok{todos}\NormalTok{, }\KeywordTok{undefined} \NormalTok{);}

    \CommentTok{// The initial length of our todos should however be zero}
    \FunctionTok{equal}\NormalTok{( }\KeywordTok{this}\NormalTok{.}\OtherTok{App}\NormalTok{.}\OtherTok{todos}\NormalTok{.}\FunctionTok{length}\NormalTok{, }\DecValTok{0} \NormalTok{);}
\NormalTok{\});}

\FunctionTok{test}\NormalTok{( }\StringTok{'Should bind Collection events to View creation.'} \NormalTok{, }\KeywordTok{function}\NormalTok{() \{}

      \CommentTok{// Set the value of a brand new todo within the input box}
      \FunctionTok{$}\NormalTok{(}\StringTok{'#new-todo'}\NormalTok{).}\FunctionTok{val}\NormalTok{( }\StringTok{'Buy some milk'} \NormalTok{);}

      \CommentTok{// Trigger the enter (return) key to be pressed inside #new-todo}
      \CommentTok{// causing the new item to be added to the todos collection}
      \FunctionTok{$}\NormalTok{(}\StringTok{'#new-todo'}\NormalTok{).}\FunctionTok{trigger}\NormalTok{(}\KeywordTok{new} \OtherTok{$}\NormalTok{.}\FunctionTok{Event}\NormalTok{( }\StringTok{'keypress'}\NormalTok{, \{ }\DataTypeTok{keyCode}\NormalTok{: }\DecValTok{13} \NormalTok{\} ));}

      \CommentTok{// The length of our collection should now be 1}
      \FunctionTok{equal}\NormalTok{( }\KeywordTok{this}\NormalTok{.}\OtherTok{App}\NormalTok{.}\OtherTok{todos}\NormalTok{.}\FunctionTok{length}\NormalTok{, }\DecValTok{1} \NormalTok{);}
 \NormalTok{\});}
\end{Highlighting}
\end{Shaded}

\subsection{Further Reading \&
Resources}\label{further-reading-resources}

That's it for this section on testing applications with QUnit and
SinonJS. I encourage you to try out the
\href{https://github.com/addyosmani/backbone-koans-qunit}{QUnit
Backbone.js Koans} and see if you can extend some of the examples. For
further reading consider looking at some of the additional resources
below:

\begin{itemize}
\itemsep1pt\parskip0pt\parsep0pt
\item
  \textbf{\href{http://tddjs.com/}{Test-driven JavaScript Development
  (book)}}
\item
  \textbf{\href{http://sinonjs.org/qunit/}{SinonJS/QUnit Adapter}}
\item
  \textbf{\href{http://cjohansen.no/en/javascript/using_sinon_js_with_qunit}{SinonJS
  and QUnit}}
\item
  \textbf{\href{http://msdn.microsoft.com/en-us/scriptjunkie/gg749824}{Automating
  JavaScript Testing With QUnit}}
\item
  \textbf{\href{http://benalman.com/talks/unit-testing-qunit.html}{Ben
  Alman's Unit Testing With QUnit}}
\item
  \textbf{\href{https://github.com/jc00ke/qunit-backbone}{Another
  QUnit/Backbone.js demo project}}
\item
  \textbf{\href{http://devblog.supportbee.com/2012/02/10/helpers-for-testing-backbone-js-apps-using-jasmine-and-sinon-js/}{SinonJS
  helpers for Backbone}}
\end{itemize}

\section{Resources}\label{resources}

\subsection{Books \& Courses}\label{books-courses}

\begin{itemize}
\itemsep1pt\parskip0pt\parsep0pt
\item
  \href{https://peepcode.com/products/backbone-js}{PeepCode: Backbone.js
  Basics}
\item
  \href{http://www.codeschool.com/courses/anatomy-of-backbonejs}{CodeSchool:
  Anatomy Of Backbone}
\item
  \href{http://recipeswithbackbone.com/}{Recipes With Backbone}
\item
  \href{http://ricostacruz.com/backbone-patterns/}{Backbone Patterns}
\item
  \href{https://learn.thoughtbot.com/products/1-backbone-js-on-rails}{Backbone
  On Rails}
\item
  \href{http://www.integralist.co.uk/posts/mvc-in-javascript-with-backbone-js/index.html}{MVC
  In JavaScript With Backbone}
\item
  \href{http://backbonetutorials.com/}{Backbone Tutorials}
\item
  \href{http://lostechies.com/derickbailey/2011/09/13/resources-for-and-how-i-learned-backbone-js/}{Derick
  Bailey's Resources For Learning Backbone}
\end{itemize}

\subsection{Extensions/Libraries}\label{extensionslibraries}

\begin{itemize}
\itemsep1pt\parskip0pt\parsep0pt
\item
  \href{http://marionettejs.com/}{MarionetteJS}
\item
  \href{https://github.com/aurajs/aura}{AuraJS}
\item
  \href{http://thoraxjs.org}{Thorax}
\item
  \href{http://walmartlabs.github.com/lumbar}{Lumbar}
\item
  \href{https://github.com/tbranyen/backbone.layoutmanager}{Backbone
  Layout Manager}
\item
  \href{https://github.com/backbone-boilerplate/backbone-boilerplate}{Backbone
  Boilerplate}
\item
  \href{https://github.com/theironcook/Backbone.ModelBinder}{Backbone.ModelBinder}
\item
  \href{https://github.com/PaulUithol/Backbone-relational}{Backbone
  Relational - for model relationships}
\item
  \href{https://github.com/janmonschke/backbone-couchdb}{Backbone
  CouchDB}
\item
  \href{https://github.com/n-time/backbone.validations}{Backbone
  Validations - HTML5 inspired validations}
\end{itemize}

\section{Conclusions}\label{conclusions-3}

I hope that you've found this introduction to Backbone.js of value. What
you've hopefully learned is that while building a JavaScript-heavy
application using nothing more than a DOM manipulation library (such as
jQuery) is certainly a possible feat, it is difficult to build anything
non-trivial without any formal structure in place. Your nested pile of
jQuery callbacks and DOM elements are unlikely to scale and they can be
very difficult to maintain as your application grows.

The beauty of Backbone.js is its simplicity. It's very small given the
functionality and flexibility it provides, which is evident if you begin
to study the Backbone.js source. In the words of Jeremy Ashkenas, ``The
essential premise at the heart of Backbone has always been to try and
discover the minimal set of data-structuring (Models and Collections)
and user interface (Views and URLs) primitives that are useful when
building web applications with JavaScript.'' It just helps you improve
the structure of your applications, helping you better separate
concerns. There isn't anything more to it than that.

Backbone offers Models with key-value bindings and events, Collections
with an API of rich enumerable methods, declarative Views with event
handling and a simple way to connect an existing API to your client-side
application over a RESTful JSON interface. Use it and you can abstract
away data into sane models and your DOM manipulation into views, binding
them together using nothing more than events.

Almost any developer working on JavaScript applications for a while will
ultimately create a similar solution on their own if they value
architecture and maintainability. The alternative to using it or
something similar is rolling your own - often a process that involves
glueing together a diverse set of libraries that weren't built to work
together. You might use jQuery BBQ for history management and Handlebars
for templating, while writing abstracts for organizing and testing code
by yourself.

Contrast this with Backbone, which has
\href{http://en.wikipedia.org/wiki/Literate_programming}{literate}
\href{http://backbonejs.org/docs/backbone.html}{documentation} of the
source code, a thriving community of both users and hackers, and a large
number of questions about it asked and answered daily on sites like
\href{http://stackoverflow.com/search?q=backbone}{Stack Overflow}.
Rather than re-inventing the wheel there are many advantages to
structuring your application using a solution based on the collective
knowledge and experience of an entire community.

In addition to helping provide sane structure to your applications,
Backbone is highly extensible supporting more custom architecture should
you require more than what is prescribed out of the box. This is evident
by the number of extensions and plugins which have been released for it
over the past year, including those which we have touched upon such as
MarionetteJS and Thorax.

These days Backbone.js powers many complex web applications, ranging
from the LinkedIn
\href{http://touch.www.linkedin.com/mobile.html}{mobile app} to popular
RSS readers such as \href{http://newsblur.com}{NewsBlur} through to
social commentary widgets such as \href{http://disqus.com/}{Disqus}.
This small library of simple, but sane abstractions has helped to create
a new generation of rich web applications, and I and my collaborators
hope that in time it can help you too.

If you're wondering whether it is worth using Backbone on a project, ask
yourself whether what you are building is complex enough to merit using
it. Are you hitting the limits of your ability to organize your code?
Will your application have regular changes to what is displayed in the
UI without a trip back to the server for new pages? Would you benefit
from a separation of concerns? If so, a solution like Backbone may be
able to help.

Google's GMail is often cited as an example of a well built single-page
app. If you've used it, you might have noticed that it requests a large
initial chunk, representing most of the JavaScript, CSS and HTML most
users will need and everything extra needed after that occurs in the
background. GMail can easily switch between your inbox to your spam
folder without needing the whole page to be re-rendered. Libraries like
Backbone make it easier for web developers to create experiences like
this.

That said, Backbone won't be able to help if you're planning on building
something which isn't worth the learning curve associated with a
library. If your application or site will still be using the server to
do the heavy lifting of constructing and serving complete pages to the
browser, you may find just using plain JavaScript or jQuery for simple
effects or interactions to be more appropriate. Spend time assessing how
suitable Backbone might be for you and make the right choice on a
per-project basis.

Backbone is neither difficult to learn nor use, however the time and
effort you spend learning how to structure applications using it will be
well worth it. While reading this book will equip you with the
fundamentals needed to understand the library, the best way to learn is
to try building your own real-world applications. You will hopefully
find that the end product is cleaner, better organized and more
maintainable code.

With that, I wish you the very best with your onward journey into the
world of Backbone and will leave you with a quote from American writer
\href{http://en.wikipedia.org/wiki/Henry_Miller}{Henry Miller} - ``One's
destination is never a place, but a new way of seeing things.''

\section{Appendix}\label{appendix}

\subsection{A Simple JavaScript MVC
Implementation}\label{a-simple-javascript-mvc-implementation}

A comprehensive discussion of Backbone's implementation is beyond the
scope of this book. We can, however, present a simple MVC library -
which we will call Cranium.js - that illustrates how libraries such as
Backbone implement the MVC pattern.

Like Backbone, we will rely on
\href{http://underscorejs.org}{Underscore} for inheritance and
templating.

\subsubsection{Event System}\label{event-system}

At the heart of our JavaScript MVC implementation is an \texttt{Event}
system (object) based on the Publisher-Subscriber Pattern which makes it
possible for MVC components to communicate in an elegant, decoupled
manner. Subscribers `listen' for specific events of interest and react
when Publishers broadcast these events.

\texttt{Event} is mixed into both the View and Model components so that
instances of either of these components can publish events of interest.

\begin{Shaded}
\begin{Highlighting}[]
\CommentTok{// cranium.js - Cranium.Events}

\KeywordTok{var} \NormalTok{Cranium = Cranium || \{\};}

\CommentTok{// Set DOM selection utility}
\KeywordTok{var} \NormalTok{$ = }\OtherTok{document}\NormalTok{.}\OtherTok{querySelector}\NormalTok{.}\FunctionTok{bind}\NormalTok{(document) || }\KeywordTok{this}\NormalTok{.}\FunctionTok{jQuery} \NormalTok{|| }\KeywordTok{this}\NormalTok{.}\FunctionTok{Zepto}\NormalTok{;}

\CommentTok{// Mix in to any object in order to provide it with custom events.}
\KeywordTok{var} \NormalTok{Events = }\OtherTok{Cranium}\NormalTok{.}\FunctionTok{Events} \NormalTok{= \{}
  \CommentTok{// Keeps list of events and associated listeners}
  \DataTypeTok{channels}\NormalTok{: \{\},}

  \CommentTok{// Counter}
  \DataTypeTok{eventNumber}\NormalTok{: }\DecValTok{0}\NormalTok{,}

  \CommentTok{// Announce events and passes data to the listeners;}
  \DataTypeTok{trigger}\NormalTok{: }\KeywordTok{function} \NormalTok{(events, data) \{}
    \KeywordTok{for} \NormalTok{(}\KeywordTok{var} \NormalTok{topic }\KeywordTok{in} \OtherTok{Cranium}\NormalTok{.}\OtherTok{Events}\NormalTok{.}\FunctionTok{channels}\NormalTok{)\{}
      \KeywordTok{if} \NormalTok{(}\OtherTok{Cranium}\NormalTok{.}\OtherTok{Events}\NormalTok{.}\OtherTok{channels}\NormalTok{.}\FunctionTok{hasOwnProperty}\NormalTok{(topic)) \{}
        \KeywordTok{if} \NormalTok{(}\OtherTok{topic}\NormalTok{.}\FunctionTok{split}\NormalTok{(}\StringTok{"-"}\NormalTok{)[}\DecValTok{0}\NormalTok{] == events)\{}
          \OtherTok{Cranium}\NormalTok{.}\OtherTok{Events}\NormalTok{.}\FunctionTok{channels}\NormalTok{[topic](data) !== }\KeywordTok{false} \NormalTok{|| }\KeywordTok{delete} \OtherTok{Cranium}\NormalTok{.}\OtherTok{Events}\NormalTok{.}\FunctionTok{channels}\NormalTok{[topic];}
        \NormalTok{\}}
      \NormalTok{\}}
    \NormalTok{\}}
  \NormalTok{\},}
  \CommentTok{// Registers an event type and its listener}
  \DataTypeTok{on}\NormalTok{: }\KeywordTok{function} \NormalTok{(events, callback) \{}
    \OtherTok{Cranium}\NormalTok{.}\OtherTok{Events}\NormalTok{.}\FunctionTok{channels}\NormalTok{[events + --}\OtherTok{Cranium}\NormalTok{.}\OtherTok{Events}\NormalTok{.}\FunctionTok{eventNumber}\NormalTok{] = callback;}
  \NormalTok{\},}
  \CommentTok{// Unregisters an event type and its listener}
  \DataTypeTok{off}\NormalTok{: }\KeywordTok{function}\NormalTok{(topic) \{}
    \KeywordTok{var} \NormalTok{topic;}
    \KeywordTok{for} \NormalTok{(topic }\KeywordTok{in} \OtherTok{Cranium}\NormalTok{.}\OtherTok{Events}\NormalTok{.}\FunctionTok{channels}\NormalTok{) \{}
      \KeywordTok{if} \NormalTok{(}\OtherTok{Cranium}\NormalTok{.}\OtherTok{Events}\NormalTok{.}\OtherTok{channels}\NormalTok{.}\FunctionTok{hasOwnProperty}\NormalTok{(topic)) \{}
        \KeywordTok{if} \NormalTok{(}\OtherTok{topic}\NormalTok{.}\FunctionTok{split}\NormalTok{(}\StringTok{"-"}\NormalTok{)[}\DecValTok{0}\NormalTok{] == events) \{}
          \KeywordTok{delete} \OtherTok{Cranium}\NormalTok{.}\OtherTok{Events}\NormalTok{.}\FunctionTok{channels}\NormalTok{[topic];}
        \NormalTok{\}}
      \NormalTok{\}}
    \NormalTok{\}}
  \NormalTok{\}}
\NormalTok{\};}
\end{Highlighting}
\end{Shaded}

The Event system makes it possible for:

\begin{itemize}
\itemsep1pt\parskip0pt\parsep0pt
\item
  a View to notify its subscribers of user interaction (e.g., clicks or
  input in a form), to update/re-render its presentation, etc.
\item
  a Model whose data has changed to notify its Subscribers to update
  themselves (e.g., view to re-render to show accurate/updated data),
  etc.
\end{itemize}

\subsubsection{Models}\label{models-4}

Models manage the (domain-specific) data for an application. They are
concerned with neither the user-interface nor presentation layers, but
instead represent structured data that an application may require. When
a model changes (e.g when it is updated), it will typically notify its
observers (Subscribers) that a change has occurred so that they may
react accordingly.

Let's see a simple implementation of the Model:

\begin{Shaded}
\begin{Highlighting}[]
\CommentTok{// cranium.js - Cranium.Model}

\CommentTok{// Attributes represents data, model's properties.}
\CommentTok{// These are to be passed at Model instantiation.}
\CommentTok{// Also we are creating id for each Model instance}
\CommentTok{// so that it can identify itself (e.g. on chage}
\CommentTok{// announcements)}
\KeywordTok{var} \NormalTok{Model = }\OtherTok{Cranium}\NormalTok{.}\FunctionTok{Model} \NormalTok{= }\KeywordTok{function} \NormalTok{(attributes) \{}
    \KeywordTok{this}\NormalTok{.}\FunctionTok{id} \NormalTok{= }\OtherTok{_}\NormalTok{.}\FunctionTok{uniqueId}\NormalTok{(}\StringTok{'model'}\NormalTok{);}
    \KeywordTok{this}\NormalTok{.}\FunctionTok{attributes} \NormalTok{= attributes || \{\};}
\NormalTok{\};}

\CommentTok{// Getter (accessor) method;}
\CommentTok{// returns named data item}
\OtherTok{Cranium}\NormalTok{.}\OtherTok{Model}\NormalTok{.}\OtherTok{prototype}\NormalTok{.}\FunctionTok{get} \NormalTok{= }\KeywordTok{function}\NormalTok{(attrName) \{}
    \KeywordTok{return} \KeywordTok{this}\NormalTok{.}\FunctionTok{attributes}\NormalTok{[attrName];}
\NormalTok{\};}

\CommentTok{// Setter (mutator) method;}
\CommentTok{// Set/mix in into model mapped data (e.g.\{name: "John"\})}
\CommentTok{// and publishes the change event}
\OtherTok{Cranium}\NormalTok{.}\OtherTok{Model}\NormalTok{.}\OtherTok{prototype}\NormalTok{.}\FunctionTok{set} \NormalTok{= }\KeywordTok{function}\NormalTok{(attrs)\{}
    \KeywordTok{if} \NormalTok{(}\OtherTok{_}\NormalTok{.}\FunctionTok{isObject}\NormalTok{(attrs)) \{}
      \OtherTok{_}\NormalTok{.}\FunctionTok{extend}\NormalTok{(}\KeywordTok{this}\NormalTok{.}\FunctionTok{attributes}\NormalTok{, attrs);}
      \KeywordTok{this}\NormalTok{.}\FunctionTok{change}\NormalTok{(}\KeywordTok{this}\NormalTok{.}\FunctionTok{attributes}\NormalTok{);}
    \NormalTok{\}}
    \KeywordTok{return} \KeywordTok{this}\NormalTok{;}
\NormalTok{\};}

\CommentTok{// Returns clone of the Models data object}
\CommentTok{// (used for view template rendering)}
\OtherTok{Cranium}\NormalTok{.}\OtherTok{Model}\NormalTok{.}\OtherTok{prototype}\NormalTok{.}\FunctionTok{toJSON} \NormalTok{= }\KeywordTok{function}\NormalTok{(options) \{}
    \KeywordTok{return} \OtherTok{_}\NormalTok{.}\FunctionTok{clone}\NormalTok{(}\KeywordTok{this}\NormalTok{.}\FunctionTok{attributes}\NormalTok{);}
\NormalTok{\};}

\CommentTok{// Helper function that announces changes to the Model}
\CommentTok{// and passes the new data}
\OtherTok{Cranium}\NormalTok{.}\OtherTok{Model}\NormalTok{.}\OtherTok{prototype}\NormalTok{.}\FunctionTok{change} \NormalTok{= }\KeywordTok{function}\NormalTok{(attrs)\{}
    \KeywordTok{this}\NormalTok{.}\FunctionTok{trigger}\NormalTok{(}\KeywordTok{this}\NormalTok{.}\FunctionTok{id} \NormalTok{+ }\StringTok{'update'}\NormalTok{, attrs);}
\NormalTok{\};}

\CommentTok{// Mix in Event system}
\OtherTok{_}\NormalTok{.}\FunctionTok{extend}\NormalTok{(}\OtherTok{Cranium}\NormalTok{.}\OtherTok{Model}\NormalTok{.}\FunctionTok{prototype}\NormalTok{, }\OtherTok{Cranium}\NormalTok{.}\FunctionTok{Events}\NormalTok{);}
\end{Highlighting}
\end{Shaded}

\subsubsection{Views}\label{views-4}

Views are a visual representation of models that present a filtered view
of their current state. A view typically observes a model and is
notified when the model changes, allowing the view to update itself
accordingly. Design pattern literature commonly refers to views as
`dumb', given that their knowledge of models and controllers in an
application is limited.

Let's explore Views a little further using a simple JavaScript example:

\begin{Shaded}
\begin{Highlighting}[]
\CommentTok{// DOM View}
\KeywordTok{var} \NormalTok{View = }\OtherTok{Cranium}\NormalTok{.}\FunctionTok{View} \NormalTok{= }\KeywordTok{function} \NormalTok{(options) \{}
  \CommentTok{// Mix in options object (e.g extending functionality)}
  \OtherTok{_}\NormalTok{.}\FunctionTok{extend}\NormalTok{(}\KeywordTok{this}\NormalTok{, options);}
  \KeywordTok{this}\NormalTok{.}\FunctionTok{id} \NormalTok{= }\OtherTok{_}\NormalTok{.}\FunctionTok{uniqueId}\NormalTok{(}\StringTok{'view'}\NormalTok{);}
\NormalTok{\};}

\CommentTok{// Mix in Event system}
\OtherTok{_}\NormalTok{.}\FunctionTok{extend}\NormalTok{(}\OtherTok{Cranium}\NormalTok{.}\OtherTok{View}\NormalTok{.}\FunctionTok{prototype}\NormalTok{, }\OtherTok{Cranium}\NormalTok{.}\FunctionTok{Events}\NormalTok{);}
\end{Highlighting}
\end{Shaded}

\subsubsection{Controllers}\label{controllers-2}

Controllers are an intermediary between models and views which are
classically responsible for two tasks:

\begin{itemize}
\itemsep1pt\parskip0pt\parsep0pt
\item
  they update the view when the model changes
\item
  they update the model when the user manipulates the view
\end{itemize}

\begin{Shaded}
\begin{Highlighting}[]
\CommentTok{// cranium.js - Cranium.Controller}

\CommentTok{// Controller tying together a model and view}
\KeywordTok{var} \NormalTok{Controller = }\OtherTok{Cranium}\NormalTok{.}\FunctionTok{Controller} \NormalTok{= }\KeywordTok{function}\NormalTok{(options)\{}
  \CommentTok{// Mix in options object (e.g extending functionality)}
  \OtherTok{_}\NormalTok{.}\FunctionTok{extend}\NormalTok{(}\KeywordTok{this}\NormalTok{, options);}
  \KeywordTok{this}\NormalTok{.}\FunctionTok{id} \NormalTok{= }\OtherTok{_}\NormalTok{.}\FunctionTok{uniqueId}\NormalTok{(}\StringTok{'controller'}\NormalTok{);}
  \KeywordTok{var} \NormalTok{parts, selector, eventType;}

  \CommentTok{// Parses Events object passed during the definition of the}
  \CommentTok{// controller and maps it to the defined method to handle it;}
  \KeywordTok{if}\NormalTok{(}\KeywordTok{this}\NormalTok{.}\FunctionTok{events}\NormalTok{)\{}
    \OtherTok{_}\NormalTok{.}\FunctionTok{each}\NormalTok{(}\KeywordTok{this}\NormalTok{.}\FunctionTok{events}\NormalTok{, }\KeywordTok{function}\NormalTok{(method, eventName)\{}
      \NormalTok{parts = }\OtherTok{eventName}\NormalTok{.}\FunctionTok{split}\NormalTok{(}\StringTok{'.'}\NormalTok{);}
      \NormalTok{selector = parts[}\DecValTok{0}\NormalTok{];}
      \NormalTok{eventType = parts[}\DecValTok{1}\NormalTok{];}
      \FunctionTok{$}\NormalTok{(selector)[}\StringTok{'on'} \NormalTok{+ eventType] = }\KeywordTok{this}\NormalTok{[method];}
    \NormalTok{\}.}\FunctionTok{bind}\NormalTok{(}\KeywordTok{this}\NormalTok{));}
  \NormalTok{\}}
\NormalTok{\};}
\end{Highlighting}
\end{Shaded}

\subsubsection{Practical Usage}\label{practical-usage}

HTML template for the primer that follows:

\begin{Shaded}
\begin{Highlighting}[]
\ErrorTok{<}\NormalTok{!doctype html>}
\KeywordTok{<html}\OtherTok{ lang=}\StringTok{"en"}\KeywordTok{>}
\KeywordTok{<head>}
  \KeywordTok{<meta}\OtherTok{ charset=}\StringTok{"utf-8"}\KeywordTok{>}
  \KeywordTok{<title></title>}
  \KeywordTok{<meta}\OtherTok{ name=}\StringTok{"description"}\OtherTok{ content=}\StringTok{""}\KeywordTok{>}
\KeywordTok{</head>}
\KeywordTok{<body>}
\KeywordTok{<div}\OtherTok{ id=}\StringTok{"todo"}\KeywordTok{>}
\KeywordTok{</div>}
  \KeywordTok{<script}\OtherTok{ type=}\StringTok{"text/template"}\OtherTok{ class=}\StringTok{"todo-template"}\KeywordTok{>}
    \NormalTok{<div>}
      \NormalTok{<input id=}\StringTok{"todo_complete"} \NormalTok{type=}\StringTok{"checkbox"} \NormalTok{<%= completed %>>}
      \NormalTok{<%= title %>}
    \NormalTok{<}\OtherTok{/div>}
\OtherTok{  </script}\NormalTok{>}
  \NormalTok{<script src=}\StringTok{"underscore-min.js"}\NormalTok{>}\KeywordTok{</script>}
  \KeywordTok{<script}\OtherTok{ src=}\StringTok{"cranium.js"}\KeywordTok{></script>}
  \KeywordTok{<script}\OtherTok{ src=}\StringTok{"example.js"}\KeywordTok{></script>}
\KeywordTok{</body>}
\KeywordTok{</html>}
\end{Highlighting}
\end{Shaded}

Cranium.js usage:

\begin{Shaded}
\begin{Highlighting}[]

\CommentTok{// example.js - usage of Cranium MVC}

\CommentTok{// And todo instance}
\KeywordTok{var} \NormalTok{todo1 = }\KeywordTok{new} \OtherTok{Cranium}\NormalTok{.}\FunctionTok{Model}\NormalTok{(\{}
    \DataTypeTok{title}\NormalTok{: }\StringTok{""}\NormalTok{,}
    \DataTypeTok{completed}\NormalTok{: }\StringTok{""}
\NormalTok{\});}

\OtherTok{console}\NormalTok{.}\FunctionTok{log}\NormalTok{(}\StringTok{"First todo title - nothing set: "} \NormalTok{+ }\OtherTok{todo1}\NormalTok{.}\FunctionTok{get}\NormalTok{(}\StringTok{'title'}\NormalTok{));}
\OtherTok{todo1}\NormalTok{.}\FunctionTok{set}\NormalTok{(\{}\DataTypeTok{title}\NormalTok{: }\StringTok{"Do something"}\NormalTok{\});}
\OtherTok{console}\NormalTok{.}\FunctionTok{log}\NormalTok{(}\StringTok{"Its changed now: "} \NormalTok{+ }\OtherTok{todo1}\NormalTok{.}\FunctionTok{get}\NormalTok{(}\StringTok{'title'}\NormalTok{));}
\StringTok{''}
\CommentTok{// View instance}
\KeywordTok{var} \NormalTok{todoView = }\KeywordTok{new} \OtherTok{Cranium}\NormalTok{.}\FunctionTok{View}\NormalTok{(\{}
  \CommentTok{// DOM element selector}
  \DataTypeTok{el}\NormalTok{: }\StringTok{'#todo'}\NormalTok{,}

  \CommentTok{// Todo template; Underscore temlating used}
  \DataTypeTok{template}\NormalTok{: }\OtherTok{_}\NormalTok{.}\FunctionTok{template}\NormalTok{(}\FunctionTok{$}\NormalTok{(}\StringTok{'.todo-template'}\NormalTok{).}\FunctionTok{innerHTML}\NormalTok{),}

  \DataTypeTok{init}\NormalTok{: }\KeywordTok{function} \NormalTok{(model) \{}
    \KeywordTok{this}\NormalTok{.}\FunctionTok{render}\NormalTok{( }\OtherTok{model}\NormalTok{.}\FunctionTok{attributes} \NormalTok{);}

    \KeywordTok{this}\NormalTok{.}\FunctionTok{on}\NormalTok{(}\OtherTok{model}\NormalTok{.}\FunctionTok{id} \NormalTok{+ }\StringTok{'update'}\NormalTok{, }\KeywordTok{this}\NormalTok{.}\OtherTok{render}\NormalTok{.}\FunctionTok{bind}\NormalTok{(}\KeywordTok{this}\NormalTok{));}
  \NormalTok{\},}
  \DataTypeTok{render}\NormalTok{: }\KeywordTok{function} \NormalTok{(data) \{}
    \OtherTok{console}\NormalTok{.}\FunctionTok{log}\NormalTok{(}\StringTok{"View about to render."}\NormalTok{);}
    \FunctionTok{$}\NormalTok{(}\KeywordTok{this}\NormalTok{.}\FunctionTok{el}\NormalTok{).}\FunctionTok{innerHTML} \NormalTok{= }\KeywordTok{this}\NormalTok{.}\FunctionTok{template}\NormalTok{( data );}
  \NormalTok{\}}
\NormalTok{\});}

\KeywordTok{var} \NormalTok{todoController = }\KeywordTok{new} \OtherTok{Cranium}\NormalTok{.}\FunctionTok{Controller}\NormalTok{(\{}
  \CommentTok{// Specify the model to update}
  \DataTypeTok{model}\NormalTok{: todo1,}

  \CommentTok{// and the view to observe this model}
  \DataTypeTok{view}\NormalTok{:  todoView,}

  \DataTypeTok{events}\NormalTok{: \{}
    \StringTok{"#todo.click"} \NormalTok{: }\StringTok{"toggleComplete"}
  \NormalTok{\},}

  \CommentTok{// Initialize everything}
  \DataTypeTok{initialize}\NormalTok{: }\KeywordTok{function} \NormalTok{() \{}
    \KeywordTok{this}\NormalTok{.}\OtherTok{view}\NormalTok{.}\FunctionTok{init}\NormalTok{(}\KeywordTok{this}\NormalTok{.}\FunctionTok{model}\NormalTok{);}
    \KeywordTok{return} \KeywordTok{this}\NormalTok{;}
  \NormalTok{\},}
  \CommentTok{// Toggles the value of the todo in the Model}
  \DataTypeTok{toggleComplete}\NormalTok{: }\KeywordTok{function} \NormalTok{() \{}
    \KeywordTok{var} \NormalTok{completed = }\OtherTok{todoController}\NormalTok{.}\OtherTok{model}\NormalTok{.}\FunctionTok{get}\NormalTok{(}\StringTok{'completed'}\NormalTok{);}
    \OtherTok{console}\NormalTok{.}\FunctionTok{log}\NormalTok{(}\StringTok{"Todo old 'completed' value?"}\NormalTok{, completed);}
    \OtherTok{todoController}\NormalTok{.}\OtherTok{model}\NormalTok{.}\FunctionTok{set}\NormalTok{(\{ }\DataTypeTok{completed}\NormalTok{: (!completed) ? }\StringTok{'checked'}\NormalTok{: }\StringTok{''} \NormalTok{\});}
    \OtherTok{console}\NormalTok{.}\FunctionTok{log}\NormalTok{(}\StringTok{"Todo new 'completed' value?"}\NormalTok{, }\OtherTok{todoController}\NormalTok{.}\OtherTok{model}\NormalTok{.}\FunctionTok{get}\NormalTok{(}\StringTok{'completed'}\NormalTok{));}
    \KeywordTok{return} \KeywordTok{this}\NormalTok{;}
  \NormalTok{\}}
\NormalTok{\});}


\CommentTok{// Let's kick start things off}
\OtherTok{todoController}\NormalTok{.}\FunctionTok{initialize}\NormalTok{();}

\OtherTok{todo1}\NormalTok{.}\FunctionTok{set}\NormalTok{(\{ }\DataTypeTok{title}\NormalTok{: }\StringTok{"Due to this change Model will notify View and it will re-render"}\NormalTok{\});}
\end{Highlighting}
\end{Shaded}

Samuel Clay, one of the authors of the first version of Backbone.js says
of cranium.js: ``Unsurprisingly, it looks a whole lot like the
beginnings of Backbone. Views are dumb, so they get very little
boilerplate and setup. Models are responsible for their attributes and
announcing changes to those models.''

I hope you've found this implementation helpful in understanding how one
would go about writing their own library like Backbone from scratch, but
moreso that it encourages you to take advantage of mature existing
solutions where possible but never be afraid to explore deeper down into
what makes them tick.

\subsection{MVP}\label{mvp}

Model-View-Presenter (MVP) is a derivative of the MVC design pattern
which focuses on improving presentation logic. It originated at a
company named \href{http://en.wikipedia.org/wiki/Taligent}{Taligent} in
the early 1990s while they were working on a model for a C++ CommonPoint
environment. Whilst both MVC and MVP target the separation of concerns
across multiple components, there are some fundamental differences
between them.

For the purposes of this summary we will focus on the version of MVP
most suitable for web-based architectures.

\subsubsection{Models, Views \&
Presenters}\label{models-views-presenters}

The P in MVP stands for presenter. It's a component which contains the
user-interface business logic for the view. Unlike MVC, invocations from
the view are delegated to the presenter, which are decoupled from the
view and instead talk to it through an interface. This allows for all
kinds of useful things such as being able to mock views in unit tests.

The most common implementation of MVP is one which uses a Passive View
(a view which is for all intents and purposes ``dumb''), containing
little to no logic. MVP models are almost identical to MVC models and
handle application data. The presenter acts as a mediator which talks to
both the view and model, however both of these are isolated from each
other. They effectively bind models to views, a responsibility held by
Controllers in MVC. Presenters are at the heart of the MVP pattern and
as you can guess, incorporate the presentation logic behind views.

Solicited by a view, presenters perform any work to do with user
requests and pass data back to them. In this respect, they retrieve
data, manipulate it and determine how the data should be displayed in
the view. In some implementations, the presenter also interacts with a
service layer to persist data (models). Models may trigger events but
it's the presenter's role to subscribe to them so that it can update the
view. In this passive architecture, we have no concept of direct data
binding. Views expose setters which presenters can use to set data.

The benefit of this change from MVC is that it increases the testability
of your application and provides a more clean separation between the
view and the model. This isn't however without its costs as the lack of
data binding support in the pattern can often mean having to take care
of this task separately.

Although a common implementation of a
\href{http://martinfowler.com/eaaDev/PassiveScreen.html}{Passive View}
is for the view to implement an interface, there are variations on it,
including the use of events which can decouple the View from the
Presenter a little more. As we don't have the interface construct in
JavaScript, we're using it more and more as a protocol than an explicit
interface here. It's technically still an API and it's probably fair for
us to refer to it as an interface from that perspective.

There is also a
\href{http://martinfowler.com/eaaDev/SupervisingPresenter.html}{Supervising
Controller} variation of MVP, which is closer to the MVC and
\href{http://en.wikipedia.org/wiki/Model_View_ViewModel}{MVVM -
Model-View-ViewModel} patterns as it provides data-binding from the
Model directly from the View. Key-value observing (KVO) plugins (such as
Derick Bailey's Backbone.ModelBinding plugin) introduce this idea of a
Supervising Controller to Backbone.

\subsection{MVP or MVC?}\label{mvp-or-mvc}

MVP is generally used most often in enterprise-level applications where
it's necessary to reuse as much presentation logic as possible.
Applications with very complex views and a great deal of user
interaction may find that MVC doesn't quite fit the bill here as solving
this problem may mean heavily relying on multiple controllers. In MVP,
all of this complex logic can be encapsulated in a presenter, which can
simplify maintenance greatly.

As MVP views are defined through an interface and the interface is
technically the only point of contact between the system and the view
(other than a presenter), this pattern also allows developers to write
presentation logic without needing to wait for designers to produce
layouts and graphics for the application.

Depending on the implementation, MVP may be more easy to automatically
unit test than MVC. The reason often cited for this is that the
presenter can be used as a complete mock of the user-interface and so it
can be unit tested independent of other components. In my experience
this really depends on the languages you are implementing MVP in
(there's quite a difference between opting for MVP for a JavaScript
project over one for say, ASP.NET).

At the end of the day, the underlying concerns you may have with MVC
will likely hold true for MVP given that the differences between them
are mainly semantic. As long as you are cleanly separating concerns into
models, views and controllers (or presenters) you should be achieving
most of the same benefits regardless of the pattern you opt for.

\subsection{MVC, MVP and Backbone.js}\label{mvc-mvp-and-backbone.js}

There are very few, if any architectural JavaScript frameworks that
claim to implement the MVC or MVP patterns in their classical form as
many JavaScript developers don't view MVC and MVP as being mutually
exclusive (we are actually more likely to see MVP strictly implemented
when looking at web frameworks such as ASP.NET or GWT). This is because
it's possible to have additional presenter/view logic in your
application and yet still consider it a flavor of MVC.

Backbone contributor \href{http://ireneros.com/}{Irene Ros} subscribes
to this way of thinking as when she separates Backbone views out into
their own distinct components, she needs something to actually assemble
them for her. This could either be a controller route (such as a
\texttt{Backbone.Router}) or a callback in response to data being
fetched.

That said, some developers do however feel that Backbone.js better fits
the description of MVP than it does MVC . Their view is that:

\begin{itemize}
\itemsep1pt\parskip0pt\parsep0pt
\item
  The presenter in MVP better describes the \texttt{Backbone.View} (the
  layer between View templates and the data bound to it) than a
  controller does
\item
  The model fits \texttt{Backbone.Model} (it isn't that different from
  the classical MVC ``Model'')
\item
  The views best represent templates (e.g Handlebars/Mustache markup
  templates)
\end{itemize}

A response to this could be that the view can also just be a View (as
per MVC) because Backbone is flexible enough to let it be used for
multiple purposes. The V in MVC and the P in MVP can both be
accomplished by \texttt{Backbone.View} because they're able to achieve
two purposes: both rendering atomic components and assembling those
components rendered by other views.

We've also seen that in Backbone the responsibility of a controller is
shared with both the Backbone.View and Backbone.Router and in the
following example we can actually see that aspects of that are certainly
true.

Here, our Backbone \texttt{TodoView} uses the Observer pattern to
`subscribe' to changes to a View's model in the line
\texttt{this.listenTo(this.model, 'change',...)}. It also handles
templating in the \texttt{render()} method, but unlike some other
implementations, user interaction is also handled in the View (see
\texttt{events}).

\begin{Shaded}
\begin{Highlighting}[]
\CommentTok{// The DOM element for a todo item...}
\OtherTok{app}\NormalTok{.}\FunctionTok{TodoView} \NormalTok{= }\OtherTok{Backbone}\NormalTok{.}\OtherTok{View}\NormalTok{.}\FunctionTok{extend}\NormalTok{(\{}

  \CommentTok{//... is a list tag.}
  \DataTypeTok{tagName}\NormalTok{:  }\StringTok{'li'}\NormalTok{,}

  \CommentTok{// Pass the contents of the todo template through a templating}
  \CommentTok{// function, cache it for a single todo}
  \DataTypeTok{template}\NormalTok{: }\OtherTok{_}\NormalTok{.}\FunctionTok{template}\NormalTok{( }\FunctionTok{$}\NormalTok{(}\StringTok{'#item-template'}\NormalTok{).}\FunctionTok{html}\NormalTok{() ),}

  \CommentTok{// The DOM events specific to an item.}
  \DataTypeTok{events}\NormalTok{: \{}
    \StringTok{'click .toggle'}\NormalTok{:  }\StringTok{'togglecompleted'}
  \NormalTok{\},}

  \CommentTok{// The TodoView listens for changes to its model, re-rendering. Since there's}
  \CommentTok{// a one-to-one correspondence between a **Todo** and a **TodoView** in this}
  \CommentTok{// app, we set a direct reference on the model for convenience.}
  \DataTypeTok{initialize}\NormalTok{: }\KeywordTok{function}\NormalTok{() \{}
    \KeywordTok{this}\NormalTok{.}\FunctionTok{listenTo}\NormalTok{( }\KeywordTok{this}\NormalTok{.}\FunctionTok{model}\NormalTok{, }\StringTok{'change'}\NormalTok{, }\KeywordTok{this}\NormalTok{.}\FunctionTok{render} \NormalTok{);}
    \KeywordTok{this}\NormalTok{.}\FunctionTok{listenTo}\NormalTok{( }\KeywordTok{this}\NormalTok{.}\FunctionTok{model}\NormalTok{, }\StringTok{'destroy'}\NormalTok{, }\KeywordTok{this}\NormalTok{.}\FunctionTok{remove} \NormalTok{);}
  \NormalTok{\},}

  \CommentTok{// Re-render the titles of the todo item.}
  \DataTypeTok{render}\NormalTok{: }\KeywordTok{function}\NormalTok{() \{}
    \KeywordTok{this}\NormalTok{.}\OtherTok{$el}\NormalTok{.}\FunctionTok{html}\NormalTok{( }\KeywordTok{this}\NormalTok{.}\FunctionTok{template}\NormalTok{( }\KeywordTok{this}\NormalTok{.}\OtherTok{model}\NormalTok{.}\FunctionTok{attributes} \NormalTok{) );}
    \KeywordTok{return} \KeywordTok{this}\NormalTok{;}
  \NormalTok{\},}

  \CommentTok{// Toggle the `"completed"` state of the model.}
  \DataTypeTok{togglecompleted}\NormalTok{: }\KeywordTok{function}\NormalTok{() \{}
    \KeywordTok{this}\NormalTok{.}\OtherTok{model}\NormalTok{.}\FunctionTok{toggle}\NormalTok{();}
  \NormalTok{\},}
\NormalTok{\});}
\end{Highlighting}
\end{Shaded}

Another (quite different) opinion is that Backbone more closely
resembles
\href{http://martinfowler.com/eaaDev/uiArchs.html\#ModelViewController}{Smalltalk-80
MVC}, which we went through earlier.

As MarionetteJS author Derick Bailey has
\href{http://lostechies.com/derickbailey/2011/12/23/backbone-js-is-not-an-mvc-framework/}{written},
it's ultimately best not to force Backbone to fit any specific design
patterns. Design patterns should be considered flexible guides to how
applications may be structured and in this respect, Backbone doesn't fit
either MVC nor MVP perfectly. Instead, it borrows some of the best
concepts from multiple architectural patterns and creates a flexible
library that just works well. Call it \textbf{the Backbone way}, MV* or
whatever helps reference its flavor of application architecture.

It \emph{is} however worth understanding where and why these concepts
originated, so I hope that my explanations of MVC and MVP have been of
help. Most structural JavaScript frameworks will adopt their own take on
classical patterns, either intentionally or by accident, but the
important thing is that they help us develop applications which are
organized, clean and can be easily maintained.

\subsection{Namespacing}\label{namespacing}

When learning how to use Backbone, an important and commonly overlooked
area by tutorials is namespacing. If you already have experience with
namespacing in JavaScript, the following section will provide some
advice on how to specifically apply concepts you know to Backbone,
however I will also be covering explanations for beginners to ensure
everyone is on the same page.

\paragraph{What is namespacing?}\label{what-is-namespacing}

Namespacing is a way to avoid collisions with other objects or variables
in the global namespace. Using namespacing reduces the potential of your
code breaking because another script on the page is using the same
variable names that you are. As a good `citizen' of the global
namespace, it's also imperative that you do your best to minimize the
possibility of your code breaking other developer's scripts.

JavaScript doesn't really have built-in support for namespaces like
other languages, however it does have closures which can be used to
achieve a similar effect.

In this section we'll be taking a look shortly at some examples of how
you can namespace your models, views, routers and other components. The
patterns we'll be examining are:

\begin{itemize}
\itemsep1pt\parskip0pt\parsep0pt
\item
  Single global variables
\item
  Object Literals
\item
  Nested namespacing
\end{itemize}

\textbf{Single global variables}

One popular pattern for namespacing in JavaScript is opting for a single
global variable as your primary object of reference. A skeleton
implementation of this where we return an object with functions and
properties can be found below:

\begin{Shaded}
\begin{Highlighting}[]
\KeywordTok{var} \NormalTok{myApplication = (}\KeywordTok{function}\NormalTok{()\{}
    \KeywordTok{function}\NormalTok{()\{}
      \CommentTok{// ...}
    \NormalTok{\},}
    \KeywordTok{return} \NormalTok{\{}
      \CommentTok{// ...}
    \NormalTok{\}}
\NormalTok{\})();}
\end{Highlighting}
\end{Shaded}

You've probably seen this technique before. A Backbone-specific example
might look like this:

\begin{Shaded}
\begin{Highlighting}[]
\KeywordTok{var} \NormalTok{myViews = (}\KeywordTok{function}\NormalTok{()\{}
    \KeywordTok{return} \NormalTok{\{}
        \DataTypeTok{TodoView}\NormalTok{: }\OtherTok{Backbone}\NormalTok{.}\OtherTok{View}\NormalTok{.}\FunctionTok{extend}\NormalTok{(\{ .. \}),}
        \DataTypeTok{TodosView}\NormalTok{: }\OtherTok{Backbone}\NormalTok{.}\OtherTok{View}\NormalTok{.}\FunctionTok{extend}\NormalTok{(\{ .. \}),}
        \DataTypeTok{AboutView}\NormalTok{: }\OtherTok{Backbone}\NormalTok{.}\OtherTok{View}\NormalTok{.}\FunctionTok{extend}\NormalTok{(\{ .. \})}
        \CommentTok{//etc.}
    \NormalTok{\};}
\NormalTok{\})();}
\end{Highlighting}
\end{Shaded}

Here we can return a set of views, but the same technique could return
an entire collection of models, views and routers depending on how you
decide to structure your application. Although this works for certain
situations, the biggest challenge with the single global variable
pattern is ensuring that no one else has used the same global variable
name as you have in the page.

One solution to this problem, as mentioned by Peter Michaux, is to use
prefix namespacing. It's a simple concept at heart, but the idea is you
select a common prefix name (in this example, \texttt{myApplication\_})
and then define any methods, variables or other objects after the
prefix.

\begin{Shaded}
\begin{Highlighting}[]
\KeywordTok{var} \NormalTok{myApplication_todoView = }\OtherTok{Backbone}\NormalTok{.}\OtherTok{View}\NormalTok{.}\FunctionTok{extend}\NormalTok{(\{\}),}
    \NormalTok{myApplication_todosView = }\OtherTok{Backbone}\NormalTok{.}\OtherTok{View}\NormalTok{.}\FunctionTok{extend}\NormalTok{(\{\});}
\end{Highlighting}
\end{Shaded}

This is effective from the perspective of trying to lower the chances of
a particular variable existing in the global scope, but remember that a
uniquely named object can have the same effect. This aside, the biggest
issue with the pattern is that it can result in a large number of global
objects once your application starts to grow.

For more on Peter's views about the single global variable pattern, read
his \href{http://michaux.ca/articles/javascript-namespacing}{excellent
post on them}.

Note: There are several other variations on the single global variable
pattern out in the wild, however having reviewed quite a few, I felt the
prefixing approach applied best to Backbone.

\textbf{Object Literals}

Object Literals have the advantage of not polluting the global namespace
but assist in organizing code and parameters logically. They're
beneficial if you wish to create easily readable structures that can be
expanded to support deep nesting. Unlike simple global variables, Object
Literals often also take into account tests for the existence of a
variable by the same name, which helps reduce the chances of collision.

This example demonstrates two ways you can check to see if a namespace
already exists before defining it. I commonly use Option 2.

\begin{Shaded}
\begin{Highlighting}[]
\CommentTok{/* Doesn't check for existence of myApplication */}
\KeywordTok{var} \NormalTok{myApplication = \{\};}

\CommentTok{/*}
\CommentTok{Does check for existence. If already defined, we use that instance.}
\CommentTok{Option 1:   if(!myApplication) myApplication = \{\};}
\CommentTok{Option 2:   var myApplication = myApplication || \{\};}
\CommentTok{We can then populate our object literal to support models, views and collections (or any data, really):}
\CommentTok{*/}

\KeywordTok{var} \NormalTok{myApplication = \{}
    \DataTypeTok{models }\NormalTok{: \{\},}
    \DataTypeTok{views }\NormalTok{: \{}
        \DataTypeTok{pages }\NormalTok{: \{\}}
    \NormalTok{\},}
    \DataTypeTok{collections }\NormalTok{: \{\}}
\NormalTok{\};}
\end{Highlighting}
\end{Shaded}

One can also opt for adding properties directly to the namespace (such
as your views, in the following example):

\begin{Shaded}
\begin{Highlighting}[]
\KeywordTok{var} \NormalTok{myTodosViews = myTodosViews || \{\};}
\OtherTok{myTodosViews}\NormalTok{.}\FunctionTok{todoView} \NormalTok{= }\OtherTok{Backbone}\NormalTok{.}\OtherTok{View}\NormalTok{.}\FunctionTok{extend}\NormalTok{(\{\});}
\OtherTok{myTodosViews}\NormalTok{.}\FunctionTok{todosView} \NormalTok{= }\OtherTok{Backbone}\NormalTok{.}\OtherTok{View}\NormalTok{.}\FunctionTok{extend}\NormalTok{(\{\});}
\end{Highlighting}
\end{Shaded}

The benefit of this pattern is that you're able to easily encapsulate
all of your models, views, routers etc. in a way that clearly separates
them and provides a solid foundation for extending your code.

This pattern has a number of benefits. It's often a good idea to
decouple the default configuration for your application into a single
area that can be easily modified without the need to search through your
entire codebase just to alter it. Here's an example of a hypothetical
object literal that stores application configuration settings:

\begin{Shaded}
\begin{Highlighting}[]
\KeywordTok{var} \NormalTok{myConfig = \{}
  \DataTypeTok{language}\NormalTok{: }\StringTok{'english'}\NormalTok{,}
  \DataTypeTok{defaults}\NormalTok{: \{}
    \DataTypeTok{enableDelegation}\NormalTok{: }\KeywordTok{true}\NormalTok{,}
    \DataTypeTok{maxTodos}\NormalTok{: }\DecValTok{40}
  \NormalTok{\},}
  \DataTypeTok{theme}\NormalTok{: \{}
    \DataTypeTok{skin}\NormalTok{: }\StringTok{'a'}\NormalTok{,}
    \DataTypeTok{toolbars}\NormalTok{: \{}
      \DataTypeTok{index}\NormalTok{: }\StringTok{'ui-navigation-toolbar'}\NormalTok{,}
      \DataTypeTok{pages}\NormalTok{: }\StringTok{'ui-custom-toolbar'}
    \NormalTok{\}}
  \NormalTok{\}}
\NormalTok{\}}
\end{Highlighting}
\end{Shaded}

Note that there are really only minor syntactical differences between
the Object Literal pattern and a standard JSON data set. If for any
reason you wish to use JSON for storing your configurations instead
(e.g.~for simpler storage when sending to the back-end), feel free to.

For more on the Object Literal pattern, I recommend reading Rebecca
Murphey's
\href{http://rmurphey.com/blog/2009/10/15/using-objects-to-organize-your-code}{excellent
article on the topic}.

\textbf{Nested namespacing}

An extension of the Object Literal pattern is nested namespacing. It's
another common pattern used that offers a lower risk of collision due to
the fact that even if a top-level namespace already exists, it's
unlikely the same nested children do. For example, Yahoo's YUI uses the
nested object namespacing pattern extensively:

\begin{Shaded}
\begin{Highlighting}[]
\OtherTok{YAHOO}\NormalTok{.}\OtherTok{util}\NormalTok{.}\OtherTok{Dom}\NormalTok{.}\FunctionTok{getElementsByClassName}\NormalTok{(}\StringTok{'test'}\NormalTok{);}
\end{Highlighting}
\end{Shaded}

Yahoo's YUI uses the nested object namespacing pattern regularly and
even DocumentCloud (the creators of Backbone) use the nested namespacing
pattern in their main applications. A sample implementation of nested
namespacing with Backbone may look like this:

\begin{Shaded}
\begin{Highlighting}[]
\KeywordTok{var} \NormalTok{todoApp =  todoApp || \{\};}

\CommentTok{// perform similar check for nested children}
\OtherTok{todoApp}\NormalTok{.}\FunctionTok{routers} \NormalTok{= }\OtherTok{todoApp}\NormalTok{.}\FunctionTok{routers} \NormalTok{|| \{\};}
\OtherTok{todoApp}\NormalTok{.}\FunctionTok{model} \NormalTok{= }\OtherTok{todoApp}\NormalTok{.}\FunctionTok{model} \NormalTok{|| \{\};}
\OtherTok{todoApp}\NormalTok{.}\OtherTok{model}\NormalTok{.}\FunctionTok{special} \NormalTok{= }\OtherTok{todoApp}\NormalTok{.}\OtherTok{model}\NormalTok{.}\FunctionTok{special} \NormalTok{|| \{\};}

\CommentTok{// routers}
\OtherTok{todoApp}\NormalTok{.}\OtherTok{routers}\NormalTok{.}\FunctionTok{Workspace}   \NormalTok{= }\OtherTok{Backbone}\NormalTok{.}\OtherTok{Router}\NormalTok{.}\FunctionTok{extend}\NormalTok{(\{\});}
\OtherTok{todoApp}\NormalTok{.}\OtherTok{routers}\NormalTok{.}\FunctionTok{TodoSearch} \NormalTok{= }\OtherTok{Backbone}\NormalTok{.}\OtherTok{Router}\NormalTok{.}\FunctionTok{extend}\NormalTok{(\{\});}

\CommentTok{// models}
\OtherTok{todoApp}\NormalTok{.}\OtherTok{model}\NormalTok{.}\FunctionTok{Todo}   \NormalTok{= }\OtherTok{Backbone}\NormalTok{.}\OtherTok{Model}\NormalTok{.}\FunctionTok{extend}\NormalTok{(\{\});}
\OtherTok{todoApp}\NormalTok{.}\OtherTok{model}\NormalTok{.}\FunctionTok{Notes} \NormalTok{= }\OtherTok{Backbone}\NormalTok{.}\OtherTok{Model}\NormalTok{.}\FunctionTok{extend}\NormalTok{(\{\});}

\CommentTok{// special models}
\OtherTok{todoApp}\NormalTok{.}\OtherTok{model}\NormalTok{.}\OtherTok{special}\NormalTok{.}\FunctionTok{Admin} \NormalTok{= }\OtherTok{Backbone}\NormalTok{.}\OtherTok{Model}\NormalTok{.}\FunctionTok{extend}\NormalTok{(\{\});}
\end{Highlighting}
\end{Shaded}

This is readable, clearly organized, and is a relatively safe way of
namespacing your Backbone application. The only real caveat however is
that it requires your browser's JavaScript engine to first locate the
todoApp object, then dig down until it gets to the function you're
calling. However, developers such as Juriy Zaytsev (kangax) have tested
and found the performance differences between single object namespacing
vs the `nested' approach to be quite negligible.

\textbf{What does DocumentCloud use?}

In case you were wondering, here is the original DocumentCloud (remember
those guys that created Backbone?) workspace that uses namespacing in a
necessary way. This approach makes sense as their documents (and
annotations and document lists) are embedded on third-party news sites.

\begin{Shaded}
\begin{Highlighting}[]

\CommentTok{// Provide top-level namespaces for our javascript.}
\NormalTok{(}\KeywordTok{function}\NormalTok{() \{}
  \OtherTok{window}\NormalTok{.}\FunctionTok{dc} \NormalTok{= \{\};}
  \OtherTok{dc}\NormalTok{.}\FunctionTok{controllers} \NormalTok{= \{\};}
  \OtherTok{dc}\NormalTok{.}\FunctionTok{model} \NormalTok{= \{\};}
  \OtherTok{dc}\NormalTok{.}\FunctionTok{app} \NormalTok{= \{\};}
  \OtherTok{dc}\NormalTok{.}\FunctionTok{ui} \NormalTok{= \{\};}
\NormalTok{\})();}
\end{Highlighting}
\end{Shaded}

As you can see, they opt for declaring a top-level namespace on the
\texttt{window} called \texttt{dc}, a short-form name of their app,
followed by nested namespaces for the controllers, models, UI and other
pieces of their application.

\textbf{Recommendation}

Reviewing the namespace patterns above, the option that I prefer when
writing Backbone applications is nested object namespacing with the
object literal pattern.

Single global variables may work fine for applications that are
relatively trivial. However, larger codebases requiring both namespaces
and deep sub-namespaces require a succinct solution that's both readable
and scalable. I feel this pattern achieves both of these objectives and
is a good choice for most Backbone development.

\subsection{Backbone Dependency
Details}\label{backbone-dependency-details}

The following sections provide insight into how Backbone uses
jQuery/Zepto and Underscore.js.

\subsubsection{DOM Manipulation}\label{dom-manipulation}

Although most developers won't need it, Backbone does support setting a
custom DOM library to be used instead of these options. From the source:

\begin{verbatim}
// For Backbone's purposes, jQuery, Zepto, Ender, or My Library (kidding) owns
// the `$` variable.
 Backbone.$ = root.jQuery || root.Zepto || root.ender || root.$;
\end{verbatim}

So, setting \texttt{Backbone.\$ = myLibrary;} will allow you to use any
custom DOM-manipulation library in place of the jQuery default.

\subsubsection{Utilities}\label{utilities}

Underscore.js is heavily used in Backbone behind the scenes for
everything from object extension to event binding. As the entire library
is generally included, we get free access to a number of useful
utilities we can use on Collections such as filtering
\texttt{\_.filter()}, sorting \texttt{\_.sortBy()}, mapping
\texttt{\_.map()} and so on.

From the source:

\begin{verbatim}
// Underscore methods that we want to implement on the Collection.
// 90% of the core usefulness of Backbone Collections is actually implemented
// right here:
var methods = ['forEach', 'each', 'map', 'collect', 'reduce', 'foldl', 'inject', 'reduceRight', 'foldr', 'find', 'detect', 'filter', 'select', 'reject', 'every', 'all', 'some', 'any', 'include', 'contains', 'invoke', 'max', 'min', 'toArray', 'size', 'first', 'head', 'take', 'initial', 'rest', 'tail', 'drop', 'last', 'without', 'indexOf', 'shuffle', 'lastIndexOf', 'isEmpty', 'chain'];

// Mix in each Underscore method as a proxy to `Collection#models`.
_.each(methods, function(method) {
    Collection.prototype[method] = function() {
        var args = slice.call(arguments);
        args.unshift(this.models);
        return _[method].apply(_, args);
    };
});
\end{verbatim}

However, for a complete linked list of methods supported, see the
\href{http://backbonejs.org/\#Collection-Underscore-Methods}{official
documentation}.

\subsubsection{RESTful persistence}\label{restful-persistence-1}

Models and collections in Backbone can be ``sync''ed with the server
using the \texttt{fetch}, \texttt{save} and \texttt{destroy} methods.
All of these methods delegate back to the \texttt{Backbone.sync}
function, which actually wraps jQuery/Zepto's \texttt{\$.ajax} function,
calling GET, POST and DELETE for the respective persistence methods on
Backbone models.

From the the source for \texttt{Backbone.sync}:

\begin{verbatim}
var methodMap = {
  'create': 'POST',
  'update': 'PUT',
  'patch':  'PATCH',
  'delete': 'DELETE',
  'read':   'GET'
};

Backbone.sync = function(method, model, options) {
    var type = methodMap[method];

    // ... Followed by lots of Backbone.js configuration, then..

    // Make the request, allowing the user to override any Ajax options.
    var xhr = options.xhr = Backbone.ajax(_.extend(params, options));
    model.trigger('request', model, xhr, options);
    return xhr;
\end{verbatim}

\subsubsection{Routing}\label{routing}

Calls to \texttt{Backbone.History.start} rely on jQuery/Zepto binding
\texttt{popState} or \texttt{hashchange} event listeners back to the
window object.

From the source for \texttt{Backbone.history.start}:

\begin{verbatim}
      // Depending on whether we're using pushState or hashes, and whether
      // 'onhashchange' is supported, determine how we check the URL state.
      if (this._hasPushState) {
          Backbone.$(window)
              .on('popstate', this.checkUrl);
      } else if (this._wantsHashChange && ('onhashchange' in window) && !oldIE) {
          Backbone.$(window)
              .on('hashchange', this.checkUrl);
      } else if (this._wantsHashChange) {
          this._checkUrlInterval = setInterval(this.checkUrl, this.interval);
      }
      ...
\end{verbatim}

\texttt{Backbone.History.stop} similarly uses your DOM manipulation
library to unbind these event listeners.

\subsection{Backbone Vs. Other Libraries And
Frameworks}\label{backbone-vs.-other-libraries-and-frameworks}

Backbone is just one of many different solutions available for
structuring your application and we're by no means advocating it as the
be all and end all. It's served the authors of this book well in
building many simple and complex web applications and we hope that it
can serve you equally as well. The answer to the question `Is Backbone
better than X?' generally has a lot more to do with what kind of
application you're building.

AngularJS and Ember.js are examples of powerful alternatives but differ
from Backbone in that there are more opinionated. For some projects this
can be useful and for others, perhaps not. The important thing to
remember is that there is no library or framework that's going to be the
best solution for every use-case and so it's important to learn about
the tools at your disposal and decide which one is best on a
project-by-project basis.

Choose the right tool for the right job. This is why we recommend
spending some time doing a little due diligence. Consider productivity,
ease of use, testability, community and documentation. If you're looking
for more concrete comparisons between frameworks, read:

\begin{itemize}
\itemsep1pt\parskip0pt\parsep0pt
\item
  \href{http://coding.smashingmagazine.com/2012/07/27/journey-through-the-javascript-mvc-jungle/}{Journey
  Through The JavaScript MVC Jungle}
\item
  \href{http://blog.stevensanderson.com/2012/08/01/rich-javascript-applications-the-seven-frameworks-throne-of-js-2012/}{Throne
  of JS - Seven JavaScript Frameworks}
\end{itemize}

The authors behind Backbone.js, AngularJS and Ember have also discussed
some of the strengths and weaknesses of their solutions on Quora,
StackOverflow and so on:

\begin{itemize}
\itemsep1pt\parskip0pt\parsep0pt
\item
  \href{http://backbonejs.org/\#FAQ-why-backbone}{Jeremy Ashkenas on Why
  Backbone?}
\item
  \href{http://www.quora.com/Ember-js/Which-one-of-angular-js-and-ember-js-is-the-better-choice/answer/Tom-Dale}{Tom
  Dale on Why Ember.js vs.~AngularJS}
\item
  \href{http://www.reddit.com/r/javascript/comments/17h22w/an_introduction_to_angular_for_backbone_developers/}{Brian
  Ford \& Jeremy Ashkenas on Backbone vs.~Angular (discussion)}
\end{itemize}

The solution you opt for may need to support building non-trivial
features and could end up being used to maintain the app for years to
come so think about things like:

\textbf{What is the library/framework really capable of?}

Spend time reviewing both the source code of the framework and official
list of features to see how well they fit with your requirements. There
will be projects that may require modifying or extending the underlying
source and thus make sure that if this might be the case, you've
performed due diligence on the code. Has the framework been proven in
production?

i.e Have developers actually built and deployed large applications with
it that are publicly accessible? Backbone has a strong portfolio of
these (SoundCloud, LinkedIn, Walmart) but not all libraries and
frameworks do. Ember is used in number of large apps, including the new
version of ZenDesk. AngularJS has been used to build the YouTube app for
PS3 amongst other places. It's not only important to know that a library
or framework works in production, but also being able to look at real
world code and be inspired by what can be built with it.

\textbf{Is the framework mature?}

I generally recommend developers don't simply ``pick one and go with
it''. New projects often come with a lot of buzz surrounding their
releases but remember to take care when selecting them for use on a
production-level app. You don't want to risk the project being canned,
going through major periods of refactoring or other breaking changes
that tend to be more carefully planned out when a framework is mature.
Mature projects also tend to have more detailed documentation available,
either as a part of their official or community-driven docs.

\textbf{Is the framework flexible or opinionated?}

Know what flavor you're after as there are plenty of frameworks
available which provide one or the other. Opinionated frameworks lock
(or suggest) you to do things in a specific way (theirs). By design they
are limiting, but place less emphasis on the developer having to figure
out how things should work on their own. Have you really played with the
framework?

Write a small application without using frameworks and then attempt to
refactor your code with a framework to confirm whether it's easy to work
with or not. As much as researching and reading up on code will
influence your decision, it's equally as important to write actual code
using the framework to make sure you're comfortable with the concepts it
enforces.

\textbf{Does the framework have a comprehensive set of documentation?}

Although demo applications can be useful for reference, you'll almost
always find yourself consulting the official framework docs to find out
what its API supports, how common tasks or components can be created
with it and what the gotchas worth noting are. Any framework worth it's
salt should have a detailed set of documentation which will help guide
developers using it. Without this, you can find yourself heavily relying
on IRC channels, groups and self-discovery, which can be fine, but are
often overly time-consuming when compared to a great set of docs
provided upfront.

\textbf{What is the total size of the framework, factoring in
minification, gzipping and any modular building that it supports?}

What dependencies does the framework have? Frameworks tend to only list
the total filesize of the base library itself, but don't list the sizes
of the library's dependencies. This can mean the difference between
opting for a library that initially looks quite small, but could be
relatively large if it say, depends on jQuery and other libraries.

\textbf{Have you reviewed the community around the framework?}

Is there an active community of project contributors and users who would
be able to assist if you run into issues? Have enough developers been
using the framework that there are existing reference applications,
tutorials and maybe even screencasts that you can use to learn more
about it?

\begin{center}\rule{3in}{0.4pt}\end{center}

Where relevant, copyright Addy Osmani, 2012-2013.

\end{document}
